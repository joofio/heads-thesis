%% UP THESIS STYLE for LaTeX2e
%% how to use upteses for MIMED dissertations
%%
%% PLEASE send improvements to jlopes at fe.up.pt and to jcf at fe.up.pt
%%

%%========================================
%% Commands: pdflatex thesis
%%           bibtex thesis
%%           makeindex thesis (only if creating an index) 
%%           pdflatex thesis
%% Alternative:
%%          latexmk -pdf thesis.tex
%%========================================

\documentclass[11pt,a4paper,twoside,openright]{report}
\usepackage[utf8]{inputenc}
%\usepackage[latin1]{inputenc}
%\usepackage[acronym]{glossaries}
\usepackage[nonumberlist]{glossaries}

\usepackage[backend=biber, style=numeric]{biblatex}
\addbibresource{references.bib}
\addbibresource{full-thesis.bib}
%%%%%%% English version 

%% MIMED options
\usepackage[mimed]{styles/upteses}                   % work version
%\usepackage[mimed,juri]{styles/upteses}             % juri verrion
%\usepackage[mimed,final]{styles/upteses}            % final version

%% Additional options for upteses.sty: 
%% - portugues: titles, etc in portuguese
%% - onpaper: links are not shown (for paper versions)
%% - backrefs: include back references from bibliography to citation place

%%%%%%% Portuguese version

%\usepackage[mimed,portugues]{upteses}        % work version
%\usepackage[mimed,portugues,juri]{upteses}   % juri version
%\usepackage[mimed,portugues,final]{upteses}  % final version

%\usepackage{pdfpages}

%% Uncomment the next lines if side by side graphics used
%\usepackage[lofdepth,lotdepth]{subfig}
%\usepackage{datagidx}
\usepackage{longtable}
\usepackage{xltabular}
\usepackage[inline]{enumitem}
\usepackage{verbatimbox}
\usepackage{lscape} 
\usepackage{makecell}
\usepackage[svgnames, table]{xcolor}
\usepackage{courier}
\usepackage{textcomp}%
\usepackage{manyfoot}%
\usepackage{booktabs}%
\usepackage{footnote}
\usepackage{color,soul}
\usepackage{algorithm2e}
\usepackage{array}
\usepackage{xurl}
\usepackage{caption}
\usepackage{colortbl}
\usepackage{float}
\usepackage{subcaption}
%%%% similarity
\usepackage{graphicx}%
\usepackage{multirow}%
\usepackage{amsmath,amssymb,amsfonts}%
\usepackage{amsthm}%
\usepackage{mathrsfs}%
\usepackage[title]{appendix}%
\usepackage{xcolor}%
\usepackage{listings}%
\usepackage{tabularx}
\usepackage{acro}
\usepackage{setspace}
%% Include color package
\usepackage{color}
\definecolor{cloudwhite}{cmyk}{0,0,0,0.025}
\usepackage{enumitem}
\usepackage[helvetica]{quotchap}  % the quotchap package to get fancy chapter styles
\usepackage{titlesec}
\usepackage{adjustbox}
%% Include source-code listings package
\usepackage{listings}
\lstset{ %
 language=C,                        % choose the language of the code
 basicstyle=\footnotesize\ttfamily,
 keywordstyle=\bfseries,
 numbers=left,                      % where to put the line-numbers
 numberstyle=\scriptsize\texttt,    % the size of the fonts that are used for the line-numbers
 stepnumber=1,                      % the step between two line-numbers. If it's 1 each line will be numbered
 numbersep=8pt,                     % how far the line-numbers are from the code
 frame=tb,
 float=htb,
 aboveskip=8mm,
 belowskip=4mm,
 backgroundcolor=\color{cloudwhite},
 showspaces=false,                  % show spaces adding particular underscores
 showstringspaces=false,            % underline spaces within strings
 showtabs=false,                    % show tabs within strings adding particular underscores
 tabsize=2,	                    % sets default tabsize to 2 spaces
 captionpos=b,                      % sets the caption-position to bottom
 breaklines=true,                   % sets automatic line breaking
 breakatwhitespace=false,           % sets if automatic breaks should only happen at whitespace
 escapeinside={\%*}{*)},            % if you want to add a comment within your code
 morekeywords={*,var,template,new}  % if you want to add more keywords to the set
}

%% Uncomment to create an index (at the end of the document)
%\makeindex

%% Path to the figures directory
%% TIP: use folder ``figures'' to keep all your figures
\graphicspath{{figures/}}

%%----------------------------------------
%% TIP: if you want to define more macros, use an external file to keep them


%%----------------------------------------
%% singular %%


\DeclareAcronym{fhir}{
short = FHIR ,
long = Fast Healthcare Interoperability Resources
}

\DeclareAcronym{hl7}{
short = HL7 ,
long = Health Level Seven
}
\DeclareAcronym{ehr}{
short = EHR ,
long = Electronic health record
}

\DeclareAcronym{ai}{
short = AI ,
long = Artifical Intelligence
}

\DeclareAcronym{api}{
short = API ,
long = Application Programming Interface
}

\DeclareAcronym{auroc}{
short = AUROC ,
long = Area Under the Receiver Operating Characteristic Curve 
}
\DeclareAcronym{roc}{
short = ROC ,
long =  Receiver Operating Characteristic 
}


\DeclareAcronym{bmi}{
short = BMI ,
long = Body Mass Index
}

\DeclareAcronym{iqr}{
short = IQR ,
long = Inter-Quartile Range
}
\DeclareAcronym{gan}{
short = GAN ,
long = Generative Adversarial Network
}

\DeclareAcronym{js}{
short = JS ,
long = Jensen-Shannon Divergence
}
\DeclareAcronym{ks}{
short = KS ,
long = \textit{Kolmogorov-Smirnov}
}

\DeclareAcronym{auprc}{
short = AUPRC ,
long = Area Under the Precision Recall  Curve 
}

\DeclareAcronym{mre}{
short = MRE ,
long = Mean Relative Error
}

\DeclareAcronym{pca}{
short = PCA ,
long = Principal Component Analysis
}
\DeclareAcronym{ml}{
short = ML,
long = Machine Learning
}
\DeclareAcronym{gdpr}{
short = GDPR,
long = General Data Protection Regulation
}

\DeclareAcronym{eu}{
short = EU,
long = European Union
}

\DeclareAcronym{hipaa}{
short = HIPAA,
long = Health Insurance Portability and Accountability Act
}
\DeclareAcronym{his}{
short = HIS,
long = Health Information System
}

\DeclareAcronym{NDCG}{
short = NDCG,
long = Normalized Discounted Cumulative Gain
}

\DeclareAcronym{DCG}{
short = DCG,
long = Discounted Cumulative Gain 
}

\DeclareAcronym{IDCG}{
short = IDCG,
long = Ideal Discounted Cumulative Gain
}

\DeclareAcronym{rbo}{
short = RBO,
long = Rank-biased overlap
}
\DeclareAcronym{IKNL}{
short = IKNL,
long = Integraal Kankercentrum Nederland
}

\DeclareAcronym{MHRA}{
short = MHRA,
long = Healthcare Products Regulatory Agency
}


\DeclareAcronym{us}{
short = USA,
long = United States of America
}

\DeclareAcronym{svm}{
short = SVM,
long = Support Vector Machines
}

\DeclareAcronym{smote}{
short = SMOTE,
long = Synthetic Minority Oversampling Technique
}
\DeclareAcronym{rmse}{
short = RMSE,
long = Root Mean Squared Error
}

\DeclareAcronym{mae}{
short = MAE,
long = Mean Absolute Error
}


\DeclareAcronym{cv}{
short = CV,
long = Cross-Validation
}

\DeclareAcronym{wcss}{
short =  WCSS,
long = within-cluster sum of squares
}

\DeclareAcronym{ri}{
short =  RI,
long =  Rand Index
}
\DeclareAcronym{cs}{
short =  C-Section,
long =  Cesarean Section
}
\DeclareAcronym{knn}{
short =  KNN,
long =  K-Nearest Neighbours
}

\DeclareAcronym{atc}{
short =  ATC,
long =  Anatomical Therapeutic Chemical
}

\DeclareAcronym{kdd}{
short =  KDD,
long =  Knowledge Discovery in Databases
}

\DeclareAcronym{ebm}{
short =  EBM,
long =  Evidence Based Medicine
}
\DeclareAcronym{rct}{
short =  RCT,
long =  Randomized Clinical Trial
}
\DeclareAcronym{rwd}{
short =  RWD,
long =  Real World Data
}
\DeclareAcronym{xai}{
short =  XAI,
long =  Explainable AI
}
\DeclareAcronym{shap}{
short =  SHAP,
long =  SHapley Additive exPlanations
}


\DeclareAcronym{lime}{
short =  LIME,
long =  Local Interpretable Model-Agnostic Explanation
}
\DeclareAcronym{bn}{
short =  BN,
long =  Bayesian Network
}

\DeclareAcronym{cml}{
short =  CausalML,
long =  Causal Machine Learning
}

\DeclareAcronym{eda}{
short =  EDA,
long =  Exploratory Data Analysis
}
\DeclareAcronym{uci}{
short =  UCI,
long =  UC Irvine Machine Learning Repository
}


\newglossaryentry{Electronic Health Record}{
  name={Electronic Health Record},
  sort={Electronic Health Record},
  description={is a digital version of a patient's medical history, maintained by the provider over time, that includes all key administrative clinical data relevant to that person's care}
}

\newglossaryentry{Bayesian Network}{
  name={Bayesian Network},
  sort={Bayesian Network},
  description={ is a probabilistic graphical model that represents a set of variables and their conditional dependencies via a directed acyclic graph}
}

\newglossaryentry{Inverse Probability of Treatment Weighting}{
  name={Inverse Probability of Treatment Weighting},
  sort={Inverse Probability of Treatment Weighting},
  description={is a statistical technique used in observational studies to adjust for confounding, where each subject is weighted inversely to the probability of receiving the treatment they actually received}
}

\newglossaryentry{Cox Regression}{
  name={Cox Regression},
  sort={Cox Regression},
  description={ is a statistical method for investigating the effect of several variables on the time a specified event takes to happen, often used in survival analysis}
}

\newglossaryentry{Causality}{
  name={Causality},
  sort={Causality},
  description={ is the relationship between causes and effects, implying that an action or event can produce a specific outcome}
}

\newglossaryentry{FHIR}{
  name={FHIR},
  sort={FHIR},
  description={ is a standard for exchanging healthcare information electronically, focusing on ease of implementation and interoperability}
}


\newglossaryentry{Clinical Decision Support System }{
  name={Clinical Decision Support System },
  sort={Clinical Decision Support System },
  description={ is a health information technology system designed to assist healthcare providers in making evidence-based clinical decisions}
}


\newglossaryentry{Explainable AI}{
  name={Explainable AI},
  sort={Explainable AI},
  description={refers to artificial intelligence and machine learning techniques that provide human-understandable explanations of their operations and decisions}
}


\newglossaryentry{Evidence-Based Medicine}{
  name={Evidence-Based Medicine},
  sort={Evidence-Based Medicine},
  description={is a clinical discipline that emphasizes the use of empirical evidence from clinical research to inform medical decision-making}
}



\newglossaryentry{Generative Adversarial Networks}{
  name={Generative Adversarial Networks},
  sort={Generative Adversarial Networks},
  description={are a class of machine learning frameworks where two neural networks, the generator and the discriminator, are trained simultaneously in a competitive manner, with the generator creating data samples and the discriminator evaluating them}
}

\newglossaryentry{Area Under the Receiver Operating Characteristic Curve}{
  name={Area Under the Receiver Operating Characteristic Curve},
  sort={Area Under the Receiver Operating Characteristic Curve},
  description={is a performance measurement for classification problems at various thresholds settings, representing the degree or measure of separability between classes by plotting the true positive rate against the false positive rate}
}

\newglossaryentry{Area Under the Precision-Recall Curve}{
  name={Area Under the Precision-Recall Curve},
  sort={Area Under the Precision-Recall Curve},
  description={is a metric used in binary classification tasks which evaluates the trade-off between precision and recall for different thresholds, particularly useful in datasets with a significant imbalance between classes}
}

\newglossaryentry{Root Mean Square Error}{
  name={Root Mean Square Error},
  sort={Root Mean Square Error},
  description={is a standard way to measure the error of a model in predicting quantitative data, representing the square root of the average squared differences between predicted values and actual values}
}

\newglossaryentry{Average Treatment Effect on the Treated}{
  name={Average Treatment Effect on the Treated},
  sort={Average Treatment Effect on the Treated},
  description={is a statistical measure that estimates the average effect of a treatment on those individuals who have received the treatment, as opposed to comparing them with a control group who did not receive the treatment}
}

\newglossaryentry{Average Treatment Effect}{
  name={Average Treatment Effect},
  sort={Average Treatment Effect},
  description={is a measure used in statistics and econometrics to estimate the mean effect of a treatment (or intervention) compared to a control condition across an entire population}
}
\newglossaryentry{F1 score}{
  name={F1 score},
  sort={F1 score},
  description={ is the harmonic mean of the precision and recall. It thus symmetrically represents both precision and recall in one metric.}
}


%some macro definitions

% format
\newcommand{\class}[1]{{\normalfont\slshape #1\/}}

% entities
\newcommand{\Up}{Universidade do Porto}

\newcommand{\svg}{\class{SVG}}
\newcommand{\scada}{\class{SCADA}}
\newcommand{\scadadms}{\class{SCADA/DMS}}

\newenvironment{myitemize}{
    \begin{itemize}[nolistsep]
}{
    \end{itemize}
}
\definecolor{SchoolColor}{rgb}{0.6471, 0.1098, 0.1882} % Crimson
\definecolor{chaptergrey}{rgb}{0.6471, 0.1098, 0.1882} % for chapter numbers
\definecolor{chaptergrey}{HTML}{8087b5}

\definecolor{mygrey}{RGB}{250, 250, 250}


\hypersetup{
 colorlinks,
 citecolor=chaptergrey,
 linkcolor=chaptergrey,
 urlcolor=Blue}


% some definitions
\def\degreeyear#1{\gdef\@degreeyear{#1}}
\def\degreemonth#1{\gdef\@degreemonth{#1}}
\def\degree#1{\gdef\@degree{#1}}
\def\advisor#1{\gdef\@advisor{#1}}
\def\department#1{\gdef\@department{#1}}
\def\field#1{\gdef\@field{#1}}
\def\university#1{\gdef\@university{#1}}
\def\universitycity#1{\gdef\@universitycity{#1}}
\def\universitystate#1{\gdef\@universitystate{#1}}
\def\programname#1{\gdef\@programname{#1}}
\def\pdOneName#1{\gdef\@pdOneName{#1}}
\def\pdOneSchool#1{\gdef\@pdOneSchool{#1}}
\def\pdOneYear#1{\gdef\@pdOneYear{#1}}
\def\pdTwoName#1{\gdef\@pdTwoName{#1}}
\def\pdTwoSchool#1{\gdef\@pdTwoSchool{#1}}
\def\pdTwoYear#1{\gdef\@pdTwoYear{#1}}


\newcommand{\copyrightpage}{
	\newpage
	\thispagestyle{empty}
	\vspace*{\fill}
	\scshape \noindent \small \copyright \small \degreeyear \hspace{3pt}-- \theauthor \\
	\noindent all rights reserved.
	\vspace*{\fill}
	\newpage
	\rm
}

% Define new section
\titleclass{\subsubsubsection}{straight}[\subsubsection]

\newcounter{subsubsubsection}
\renewcommand\thesubsubsubsection{\thesubsubsection.\arabic{subsubsubsection}}
\titleformat{\subsubsubsection}
  {\normalfont\normalsize\bfseries}{\thesubsubsubsection}{1em}{}
\titlespacing*{\subsubsubsection}
  {0pt}{3.25ex plus 1ex minus .2ex}{1.5ex plus .2ex}

% Update the list of contents to include the new level
\makeatletter
\def\toclevel@subsubsubsection{5}
\def\l@subsubsubsection{\@dottedtocline{5}{8.0em}{5.1em}}
\makeatother

\setcounter{secnumdepth}{5} % How deep to number within sections
\setcounter{tocdepth}{5} % How deep to show in table of contents


%%========================================
%% Start of document
%%========================================
\begin{document}
%\input{personalize}
\onehalfspacing
%%----------------------------------------
%% Information about the work
%%----------------------------------------
% \title{Knowledge Discovery in Healthcare: Exploring the role of real-world data to leverage clinical practice.}
\title{Knowledge Discovery in Healthcare: Exploring the role of real-world data to leverage clinical practice}
%Advancing Healthcare Outcomes: A Multidimensional Analysis of Distributed Learning, Synthetic Data, Clinical Decision Support Systems, Data Quality, and Real-World Drug Efficacy

\author{João Filipe Coutinho de Almeida}

%% Uncomment next line for date of submission
%\pdisdate{July 31, 2008}
%\thesisdate{July 31, 2008}

%%Uncomment next line for copyright text if used
\copyrightnotice{João Almeida, 2023}

\supervisor{Supervisor}{Pedro Pereira Rodrigues}
\supervisor{Second Supervisor}{Ricardo Correia}

%% Uncomment committee stuff in the final version 
%\committeetext{Approved in oral examination by the committee:}
%\committeemember{Chair}{Prof.\ Name of the President}
%\committeemember{External Examiner}{Prof.\  Name of the Examiner}
%\committeemember{Supervisor}{Prof.\ Name of the Supervisor}

%\committeetext{Aprovado em provas públicas pelo Júri:}
%\committeemember{Presidente}{Prof.\ Nome do presidente do júri}
%\committeemember{Arguente}{Prof.\ Nome do arguente do júri}
%\committeemember{Vogal}{Prof.\ Nome do vogal do júri}

%% Specify cover logo (in folder ``figures'')
% \logo{uporto-feup.pdf}
\logo{uporto-fmup.png}
\logo{fmup_heads.png}

%% Uncomment next line for additional text below the author's name (front page)
\additionalfronttext{Preparação da Dissertação}

%%----------------------------------------
%% Preliminary materials
%%----------------------------------------

% remove unnecssary \include{} commands
\begin{Prolog}

  \chapter*{Cover page}
% \chaptermark{COVER PAGE}  % the cover page
%  \copyrightpage harvard style

  \chapter*{Integrity Declaration}
% \chaptermark{INTEGRITY DECLARATION}
%necessario?
I declare on my honour that what is written in this work has been written exclusively by me and that, excluding quotations, no part has been copied from scientific publications, Internet (any type of programs, set of tools and others included) or research works - or, more generally, any other source - already presented in the academic field (but not only) by me, other students or third parties.  % the declaration of integrity
  \chapter*{Reproducibility}
% \chaptermark{REPRODUCIBILITY}


The code for all the experiments conducted in this thesis is available online on GitHub at the following link: \url{https://github.com/joofio/heads-thesis}. From there, you can access the list of all repositories involved in this thesis. Data is also available where possible; however, most of the data used was directly retrieved from Electronic Health Records and is restricted from sharing with third parties by ethical committees.  % the declaration of reproducibility
  \cleardoublepage
\thispagestyle{plain}

\vspace*{20cm}

\begin{flushright}
    {\calligra  To my and my}


\end{flushright}

%qcr
%frc % personal acknowledgments
  \chapter*{List of Publications}
% \chaptermark{REPRODUCIBILITY}



\textbf{Core Research Papers} \\
The 8 papers described below are the core structure of this thesis (3 were already published, 3 are accepted and waiting for publication and 2 are under review). The manuscripts are listed by order of appearance in the thesis.
\\
\begin{itemize}
    \item Coutinho-Almeida, J., Rodrigues, P., \& Cruz-Correia, R. (2021). GANs for Tabular Healthcare Data Generation: A Review on Utility and Privacy. In Discovery Science (pp. 282-291). Springer International Publishing.

    \item Coutinho-Almeida, J., Cruz-Correia, R., \& Rodrigues, P. (2022). Dataset Comparison Tool: Utility and Privacy. Stud Health Technol Inform, 294, 23-27.
    

    \item (accepted) Using Machine Learning Models' feature importance to assess dataset similarity 


    \item (in review) Development and Validation of a Data Quality Evaluation Tool in Obstetrics Real-World Data through \ac{hl7} \ac{fhir} interoperable Bayesian Networks and Expert Rules
    
    \item (accepted) Coutinho-Almeida, J., Cardoso, A., Cruz-Correia, R., \& Pereira-Rodrigues, P. (2024). Evaluating distributed-learning on real-world obstetrics data: Comparing distributed, centralised and local models Scientific Reports 
    
    \item (in review) Benchmarking institutions' health outcomes with clustering methods 

    
    \item (accepted) Comparative Analysis of Palbociclib and Ribociclib: A real world data and Propensity Score-Adjusted Evaluation with endocrine therapy
     
    
    \item Coutinho-Almeida, J., Cardoso, A., Cruz-Correia, R., \& Pereira-Rodrigues, P. (2024). Fast Healthcare Interoperability Resources-Based Support System for Predicting Delivery Type: Model Development and Evaluation Study. JMIR Formative Research, 
    8, e54109. https://doi.org/10.2196/54109

\end{itemize}



\textbf{Other Publications and activities}\\
In addition, during the duration of this thesis conduction, the candidate was also the author and co-author of other papers. Although these studies were not part of the thesis core structure, they were important to improve the researcher's knowledge of the field and/or to present the results to the community. They are listed below:\\

\begin{itemize}

\item Coutinho-Almeida, J., \& Cruz-Correia, R. (2022). Developing a Process Mining Tool Based on HL7. Procedia Computer Science, 196, 501-508.\\


\item Holmgren, A., Esdar, M., Hüsers, J., \& Coutinho-Almeida, J. (2023). Health Information Exchange: Understanding the Policy Landscape and Future of Data Interoperability. Yearbook of Medical Informatics, s-0043-1768719.\\



\item Costa, P., Almeida, J., Araujo, S., Alves, P., Cruz-Correia, R., Saranto, K., \& Mantas, J. (2023). Biomedical and Health Informatics Teaching in Portugal: Current Status. Heliyon, 9(3).\\

\end{itemize} %% list of my pubs
  \chapter*{Abstract}

This thesis delves into the intricate process of extracting knowledge from healthcare data, a task fraught with challenges yet brimming with potential. Central to our investigation is the acknowledgment, inspired by Richard P. Feynman, that absolute certainty is elusive in scientific inquiry; instead, our journey is marked by continual learning and improvement. We confront various obstacles, including data accessibility, quality concerns, and the integration of real-world evidence (RWE) into clinical practice, while also exploring innovative solutions like synthetic data and distributed data analysis paradigms.

A significant portion of our work is dedicated to addressing the dual challenges of data access and quality. The stringent requirements of ethics committees and Data Protection Officers (DPOs), designed to safeguard patient privacy, often impede timely data access. We propose synthetic data as a potential workaround, offering a secure and legal avenue for algorithm development and testing. Additionally, we underscore the importance of distributed data analysis, allowing for secure, location-based data analysis, thereby enhancing the timeliness and security of the process.

Quality of healthcare data emerges as a complex and elusive concept, demanding extensive data preprocessing to manage missing values, outliers, and inconsistencies across different health information systems (HIS). The thesis highlights the criticality of clear functional and clinical data descriptions, advocating for comprehensive data dictionaries and governance tools to facilitate effective data utilization.

Moreover, we emphasize the necessity of a collaborative approach with clinicians, who are the end-users of the developed tools. Understanding their needs and workflows is paramount, necessitating user-friendly tools that clinicians can seamlessly integrate into their practice without the need for extensive data science training.

Another pivotal aspect of this thesis is the exploration of a legal and technical framework for healthcare data science, akin to the rigorous approval processes for pharmaceuticals. Such a framework should balance safety and innovation, ensuring that new tools and methodologies are both effective and ethical. This approach is complemented by a strong focus on biomedical informatics and the development of robust data infrastructures, both locally and internationally, to enhance data availability and quality.

The thesis also explores the potential of RWE to support clinical decisions in real-time, emphasizing the need for a trust framework that ensures transparency and explainability in evidence production. This is crucial for building clinician and patient trust in the data and decision-making processes.

In conclusion, this thesis contributes to the field of healthcare data science by highlighting the multifaceted challenges and proposing innovative approaches for effective knowledge extraction from healthcare data. It underscores the importance of cross-disciplinary collaboration, robust data infrastructures, and a balanced legal and technical framework to harness the full potential of healthcare data, ultimately driving innovation and improving patient outcomes.

\vspace*{10mm}\noindent
\textbf{Keywords}: real-world data, \ac{heads}, distributed-learning, machine-learning, data quality

\chapter*{Resumo}

Esta tese mergulha no processo intrincado de extrair conhecimento dos dados de saúde, uma tarefa repleta de desafios, mas também de potencial. Central para nossa investigação é o reconhecimento, inspirado por Richard P. Feynman, de que a certeza absoluta é ilusória na pesquisa científica; em vez disso, nossa jornada é marcada por aprendizado contínuo e melhoria. Confrontamos vários obstáculos, incluindo acessibilidade dos dados, preocupações com a qualidade e a integração de evidências do mundo real (RWE) na prática clínica, explorando também soluções inovadoras como dados sintéticos e paradigmas de análise de dados distribuídos.

Uma parte significativa do nosso trabalho é dedicada a enfrentar os desafios duplos de acesso e qualidade dos dados. Os requisitos rigorosos dos comités de ética e dos Responsáveis pela Proteção de Dados (DPOs), projetados para salvaguardar a privacidade do paciente, muitas vezes impedem o acesso oportuno aos dados. Propomos dados sintéticos como uma solução potencial, oferecendo um caminho seguro e legal para o desenvolvimento e teste de algoritmos. Além disso, sublinhamos a importância da análise de dados distribuída, permitindo uma análise de dados segura e baseada na localização, aumentando assim a pontualidade e a segurança do processo.

A qualidade dos dados de saúde surge como um conceito complexo e elusivo, exigindo extenso pré-processamento de dados para gerir valores ausentes, outliers e inconsistências entre diferentes sistemas de informação de saúde (HIS). A tese destaca a criticidade de descrições claras de dados funcionais e clínicos, defendendo ferramentas abrangentes de dicionários de dados e governança para facilitar a utilização eficaz dos dados.

Além disso, enfatizamos a necessidade de uma abordagem colaborativa com os clínicos, que são os utilizadores finais das ferramentas desenvolvidas. Entender as suas necessidades e fluxos de trabalho é fundamental, necessitando de ferramentas de fácil utilização que os clínicos possam integrar sem problemas na sua prática, sem a necessidade de extensa formação em ciência de dados.

Outro aspecto fundamental desta tese é a exploração de um quadro legal e técnico para a ciência de dados de saúde, semelhante aos rigorosos processos de aprovação para produtos farmacêuticos. Tal quadro deve equilibrar segurança e inovação, garantindo que novas ferramentas e metodologias sejam eficazes e éticas. Esta abordagem é complementada por um forte enfoque na informática biomédica e no desenvolvimento de infraestruturas de dados robustas, tanto a nível local como internacional, para melhorar a disponibilidade e qualidade dos dados.

A tese também explora o potencial da RWE para apoiar decisões clínicas em tempo real, enfatizando a necessidade de um quadro de confiança que garanta transparência e explicabilidade na produção de evidências. Isso é crucial para construir confiança dos clínicos e pacientes na precisão, confiabilidade e transparência dos dados e dos algoritmos utilizados. A construção dessa confiança implica garantir que os processos de tratamento de dados e de tomada de decisão sejam transparentes e explicáveis, fomentando um sentido de responsabilidade e confiabilidade no sistema.

Em conclusão, esta tese contribui para o campo da ciência de dados de saúde, destacando os desafios multifacetados e propondo abordagens inovadoras para a extração eficaz de conhecimento dos dados de saúde. Sublinha a importância da colaboração interdisciplinar, infraestruturas de dados robustas e um quadro legal e técnico equilibrado para aproveitar todo o potencial dos dados de saúde, impulsionando a inovação e melhorando os resultados dos pacientes.

\vspace*{10mm}\noindent
\textbf{Keywords}: keyword1, Keyword2, keyword3
  % the abstract
  \chapter*{Acknowledgements}
%\chapter*{Agradecimentos}

Queria aqui agradecer, fachabor, fachabor. 

\vspace{10mm}
\flushleft{Author's Name}
   % the acknowledgments
  \cleardoublepage
\thispagestyle{plain}

\vspace*{8cm}

\begin{flushright}
  \textsl{``If you ain't aim too high, \\
    Then you aim too low.''}\\
\vspace*{1.5cm}
    Jermaine Lamarr Cole
\end{flushright}


%\begin{flushright}
%  \textsl{``Until I began to learn to draw, \\
%    I was never much interested in looking at art.''}\\
%\vspace*{1.5cm}
%    Richard P. Feynman
%\end{flushright}     % initial quotation if desired
    \chapter*{Outline}

% \chaptermark{REPRODUCIBILITY}

\doublespacing
The idea for this thesis first formed in my mind during a mental process of understanding how clinical knowledge could be improved in terms of quality, quantity and speed of generation. The feeling was that new technology, especially the digital, informatics domain took years to fully implement in practice and harness the potential benefits they provided. I felt that healthcare, like other domains, had a serious gap between academia and industry. So the potential of all these discoveries was lost in "translation".
So how could we leverage this? \\
So, this thesis is organized as follows:\\
%improve to better suit papers
Chapter \ref{chap:intro} synthetizes the aim and specific objectives of this thesis.
Chapter \ref{chap:sota} presents a brief introduction to core concepts for the thesis, like \ac{kdd}, \ac{ebm} and privacy and ethical concerns.\\
%Chapters \ref{chap:usecase}, 4 and 5 present the main results of the different studies developed to achieve this thesis’s three main objectives. When a chapter has more than one article with conclusions, a summary is made.
Chapter \ref{chap:usecase} corresponds to the papers published. The papers cover a wide range of \ac{kdd} processes so they are divided into subsections related to the phases they represent the most.\\

Chapter \ref{chap:disc} represents the overall discussion of all of the papers and experiments done in the thesis.\\

Chapter \ref{chap:conclusion} communicates the conclusion, limitations and future work.\\

Attachments include ethical permissions and supplementary data to some of the papers.
 %%narrative description of each phase/chapter

  \cleardoublepage
  \pdfbookmark[0]{Table of Contents}{contents}
  \tableofcontents
  \cleardoublepage
  \pdfbookmark[0]{List of Figures}{figures}
  \listoffigures
  \cleardoublepage
  \pdfbookmark[0]{List of Tables}{tables}
  \listoftables
  \chapter*{Abbreviations}
\chaptermark{ABBREVIATIONS}
%\chapter*{Acrónimos}
%\chaptermark{ACRONIMOS}

\printacronyms[heading=none]
  % the list of abbreviations used
  \cleardoublepage
    \chapter*{Glossary}
\chaptermark{GLOSSARY}
%\chapter*{Acrónimos}
%\chaptermark{ACRONIMOS}
\glsaddall %to add all to be printed

\renewcommand{\glossarysection}[2][]{} %remove header from glossary


\printglossary

  % glossary

\end{Prolog}

%%----------------------------------------
%% Body
%%----------------------------------------
\StartBody

%% TIP: use a separate file for each chapter


\begin{savequote}[75mm]
If we knew what it was we were doing, it would not be called research, would it?
\qauthor{Albert Einstein}
\end{savequote}
\chapter{Introduction} \label{chap:intro}

%reset de nomeclaturas
\acresetall

The practice of healthcare is deeply intertwined with technological advancements. Technology, in its broadest definition, encompasses \textit{"methods, systems, and devices which are the result of scientific knowledge being used for practical purposes"}. In essence, healthcare and medicine represent applied sciences, utilizing principles from biology, physics, chemistry, and mathematics to develop treatments, diagnostic methods, and medical procedures. Over the past two to three decades, the fields of computer science and informatics have increasingly integrated into the healthcare domain, significantly influencing its evolution and methodologies. \cite{adler-milsteinHITECHActDrove2017}. A paper-based industry is now being digitalized and computerized. This has been leading to an increase in the amount of data generated by healthcare systems \cite{kruseUseElectronicHealth2018,palabindalaAdoptionElectronicHealth2016}. 
This data has the potential to greatly improve the current methods and practices in healthcare. However, it is still not being used to its full potential \cite{kruseUseElectronicHealth2018,dicamilloGuestEditorialData2020}. This is especially important when we note that the gold standard of evidence creation is \acp{rct} which can vary on quality, time, and resources. A \ac{rct} may cost no less than 20 million euros to run, and according to a report submitted to the \ac{us} Department of Health and Human Services \cite{sertkayaaylinEXAMINATIONCLINICALTRIAL2014} can cost as much as 100 million \ac{us} dollars. This is indeed a very steep price to get the information we need to innovate. Parallel to this, usually supported by these \acp{rct} are systematic reviews and meta-analyses, highly supported and promoted by \ac{ebm} which are estimated to cost around 140 thousand dollars each \cite{michelsonSignificantCostSystematic2019}. Additionally, we must take into account the time that it takes to create and publish a good paper on evidence synthesis, often making it hard to keep up with the pace of innovation.

So, we are now being faced with huge amounts of clinical data generated by \acp{ehr} and \acp{his}. But which tools are the most suited for harvesting the potential of this data? 
The capabilities and assumptions behind modern \ac{kdd}, \ac{ml}, and \ac{ai} seem like a good approach for harvesting this potential. However, they are very different from the traditional statistical methods that are usually used in healthcare.
So, in order to properly use these methods in healthcare and actually provide value to the patients, we need to understand the differences between these methods and how they can be used to complement each other.

Currently, we already have an idea of what are the major key areas that hinder the adoption of \ac{ai} in healthcare like problems related to data privacy and security, data quality and integrity, interoperability, ethical considerations, and the fact that the hype of \ac{ai} is far greater than the \ac{ai} science, the acceptance, and trust of healthcare practitioners of \ac{ai} based systems \cite{muhiyaddinImpactClinicalDecision2020,kilsdonkFactorsInfluencingImplementation2017}, and how to proper evaluate the potential risks of \ac{ai} in healthcare, just to mention a few \cite{topolHighperformanceMedicineConvergence2019a}.
This is a very complex problem that requires a lot of different approaches and solutions. It is a popular assumption that 87\% of data science projects never get into production \cite{Why87Data2019}. Even if numbers for the healthcare domain are not available at this time, it is safe to assume that the number is not much different, if not higher. And those that actually do, may never actually create any impact due to the lack of adoption by the healthcare practitioners or the lack of trust in the system \cite{walkerModelGuidedDecisionMakingThromboprophylaxis2023}.

So, with this introduction, we have there is still a long way to go to harvest all the potential healthcare data has to offer. And so, our research objectives are focused on powering up this adoption. What can be done to improve these chances? What can we bring to the table to enhance the rate of success?

%cost of systematics review
%https://www.ncbi.nlm.nih.gov/pmc/articles/PMC6722281/

%cost of trials
%https://www.sofpromed.com/how-much-does-a-clinical-trial-cost


%%% falar de perfil hibrido e saber de saude e de machine leraning e estatistica sera complicado


%%falar da pouca adoção de ML e CDSS
%%%TODO finish

\section{Research Objectives}
%%rework
This thesis has three main goals:


\begin{itemize}
    \item Goal 1: Research methods to improve data quality, whether using synthetic data generation to enlarge data volume and protect privacy (sections \ref{subsec:gans}, \ref{subsec:tabular} and \ref{subsec:similarity}) or by creating automatic data quality assessment methods (section \ref{subsec:dq})

    \item Goal 2: Assess health data science  methods with limited data access (sections \ref{subsec:distributed} and \ref{subsec:benchmark}).

    \item Goal 3: Explore strategies to transform health data into actionable decisions and policies (sections  \ref{subsec:ipop} and \ref{subsec:obs}).
\end{itemize}


\section{Research Questions}

\begin{itemize}
    \item Goal 1: How can we improve the quality of data used in health data science?
    \begin{itemize}
        \item RQ1.1: Can GANs Help Create Realistic Datasets?
        \item RQ1.2: How Can We Compare Two Tabular Datasets?
        \item RQ1.3: Can We Use Machine Learning Feature to Compare Datasets?
        \item RQ1.4: Can We Use Machine Learning to Create Automatic Data Quality Assessments?
    \end{itemize}
\end{itemize}

For this goal, our aim is to improve the quality of data used in health data science. We will start by exploring the use of \ac{gan} to create realistic tabular datasets (section \ref{subsec:gans}). We have seen the advent and deep learning in creating data, namely for image, video and sound. From these examples, \acp{gan} have been providing excellent results, but do we know if this performance is matched in tabular datasets? Then we will explore the methods for comparing datasets (section \ref{subsec:tabular}), which relates to the first point since we cannot create good synthetic data if we do not have good metrics to assess how equal they are. So we tried to compile the current state-of-the-art metrics for such. And since we did not find, in our opinion, a good set of metrics in the 2nd point, we tried to apply \ac{ml} to create a good evaluation metrics for comparing two tabular datasets (section \ref{subsec:similarity}). Finally,  we will then explore the use of \ac{ml} to create automatic data quality assessments. This is a very important point since we need to know if the data we are using is good enough for the task at hand. This is especially important when we are using \ac{ml} methods since they are very sensitive to data quality (section \ref{subsec:dq}).

\begin{itemize}

    \item Goal 2: How can we assess health data science methods with limited data access?
    \begin{itemize}
        \item RQ2.1: Leveraging Distributed Systems in Healthcare: is it Advisable?
        \item RQ2.2: Can Institutions Share Their Performance Metrics Without Hesitation of Retaliation?
    \end{itemize}
\end{itemize}

For this goal, we tried to overcome the data access limitations. If we do not have access to data, how can we explore its potential? There are several issues that are set in place to guarantee the good, safe and ethical usage of data, but sometimes it creates hurdles to a timely usage of the data. So we tried to evaluate if distributed learning was a fitting option to overcome this limitation (section \ref{subsec:distributed}). Then we tried to come up with an alternative to compare health institutions without actually sharing their true metrics, but still be able to position themselves in a scale (section \ref{subsec:benchmark}). 

\begin{itemize}
    \item Goal 3: How can we convert health data into decisions and policies?
    \begin{itemize}
        \item RQ3.1: How Can We Leverage Data to Create Clinical Decision Support Systems?
        \item RQ3.2: How Can We Leverage Data to Assess Treatment Efficacy?
    \end{itemize}    
\end{itemize}

For the last goal, we tried to actually go from end-to-end. This means that we tried to leverage real world data in its raw form and transform it into actionable insights. Our major objective was to check the challenges that block the development of such tools. We tried to assess the real-world effect of two drugs for breast cancer and compare them among themselves and with the previous gold standard: endocrine therapy (section \ref{subsec:ipop}). Then we tried to create a \ac{cdss} that could be used in real-time clinical environments and be able to provide support for the need of C-sections (section \ref{subsec:obs}).  
\begin{savequote}[75mm]
The most exciting phrase to hear in science, the one that heralds new discoveries, is not 'Eureka!' but 'That's funny...'
\qauthor{Isaac Asimov}
\end{savequote}

\chapter{State of the art} \label{chap:sota}

\section{Extracting Knowledge of data}\label{sec:kdd}
%%%TODO: explorar mais artigos recolhidos e ver zotero o que ja tenho, e criar mais ligacoes entre as frases e conceitos

%fayyad explorar mais e melhor intro
\cite{Fayyad_Piatetsky-Shapiro_Smyth_1996}
\ac{kdd} plays a pivotal role in the healthcare industry. The complexity and vastness of healthcare data, encompassing electronic health records, genomic data, medical imaging data, and various other types of data, call for the adoption of intelligent systems that can mine this data for useful insights. The \ac{kdd} process, comprising data cleaning, integration, selection, transformation, data mining, pattern evaluation, and knowledge presentation, can effectively help discover patterns and relationships in healthcare data, which are often not apparent to traditional analysis methods. This process facilitates the prediction of disease outbreaks, the identification of high-risk patient groups, the optimization of treatment plans, and the enhancement of healthcare service delivery. The generic process for \ac{kdd} is shown in figure \ref{fig:kdd-generic}.

\begin{figure}
\centering
%\includegraphics[width=\textwidth]{image.png}
\includegraphics[scale=0.55]{figures/imagens-tese.jpg}

\caption{\ac{kdd} Process, adapted from \cite{Fayyad_Piatetsky-Shapiro_Smyth_1996}} \label{fig:kdd-generic}
\end{figure}

Several frameworks have been proposed to implement the \ac{kdd} process. One such prominent framework is \ac{crispdm}, which comprises business understanding, data understanding, data preparation, modeling, evaluation, and deployment. \ac{crispdm} was conceived in 1996 and became a \ac{eu} project under the ESPRIT funding initiative in 1997 \cite{Chapman2000CRISPDM1S}. %extend 
%Another significant framework is \ac{kddf}, which is designed explicitly for healthcare applications. It focuses on the acquisition and integration of data from diverse healthcare sources, data preprocessing, data mining, data interpretation, and knowledge utilization. %%FAKE
%semma é proprietário

Furthermore, \ac{d3m} 
\cite{huangDataMiningIntegrated2009} and \ac{semma} \cite{rohanizadehProposedDataMining2009} are also important frameworks for applying \ac{kdd} in healthcare, helping align data mining processes with specific healthcare domains and facilitating more accurate and valuable knowledge discovery.

In the context of \ac{kdd} in healthcare, various classes of algorithms can be utilized, each best suited for different kinds of tasks.

Classification Algorithms: These are used to predict categorical class labels. Examples include Decision Trees, Naive Bayes, \ac{svm}, \ac{knn}, and various types of Neural Networks. These are used in disease diagnosis, patient risk prediction, and readmission prediction.

Clustering Algorithms: These are unsupervised methods used to group similar data points together. K-Means, Hierarchical Clustering, DBSCAN, and Self-Organizing Maps (SOM) are common clustering algorithms used in patient segmentation and anomaly detection.

Regression Algorithms: These are used to predict continuous output variables. Examples include Linear Regression, Logistic Regression, and Regression Trees. These algorithms find application in predicting disease progression and healthcare costs.

Association Rule Mining Algorithms: These discover associations or patterns among a set of items in large databases. Apriori and FP-Growth are commonly used algorithms in this class, helping in discovering co-occurring health conditions or drug interactions.

Sequential Pattern Mining Algorithms: These help discover or predict specific sequences of events, which is particularly useful in medical trajectory analysis.

Most recently, more sophisticated architectures and algorithms appeared with neural networks, generative \ac{ai}, and reinforcement learning....

%-tipos de algortimos paraa




\subsection{Evidence Based Medicine}
\ac{ebm} is a relatively recent concept in healthcare, which entails integrating the best available research evidence with clinical experience and patient values to make decisions about patient care. The term "evidence-based medicine" was first coined by a team at McMaster University in Canada in the 1980s, but the concept has historical antecedents dating back to at least the 19th century. This was a time when clinical decision-making was mostly based on untested observations and physicians' experience, leading to variability in treatment strategies. The birth of \ac{ebm} marked a pivotal moment in medical history, aiming to standardize patient care and improve outcomes.

The advent of \ac{ebm} was closely tied to the development of clinical epidemiology, the study of disease patterns, causes, and effects in populations. This field, which emerged in the 20th century, focused on statistical and methodological tools to rigorously evaluate treatments. The early proponents of \ac{ebm} aimed to counter anecdotal and unsystematic approaches to clinical decision-making by insisting on rigorous scientific evidence as a basis for decisions. This transitioned medicine from a largely experience-based discipline to scientific, data-driven practice.

The main concept of \ac{ebm} is the hierarchy of evidence, which classifies different types of research studies based on their methodological quality and applicability to patients. At the top of this hierarchy are \acp{rct} and systematic reviews of \acp{rct}, which are considered to provide the most robust evidence. Observational studies, case series, and expert opinions are further down the hierarchy due to their inherent limitations. \ac{ebm} advocates for the application of the highest level of evidence available in clinical decision-making.

However, \ac{ebm} is not just about applying research findings mechanically to patient care. It integrates these findings with the clinician's expertise and the patient's individual circumstances, values, and preferences. This is known as the triad of \ac{ebm}: best available evidence, clinical expertise, and patient values. The concept recognizes that while evidence can guide decisions, it cannot replace the clinical judgment required in individual cases or the need to consider the patient's personal circumstances and preferences. Thus, \ac{ebm} aims to enhance, not replace, the clinician's traditional skills and roles, adding a new dimension to patient care.


\subsection{Health Data Science}
%sacket 
%- graph of evidence based medicine

%- inconvenient truth ai
%Three controversies in health data science
Health Data Science is an interdisciplinary field that applies rigorous methods to transform healthcare data into actionable knowledge for improving health outcomes. It involves the collection, interpretation, and application of vast amounts of biological, clinical, population, and health system data to improve patient care and public health. The advent of electronic health records, genomics, mobile health technologies, and other forms of big data have fueled the growth of this discipline.

In practice, Health Data Science involves the use of statistical and machine learning methods to analyze healthcare data. This data can be patient records, genomic data, demographic data, and more. It includes elements from various disciplines like biostatistics, epidemiology, informatics, and health economics. The ultimate goal is to provide a data-driven foundation for health decision-making for clinicians, health administrators, policymakers, and researchers.

An integral part of Health Data Science is predictive modeling and hypothesis testing. Predictive modeling involves the creation and use of statistical models or machine learning algorithms to predict future outcomes based on historical data. Hypothesis testing, on the other hand, is used to test the validity of a claim or theory about a population based on sample data. These are crucial for health data science as they allow us to make educated guesses about health trends and outcomes.

Importantly, Health Data Science has significant ethical and privacy considerations. Health data is often sensitive and personal, so maintaining privacy and confidentiality is crucial. This requires secure data handling and storage practices, as well as careful consideration of ethical implications when designing studies and algorithms. Health Data Scientists must also be wary of algorithmic bias and must ensure their models do not perpetuate or amplify health disparities. The ultimate goal of Health Data Science is to improve patient outcomes and health equity using the best available data and methods.


The potential of using systematically created data in healthcare, which reaches the PB, has certainly a lot of potential. However, we have seen in the past as well, that the hype of \ac{ai} and \ac{ml} usually are not supported by truth. There are currently six main aspects that hinder the potential of health data science \cite{panchInconvenientTruthAI2019,peekThreeControversiesHealth2018}:
\begin{myitemize}
    \item interoperability
    \item semantic
    \item secondary usage
    \item data quality
    \item privacy and ethics
    \item observational data
\end{myitemize}

\textbf{Interoperability} is defined by \textit{Interoperability is ability of two or more systems or components to exchange information and to use the information that has been exchanged}\cite{182763}. In the context of healthcare, this means that different systems should be able to exchange data and interpret the data that has been exchanged. This is a very important aspect of health data science, since the data is usually stored in different systems, with different structures and different purposes. So if systems are locked inside themselves and no export is possible, data becomes inaccessible. So, it is only natural that interoperability has been a key factor in gathering data. With tens or hundreds of different systems in every health institution, the possibility of exchanging data between \acp{ehr} plays a vital role. The usage of interoperable standards is of extreme importance in order to tackle the need to get data with a predefined structure.


\textbf{Semantic} adds a layer to the previous points, being sometimes related to interoperability as well. The fact that several institutions and \acp{ehr} are involved in creating knowledge from data, raises the problem that not all have data coded in clinical terminologies, or if they do, it is seldom the same across systems, since semantics has a very tight relationship with domain, especially in healthcare. So the normalization of uncoded terms is often required and mapping across terminologies is also very common, which is time-consuming and requires expertise in several fields.
%mts ehrs
%mts 
\textbf{Secondary usage} is related to the fact that we are aiming to use data for a purpose for which the data was not created. The main goal of the healthcare data is to provide care. It is not meant for analysis and gaining insights. More than that, is already pretty well documented that the usage of \ac{ehr} is very different from institution to institution and from country to country \cite{anckerHowElectronicHealth2014,weiskopfMethodsDimensionsElectronic2013a,peekThreeControversiesHealth2018}. This means that the context where data was collected, even the actual person who inserted the information could be key to interpreting the results. To make things more complicated, the degree of precision of the data inserted varies highly on the type of information and context, as reported in \cite{cruz-correiaDataQualityIntegration2009}. 

\textbf{Data Quality} stems from the secondary usage. If the data is not reliable, how can we use it to gather useful knowledge from it? In order to, at least, try to counter this, we can apply several statistical methods and machine learning algorithms to try to clean the data. However, this is not a trivial task, since the data is usually very heterogeneous and the context where it was collected is not always available. So, data quality is a very important aspect of health data science, since it can be the difference between a good and a bad model.


\textbf{privacy and ethics} adds yet another layer to the problems of health data science. The fact that we are dealing with sensitive private data, which is not meant to be used for secondary purposes, raises the question of privacy and ethical concerns. Anonymization techniques and privacy-preserving methods are key to tackling this problem. However, they are not problem-free and are often complicated to assess. Moreover, the risks are very high, since the data is very sensitive and the consequences of a breach of privacy can be very serious, undermining public trust in clinicians, healthcare institutions and the healthcare system as a whole.

%%%LLM
%Health data science and evidence-based medicine are closely intertwined fields that leverage the power of data analysis and research to improve patient outcomes and inform clinical decision-making. Health data science involves the collection, management, and analysis of vast amounts of health-related data, including electronic health records, medical imaging, genomics, and wearable devices. By applying advanced statistical and machine learning techniques to these datasets, health data scientists can uncover patterns, trends, and insights that can enhance our understanding of diseases, treatment effectiveness, and population health.

%Evidence-based medicine, on the other hand, is an approach to clinical practice that integrates the best available research evidence, clinical expertise, and patient values and preferences. It aims to guide healthcare professionals in making informed decisions about patient care by using rigorous scientific evidence. Health data science plays a crucial role in evidence-based medicine by providing the necessary data and analytical tools to generate high-quality evidence. Through the analysis of large-scale health datasets, researchers can identify associations, evaluate treatment effectiveness, and uncover potential risk factors, all of which contribute to the evidence base that informs clinical practice guidelines and treatment recommendations.
\textbf{Observational Data} relates to the fact that all health data science will be based on observational data. This means that the data is not collected in a controlled environment, which is the case for \acp{rct}. This means that the data is subject to several biases, which are not always possible to control. The corner-stone of \acp{rct} is simply not possible to apply here, unabling a proper comparasion between groups. And even though there are techniques to tackle unbalance in the measured variables, there is no way to control the unmeasured variables, which can be the cause of the observed effect. This is particular importance and a major area of research at the moment, like we will see in the section \ref{subsec:xai} and \ref{causalml}.


The integration of health data science and evidence-based medicine has the potential to revolutionize healthcare delivery and improve patient outcomes. By leveraging the power of data analysis and advanced algorithms, health data scientists can identify novel biomarkers, develop predictive models, and personalize treatment plans based on individual patient characteristics. This not only enhances clinical decision-making but also enables precision medicine, where treatments can be tailored to the specific needs of each patient. Additionally, the use of health data science in evidence-based medicine allows for the continuous monitoring of treatment effectiveness and safety, facilitating the identification of best practices and the refinement of clinical guidelines over time.

In conclusion, health data science and evidence-based medicine are interconnected fields that rely on each other to drive advancements in healthcare. Health data science provides the necessary tools and techniques to analyze large and complex datasets, enabling the generation of high-quality evidence that informs evidence-based medicine. The integration of these disciplines holds immense potential to transform healthcare by enabling personalized medicine, improving patient outcomes, and enhancing clinical decision-making.



\subsection{Artificial Intelligence}
The idea behind the issue is that \ac{ai} is fundamental to leveraging the potential of the data. \ac{ai} has already been under public focus for a few years now, but its concept is still elusive. 
Mainly because the definition has been changing rapidly as well. If in the 90s, the computer program that played chess could be considered \ac{ai} at the time, nowadays is a very simple and common concept to develop. This is connected with the concept that \ac{ai} is tightly connected with the subjective nature of humans. If we feel that \ac{ai} is something related to what a human can do, it can be widely diverse from person to person.
....
however, there are some definitions that could be interesting to explore in order to get the concept for the purpose of this thesis cleared up.

From \cite{DBLP:books/aw/RN2020}

According to the European Commission \cite{DefinitionAIMain2019} and \cite{EthicsGuidelinesTrustworthy2019}

%%what is ai - livro ai verde + european comission
%what is ai - european comission



\subsection{Explainable Artificial Intelligence}\label{subsec:xai}
\ac{ai} has experienced unprecedented advancements in the last decade, leading to its integration in various domains, including medicine. It has been instrumental in transforming clinical decision-making, drug discovery, patient monitoring, and predicting disease trajectories. Despite these advancements, the "black box" nature of complex \ac{ai} models poses interpretability challenges, limiting their widespread adoption in healthcare, a field where transparency, reliability, and understanding of decision-making processes are vital. This lack of interpretability, also known as opacity, can lead to misdiagnoses, inappropriate treatment plans, and, most importantly, breaches in trust among clinicians, patients, and \ac{ai} systems.

As such, the concept of \ac{xai}, which aims to create a suite of techniques that produce more explainable models while maintaining a high level of predictive accuracy, has gained significant attention in medical \ac{ai} research. \ac{xai} seeks to bridge the gap between \ac{ai} opacity and human interpretability, and in doing so, it can enhance the transparency, reliability, and acceptance of \ac{ai} applications in the healthcare setting.

So, for this to happen, we need a new framework for applying such mechanisms. A new step that could be attached to the ones seen before in section \ref{sec:kdd} will enable human comprehension of the model's output.

Even though several grouping and taxonomies of \ac{xai} are available mentioned in \cite{adadiPeekingBlackBoxSurvey2018,linardatosExplainableAIReview2020,barredoarrietaExplainableArtificialIntelligence2020,linardatosExplainableAIReview2020,kamath2021explainable}, a simplified approach based on \cite{kamath2021explainable} will be used in order to contextualize this concept.

We can divide it into two main categories. Firstly the explanation type, which is divided into global and local. Local and global explanations are methods used to interpret machine learning models, especially those that are considered "black box" models, such as deep learning networks. These methods help us understand why and how a model makes certain decisions, which can be crucial in many settings for ethical, legal, and practical reasons.

Local Explanations: These involve understanding the prediction of a \ac{ml} model for a specific individual instance. They help to answer questions like: "Why did the model predict that this particular patient has cancer?" or "Why was this specific transaction flagged as fraudulent?". 

Global Explanations: These focus on understanding the model behavior across all instances, or more broadly on a dataset-wide level. They help to answer questions like: "What features are generally important for prediction in the model?" or "What is the overall logic of the model?". 

Secondly, we have the method type, where we have 3 main subcategories related to the stage of the data science process it is applied, pre, during and post-model training.

\textbf{Pre-Model \ac{xai}:} These methods involve improving the transparency and interpretability of models before they are even trained. This includes thoughtful feature engineering, \ac{eda}, and applying domain knowledge to create meaningful variables. The goal is to design a model that will be more interpretable from the onset.

\textbf{Intrinsic \ac{xai}:} This involves using machine learning models that are intrinsically explainable. These models are designed in such a way that their decision-making process is understandable by default. Examples include linear and logistic regression, cox regressions, decision trees, Naïve Bayes, \ac{bn} and rule-based models. While these models may sometimes lack the predictive power of more complex models, they provide clear interpretability: you can directly examine the impact of the variables and understand how the model makes its predictions.
\\
\textbf{Linear Regression}
Linear regression is a linear approach to modeling the relationship between a dependent variable and one or more independent variables. It assumes that the relationship between these variables is linear and can be represented by a straight line. The goal is to fit the best possible line that describes this relationship by minimizing the sum of the squared differences (errors) between the observed values and the values predicted by the line. Linear regression is widely used in various fields for prediction, modeling, and determining the strength and character of the relationship between variables. It forms the basis of many more complex statistical modeling techniques.
\\
\textbf{Logistic Regression}
Logistic regression is used to model the probability of a binary outcome that depends on one or more independent variables. Unlike linear regression, which predicts a continuous outcome, logistic regression predicts the probability of a categorical outcome (e.g., success/failure, yes/no, 1/0). The logistic function is applied to the linear combination of independent variables to ensure that the estimated probabilities are between 0 and 1. It's often used in fields like medicine, economics, and social sciences to predict the likelihood of an event occurring based on various factors.
\\
\textbf{Cox Regression}
Cox regression, or the Cox proportional-hazards model, is a statistical technique used for investigating the effect of several variables on the time a specified event takes to happen. In medical research, this often refers to survival times. The model allows for the estimation of hazard ratios, which describe how the hazard changes with a one-unit change in the predictor variable. The Cox model makes an assumption that the hazard ratios are constant over time, known as the proportional hazards assumption. This model is vital for understanding how different factors influence survival or failure time and is commonly applied in epidemiological and medical research.

\textbf{Bayesian Networks}
A \ac{bn}, also known as a belief network or \ac{dag} model, is a probabilistic graphical model that represents a set of random variables and their conditional dependencies via a \ac{dag}. 

Given a set of variables \(X = \{X_1, X_2, ..., X_n\}\), the joint probability distribution is given by:

\[
P(X_1, X_2, ..., X_n) = \prod_{i=1}^{n} P(X_i | Parents(X_i))
\]

where \(Parents(X_i)\) is the set of parent variables of \(X_i\) in the network.

This formula represents the factorization of the joint distribution over \(X\), based on the graphical structure of the Bayesian network.

Now, in the Bayesian network, each node is conditional independent of its non-descendants given its parents. If we denote \(ND(X_i)\) as the set of non-descendants of \(X_i\) and \(Pa(X_i)\) as the parents of \(X_i\), the conditional independence is described as:

\[
X_i \perp ND(X_i) | Pa(X_i)
\]

This means that \(X_i\) is conditionally independent of its non-descendants given its parents. 

A common task for Bayesian networks is inference, which means computing the posterior probability of a set of query variables \(Q\), given some observed variables \(E\). That is, we want to compute \(P(Q|E)\). According to the Bayes rule, we have:

\[
P(Q|E) = \frac{P(Q,E)}{P(E)} = \frac{P(Q,E)}{\sum_{q \in Q} P(Q=q,E)}
\]

where the denominator is a normalization constant ensuring the result is a valid probability distribution. Note that performing this inference is NP-hard, which is why various approximation algorithms have been developed.

\textbf{Tree based methods}
Tree-based machine learning methods are a subset of algorithms that use a tree-like graph structure for making decisions or predictions. The most basic type is the Decision Tree, where the tree is used to go from observations about an item to conclusions about the item's target value (classification or regression). Each node in the tree represents a feature in the dataset, each branch represents a decision rule, and each leaf node represents the output value. More advanced tree-based methods include Random Forests, which build multiple Decision Trees and average their predictions for better accuracy and generalization, and gradient-boosted trees, which build trees sequentially, each one correcting the errors from the previous one.

The major advantage of tree-based methods is their ease of interpretation and understanding, especially for Decision Trees. However, a single tree is often prone to overfitting, where it performs well on the training data but poorly on unseen data. This is why ensemble methods like Random Forest and Gradient Boosting are popular; they aim to increase robustness and predictive power by combining multiple trees. These methods are widely used in various domains including but not limited to finance, healthcare, and natural language processing for tasks like classification, regression, and even unsupervised learning tasks like clustering.

\textbf{Post-Hoc \ac{xai}:} Post-hoc methods are applied after a model has been trained, to try to explain its decisions. This includes techniques like feature importance analysis, partial dependence plots, \ac{lime}, \ac{shap}, and counterfactuals. For instance, \ac{lime} can be used to create local explanations for individual predictions made by any model, and \ac{shap} values can be used to interpret the impact of features on the model's output both locally and globally. Counterfactuals try to explain a model by example, providing possible changes that would alter the outcome provided by the model.

It is to be noted that a methodology can be classified into two categories. For example, \ac{lime} is a local explanation model in a \textit{post-hoc} manner.

Finally, we can assess how these three types of model are an explanation of the model, we could argue, as stated in \cite{rudinStopExplainingBlack2019} that only an intrinsically transparent model can really be the basis of \ac{xai} and applying \textit{post-hoc} methods is only a potentially wrong proxy for an explanation.
%explicar mais disto e fechar.

\subsection{Causality}\label{causalml}

In order to merge the \ac{ebm} and \ac{xai}, we can follow on to the realm of \ac{cml}.

%- causality vs observation

\ac{cml} is a branch of machine learning that focuses on understanding and quantifying causal relationships from data. Instead of just finding patterns or correlations in data, \ac{cml} aims to uncover the cause-and-effect relationships that explain these patterns.
This is especially important since current or traditional \ac{ml} and \ac{ai} methodologies rely heavily on association and not causation. So, \ac{cml} can support traditional algorithms to solve its limitations \cite{pearlTheoreticalImpedimentsMachine2018}.
There are currently two main frameworks for trying to unveil causality in data: the \ac{scm}  and the \ac{pof} \cite{shiLearningCausalEffects2022b}.

\ac{scm} relies on \acp{dag} and structual equations.
Causal Graphs are based on \acp{dag}: These are graphical models used to represent causal relationships between different variables. The nodes in the graph represent variables, and the edges (arrows) between nodes represent causal relationships. For instance, an edge from Node A to Node B signifies that A has a causal effect on B. We should not confuse causal graphs with \ac{bn}. Even though both rely on \acp{dag}, Causal Graphs represent causal relationships, and \ac{bn} represent conditional dependencies.
%% SCM???
strucurual equations....


The \ac{pof} model centers on the concept of potential outcomes which can be understood as all of the possible outcomes for a patient.  Each unit (e.g., a patient or a sample) has a set of potential outcomes, each corresponding to one of the possible treatments the unit could receive. The causal effect is defined as the difference between these potential outcomes. This framework allows for the formal definition and estimation of causal effects. In this approach, we consider the potential outcomes for each unit (for example, a patient in a healthcare context) under each possible treatment or intervention. Each unit has a set of potential outcomes corresponding to each possible intervention. However, we can only observe one of these outcomes for each unit, corresponding to the intervention that was actually received. The other outcomes, which would have occurred had different interventions been implemented, remain latent. These are known as counterfactual outcomes.

The difference between potential outcomes under different treatments represents the causal effect of the treatments. For instance, in a healthcare scenario, if we are studying the effect of a drug, we might consider two potential outcomes for each patient: the outcome if the patient is given the drug, and the outcome if the patient is not given the drug. The difference between these outcomes represents the causal effect of the drug on the patient. However, as we can only observe one of these outcomes for each patient (the one corresponding to the treatment they actually received), a key challenge in causal inference is estimating the unobserved potential outcomes. Various statistical methods, including randomized experiments, matching methods, and instrumental variable methods, can be used to estimate these unobserved potential outcomes.


Counterfactuals: This is a concept rooted in the idea of "what-if" scenarios. A counterfactual outcome for a given individual is the outcome that would have occurred had the individual been exposed to a different treatment or condition.
%%

Counterfactuals play a pivotal role in the field of causal machine learning, offering a sophisticated approach to understanding cause-and-effect relationships. In essence, a counterfactual is a conceptual device used to contemplate what would have happened under a different set of circumstances than what actually occurred. This hypothetical scenario is created by altering some aspect of the actual situation, providing a means of comparison to evaluate the effect of a particular variable or intervention.

For instance, in the context of healthcare, consider a scenario where a patient was given a particular drug and recovered. The counterfactual question here would be: "What would have happened to the patient if they hadn't been given the drug?" Answering this question allows us to estimate the causal effect of the drug on the patient's recovery. While the true counterfactual outcome is unobservable (since we cannot rewind time and alter the decision), various statistical techniques, machine learning algorithms, and experimental designs are employed in causal inference to estimate this effect as accurately as possible. The ability to make such counterfactual inferences is crucial in numerous fields, including medicine, economics, social sciences, and policy-making, where understanding causal relationships is paramount.


%Confounding Variables: These are variables that both affect the treatment assignment and the outcome. Confounding variables can create spurious associations that can be mistaken for causal relationships, so they must be carefully controlled for in causal analyses.
%%%
%Confounding variables, also known as confounders, represent a major challenge in the study of causal relationships. A confounding variable is a variable that influences both the dependent variable and independent variable, causing a spurious association. This can lead to false conclusions about the relationship between the independent and dependent variables, obscuring the true causal effect.

%For instance, imagine a study investigating the relationship between physical exercise (independent variable) and health outcomes (dependent variable). Age could act as a confounding variable in this scenario. Older individuals might exercise less than younger ones, and they also tend to have more health problems. Without proper adjustment, the study might falsely attribute the poorer health outcomes solely to a lack of exercise, while in reality, age is influencing both exercise habits and health outcomes.

%Confounding variables can be addressed in a number of ways. In the design stage of a study, randomized controlled trials (RCTs) are used to ensure that confounding variables are evenly distributed across treatment and control groups, thereby reducing their impact. In observational studies where random assignment is not possible, statistical techniques such as regression adjustment, stratification, matching, and propensity score methods can be employed to control for confounding. Importantly, the success of these methods depends on the ability to measure all relevant confounders, a challenge known as unobserved confounding, which is a persistent issue in causal inference.



Instrumental Variables: These are variables that are related to the treatment but not the outcome, except through their effect on the treatment. They can be used to control for unmeasured confounding variables.
%%%%

Instrumental variables (IVs) are a powerful tool used in causal inference to help address the problem of confounding variables, especially in situations where randomization is not feasible. An instrumental variable is a variable that is correlated with the independent variable (the treatment) but does not directly affect the dependent variable (the outcome), except through its effect on the treatment. In other words, it is a variable that induces changes in the explanatory variable but is otherwise unrelated to the outcome of interest.

The idea behind using an instrumental variable is to isolate the portion of the variation in the treatment that is independent of the confounders and therefore provides a "natural" form of randomization. The causal effect of the treatment on the outcome can then be estimated based on this variation.

For example, in a study assessing the impact of education on income, it's challenging to identify causal effects because numerous unobserved factors (like ability or motivation) could affect both education and income, thus confounding the relationship. If we find an instrumental variable – say, distance to the nearest college (which affects the likelihood of getting higher education but doesn't directly affect income) – we can use this to isolate the part of the variation in education that is unrelated to the unobserved confounders, and thereby get a more accurate estimate of the causal effect of education on income.

It's crucial, however, to remember that the use of instrumental variables relies on certain assumptions, such as the relevance and exogeneity of the IV. The relevance assumption requires that the IV is correlated with the treatment, and the exogeneity assumption requires that the IV affects the outcome only through the treatment and is not related to the unobserved confounders. Violations of these assumptions can lead to biased and inconsistent estimates of causal effects.



Propensity Score: This is the probability of a unit (e.g., a patient) being assigned to a particular treatment given a set of observed characteristics. Propensity scores are used to balance the characteristics of treatment and control groups, mimicking the conditions of a randomized experiment.
%%%%

The propensity score is a statistical concept widely used in causal inference, particularly in the field of observational studies where random assignment of treatment is not possible. The propensity score for an individual is the probability of receiving the treatment given the observed characteristics of that individual. In other words, it's the likelihood that a particular individual would be assigned to the treatment group based on their observed features.

The key idea behind propensity scores is to create a balance between the treatment and control groups based on these observed characteristics, thus mimicking the conditions of a randomized controlled trial. This balance helps to eliminate bias caused by confounding variables, allowing for a more accurate estimate of the treatment effect. Once propensity scores are calculated, they can be used in several ways including matching, stratification, inverse probability of treatment weighting (IPTW), and as covariates in regression adjustment.

For example, consider a study investigating the effect of a training program on job outcomes. Individuals might self-select into the training program based on characteristics like motivation or prior education, which are also related to job outcomes, creating confounding. The propensity score, calculated based on these observed characteristics, can be used to match each participant in the training program with a similar non-participant or to weight the observations, such that the distribution of observed characteristics is similar between the groups. This helps to isolate the effect of the training program on job outcomes.

However, it's important to note that propensity scores only account for observed confounders. If there are unobserved confounders that influence both treatment assignment and the outcome, propensity score methods may still produce biased estimates of the causal effect.

%The importance of quality data in observational studies cannot be overstated. Unlike randomized controlled trials (RCTs), where randomization helps to balance both observed and unobserved covariates between the treatment and control groups, observational studies are often fraught with selection biases, confounding variables, and imbalances in baseline characteristics. Researchers typically have no control over the assignment of subjects to treatment or control groups, leading to potential biases that can significantly skew results. Well-structured and rich datasets can provide a wealth of information that allows for more accurate control of these confounding factors. By including a variety of variables that might influence the outcome, data richness enables the use of statistical techniques like matching, stratification, or weighting to create comparable treatment and control groups, thereby mimicking the conditions of an RCT to some extent.

%One of the critical ways to partially mitigate the issues inherent in observational studies is through the use of propensity score methods, such as Average Treatment Effect (ATE) and Average Treatment effect on the Treated (ATT) weighted Kaplan-Meier curves. These methods seek to balance the distribution of observed covariates between treatment and control groups, thereby reducing selection bias. Once balanced, the survival curves can more accurately reflect the true impact of the treatment, providing results that are closer to what might be observed in a randomized study. In essence, propensity score methods help to level the playing field by reweighting or resampling the original data based on the probability of receiving treatment, allowing for a more fair comparison between the treatment and control groups.

%That being said, it's crucial to remember that even the most sophisticated statistical techniques can only control for observed confounders; hidden biases due to unmeasured or unknown variables can still persist. Additionally, the quality of the propensity score model is heavily reliant on the data available, underlining the need for comprehensive data collection and thorough exploratory data analysis. The application of methods like ATE and ATT weighted Kaplan-Meier curves is not a substitute for good data but a complement to it. In sum, while quality data and sophisticated statistical methods can't fully replicate the conditions of a randomized trial, they can substantially improve the validity and reliability of findings from observational data.


%These concepts and others are used in methods like matching, instrumental variable analysis, difference in differences, regression discontinuity, and others to estimate causal effects. The goal of all these techniques is to create a situation as close to a randomized control trial as possible, as this type of experiment is considered the "gold standard" for estimating causal effects.



\subsection{All of this brought together}

So....



\section{Legal and ethical considerations}

%GDPR, health european data space, questoes eticas


As data science and \ac{kdd} become increasingly prevalent in the healthcare sector, the legal and ethical considerations related to these practices also become critically important. Ensuring the proper use of healthcare data is key to preserving public trust and ensuring the long-term viability of data-driven health initiatives.

One of the primary legal considerations is data privacy. Laws such as the \ac{hipaa} in the \ac{us}, and the \ac{gdpr} in the \ac{eu}, set stringent rules on how healthcare data should be stored, shared, and processed. They require data scientists and healthcare providers to take steps to anonymize data and limit the scope of data usage. Breaching these regulations can lead to severe penalties, including fines and imprisonment.

On the ethical front, considerations include ensuring data fairness and avoiding bias. Given the diversity of patients in terms of age, race, sex, socioeconomic status, etc., algorithms should be designed and validated to ensure that they don't unintentionally perpetuate or amplify societal biases. For instance, a predictive model for disease risk should not unfairly disadvantage certain demographic groups. Furthermore, the informed consent of patients is another significant ethical consideration. Patients should be fully informed about how their data will be used, and they should have the right to opt-out if they wish.

Transparency is another crucial aspect that straddles both legal and ethical dimensions. It involves explaining how decisions or predictions are made by complex algorithms, particularly when they have significant implications for patient care. For instance, if an AI model is used to prioritize patients for treatment, it should be transparent about how the model makes its decisions. The explainability of machine learning models can help achieve this transparency, which aids in maintaining accountability and trust.

Lastly, there's the matter of data security. With the rise of cyber-attacks, ensuring the robustness of the system against such breaches is both a legal requirement and an ethical obligation. Security breaches could lead to sensitive patient data being stolen, with severe implications for the individuals involved and for the trust in the healthcare system as a whole.

%In conclusion, while data science and KDD offer immense potential to improve healthcare, it is crucial that these technologies are implemented in a way that respects legal regulations and ethical principles. This will help to ensure the sustainability and public acceptance of these technologies in the long run.

In a parallel layer to this one, we have ethical issues. If we use data to derive knowledge and create  \acp{cdss} that orient and support clinical practice, they can be biased by the type of data that originated said knowledge.
This could lead to biased and discriminatory systems....

%%%chat
The importance of ethics in \ac{ai} cannot be overstated, primarily because the decisions that these systems make can have profound implications on individuals and society. These decisions may affect anything from employment opportunities to legal outcomes, and increasingly, health outcomes. As AI models grow in complexity and application, they possess an enormous power that needs to be harnessed responsibly. This necessitates rigorous ethical considerations to ensure fair, unbiased, and transparent operations. Ethical lapses can result in discrimination, loss of privacy, and unjust outcomes, among other issues, which erode public trust in these technologies.

Equally important in the realm of \ac{ai} is the understanding of why a model works the way it does. This concept, known as "explainability" or "interpretability", is central to AI ethics. It concerns the transparency of AI algorithms and the ability to understand and interpret their inner workings and decisions. Without this understanding, we run the risk of blind reliance on AI's 'black box' that may lead to erroneous or biased outcomes. It is critical to scrutinize AI models' reasoning processes, ensuring they align with human values and principles and are not based on inappropriate or discriminatory features.

In the context of healthcare, these considerations take on an even greater significance. AI applications in healthcare, such as diagnostic tools or treatment recommendation systems, directly impact human lives. They may influence critical decisions such as who gets treatment, what kind of treatment is administered, and when it should be given. These systems must not only be accurate but also transparent, fair, and accountable. They should be designed and implemented in a way that respects patient rights, including privacy, autonomy, and informed consent.

Therefore, in healthcare, the need for ethical \ac{ai} and model explainability is not just a matter of good practice, it's a matter of life and death. Bias or errors in AI could lead to misdiagnoses or inappropriate treatment recommendations, with potentially fatal consequences. Similarly, if AI-based systems make decisions that healthcare professionals or patients can't understand, it may lead to mistrust and potential harm. It's crucial for the advancement of AI in healthcare to ensure ethical considerations and explainability are at the core of AI model design, development, and deployment. This will build trust in AI systems and ultimately lead to better health outcomes.


The European Health Data Space (EHDS) refers to a strategic initiative by the European Union (EU) aimed at creating a unified and secure platform for sharing and accessing health-related data across member states. Artificial Intelligence (AI) is expected to have a significant impact on the EHDS in several ways:

\begin{itemize}
    \item Improved Diagnostics and Personalized Medicine: AI can analyze vast amounts of health data, including medical records, imaging, and genetic information, to enhance diagnostic accuracy and tailor treatments to individual patients. This can lead to more effective and efficient healthcare delivery.
\item Data Integration and Interoperability: AI can help harmonize data from various sources within the EHDS, including electronic health records, wearable devices, and clinical databases. This promotes interoperability, allowing healthcare professionals to access comprehensive patient information seamlessly.

\item Predictive Analytics: AI-powered predictive models can help forecast disease outbreaks, patient admission rates, and healthcare resource utilization. This enables better resource allocation and proactive healthcare planning.

\item Drug Discovery and Development: AI can accelerate drug discovery by analyzing genetic data, identifying potential drug candidates, and predicting their efficacy and safety profiles. This can expedite the development of new treatments and therapies.

\item Enhanced Clinical Decision Support: AI can provide healthcare providers with real-time decision support, offering recommendations based on the latest medical evidence and patient-specific data. This can lead to more informed clinical decisions and better patient outcomes.

\item Data Security and Privacy: The EHDS must ensure the privacy and security of health data. AI can help by implementing robust encryption, access controls, and anomaly detection systems to safeguard sensitive information.

\item Research and Insights: AI can facilitate large-scale data analysis for medical research, enabling researchers to identify patterns, correlations, and potential breakthroughs in healthcare. This can lead to advancements in medical knowledge and treatments.

\item Patient Engagement and Monitoring: AI-driven apps and wearable devices can empower patients to take a more active role in managing their health. These technologies can monitor vital signs, offer health advice, and send alerts to healthcare providers when necessary.

\item Reduced Healthcare Costs: By optimizing healthcare processes, improving diagnosis accuracy, and preventing medical errors, AI can contribute to cost savings within the healthcare system, making it more sustainable.

\item Regulatory Challenges: Implementing AI in healthcare requires navigating complex regulatory frameworks, ensuring ethical use, and addressing issues related to bias and fairness in AI algorithms. The EHDS will need to establish guidelines and standards to address these challenges.

\end{itemize}


In summary, AI is poised to revolutionize the European Health Data Space by enhancing the quality and accessibility of healthcare data, improving patient outcomes, and fostering innovation in medical research and treatment. However, it also presents challenges related to data privacy, security, and ethical considerations that must be carefully addressed in the implementation process.
\begin{savequote}[85mm]
    Nothing travels faster than the speed of light, 
    with the possible exception of bad news, which obeys its own special laws.
    \qauthor{Douglas Adams}
    \end{savequote}


\chapter{Research Methods to Improve Data Quality}\label{chap:goal1}
This chapter focuses on enhancing data quality through innovative research methods. Firstly, the utilization of synthetic data generation, as detailed in sections \ref{subsec:gans}, \ref{subsec:tabular}, and \ref{subsec:similarity}, serves a dual purpose: expanding the volume of available data while simultaneously safeguarding privacy. This approach focuses on  techniques such as \acp{gan} to create realistic, yet non-sensitive data sets. Secondly, the development of automatic data quality assessment methods, explored in section \ref{subsec:dq}, marks a significant stride in ensuring the integrity and reliability of data. These methods aim to automate the process of evaluating data quality, thus reducing manual effort and increasing the efficiency and accuracy of data analysis.



\section{Can GANs Help Create Realistic Datasets?}\label{subsec:gans}
This section is based on the paper entitled "GANs for Tabular Healthcare Data Generation: A Review on Utility and Privacy". It focuses on a review of the \ac{gan} framework for creating synthetic data for healthcare. Tries to compile the metrics used for comparing and assessing synthetic data in terms of utility - or how similar they are to the original data and privacy - how protective of the patient's data it is. 

% !TeX root = ../thesis.tex

\subsection{Introduction}

With the growing technological advances, the quantity of healthcare-related data produced around the world increased exponentially \cite{choi_generating_2017,henry_adoption_2016}.
Consequently, the potential for harvesting this data also increases. The value locked within
this data could help provide better healthcare with new information about diseases,
drugs, and preventive therapies. It can also help create better \acp{his}, meaning an overall better clinical practice \cite{ISI:000502534100049}. But for this to happen, data must reach capable hands at the right time.
But the release of clinical data has several barriers attached and rightly so. The leakage of patient’s privacy can break the confidence of the population in healthcare
professionals and institutions. Patient safety
and privacy should be kept at all costs. However, the current mechanisms for privacy maintenance are very long, bureaucratic, and time-consuming, nationally \cite{comissao_nacional_protecao_de_dados_principios_2015}, and internationally \cite{office_for_civil_rights_guidance_2013}. The current scenario and general methods for privacy safeguards are related to pseudo-anonymisation techniques.
The removal of certain attributes, identifier modification, code grouping, or discretization are some methodologies. But not even these are totally safe \cite{el_emam_systematic_2011}.
Synthetic data appear as an alternative for clinical data sharing, promising great data
utility with minimal privacy concerns. Synthetic data is data that is generated automatically through programmatic processes. This is especially impactful for the case at hand
since synthetic data has no explicit connection with the original data. There are several
mechanisms for data synthesis postulated by \cite{goncalves_generation_2020}, there are
process-driven methods and data-driven methods. Process-driven methods generate
data through pre-determined models inputted into the generator. Data-driven methods
produce new data based on inputted source data. With this, it is possible to create new
patient data that has no relation to reality while providing the same statistical relations
between variables. This provides the basis for quality clinical research on top of this
new data. Even though these techniques are still new and in rapid development, the
results seem interesting \cite{goncalves_generation_2020}, but not without questions and doubts
\cite{stadler_synthetic_2020}.
Creating a thorough survey based on the generation of synthetic data is seldom a simple task when compared to other surveys since synthetic data is present across several domains and has several uses, like software testing, assessing methods, or generating hypotheses. Moreover, synthesis has
the double meaning of summing up information and generating something, easily wielding hundreds of results per query. Finally, trying to filter
algorithms aimed at tabular data is also burdensome, since not always it is easy to discriminate input types. These factors make the survey interesting to focus on the state-of-the-art mechanisms of generating tabular data.

\subsection{Theoretical background}
First introduced in 2014, \acp{gan} \cite{goodfellow_generative_2014} have been under the scope and have been proven very good for generating complex data. Images, text, and video have been successfully generated with very good performances. %cite?
The original architecture is based on two artificial neural networks trained simultaneously in a competitive manner. One of them, the generator, has the objective of generating the most realistic possible data, while the second network – the discriminator, has the opposite aim of aiming to distinguish the realistic data from the synthetic
data the best it can. So, the elegance of this architecture is that each network tries to
make the other perform better every time. The \ac{gan} architecture is shown in \ref{fig:gan-arch}.
% w - \omega
% θ - \theta
% G - generator
% D - discriminator

\begin{figure}
\centering
%\includegraphics[width=\textwidth]{image.png}
\includegraphics[scale=0.75]{figures/image.png}

\caption{\ac{gan} framework} \label{fig:gan-arch}
\end{figure}
The generator is represented by $G_{\theta}$ where the parameter $\theta$ represents the weights of
the neural network. It takes as input, a Gaussian random variable, and outputs $G_{\theta}$(Z).
Distribution of $G_{\theta}$(Z) is denoted by $P_{\theta}$. The goal of the generator is to choose $\theta$ such that the output $G_{\theta}$(Z) has a distribution close to the real data. The discriminator is represented by $D_{\omega}$, parametrized by weights $\omega$. The goal of the discriminator is to assign 1 to the samples from the real distribution $P_{X}$ and 0 to the generated samples ($P_{\theta}$). So, \acp{gan} can be mathematically represented by a \textit{MinMax} game identified by:
\begin{equation}
\min_{G}\max_{D} \; E [log(D_{\omega}(X)) + log(1-D_{\omega}(G_{\theta}(Z))]
\end{equation}
So, $G$ must minimize this equation and $D$ must maximize it, each one tweaking the weights of its network ($\theta$ and $\omega$) to do so. This is the loss function on the initial \ac{gan} architecture. After the classification of $D$, the $G$ is trained again with the error signal from $D$ through backpropagation. This equation is the log of the probability of $D$ predicting that the real data is genuine and the log probability of $D$ classifying synthetic data as not genuine. The equation is essentially the same as minimising the \ac{jsd} \cite{goodfellow_generative_2014}:
\begin{equation}
\min_{G} JS(P_{x}||P_{\theta})
\end{equation}
Where the JS means the \acl{jsd} between the probability of the real data and the probability of the generated data. The JS divergence provides a measure of the distance between two probability distributions. Therefore, the minimization over $\theta$ means, choosing the $P_{\theta}$ that is closest to the target distribution $P_{X}$ in the JS divergence distance. Despite the significant results provided by \acp{gan} with continuous real values, categorical values still seem to be a problem for this approach \cite{kusner_gans_2016}, since it is not directly applicable for calculating the gradients of latent categorical variables in order to train these networks through backpropagation. This happens since the output of the generator, even though can be transformed into a multinomial distribution with a \textit{softmax} layer, sampling from it is not a differentiable operation, limiting the backpropagation process of the \ac{gan}.

%One of the most famous alternative GAN architectures is the \textit{Wasserstein}  GAN (WGAN)
%\cite{arjovsky_wasserstein_2017} . It improves GAN by using the \textit{Wasserstein} distance instead of the \textit{Jensen-shannon divergence}. It is represented by the
%following minimax game:
%\begin{equation}
%\min_{G}\max_{\omega \; \epsilon \;W} \;\; E[f_{\omega}(X)] - E[f_{\omega}(G_{\theta}(Z))] 
%\end{equation} 
%Where functions $f_{\omega} \;  \omega \; \epsilon \;W$  are all \textit{K-Lipschitz} (with respect to $x$) for some K.
%This was proposed because it helps prevent mode collapse, where the generator maps
%several inputs to the same output. An example of this is a generator that starts creating
%several images with the same patterns or colours. The fact that this loss function can be
%very customisable in distinct GAN implementation, applying different functions can
%lead to different outcomes and architectures.

%\iffalse
%\subsubsubsection{Differential privacy}
%The privacy model most used in generative models is differential privacy \cite{dwork_differential_2006}. It is a mathematical model for measuring privacy loss. The basis of it is a mechanism that adds noise to a dataset through randomisation. Applying differential privacy is a method for guaranteeing that adding or removing any record from the original data does not influence the generation output. The formal definition, as stated in \cite{dwork_differential_2006} is:
%\begin{equation}
%Pr[M(x)\; \epsilon \; S] \leq e^\varepsilon Pr[M(y)\; \epsilon \; S] + \delta
%\end{equation} 
%Where M is a randomised mechanism, $\varepsilon$ is the privacy budget (balance between privacy and utility). The $x$ and $y$ represent adjacent datasets that differ on only one record. So, we say that M gives $\varepsilon$-differential privacy if the probability of seeing an event S in the $x$ dataset is at most equal to $e^\varepsilon$ multiplied by the probability that we see S when the dataset is $y$. Variable $\delta$ is the probability of an uncontrolled breach. With this, a smaller $\varepsilon$ will yield better privacy but a less accurate response, so it can be used to tweak the utility and privacy of the dataset.
%\fi

\subsection{Methods}
This search was made between December 2020 and January 2021. It was made on “Web of Science”, IEEE, PubMed, Arxiv and finally GitHub. The terms searched were related to \acp{gan}, synthetic data generation, electronic health records, patient data, or tabular data. Applications of \acp{gan} to non-tabular data were filtered, like image, sound, video, or graphs. Time series and text data were also removed since the methodology for synthesizing this type of data has specific functions related to the nature of the data. The filter for date was after 2014 since \acp{gan} were introduced at that time. The queries used were similar to the one below, adapted for the search mechanics for each website.


\begin{scriptsize}

{\fontfamily{courier}\selectfont
("generation" OR "creation" OR "synthesis" OR "synthesizing" OR "generating" OR "creating") AND ("synthetic data" OR "synthetic patient" OR "synthetic electronic health record" OR "synthetic EHR" OR "realistic patient data" OR "realistic health record" OR ("synthetic" AND "privacy" AND "utility"))  AND ("GAN" OR "Generative Adversarial Network")}
\end{scriptsize}
%\begin{verbnobox}[\scriptsize]
%("generation" OR "creation" OR "synthesis" OR "synthesizing" OR
%"generating" OR "creating") AND ("synthetic data" OR
%"synthetic patient" OR "synthetic electronic health record" 
%OR "synthetic EHR" OR "realistic patient data" OR "realistic 
%health record" OR ("synthetic" AND "privacy"  AND "utility")) 
%AND ("GAN" OR "Generative Adversarial Network")
%\end{verbnobox}

From the total articles found (1165) with all the queries, 100 articles were chosen for full text and in the end, 22 papers with \ac{gan} implementations that were tested on tabular data were selected. 
%These implementations were studied and evaluated in terms of privacy concerns and utility performance and compared when possible. 
%For selecting the articles and helping with paper selection, RAYYAN QCRI online \cite{rayyan:14:cochrane} was used along with Mendeley Reference Manager for paper management.


\subsection{Results}
The selected papers ranged from 2017 to 2020. Being that 2 are from 2017, 4 from 2018, 8 from 2019 and 8 from 2020. All authors showed original \ac{gan} implementations, apart from 2 papers. Beaulieu-Jones et al. \cite{beaulieu-jones_privacy-preserving_2019} used a
\ac{gan} architecture that was originally published with usage on image datasets \cite{odena_conditional_2017}. Additionally, Vega-Marquez et al.  \cite{ISI:000490706700022} used an already known implementation of conditional \acp{gan} \cite{mirza_conditional_2014}. We classified papers regarding 3 metrics: utility, privacy and clinical. For utility, we looked for  methods for measuring the generated data's quality. As for privacy, we aimed for some mechanism for measuring the privacy loss of the new data. Concerning clinical metrics, any kind of evaluation from healthcare professionals was considered. This can be seen in table \ref{tab:review_gans}.


\begin{table}[btph]
\caption{Summary of the articles selected.}\label{tab:review_gans}
\centering
\begin{tabular}{l|llclc}
%\begin{tabular}{l|llclc}

\toprule
 ID & year  & Acronym & Article & Metric & Code \\

% & \multicolumn{1}{|c}{Year \hspace{2 mm}}  & \multicolumn{1}{|c|}{Acronym} & \multicolumn{1}{c}{\hspace{2 mm} Article \hspace{1 mm}} & \multicolumn{1}{|c|}{Metric} & \multicolumn{1}{c|}{\hspace{2 mm}Code \hspace{2 mm}}  \\

\midrule
1 & 2017 & medGAN  &\cite{choi_generating_2017}          & Utility, Privacy, Clinical \hspace{1 mm} &  \cite{medGANurl}  \\

2 & 2017 & POSTER & \cite{ISI:000440307700174} & Utility, Privacy  &   \cite{POSTERurl}\\

3 & 2018 & table-GAN & \cite{park_data_2018} & Utility, Privacy         &   \cite{table-GAN-url} \\

4 & 2018 & dp-GAN & \cite{xie_differentially_2018}& Utility, Privacy &   \cite{dp-GAN-url}  \\

5 & 2018 & mc-medGAN & \begin{tabular}[c]{@{}l@{}}\cite{camino_generating_2018}\end{tabular}    & Utility &   \cite{mcmedgan-url}\\

6 & 2018 & TGAN &  \cite{xu_synthesizing_2018}& Utility &  \cite{tgan-url}\\

7 & 2019 & PATE-GAN & \begin{tabular}[c]{@{}l@{}}\cite{jordon_pate-gan_2019}\end{tabular}    & Utility, Privacy   & --\\

8 & 2019 & SPRINT-GAN & \cite{beaulieu-jones_privacy-preserving_2019}           & Utility, Privacy, Clinical &  \cite{sprint-GAN-url} \\

9 & 2019 & GAN-based &     \cite{ISI:000524576200016} & Utility, Privacy & --\\

10 & 2019 & CTGAN &  \cite{xu_modeling_2019} & Utility & \cite{CTGAN-url}\\

11 & 2019 & WGAN-DP & \begin{tabular}[c]{@{}l@{}}\cite{brenninkmeijer_generation_2019}\end{tabular} & Utility, Privacy    & \cite{WGAN-DP-url} \\

12 & 2019 & PPGAN &  \cite{liu_ppgan_2019} & Utility, Privacy &  \cite{ppgan-url}\\

13 & 2019 & medBGAN &   \cite{ISI:000502534100049} & Utility & --\\

14 & 2019 & medWGAN& \cite{medwgan}& Utility &  \cite{medWGAN-url}  \\

15 & 2020 & ADS-GAN& \begin{tabular}[c]{@{}l@{}}\cite{ISI:000557358500024}\end{tabular} & Utility, Privacy& --\\

16 & 2020 & corGAN & \cite{2001.09346}& Utility, Privacy &  \cite{corGAN-url}\\
17 & 2020 & CGAN &  \cite{ISI:000490706700022}& Utility& --\\

18 & 2020 & DPAutoGAN&  \cite{tantipongpipat2020differentially}& Utility, Privacy & \cite{DPAutoGAN-url}\\

19 & 2020 & GAN Boosting& \cite{2007.11934} & Utility, Privacy &  \cite{postganurl}\\

20 & 2020 & RDP-CGAN&  \cite{rdp-gan}& Utility, Privacy &  \cite{rdp-cgan-url}\\

21 & 2020 & WCGAN-GP&  \cite{WCGAN-GP}& Utility, Privacy & --\\
22 & 2020 & SMOOTH-GAN &  \cite{smooth-gan} & Utility & \cite{smooth-gan-url}\\
\hline
\end{tabular}
\end{table}


The metrics the authors used are exhibited in table \ref{tab:results_review}. 



\begin{landscape}
\renewcommand{\arraystretch}{1.02} %add more space between cells (1.2 is a factor)
\begin{table}[htbp]
\caption{Metrics utilised for evaluation} \label{tab:results_review} 

\begin{tabular}{p{26mm} p{84mm} p{60mm}}
\hline
Acronym & Utility & Privacy \\
\hline
medGAN	& \begin{enumerate*}
    \item Bern.
    \item Pred F1
\end{enumerate*} & \begin{enumerate*}
\item Attrib. disc. \item Memb. inf. \item KNN	\end{enumerate*}  \\

%\arrayrulecolor{mygrey}\hline

POSTER &	\begin{enumerate*}
\item Pred Acc.
\item Corre. Mat. \item BD
\end{enumerate*} & DP\\
table-GAN &	\begin{enumerate*}
\item Cumul. Dist.
\item Pred F1$\vert$MRE
 \end{enumerate*} &	\begin{enumerate*} \item Eucl.
\item Member. inf.   \end{enumerate*}  \\

dp-GAN &	\begin{enumerate*} \item Pred AUC \item Bern. \end{enumerate*} &	DP	 \\

mc-medGAN & 	\begin{enumerate*} \item Pred F1$\vert$AUC \item Bern. \item ME F1$\vert$Acc\end{enumerate*} & -- 	\\

TGAN &	\begin{enumerate*} \item KNN \item NMI \item Pred F1 \end{enumerate*} & --\\

PATE-GAN & \begin{enumerate*} \item  Pred AUC$\vert$AUPRC \end{enumerate*}	& DP \\
SPRINT-GAN & \begin{enumerate*}	\item Pred AUC \item Corre. Mat. \end{enumerate*} &	DP \\

GAN-based &	  \begin{enumerate*} \item Pred Acc. \item Corre. Mat. \end{enumerate*} & \begin{enumerate*} \item Hit. Rate \item R. Linkage  \item Eucl. \end{enumerate*} \\

CTGAN &	\begin{enumerate*} \item Pred F1$\vert$R2$\vert$Acc. \end{enumerate*} &  --\\

WGAN-DP &	\begin{enumerate*} \item Corre. Mat.
 \item PCA \item  Pearson RMSE\newline\item Pred F1$\vert$RMSE$\vert$1-MAPE(F1)  \end{enumerate*}  & \begin{enumerate*} \item Eucl. \item Dupl. \item DP \end{enumerate*}	\\

PPGAN &	\begin{enumerate*} \item GS \end{enumerate*}	& DP \\

medBGAN	& \begin{enumerate*}  \item Assoc. Rul.
 \item  CCS Pred F1
\item KS
 \end{enumerate*}	& -- \\

medWGAN	& \begin{enumerate*} \item Assoc. Rul.\item  CCS Pred F1 \item KS \end{enumerate*} & -- \\

ADS-GAN &  \begin{enumerate*} \item $\chi^{2}$ \item JSD \item WD
\item t-test \item Pred AUROC\newline
\item Corre. Mat. \end{enumerate*}	& DP \\

CorGAN &	\begin{enumerate*} \item Pred F1 \item Bern. \end{enumerate*}
 &	Member. Inf. \\

CGAN &	\begin{enumerate*} \item Pearson
\item Spearman \item Pred F1$\vert$AUC$\vert$Acc \end{enumerate*}  &	-- \\

DPAutoGAN & \begin{enumerate*} \item Pred AUROC$\vert$R2  \item Bern. \end{enumerate*}	
 &	DP \\
GAN Boosting & \begin{enumerate*} \item pRMSE \item Pred AUROC$\vert$AUPRC$\vert$Acc. \end{enumerate*}	
	& DP \\
RDP-CGAN & \begin{enumerate*} \item Pred F1$\vert$AUROC$\vert$AUPRC \item MMD  \end{enumerate*}	
	& DP \\
WCGAN-GP & \begin{enumerate*} \item Corre. Mat.
 \item Pred F1  \end{enumerate*} & \begin{enumerate*} \item Dupl.
\item Eucl. \end{enumerate*} \\
SMOOTH-GAN & \begin{enumerate*} \item DW MAE \item Pearson  \item Pred AUROC$\vert$AUPRC \end{enumerate*}	
	& -- \\
\hline

\end{tabular}
\end{table}
\end{landscape}


Regarding privacy, 15 papers assessed it or included some kind of mechanism to improve data protection. The most common was including Differential Privacy (DP) in the generation process. Other mechanisms for measuring privacy loss were Membership Inference (Member. Inf.), Attributes Disclosure (Attrib. Disc.), Euclidean distance (Eucl.), record-linkage (R. Linkage) and \ac{knn}.
As for utility, all papers assessed it. There were 3 major areas of utility assessment: Dimension-wise (DW) probability, cross-testing, and distance metrics. The most basic one was dimension-wise probability, which is important for making sanity checks for the generated data, comparing the distributions of each column between real and synthetic. In this category, we can find Bernoulli (Bern.), cumulative distributions (Cumul. Dist.), Pearson correlation (Pearson) and Spearman correlation (Spearman), correlation coefficients (CCS), chi-squared test ($\chi^{2}$),  \ac{ks} or Correlation Matrices (Corre. Mat.).
Cross-testing was about training machine-learning algorithms with both datasets in order to compare the results. The key factor is generating a synthetic dataset based on the training set and then training models on the original training set and the generated dataset. Then the models are compared regarding their predictive capability on the (real) test set. This was a way of assessing if the generator models were capturing inter-variable relationships. The authors applied different metrics from AUC, F1, \ac{auprc}, Accuracy (Acc.) to \ac{mre}. Finally, there was also the application of distance metrics, for measuring the difference between column distribution in both datasets. \acl{jsd}, Wasserstein Distance (WD), Bhattacharyya Distance (BD) or Generate Scores (GS) that was a metric implemented by the authors of \cite{liu_ppgan_2019} that creates a metric based on the sum of the mean of \textit{kullblack-leibler}  distance of all columns. Other less used methods were \ac{pca}, propensity score mean squared error ratio (pMSE). NMI (Normalised Mutual Information), which is the ability to capture correlations between columns by computing the pairwise mutual information and MMD (Maximum Mean Discrepancy), which is similar to distance metrics were also used. Regarding datasets utilized, the most used was MIMIC-III \cite{mimiciii} (9 times). The papers used 27 different datasets, being 16 healthcare-related and 11 non-healthcare related. 
Finally, regarding clinical evaluation, only two papers assessed it, as it is possible to see in table \ref{tab:review_gans}. Both had a group of clinicians assessing a sample of both real and synthetic information and evaluating from 0 to 10, where 10 is the most realistic.
One major point preventing a larger comparison is that despite some papers using the same dataset and same methodologies, the presented values are different, making it difficult for a clear comparison of results. One example is a dimension-wise prediction with F1 score for MIMIC-III. CorGAN presents the mean difference between the two classifications (real on real and synthetic on synthetic), while medBGAN presents the correlation coefficients of the two, and medGAN only presents the visual comparisons. Regarding code availability, 16 papers had the code publicly available in some form. As of January 2021, papers pointed in table \ref{tab:review_gans} have public code.

%\begin{figure}[h]
%\centering
%\includegraphics[scale=0.50]{Paper_plot.png}
%\includegraphics[scale=0.54]{Rplot01.jpeg}

%\caption{Datasets used for measuring Utility and Privacy.} \label{fig2}
%\end{figure}



\subsection{Implications for future research}
From the work done in this paper, it is clear that synthetic data generation is a growing field. The increasing number of papers through the years as the growing quality in the mechanisms of generating data and assessing its quality is clear proof. 
It also became apparent that privacy and utility in synthetic data represent a delicate balance. The very same definition of differential privacy represents it. The compromise between privacy and utility is real and should be taken into account when creating privacy-demanding datasets.
Creating statistically good tabular datasets is already possible, but that task becomes increasingly difficult if privacy concerns are added. 
However, privacy is also a complex subject, and the context of the setting is important for privacy assessment, which explains the different approaches for evaluating privacy protection of synthetic data.
From this review, we believe that a proper evaluation of synthetic data generators in the healthcare setting with privacy concerns should at least include utility and privacy evaluations. For utility, we believe that evaluating column-wise is a nice first check but insufficient alone.
For table-wise, since there is no fundamental metric for assessing the inter-column correlations between mixed-type variables, cross-testing is the best next thing. Distance metrics are nice to have and seem to have the potential for creating a table-wise metric \cite{metrics}, so presenting them is important. Second, for privacy evaluation, we believe that Differential Privacy in itself is not a guarantee of protection for real patients. More research and depth should be employed when presenting results for such generators; record-linkage and attribute disclosure can provide extra guarantees.
Thirdly, a clinical evaluation should be done as well to understand if the synthetic patients are a reality in the clinical setting. Since the correlations could be correct but clinically (or biologically) they might not make sense. Finally, in the scope of this paper, only \acp{gan} were assessed, but there are more mechanisms for generating data, and could be interesting to assess how all of them perform on the same datasets. There are other methods for handling the mixed data types that regularly appear in clinical settings, like \acp{vae} Gaussian Mixtures, \ac{bn}, and imputation mechanisms, making them excellent candidates for this assessment.



\subsection{Conclusion}
In this paper, we had the opportunity to survey the current framework for generating tabular data using \acp{gan} and which ones were already tested in the healthcare setting. We summarised the utility and privacy metrics employed, and the datasets used to measure them. We analysed the code availability and made suggestions for further work on cataloguing, comparing, and assessing synthetic health data generators. A survey with a global benchmark of methodologies, despite being arduous, could yield great results for the community and take the aim of this paper further.

\section{How Can We Compare Two Tabular Datasets?}\label{subsec:tabular}
This section is based on the paper entitled "Dataset Comparison Tool: Utility and Privacy". This work followed the work on section \ref{subsec:gans}, where we compiled ways of assessing the utility of synthetic data. We understood that the mechanisms were far from consensual and a tool could be of use to merge all of this into a single file and report about data. Our purpose was to facilitate health data owners and legal responsible to understand how similar and protective a dataset was regarding the original one.


\subsubsection{Introduction}
Synthetic data can be defined as data that has no connection with a real-world phenomenon or event. It was not originated from a process in the real world, but rather a synthetic one. The idea is that synthetic data can have similar properties with real data, without needing to have an independent process for its generation.
Synthetic data has been used over the years for several usages, but in healthcare is still not very used. However, this scenario seems to be changing. It can be used for several use cases namely \cite{synthetic-data-usage}; i) Software testing, ii) educational purposes, iii) \ac{ml}, iv) regulatory, v) retention, vi) secondary and vii) enhanced privacy.

Software testing relates to using synthetic data to create use cases for software testing. This can be used for the development or pre-production stages for example. Often the data needed is not available on-demand and a synthetic generator of reliable data could be useful. Educational purposes relate to, at least, two different scenarios. One is for onboarding of employees \cite{synthetic-data-usage}, other is related to healthcare students for using health information systems and creating mechanisms for providing reliable data on-demand.
\ac{ml} is one of the areas where synthetic data has more widespread usage, where data augmentation through data synthesis is already common. It can be used for class imbalance, sample-size boosting or machine-learning algorithms testing. Regulatory purposes could be important as well, with the rise of \ac{ai} as medical device systems and synthetic data could be used to properly evaluate these systems under controlled environments. Retention can be an important case for synthetic data as well, since personal data must not be kept more than it would be necessary. Synthetic data generators can be of use, where the original data can be deleted and a generator kept for further usage, given that the privacy mechanisms are properly employed. Secondary uses relate to using synthetic data to share data with academia or industry. Simulacrum \cite{simulacrum} is a nice example of how the NHS uses these mechanisms to help scientists get a better grasp of data before having to fill documentation to query the real data. The same occurs for \ac{IKNL}, which has a synthetic version of the cancer registry for scientific purposes \cite{synthetic_2} and the \ac{MHRA} that uses synthetic data as well for its CPRD real-word evidence \cite{mhra_cprd}.

Finally, an aspect that is underlying to all these applications is the promise that synthetic data can be used to improve privacy. Even though specially tweaked data generators can be used to create more privacy-aware datasets, it will be inherently always at the cost of some utility \cite{stadler_synthetic_2020}. So, even though synthetic data it is not the silver bullet as primarily thought, synthetic data generation can be undeniably used to help create more private data for all the use cases seen above, at the cost of its utility. As for proper methods of evaluating security and utility, are, for now, open research questions. At the present time, it is still complicated to properly assess the utility of the generated data. We have qualitative and quantitative methods. Qualitative methods are related to plots, quantitative are related to some value that defines a evaluation metric. These quantitative metrics can be applied to equal columns from each data set, pair of column from each dataset or applied over the whole datasets. As for privacy metrics, the metrics rely on duplicates. Full duplicates or membership inference related. 

So in this paper, we developed a data pipeline for data analysis in order to create a report for providing several metrics for data utility and privacy.


\subsubsection{Methods}

The pipeline relies on python and latex for creating the document. It relies also on several packages that implemented methods for evaluating data, namely scipy \cite{scipy}, sdmetrics \cite{sdv} and scikit-learn \cite{scikit-learn} and mlxtend \cite{mlxtend}. Its basis is related to uploading 2 datasets, and a report in pdf is produced. Being that is based on programmatic code, it can be easily converted into API.
The report has a section for dataset description, columns removed due to high-null and brief variable overview. Then a null comparison is done by column and dataset. Following this is the utility subsection. Firstly by visual methodologies: heat maps for the correlation, bar plots for categorical, density plots for continuous and a pair plot for an overview. As for the quantitative utility evaluation, we divided it column-wise, pair-wise and table-wise. The first comprehends the \textit{Kolmogorov–Smirnov} test for continuous and $\chi^2$ test for categorical variables.  Distance metrics were also applied to categorical columns. First, they are transformed into distributions and then distance metrics are applied. The results is a descriptive overview of the distance metrics, having minimum value, average, max value and standard deviation. The distance metrics selected were \textit{Jensen-Shannon Divergence}, \textit{Wasserstein distance}, \textit{Kullback–Leibler divergence} and entropy.
As for pair-wise metrics, we used a discrete and continuous \textit{Kullblack-Leibler divergence}. In this, two pairs of continuous columns are compared using \textit{Kullback–Leibler divergence}. For this, they are put into bins for further application. The same is applied to categorical columns without binning.
As for table-wise metrics, first, we used likelihood metrics. We fitted several Gaussian Mixture models or Bayesian network models to the real data and then calculate the likelihood of the synthetic data belonging to the same distribution. The metrics are likelihood for Gaussian mixture and Bayesian models and log-likelihood for the Bayesian model as well.

\begin{center}
\begin{table}[htpb]
    \caption{Metrics Assessed}
    \label{tab:variables}
\begin{tabular}{@{}lll@{}}   \toprule
Metric                       & Method       & Context         \\\midrule
Bar Plot                     & visual       & utility         \\ 
KDE Plot                     & visual       & utility         \\ 
Heat-map                     & visual       & utility         \\ 
Pair-plot                    & visual       & utility         \\ 
KS test                      & column-quantitative & utility         \\ 
ChiSquared Test              & column-quantitative & utility         \\ 
Kullback–Leibler divergence  & column-quantitative & utility         \\ 
Jensen-Shannon Divergence    & column-quantitative & utility         \\ 
Wasserstein distance         & column-quantitative & utility         \\ 
Entropy                      & column-quantitative & utility         \\ 
DiscKLD                      & table-quantitative & utility         \\ 
ContinuousKLD                & table-quantitative & utility         \\ 
BNLikelihood                 & table-quantitative & utility         \\ 
BNLogLikelihood              & table-quantitative & utility         \\ 
GMLogLikelihood              & table-quantitative & utility         \\ 
Same dataset ratio           & table-quantitative & utility         \\ 
Support rules                & table-quantitative & utility         \\ 
Different dataset validation & table-quantitative & utility         \\ 
Duplicates                   & quantitative & privacy         \\ 
Duplicate minus 1            & quantitative & privacy         \\ 
Record Linkage             & quantitative & privacy         \\ 
Matrix distance              & quantitative & privacy/utility  \\ 
Cosine distance              & quantitative & privacy/utility \\ 
Euclidean distance           & quantitative & privacy/utility \\ \bottomrule

\end{tabular}
\end{table}
\end{center}


Then we used machine-learning models (linear regression and decision trees) to assess how similar models behave on both datasets. First, we tested on the same dataset in order to compare evaluation metrics. Then we cross-tested, meaning that the training set was drawn from one dataset and the test set was drawn from the second dataset. Finally, data privacy constraints duplicate evaluation, duplicate existence by removal of a single column and a record linkage approach. With the record linkage, we define a record linkage blocking ("age" in the example) and then try to match rows from the synthetic dataset to the real, with varying known attributes. Then matrix, euclidean and cosine distance was also calculated. Even though it is used for privacy evaluation, by definition, we could also use it for utility assessment. For proper assessment, the continuous and categorical variables should be indicated at the start of the code. The metrics are listed in the table \ref{tab:variables}.




%class GMLogLikelihood(SingleTableMetric):
%    """GaussianMixture Single Table metric.
 %   This metric fits multiple GaussianMixture models to the real data and then
  %  evaluates how likely it is that the synthetic data belongs to the same
%    distribution as the real data.


%class BNLikelihood(SingleTableMetric):
%    """BayesianNetwork Likelihood Single Table metric.
%    This metric fits a BayesianNetwork to the real data and then evaluates how
%    likely it is that the synthetic data belongs to the same distribution.
%    The output is the average probability across all the synthetic rows.


%class BNLogLikelihood(BNLikelihood):
%    """BayesianNetwork Log Likelihood Single Table metric.
%    This metric fits a BayesianNetwork to the real data and then evaluates how
%    likely it is that the synthetic data belongs to the same distribution.
%    The output is the average log probability across all the synthetic rows.
    
%CSTest: Chi-Squared test to compare the distributions of two categorical columns.
%KSTest: Kolmogorov-Smirnov test to compare the distributions of two numerical columns %using their empirical CDF.


%
%class ContinuousKLDivergence(ColumnPairsMetric):
%	 """Continuous Kullback–Leibler Divergence based metric.%
	
%	 This approximates the KL divergence by binning the continuous values
%	 to turn them into categorical values and then computing the relative
%	 entropy. Afterwards normalizes the value applying ``1 / (1 + KLD)``.




\subsubsection{Results}
A trial example of comparing data is available for data in the UCI repository, namely the heart disease dataset \cite{misc_heart_disease_45}. The synthetic data was created by using the synthpop package \cite{synthpop}. The variables evaluated are listed in table below. The code can be seen in https://github.com/joofio/dataset-comparasion-report. As an example. The image for visual analysis for categorical (figure \ref{fig:catgorical}) and continuous variables (figure \ref{fig:continuous}).


\begin{figure}[H]
    \centering
    \includegraphics[scale=0.23]{figures/cat.png}
    \caption{Categorical Variables plotted}
    \label{fig:catgorical}
\end{figure}

\begin{figure}[t]
    \centering
    \includegraphics[width=\textwidth]{figures/continuous_plot_0.png}
    \caption{Continuous Variables plotted}
    \label{fig:continuous}
\end{figure}



\subsubsection{Discussion \& Conclusion}

The data possible to create to evaluate similarities between two datasets is important not only for synthetic vs real datasets. For example in distributed learning, where different silos exist, with similar or even equal features, a method for evaluating the similarities can be useful for understanding how the populations are similar between them, trying to shed light on the most similar among them, or different in order to understand the differences in the silos or data acquisition inside them.
Furthermore, the differences can be assessed on a more granular level. The column-wise similarities can be different from the inter-columns similarities and this in itself, can be a metric of interest regarding the quality of the synthetic data and its generator.

With this work, we hope to help institutions and academics getting to access to a benchmark of the datasets provided in order to leverage synthetic data in the healthcare space. Finally, we hope this work helps other researchers reach an evaluation metric that could be a unique and clear response to the question of how similar two datasets are.




\section{Can We Use Machine Learning Feature to Compare Datasets?}\label{subsec:similarity}
This section is based on the paper entitled "Using Machine Learning Models' feature importance to assess dataset similarity". The reasoning behind this paper was the results of section \ref{subsec:gans}, where we felt that evaluation metrics for synthetic data could be improved. Better yet, we felt that the comparison of two datasets (that shared the same columns) could be done in a more robust way. Being that the current gold-standard was cross-validation which was not bound to any number range and the significance of the result could not be easily interpretable. We used the feature importance of several \ac{ml} models to compare datasets and concluded that it was a valid alternative to the traditional metrics.

\subsection{Introduction}
% !TeX root = ../../thesis.tex

In recent years, the use of \ac{ai} and \ac{ml} algorithms has gained increasing prominence in healthcare research and practice. One of the key requirements for the successful application of these methods is access to large, high-quality datasets. However, in many cases, the availability of such datasets can be limited due to issues around data privacy, security, and ethical concerns \cite{chingOpportunitiesObstaclesDeep2018a}. To address this challenge, synthetic data has emerged as a promising solution. Synthetic data refers to artificially generated data that closely mimic the statistical properties and patterns of real-world data \cite{mullerEvaluationSyntheticElectronic2022}.

Synthetic data has the potential to overcome many of the limitations associated with real-world data, such as the lack of sufficient data volume, noise, and privacy concerns. Even though there are still doubts if the privacy part is the silver bullet sometimes referred to \cite{stadlerSyntheticDataPrivacy2020}, the upsampling part is a standard use for years now. However, the quality of synthetic data generated by various techniques can vary significantly, and it is essential to assess the quality of synthetic data before its usage. In healthcare, the assessment of synthetic data is crucial to ensure that it can provide valid insights and inform decision-making processes.

The assessment of synthetic data in healthcare is essential for its successful use in various applications, such as developing predictive models, testing algorithms, and conducting clinical trials. The use of synthetic data can significantly enhance the efficiency and effectiveness of healthcare research and practice. However, it is crucial to ensure that the synthetic data used in these applications are of high quality and validated to provide reliable and valid insights. The evaluation of synthetic data quality involves comparing its statistical properties and patterns with those of the original data. We can assess how similar columns are to each other through several statistical tests, and then we can infer some inter-column properties with methods like cross-validation, where two datasets are split into train tests and cross-tested and then the ratio between the evaluation result of both datasets is used as a metric \cite{mullerEvaluationSyntheticElectronic2022,goncalvesGenerationEvaluationSynthetic2020a}. However, this methodology is a big proxy for such an inter-column relationship. Can we try to provide a better metric than this one to evaluate how similar are the inter-column relationship of two distinct datasets? In this paper, we suggest using feature importance values to create a more explainable and reasonable metric for inter-column relationships.

\subsection{Rationale and Related Work}
% !TeX root = ../../thesis.tex

Recently there has been a series of works related to assessing how synthetic data generators behave with data like the work of Emam et al. \cite{emamUtilityMetricsEvaluating2022} that especially focused on utility metrics for synthetic data generators. At the moment, comparing data is based on intra-columns and inter-columns relationship. The intracolumn relationship is assumed as something that compares equal columns between datasets, with highly known statistical methods like chi-squared or \ac{ks} like done in the works of \cite{combrinkComparingSyntheticTabular2022} among many others, acting more like sanity checks than anything else. 
Other known metrics are distance-based metrics like \ac{jsd}, Wasserstein Distance, Bhattacharyya Distance or Hellinger distance, which are based on the calculation of the distance between distributions like seen in the works of several teams \cite{ISI:000557358500024,choiGeneratingMultilabelDiscrete2017,Baowaly2019}.

However, regarding inter-column relationships, the metrics applied are often very different across papers. One example of trying to capture inter-column relationship is about the use of propensity score \cite{rosenbaumCentralRolePropensity1983,mullerEvaluationSyntheticElectronic2022} where a classifier is trained to the merged datasets, with the added variable of the original dataset (i.e., 1 for real and 0 for synthetic). The model is trained and the propensity Mean square error is the  mean squared difference of the estimated probability from the average prediction
Most recently, a unified metric appeared as the sum of other metrics known as described in the work of Chundawat et al., \cite{chundawatTabSynDexUniversalMetric2022}, known as TabSynDex. Other examples are likelihood of fitness like in the works of \cite{xuModelingTabularData2019b}, coverage support \cite{goncalvesGenerationEvaluationSynthetic2020a} or very specific metrics implemented for evaluating specific data generators.
However, the most used metric is cross-validation, which takes two datasets, one that is real and a second which is synthetic and we split both into train and test and train a machine learning model on the real data training set, then we test the model on both test sets. Then a ratio is created, rendering the actual value. This methodology, even if gold-standard at the moment for this type of study, has some liabilities since this value can be a bit erratic, and even above one since the evaluation metric could be better on the second dataset and we don't have a clear grasp of what that can represent in terms of dataset similarity. The image \ref{fig:1} represents this in detail. Several works used this metric as the comparing metric \cite{mullerEvaluationSyntheticElectronic2022}.

%TC:ignore
\begin{figure}[tbph]
\centering
\caption{Cross-Validation of datasets}\label{fig:1} 
\includegraphics[scale=0.35]{figures/imagem1.pdf}
\end{figure}
%TC:endignore




%Utility Metrics for Evaluating Synthetic Health Data Generation Methods: Validation Study



\subsection{Materials \& Methods}
\subsubsection{Materials}
% !TeX root = ../../thesis.tex

We used 5 datasets from the \ac{uci} repository. The ones chosen were related to healthcare and were heart disease \cite{misc_heart_disease_45}, thyroid disease \cite{misc_thyroid_disease_102}, liver disorders \cite{misc_liver_disorders_60}, breast cancer \cite{misc_breast_cancer_wisconsin_diagnostic_17} and the primary tumour dataset \cite{misc_primary_tumor_83}. We made minimal preprocessing on the datasets, namely removing the missing variables by imputing the mean on continuous variables and mode on categorical.
We also created a synthetic dataset by applying the \textit{synthpop} package to this data \cite{synthpop}. With this package, all variables were synthesised using the "cart" method, which is rpart implementation of a CART model.


\subsubsection{Method Overview}
% !TeX root = ../../thesis.tex

For this work, our goal is to test several metrics based on the ranking of feature importance of a trained model. Normalized Discounted Cumulative Gain (NDCG) \cite{wangTheoreticalAnalysisNDCG} which is the sum of the true scores ranked in the order induced by the predicted scores, after applying a logarithmic discount. Then divide by the best possible score to obtain a score between 0 and 1. It is calculated by
\[
\text{{NDGC}} = \frac{{\text{{DCG}}(P)}}{{\text{{IDCG}}(P)}}
\]
where $\text{{DCG}}(P)$ is the Discounted Cumulative Gain and $\text{{IDCG}}(P)$ is the Ideal Discounted Cumulative Gain. 

\begin{small}    
    \begin{table}[H]
        \footnotesize
            \caption{Descriptive statistics of datasets used. Mean (Standard Deviation) for continuous variables. Mode [nr categories] for categorical variables.}\label{tab:descrptive_feature}
                \begin{tabularx}{\textwidth}{lXll|lXll}
                    \toprule
                Dataset & Column                         & Statistic    & \% Nulls & Dataset & Column          & Statistic    & \% Nulls \\
                \midrule
                heart   & Age                            & 54.4 (9.0)   & 0.0      & liver   & gammagt         & 38.3 (39.3)  & 0.0      \\
                heart   & sex                            & 1.0 {[}2{]}  & 0.0      & liver   & drinks          & 3.5 (3.3)    & 0.0      \\
                heart   & cp                             & 4.0 {[}4{]}  & 0.0      & liver   & Selector        & 2 {[}2{]}    & 0.0      \\
                heart   & trestbps                       & 131.7 (17.6) & 0.0      & thyroid & Class           & 1 {[}3{]}    & 0.0      \\
                heart   & chol                           & 246.7 (51.8) & 0.0      & thyroid & T3              & 109.6 (13.1) & 0.0      \\
                heart   & fbs                            & 0.0 {[}2{]}  & 0.0      & thyroid & TST             & 9.8 (4.7)    & 0.0      \\
                heart   & restecg                        & 0.0 {[}3{]}  & 0.0      & thyroid & TSTRI           & 2.1 (1.4)    & 0.0      \\
                heart   & thalach                        & 149.6 (22.9) & 0.0      & thyroid & TSH             & 2.9 (6.1)    & 0.0      \\
                heart   & exang                          & 0.0 {[}2{]}  & 0.0      & thyroid & TMAX            & 4.2 (8.1)    & 0.0      \\
                heart   & oldpeak                        & 1.0 (1.2)    & 0.0      & tumour   & class           & 1 {[}21{]}   & 0.0      \\
                heart   & slope                          & 1.0 {[}3{]}  & 0.0      & tumour   & age             & 2 {[}3{]}    & 0.0      \\
                heart   & ca                             & 0.0 {[}4{]}  & 1.3      & tumour   & sex             & 2 {[}2{]}    & 0.3      \\
                heart   & thal                           & 3.0 {[}3{]}  & 0.7      & tumour   & histologic-type & 2 {[}3{]}    & 19.8     \\
                heart   & num                            & 0 {[}5{]}    & 0.0      & tumour   & degree-of-diffe & 3 {[}3{]}    & 45.7     \\
                breast  & Clump Thickness               & 4.4 (2.8)    & 0.0      & tumour   & bone            & 2 {[}2{]}    & 0.0      \\
                breast  & Uniformity of Cell Size     & 3.1 (3.1)    & 0.0      & tumour   & bone-marrow     & 2 {[}2{]}    & 0.0      \\
                breast  & Uniformity of Cell Shape    & 3.2 (3.0)    & 0.0      & tumour   & lung            & 2 {[}2{]}    & 0.0      \\
                breast  & Marginal Adhesion             & 2.8 (2.9)    & 0.0      & tumour   & pleura          & 2 {[}2{]}    & 0.0      \\
                breast  & Single Epithelial Cell Size & 3.2 (2.2)    & 0.0      & tumour   & peritoneum      & 2 {[}2{]}    & 0.0      \\
                breast  & Bare Nuclei                   & 3.5 (3.6)    & 2.3      & tumour   & liver           & 2 {[}2{]}    & 0.0      \\
                breast  & Bland Chromatin               & 3.4 (2.4)    & 0.0      & tumour   & brain           & 2 {[}2{]}    & 0.0      \\
                breast  & Normal Nucleoli               & 2.9 (3.1)    & 0.0      & tumour   & skin            & 2 {[}2{]}    & 0.3      \\
                breast  & Mitoses                        & 1.6 (1.7)    & 0.0      & tumour   & neck            & 2 {[}2{]}    & 0.0      \\
                breast  & Class                          & 2 {[}2{]}    & 0.0      & tumour   & supraclavicular & 2 {[}2{]}    & 0.0      \\
                liver   & mcv                            & 90.2 (4.4)   & 0.0      & tumour   & axillar         & 2 {[}2{]}    & 0.3      \\
                liver   & alkphos                        & 69.9 (18.3)  & 0.0      & tumour   & mediastinum     & 2 {[}2{]}    & 0.0      \\
                liver   & sgpt                           & 30.4 (19.5)  & 0.0      & tumour   & abdominal       & 2 {[}2{]}    & 0.0      \\
                liver   & sgot                           & 24.6 (10.1)  & 0.0      &         &                 &              &          \\
                  \bottomrule
                \end{tabularx}
    
            \end{table}
        \end{small}
Cohen's kappa coefficient \cite{doi:10.1177/001316446002000104}  is a statistic that is commonly used to assess the level of agreement between two or more raters or evaluators who are providing categorical ratings or rankings of a set of items. So, we want to use to assess if it could be of use to check how similar the ranking of the features is, using the numbers as categorical.
\[
\kappa = \frac{{P_o - P_e}}{{1 - P_e}}
\]

where \(P_o\) is the observed agreement between the two raters and \(P_e\) is the expected agreement between the two raters by chance.



We also intend to use the $R^2$ to check if the explainability changes across datasets.
\[
R^2 = 1 - \frac{{\sum_{i=1}^n (y_i - \hat{y}_i)^2}}{{\sum_{i=1}^n (y_i - \bar{y})^2}}
\]

where \(y_i\) are the observed values of the dependent variable, \(\hat{y}_i\) are the predicted values of the dependent variable, \(\bar{y}\) is the mean of the observed values of the dependent variable and \(n\) is the number of data points.

Then we intend to use ranking metrics, namely Kendall tau, weighted Kendall tau and \ac{rbo}.
Kendall tau is a measure of correlation that measures the similarity between two rankings. It is commonly used in statistics and data analysis to evaluate the agreement or disagreement between two sets of rankings.

The Kendall tau coefficient \cite{kendallTreatmentTiesRanking1945} is defined as the difference between the number of concordant and discordant pairs of observations, divided by the total number of pairs. A concordant pair is a pair of observations that have the same ranking order in both sets, while a discordant pair is a pair of observations that have opposite ranking orders. The Kendall tau coefficient ranges from -1 to 1, where -1 represents perfect negative correlation, 0 represents no correlation, and 1 represents perfect positive correlation. 
\[
\tau = \frac{{\text{{number of concordant pairs}} - \text{{number of discordant pairs}}}}{{\text{{total number of pairs}}}}
\]


Weighted Kendall tau  \cite{vignaWeightedCorrelationIndex2015} is an extension of Kendall tau that takes into account the importance or weight of each observation in the rankings. In some cases, some observations may be more important than others, and their positions in the ranking may have a greater impact on the overall correlation. Weighted Kendall tau assigns a weight to each observation, and the correlation is calculated based on the weighted concordant and discordant pairs.
\[
\tau_w = \frac{{\sum_{i<j} w_{ij} \cdot sgn(x_i - x_j)}}{{\sum_{i<j} w_{ij}}}
\]
where $w_{ij}$ is the weight associated with the pair $(x_i, x_j)$ and $sgn(\cdot)$ is the sign function.

The \ac{rbo} \cite{webberSimilarityMeasureIndefinite2010} is a measure used to compare the similarity of two ranked lists, especially when these lists are of different lengths or have only partial overlap. The \ac{rbo} value ranges from 0 (no overlap) to 1 (complete agreement). One of the key features of \ac{rbo} is that it gives more weight to the top-ranked items. The formula for \ac{rbo} at a given depth \( d \) is as follows:

\[
RBO_d = (1 - p) \sum_{k=1}^{d} \left[ p^{k-1} \cdot \frac{|S_{k} \cap T_{k}|}{k} \right]
\]

Where  \( S_{k} \) and \( T_{k} \) are the sets of elements in the top \( k \) positions of the two ranked lists \( S \) and \( T \) respectively, \( |S_{k} \cap T_{k}| \) is the size of the intersection of these top \( k \) elements, \( p \) is a persistence parameter (usually between 0 and 1) that determines the weight given to the rankings at different depths. A lower value of \( p \) gives more weight to the top-ranked items and \( d \) is the depth to which you are calculating the \ac{rbo}, which can be up to the length of the longest list. This formula calculates the \ac{rbo} up to a finite depth \( d \). The parameter \( p \) is crucial as it models the user's persistence in considering the rankings down the list. The higher the value of \( p \), the more the metric considers items further down the list. 


Finally, we intend to use text-distance metrics. The theory behind this experiment is to treat the ordered columns in a ranking manner and apply text-distance metrics to check the distance between the two. Levenshtein distance \cite{navarroGuidedTourApproximate2001} is the minimum number of single-character insertions, deletions, or substitutions required to transform one string into another. Damerau-Levenshtein distance \cite{navarroGuidedTourApproximate2001} is similar to Levenshtein distance but also includes the transposition of two adjacent characters as an allowable operation. The hamming distance \cite{6772729} is a measure of the difference between two strings of equal length, defined as the number of positions at which the corresponding symbols are different. Jaro-Winkler distance \cite{navarroGuidedTourApproximate2001} is a string similarity measure that takes into account the number of matching characters, the number of transpositions, and the length of common prefixes, with a higher weight given to the common prefix.




\begin{algorithm}[hbtp]
\small
\SetAlgoLined

\For {dataset in datasets list}{
    create two copies of dataset, one that remains the same and a second to be disturbed with permutation \\
\For {ML algorithms in ML algorithms}{
\For {i in number of columns to test}{
\For {rep in  10 repetitions}{ 
create second dataset by permutating values in i columns in the first dataset
\For {target in dataset columns}{
\begin{itemize}
    \item Train-Test Split (95:5) for both
    \item model fit to train for both
    \item get feature importance per column for both
    \item Create an ordered rank of features for both
    \item Create new metrics values by comparing the results from both
    \item make cross-classification to compare with feature importance metrics
    \item aggregate results per metric 
\end{itemize}

 }
 }}}}

 \caption{Testing similarity scores in tabular datasets. Dataset list is the 5 datasets used in this work. \Ac{ml} algorithms are the 6 algorithms used in this work. Number of columns to test is the number of columns in the dataset. 10 repetitions is the number of times the columns are permutated.}\label{alg:simil_1}
\end{algorithm}


Like seen in algorithm~\ref{alg:simil_1}, the \ac{cc} was performed several times and according to the image~\ref{fig:cross_classi1}, we trained the model on dataset1 and tested on dataset1 and 2 and compared the results. However, we applied this twice; 1) where dataset1 is the original dataset and the dataset2 is the permutated/synthetic (RS) and 2) where dataset1 is the permutated/synthetic and dataset2 is the real one or original (SR). Both are examples of \acl{cc}.

The algorithms chosen were decision trees with \textit{gini} entropy function for decision making on splits on classification and squared error for regression; random forests with 100 trees and \textit{gini} criterion for classification and squared error for regression; support vector machines with C-support Vector classification and Epsilon-Support Vector Regression, \ac{knn} with 5 neighbours and uniform weights, linear regression/logistic regression and gaussian naive bayes for classification and Bayesian ridge with 300 maximum iterations and $\alpha$ and $\lambda$ of $1e^{-6}$. All of these were used as implemented in the \textit{scikit-learn} package \cite{scikit-learn}. The hyperparameters chosen were the default ones. We felt that tuning was not necessary here to test our hypothesis, since it is based on the ratio of results.
The text distance metrics were implemented by the text-distance package \cite{orsiniumTextdistanceComputeDistance}. Kendall tau, weighted Kendall tau were used as implemented by \textit{scipy} \cite{virtanenSciPyFundamentalAlgorithms2020a} and \ac{rbo}, as implemented in \cite{chenRankbiasedOverlapRBO2023}.
The methods chosen for creating several synthetic datasets were the synthpop package \cite{synthpop} with "cart" method, which is rpart implementation of a CART model. We also used  the SDV package \cite{SDV} to leverage their implementation of the CTGAN and Gaussian Copula to create 2 more synthetic datasets to test different methodologies of synthetic data creation.








\subsection{Results}
% !TeX root = ../../thesis.tex

With the method described in the algorithm \ref{alg:simil_1}, we created a figure where the metrics are presented with increasingly different datasets: Figure \ref{fig:lineplot}.
%Then we compared the difference in the metric across iterations, rendering figure \ref{fig:boxplot}.

%TC:ignore
\begin{figure}[htbp]
\centering
\caption[Plot showing the variation of different metrics over increasingly changed datasets.]{Plot showing the decrease of the metric over increasingly changed datasets. The X axis represents the number of columns mutated. The Y axis represents the value of the metric and the hue represents the algorithm used to calculate the metric.}\label{fig:lineplot} 
\includegraphics[scale=0.37]{figures/multiple_datasets.png}
\end{figure}
%TC:endignore

The number of repetitions and how that impacts the variance of the scores is shown in Figure~\ref{fig:facet_plot}.


%TC:ignore
\begin{figure}[htbp]
    \centering
    \caption{Heatmap showing the variance of different repetitions for every metric and the number of different columns changed. X is the metric. Y is the number of repetitions and the number of columns. This was obtained by getting the variance of all values from all datasets.  }\label{fig:facet_plot} 
    \includegraphics[scale=0.60]{figures/heatmap-runs.png}
    \end{figure}
    %TC:endignore

As for the test for the synthetic and real dataset, the results are displayed in figure~\ref{fig:synth_result} and ~\ref{fig:synth_heat}. This is the metrics distribution for the comparison of the 5 mentioned datasets and the synthetic counterpart generated as stated in the methods section.
    %TC:ignore
\begin{figure}[htbp]
    \centering
    \caption{Distributions of the metrics results comparing 5 synthetic and real datasets across 3 different generation methods}\label{fig:synth_result} 
    \includegraphics[scale=0.60]{figures/synthetic_violin_swarm_colored_by_group_custom.png}
    \end{figure}
    
    \begin{figure}[htbp]
        \centering
        \caption{Values and comparison of the metrics results comparing 5 synthetic and real datasets across 3 different generation methods}\label{fig:synth_heat} 
        \includegraphics[scale=0.60]{figures/heatmap-synth.png}
        \end{figure}
    %TC:endignore
\subsection{Discussion}
% !TeX root = ../../thesis.tex

With the results found, we feel that are better alternatives to cross-validation. At least  Kendall tau, Weighted Kendall tau, and \ac{rbo} seem like alternatives to cross-validation.
Firstly, they seem to be directly connected to a difference in the dataset, secondly, they are a 0-1 metric and thirdly, variance across different iterations is also lower.
From these, the metrics based on ranking metrics seem to work best, where Kendall tau, Weighted Kendall tau and \ac{rbo} have better performance than the rest. As for the variance with the number of repetitions, we also see that the ranking-based metrics have good stability, while the cross-validation and text-based metric have higher variability with a low number of repetitions (even if low) - figure \ref{fig:facet_plot}. 
Text metrics also have a suitable performance, even though they have a drastic drop with only one column mutated (figure \ref{fig:lineplot}).


\subsection{Conclusion}
% !TeX root = ../../thesis.tex

Comparing two tabular datasets has been growing in demand in the past year mainly because of the increase in popularity of tabular data synthesis methods which have exhibited the potential in generating valuable synthetic data. However, due to the absence of a uniform metric, evaluating different methods has been inconsistent. This research proposes some alternatives for assessing synthetic tabular data's utility. \ac{rbo} seems to have the potential to capture inter-column relationships in a more consistent way than cross-classification. They could become a useful tool for comparing statistical methods of generating synthetic tabular data. Furthermore, this metric can aid in evaluating these generators' training, providing insights into improving synthetic data quality. The proposed metrics open up possibilities for future research to enhance tabular data synthesis methods and compare two datasets overall. Future research could be expanded in new comparison with others evaluation metrics, other datasets and other synthetic data generators.


\section{Can We Use Machine Learning to Create Automatic Data Quality Assessments?}\label{subsec:dq}
This section is based on the paper entitled "Development and Validation of a Data Quality Evaluation Tool in Obstetrics Real-World Data through \ac{hl7} \ac{fhir} interoperable Bayesian Networks and Expert Rules" This paper focuses on the fact that data quality is a major concern in healthcare. We developed a tool that could be used to assess the quality of data in a \ac{ehr} and provide a report on the quality of the data. We used a combination of \ac{bn} and expert rules to assess the quality of the data. Furthermore, we tested the tool on 9 real-world datasets of obstetrics \acp{ehr} and concluded that the tool was a valid alternative to the traditional methods of assessing data quality.
%\input{chapters/data-quality-paper}
\subsection{Introduction}
With the wide spreading of healthcare information systems across all contexts of healthcare practice, the production of health-related data has followed this incremental behaviour. The potential for using this data to create new clinical knowledge and push medicine further is tempting \cite{martin-sanchezBigDataMedicine2014}.
However, to correctly use the data stored in \acp{ehr}, the quality of the data must be robust enough to sustain the clinical decisions made based on this data. Data quality cannot be construed as a linear concept; it is intrinsically dependent on the context in which it is evaluated. The quality thresholds and dimensions required to classify the quality of the data depend on the purpose that we intend to use that very same data \cite{waljiElectronicHealthRecords2019}. These uses can be very distinct and have different impacts as well. For one, we can use data to support day-to-day decisions regarding individual patients’ care \cite{verheijPossibleSourcesBias2018}. These decisions can include ones based on recorded information to understand a patient’s history, clinical decision support systems based on this data, or even using the data to help support a more macro, public health-oriented decision. Another area is using information for management purposes. The data can be used by management bodies and regulatory authorities to extract metrics regarding the quality of care or reimbursement purposes. Thirdly, data can be used for research purposes, namely observational studies and, more recently, to support clinical trials through real-world evidence analysis \cite{coreyAssessingQualitySurgical2020,verheijPossibleSourcesBias2018,wengClinicalDataQuality2020}. 
So, all the \ac{ehr} data-based decisions can only be as good as the data supporting them. Several studies have already warned about the lack of data quality in \acp{ehr} and how this can be a significant hurdle to an accurate representation of the population and potentially lead to erroneous healthcare decisions \cite{reimerDataQualityAssessment2016a,joukesImpactElectronicPaperBased2019a,huserMultisiteEvaluationData2016,zhangUnderstandingDetectingDefects2020,kramerImpactDataQuality2021,gigantiImpactDataQuality2019}.

There are several steps in the data lifecycle that can be prone to error, from data generation, where the data is registered by healthcare professionals, passing by data processing, whether inside healthcare institutions or by software engineers aiming to reuse data, to data interpretation and reuse, where investigators
try to interpret the meaning of registered data\cite{wengClinicalDataQuality2020}.
So, with all of the data’s possible uses added to the several steps that can introduce errors throughout the data lifecycle, data quality frameworks and sequential implementations can have very distinct approaches and methodologies to assess data quality. Data quality tools for checking data being registered live to support day-to-day decisions will be significantly different from one whose only purpose is to provide quality checks for research purposes. So, methodologies to tackle these issues are necessary for guaranteeing the quality of healthcare practice and the knowledge derived from \ac{ehr} data. Consequently, in this paper, we propose:
\begin{myitemize}
    \item Create a tool for identifying data quality issues in obstetrics \acp{ehr};
    \item Enlighten on the issues that can appear with a full deployment of such a tool
    \item Suggestion of a creation of a single score for data quality for comparison of high-quality and low-quality records in a database.
    \item Assess how such a tool can work in early-stage real-world scenarios and how to work with obstetricians to improve data quality.
    \item Identify data quality issues on obstetrics data
\end{myitemize}


%Data quality is a crucial aspect of the healthcare industry, as it impacts the accuracy of diagnoses, treatment plans, and patient outcomes. The reliability and accuracy of healthcare data have far-reaching consequences, including financial implications, patient safety, and legal ramifications. Inaccurate or incomplete data can lead to incorrect diagnoses, inappropriate treatments, and ultimately harm to patients. Therefore, ensuring the quality of healthcare data is essential to providing effective and safe healthcare services.

%One of the main reasons why data quality is so critical in healthcare is that healthcare data is often used to make important decisions, such as treatment plans, patient management, and resource allocation. Inaccurate or incomplete data can lead to misdiagnosis, inappropriate treatment, and increased healthcare costs. Furthermore, inaccurate data can hinder research efforts and impede the development of new treatments and therapies.

%Another key aspect of data quality in healthcare is its role in patient safety. Accurate and reliable data is essential for ensuring patient safety, particularly in areas such as medication management, clinical decision-making, and adverse event reporting. Poor data quality can lead to medication errors, adverse drug reactions, and other types of harm to patients.

%Finally, data quality is also important for legal and regulatory compliance in healthcare. Accurate and complete data is required for compliance with regulations such as HIPAA, the Affordable Care Act, and other regulatory requirements. Poor data quality can result in legal and financial penalties, as well as reputational damage for healthcare organizations.

%Overall, data quality is a crucial aspect of healthcare, with far-reaching implications for patient safety, healthcare costs, and regulatory compliance. Ensuring the quality of healthcare data requires a comprehensive approach that includes data governance, data management, data quality assurance, and ongoing monitoring and improvement efforts. By prioritizing data quality, healthcare organizations can provide better patient care, improve outcomes, and reduce costs.





\subsection{Background and Related Work}
There is already a significant number of papers trying to define data quality assessment frameworks for \ac{ehr} data, all of them plausible and recommendable, already described in other papers \cite{bianAssessingPracticeData2020}. The literature has over 20 different methods, descriptions, and summaries of  different frameworks over the years. Some may be highlighted from the review from Weiskopf et al., \cite{weiskopfMethodsDimensionsElectronic2013}, where five data quality concepts were identified over 230 papers: Completeness, Correctness, Concordance, Plausibility and Currency. 



The work of Saez et al. defined a unified set of \ac{dq} dimensions: completeness, consistency, duplicity, correctness, timeliness, spatial stability, contextualization, predictive value, and reliability \cite{saezOrganizingDataQuality2012}. Then a review of Bian et al. \cite{bianAssessingPracticeData2020} expanded on the previous ones, categorizing data quality into 14 dimensions and mapping them to the previous most known definitions. These were: currency, correctness, plausibility, completeness, concordance, comparability, conformance, flexibility, relevance, usability, security, information loss, consistency, and interpretability.

Finally, the work of Khan et al. tried to harmonize data quality assessment frameworks, which simplified all previous concepts into three main categories: Conformance, Completeness and Plausibility and two assessment contexts: Verification and Validation \cite{kahnHarmonizedDataQuality2016a}.
Despite all of these comprehensive works, there is still no consensus regarding which one is best or which has taken the lead in usage. Moreover, looking at all of the descriptions related in the literature, a significant portion of concepts are overlapping, and sometimes hard to conceptualize such dimensions in practice.

As for implementations, there are already some available, such as the work from \cite{phanAutomatedDataCleaning2020} where a tool created by primary care in the Flanders was built to assess completeness and percentage of values within the normal range.
The work from Liaw et al. \cite{liawQualityAssessmentRealworld2021} already reviewed some data quality assessment tools, like tools from OHDSI \cite{hripcsakObservationalHealthData2015} or TAQIH \cite{alvarezsanchezTAQIHToolTabular2019}. 
Additionally, we found some others with similar purposes and characteristics like the work presented data dataquieR \cite{schmidtFacilitatingHarmonizedData2021}, an R language-based package that can assess several data quality dimensions in observational health research data. 
Also, the work from Razzaghi et al. developed a methodology for assessing data quality in clinical data \cite{razzaghiDevelopingSystematicApproach2022}, taking into account the semantics of data and their meanings within their context. Furthermore, the work from Rajan et al. \cite{rajanContentAgnosticComputable2019} presented a tool that can assess data quality and characterize health data repositories. Parallel to this, Kaspner et al. created a tool called DQAStats that enables the profiling and quality assessment of the MIRACUM database, being possible to integrate into other databases as well \cite{kapsnerLinkingConsortiumWideData2021a}.

Regarding data quality assessment as a whole, the works of \cite{estiriSemisupervisedEncodingOutlier2019}, focused on outlier detection in large-scale data repositories. The works of \cite{saezEHRtemporalVariabilityDelineatingTemporal2020} focused on the exploration and identification of dataset shifts, contributing to the broad examination and repurposing of large, longitudinal data sets. The works of García-de-Léon-Chocano \cite{saStandardizedDataQuality2017,garcia-de-leon-chocanoConstructionQualityassuredInfant2016,garci;a-de-leon-chocanoConstructionQualityassuredInfant2015} are the only ones focused on obstetrics data, but aimed to improve the process of generating high quality data repositories for research and best practices monitoring. These are similar and complementary works to this one. Finally, the work of \cite{springateREHRPackageManipulating2017} focused on the manipulation of \ac{ehr} data, including data quality assessment, data cleaning, and data extraction. However, these tools are not meant to be used at the production level, assessing data as it is being registered or outputs reports for human consumption and not a quantitative metric for metric comparison. Furthermore, none of these tools had standard-based interoperability in mind. Finally, we have not seen, until the moment of this paper, any implementation that used machine learning to evaluate the correctness of the value.

\subsection{Materials}
The data was gathered from 9 different Portuguese hospitals regarding obstetric information: data from the mother, several data points about the fetus and delivery mode. The data is from 2019 to 2020. The software for collecting data was the same in every institution, and the columns were the same, even though the version of each software differed across hospitals. Across the different hospitals, data rows ranged from 2364 to 18177. The sum of all rows is 73351 rows.  The data dictionary is in appendix \ref{appendix:data_dict}. This study received Institutional Review Board approval from all hospitals included in this study with the following references: Centro Hospitalar São João; 08/2021, Centro Hospitalar Baixo Vouga; 12-03-2021, Unidade Local de Saúde de Matosinhos; 39/CES/JAS, Hospital da Senhora da Oliveira; 85/2020, Centro Hospitalar Tâmega Sousa; 43/2020, Centro Hospitalar Vila Nova de Gaia/Espinho; 192/2020, Centro Hospitalar entre Douro e Vouga; CA-371/2020-0t\_MP/CC, Unidade Local de saúde do Alto Minho; 11/2021. All methods were carried out in accordance with relevant guidelines and regulations.
Data was anonymized before usage. 
For this purpose, we took the Khan harmonized framework since we understood it as simpler to communicate, we feel that the three main categories are indeed non-reducible, which makes sense from an organizational standpoint. Furthermore, the work done by Khan et al. with mapping to already existing frameworks could help compare this work with others who felt the need to use other frameworks. With this in mind, we will use three main categories, Completeness, Plausibility and Conformance. Completeness relates to missing data. Plausibility relates to how believable the values are. Conformance relates to the compliance of the data representation, like formatting, computational conformance and other data standards implemented. 

%TC:ignore
\begin{figure}[htbp]
\centering
\caption{Dimensions of data quality}\label{fig:categories} 
\includegraphics[scale=0.29]{figures/data-quality-v1.png}
\end{figure}
%TC:endignore
\subsection{Methods}
% !TeX root = ../../thesis.tex

For completeness, we used the inverse of the percentage of nulls in the training set. For plausibility, several methods were applied. The first was a Bayesian network. 

In our approach, Bayesian networks, which are probabilistic graphical models, play a pivotal role in predicting the plausibility of different elements. These networks are structured as directed acyclic graphs, where each node represents a variable and edges denote conditional dependencies among these variables \unskip~\cite{pearl1988probabilistic}. This structure allows the network to efficiently manage and represent the probabilistic relationships between multiple variables. The core strength of Bayesian networks in our context lies in their ability to predict the plausibility of various elements by analyzing these interdependencies. By integrating the conditional probabilities of variables and their dependencies, the network can infer the likelihood of certain outcomes or states, thereby assessing the plausibility of different columns in our dataset, when compared with the registered value. 

With this, we hope to capture the heterogeneous essence of the data, as well as possible outliers that are also plausible. We chose this model for its dual advantages: its capability to classify the plausibility of all columns within a single unified framework, and its interpretability, which allows for a clearer understanding of how each variable influences the overall plausibility prediction. The networks were created with the pgmpy package \unskip~\cite{pgmpy}. 

Secondly, we added the outlier-tree method\unskip~\cite{cortesExplainableOutlierDetection2020} which tries to integrate a decision tree that ''predicts'' the values of each column based on the values of each other column. In the process, every time separation is evaluated, it takes observations from each branch as a homogeneous cluster to search for outliers in the predicted 1-d distribution of the column. Outliers are determined according to confidence intervals in this 1-d distribution and need to have large gaps in order to be marked as outliers in the next observation. Because it looks for outliers in the branch of the decision tree, it knows the conditions that make it a rare observation relative to other observation types corresponding to the same conditions, and these conditions are always related to target variables (as predicted by them).  As such, it can only detect outliers described by decision tree logic, and unlike other methods such as isolation forests, it can not assign outlier points to each observation, or detect outliers that are generally rare, but will always provide human-readable justification when it recognizes outliers. Therefore, these methods not only identify anomalies based on a single column/variable but also consider the context of the data, providing a more nuanced understanding of what constitutes an outlier. This contextual awareness ensures that the outliers are not merely statistical deviations but are also substantively significant within the specific framework of the target variables. 

We added also elliptic envelope and Local Outlier Factor as complementary models to these two. Elliptic envelope is a method that assumes a Gaussian distribution of data, fitting an ellipse to the central data points to identify outliers. It works best with normally distributed data but is less effective in higher dimensions or non-normal distributions. Local Outlier Factor measures the local density deviation of a data point relative to its neighbors, identifying outliers without assuming a specific data distribution. It is versatile for different data structures but sensitive to parameter settings, like the number of neighbors. 

An Interquartile Range (IQR) based metric was also added as a supportive metric. This metric used the difference between Q1 and the triple of IQR  to define a lower threshold and Q3 + 3IQR to define an upper threshold. We only categorized as outlier the values that fell outside these margins. Finally, a rule system was implemented to leverage domain knowledge in the overall scoring. The system is based on great expectations package \unskip~\cite{GXProactiveCollaborative}. A set of 17 rules was defined by the team, focusing on impossible numbers or relationship between variables  or value format. The rules covered plausability and conformance. 

The Conformance-based were related to technical issues like the format of dates (date of birth like d/m/y), and conformance to the value set (i.e. Robson group, bishop scores, or delivery types). Plausibility rules were linked to expected values for BMI, weight, and gestational age (gestational age between 20 and 44). We also added plausibility for the relationship between columns, namely weight across different weeks of gestation (weight week 35 {\textgreater} weight week 25). We have also added a relationship of greatness between ultrasound weights more than 5 weeks apart. 

As for preprocessing, all null representations were standardized, we also removed features with high missing rates ({\textgreater} 80\% ). The imputation process was performed with the median for continuous and a new category (NULLIMP) for categorical variables.

For the usage of the Bayesian network in particular, the continuous variables were discretized into three bins defined by quantile. We defined three as the number of bins in order to reduce the number of states in each node of the network. The evaluation was done with cross-validation with 10 splits and two repetitions for each column as the target.

The API for serving the prediction models was developed with FastAPI. So, the methods applied in terms of the DQA framework shown in figure \ref{fig:categories} are described in the table \ref{tab:methods}.

\begin{table}[htpb]
\caption{Implemented Methods in the tool. The first column is the category or data quality dimension. The second is a subcategory of the first column if applicable and the third column is the actual method used to assess such a dimension.} \label{tab:methods}
\renewcommand{\arraystretch}{1.4}
\setlength{\tabcolsep}{10pt}

\begin{tabularx}{\textwidth}{ p{2cm} p{3.5cm} X }
\hline
 Category   & Subcategory           & Method   \\ \hline
Completeness     & N/A               & Score by the inverse percentage of missing in the train data         \\ 
Plausibility & Atemporal Plausibility & Bayesian model prediction based on the other values of row \\ 
Plausibility & Atemporal Plausibility         & Z-score for column value based on IQR train data       \\    
Plausibility & Atemporal Plausibility           & Elliptic Envelope                       \\ 
Plausibility & Atemporal Plausibility           & Local Outlier Factor                \\ 
Conformance & Value Conformance           & Manual Rule engine                           \\ 
Plausibility & Atemporal Plausibility           & Manual Rule engine                      \\ 
Plausibility & Atemporal Plausibility           & outlier-tree                      \\ 
Conformance & Value Conformance & Manual Rule engine\\
\hline
\end{tabularx}

\end{table}


For trying to compile all of these models into a single value, that could grasp the quality of the row or patient, a scoring method was created. The method of calculating the final score is stated in figure \ref{fig:scoring_method}. 
%TC:ignore
\begin{figure}[htbp]
    \centering
    \caption{Workflow and weights used for creating the final score and which elements are used to do so.}\label{fig:scoring_method} 
    \includegraphics[scale=0.29]{figures/score-method.png}
    \end{figure}
    %TC:endignoregrama

To assess the tool's usefulness, we implemented it in a production environment and collect metrics regarding the data being produced. Then we presented some rows (or patient's records) to selected obstetrics clinicians for them to assess how likely the information is to be suitable for usage and rank it according to the perceived quality of the record. This was done through questionnaire, presenting the data and asking the clinicians to rank them from 1-10 and to describe the most important feature for the decision. We then compared the results with the ones from the model to make sanity checks regarding the model's performance and adequacy. We used Kendal Tau and Average Spearman's Rank Correlation Coefficient. Kendall Tau is a non-parametric statistic used to measure the strength and direction of the association between two ordinal variables. It calculates the difference between the number of concordant and discordant pairs of observations, normalized to ensure a value between -1 (perfect disagreement) and 1 (perfect agreement). Spearman's rank correlation coefficient is a non-parametric measure that assesses the strength and direction of a monotonic relationship between two ranked variables. It is based on the ranked values of the variables rather than their raw data, producing a value between -1 (perfect inverse relationship) and 1 (perfect direct relationship). Finally, we tested the capability of the model to discriminate bad quality records from good quality records, testing various thresholds of rank of quality, taken into account physicians responses.
We wrote all the code in Python 3.10.6 with the usage of the scikit-learn library for preprocessing, and evaluation\unskip~\cite{scikit-learn}.

\subsection{Results}
A Bayesian network with structure and parameters learned from the training dataset reached an average of \ac{auroc} of 0.89. The results are in the table \ref{tab:result_auc}.

\begin{table}[htbp] 
 \caption{Validation Results (Column AUROC)} 
 \label{tab:result_auc} 

\renewcommand{\arraystretch}{1.2}
\setlength{\tabcolsep}{18pt}

\begin{tabularx}{\textwidth} { X X X X  }
\hline
IA & 0.674 & AA & 0.787 \\
PI & 0.873 & TP & 0.868 \\
IMC & 0.872 & A30 & 0.863 \\
NRCPN & 0.703 & GR & 0.936 \\
IGA & 0.955 & ANP & 1.000 \\
SGP & 0.962 & VCS & 0.730 \\
EPC30 & 0.946 & TG & 0.836 \\
APARA & 0.989 & TPEE & 0.842 \\
AGESTA & 0.931 & V & 0.987 \\
EA & 0.993 & VNH & 0.849 \\
VA & 0.962 & TPNP & 0.928 \\
FA & 0.962 & VP & 0.793 \\
CA & 0.998 &  \\
\hline
 \multicolumn{4}{c}{\textbf{Average}  \textbf{0.890}} \\

\hline
\end{tabularx}
\end{table}


The network is as represented in figure \ref{fig:network}.
%TC:ignore
\begin{figure}[htbp]
\centering
\caption{Network learned}\label{fig:network} 
\includegraphics[scale=0.68]{figures/network.png}
\end{figure}
%TC:endignore

As for the rules created, they were conformance based, like the format of dates, and conformance to the value set (i.e. Robson group, bishop scores, or delivery types). We also added plausibility rules, like expected values for BMI, weight and gestational age. We also added plausibility for the relationship between columns, namely weight across different weeks of gestation. We added a relationship of greatness between weights more than 5 weeks apart. 





\subsubsection{Deployment \& Validation} 
The purpose of this model is to be served as an API for usage within a healthcare institution and act as a supplementary decision support tool for obstetrics teams. Although a concrete, vendor-specific information model and health information system were initially used, our goal is to develop a more universal clinical decision support system. This system should be usable across all systems involved in birth and obstetrics departments. Therefore, we constructed it using the \ac{hl7} \ac{fhir}  R5 version standard. This approach simplifies the process of \ac{api} interaction.
Rather than utilizing a proprietary model for the data, we based our decision on the use of \ac{fhir} resources: Bundle and Observation. These resources handle the request and response through a customized operation named "\$quality\_check". Our intention is to publish the profiles of these objects to streamline API access via standardized mechanisms and data models. The current version of the profiles can be accessed at this URL: \url{https://joofio.github.io/obs-cdss-fhir/}. 


For validation, we deployed the tool in docker format in a hospital to gather new data. We gathered 3231 new cases and returned a score for quality as exemplified in figure \ref{fig:scores}. Being that the score is from 0 to 1, the average score was 0.12 and \acp{iqr} was 0.15. We also used the clinician from one of the hospitals that we get data from and asked this clinician to assess 10 records in terms of quality. We gathered the 10 records at random and asked the clinician to assess them in terms of quality. Our purpose was then to compare the rankings of each evaluator; the model and the clinician, in order to assess how similar they were.



%TC:ignore
\begin{figure}[htbp]
\centering
\caption{Scores}\label{fig:scores} 
\includegraphics[scale=0.78]{figures/Scoring.png}
\end{figure}
%TC:endignore

\subsection{Discussion}
% !TeX root = ../../thesis.tex
This work adds several pieces of information to the state of the art of data quality analysis. First we tried to map the output of an automatic assessment tool to the human perception of quality and the issues linked to doing so. Secondly, the fact that we applied explainable machine learning methods such as bayesian networks to leverage the potency of advanced data analysis without compromising interpretability and explainability. Furthermore, a single model was able to reach high performance metrics for almost all variables. Thirdly, the fact that interoperability standard such as FHIR can be adopted to facilitate the usage and information exchange of such tools. However, there are also shortcoming and challenges to address. The first is that data quality is still an elusive concept since it has a contextual dimension and the quality of the record depends on the usage of the information. For example, data aimed at primary usage and day-to-day healthcare decisions about a patient will have different requirements regarding the importance of some variable or completeness of information very different from data needed to create summary statistics for key performance indicators extraction. Moreover, the data is still very vendor-specific. Even though we used an interoperability standard, the semantic layer, more connected with terminology is still lacking. This is an issue to be addressed in order to improve the interoperability of the standard. Moreover, we do not know how the training done with this data is generalizable to other vendors. One opportunity arises of mapping all of this data to a widely used terminology like SNOMED CT or LOINC. Nevertheless, the usage of FHIR and the fact that the data is mapped to a standard terminology, makes it easier to use the data in other systems and to compare the results with other studies. Furthermore, being available freely and online makes it easier to understand how to map vendor-specific datasets to the model and use it in other contexts. Regarding the model, the usage of explainable methodologies like outlier-tree and transparent models like Bayesian networks are vital for clinical application. Since we use a single model to classify possible errors in the records, the ability to try to show clinicians why that value was tagged is of uttermost importance in order to get feedback and action from humans. From the experience gathered with the study, we believe that a weaker but transparent model could have more impact than better performant but opaque ones. If explainability and interpretability are important for any ML problem, this need only increases when we are dealing with such subjective concepts as data quality.

Regarding the clinical evaluation, we found that asking clinicians to purely assess the quality of a record in an EHR is not an easy task. We discovered that for a proper assessment, a context and objective must be defined in order to make the evaluation more objective and manageable. Moreover, the ranking methodology, though very useful for comparison with the model, presents challenges for clinicians who find it difficult to order 10 records when some appear to be of equal quality. This is a very important aspect to consider when designing an evaluation method for data quality. Perhaps a categorical evaluation of yes/no would be more effective than ordering several records. These reasons might explain the great variability between clinicians (figure \ref{fig:clinical-dq}) and between clinicians and the model (Spearman and Kendall tau). Despite that, our preliminary results are promising, demonstrating an AUROC curve for categorizing bad quality records as high as 88\% and low as 56\%. The highest value was achieved by classifying all record with a mean rank of 4 or above as bad quality and the others as good quality records. However, these results rely on very few samples, so more data and research are needed in this area since it is a very subjective decision, and it should take into account the context and the objective of the evaluation. For example, if the objective is research use, the weights given to each dimension can be a set. On the other hand, if the objective is to use the data for day-to-day clinical decisions, another set of weights could be used. 

For the next steps, a promising research direction would be identifying contexts for applying data quality checks like primary usage, research purposes, and aggregated analysis for decision-making among others. This could enhance targeting those contexts and understanding the importance of each variable for those use cases. Incorporating this approach into the tool to weigh the different variables according to the context would be beneficial.  Finally, gaining access to more data and clinician evaluation of records, although challenging, is important to thoroughly assess the performance of the tool.


\subsection{Conclusion}
This work is still an early draft of a production-ready tool. However, we feel the work done is already a valuable insight into how to use data quality frameworks and several statistical tools in order to assess ehr data quality. This is a fundamental process not only to guarantee the quality of data for primary usage on a day-to-day but also for securing quality for secondary analysis and usage.
We believe the fact that we created an interoperable tool that was trained on real obstetrics data from 9 different hospitals and has the ability to provide a single score for a clinical record can help institutions, academics, and ehr vendors implement data quality assessment tools in their own systems and institutions.

For the next steps, we would like to further evaluate the score and its relationship with clinical usefulness. This would also include a further assessment of a  threshold for the score for defining a record that would require human attention.




% !TeX root = ../thesis.tex

\begin{savequote}[75mm]
    The most exciting phrase to hear in science, the one that heralds new discoveries, is not 'Eureka!' but 'That's funny...'
    \qauthor{Isaac Asimov}
    \end{savequote}

\chapter{Assess Health Data Science Methods With Limited Data Access}\label{chap:goal2}
\initial{T}his chapter delves into the assessment of health data science methods under the constraints of limited data access, as outlined in sections \ref{subsec:distributed} and \ref{subsec:benchmark}. It emphasizes the importance of developing and evaluating data science techniques that can operate effectively even when direct access to comprehensive datasets is restricted. In section \ref{subsec:distributed}, the focus is on distributed data approaches, which allow for the analysis of health data across multiple locations without the need to centralize the information. This method is crucial for maintaining privacy and security, especially in sensitive health data contexts. Meanwhile, section \ref{subsec:benchmark} discusses benchmarking strategies for these methodologies, providing a framework to evaluate their effectiveness and reliability. These benchmarks are essential to ensure that the methods yield accurate and useful insights, despite the limitations in data accessibility.


\section{Leveraging Distributed Systems in Healthcare: is it Advisable?}\label{subsec:distributed}
This section is based on the paper entitled "Evaluating distributed-learning algorithms on real-world healthcare data". This paper was focused on the fact that access to healthcare data is often laboursome and time-consuming. So we evaluated the distributed paradigm to its gold-standard, the centralised paradigm. We used 9 real-world datasets of obstetrics \acp{ehr} and compared the performance of several \ac{ml} algorithms in both paradigms. We concluded that the distributed paradigm is a valid alternative to the centralised paradigm, with the added benefit of not requiring heavy data sharing.

\subsection{Introduction}
% !TeX root = ../../thesis.tex

%As the use of \ac{ai} is increasing in the healthcare space \cite{deep_learning_increase_health}, increased demand for ethical usage of personal patient data is occurring as well \cite{ehtical_use_ml}. This has been happening both on the governmental side, with several regulations passed to protect citizens' data and personal information (such as \ac{gdpr} in the \ac{eu} \cite{gdpr_article} and \ac{hipaa} in the \ac{us} \cite{hippa}), and on the public side, with an increased concern with continuous data breaches across institutions \cite{abdulrahmanSurveyFederatedLearning2021}. So,  we are now faced with a dilemma on a compromise between what is possible to do with the available data and what should be done regarding patient privacy \cite{swarm_learning}. This is the main reason why health institutions implement burdensome processes and methodologies for sharing patient data, often costing a great deal of time, money, and human resources, seldomly overtaking the ideal time frame for analysing such data.
%Due to these privacy concerns, the traditional method for using data in healthcare is, nowadays, by focusing on data from a single institution in order to predict or infer something regarding those patients; this could be understood as local learning. This approach has some drawbacks, namely data quantity, data quality and possible class imbalance \cite{rajkomarMachineLearningMedicine2019}, never quite raising into its full potential for promoting best healthcare practices
%\cite{federated_healthcare_informatics,usage_ai_healthcare,wangAIHealthState2019} with data sharing between institutions.
%In order to overcome this issue, there are a few, more complex, systems that aggregate data from several institutions, so more robust algorithms could be trained. However, this globally centralised aggregation of data encompasses a very important data breach hazard. 

%This is the setting where distributed learning could create a greater impact. A halfway point between local and centralised learning is where we train several models, one in each institution (or silo), and where the sole information that leaves the premises is a trained model or its metadata. A distributed model is built as the aggregation of all the local models, consequently aiming to create a model similar to one globally trained with all the data in a centralised server. However, the distributed model never contacted with any data, only the local models did. This provides the opportunity to create better models, improve data protection, reduce training time and cost and provide better scaling capabilities  \cite{jatainContemplativePerspectiveFederated2021}.

%including federated-learning approaches, where a central system orchestrates the operation \cite{federated_learning_intro} or a swarm/peer-to-peer framework where silos communicate with each other. However,
%There are already some implementations of distributed systems in the healthcare space, but we lack a robust understanding of how these models behave with real data, when compared with the classical models built with all the aggregated data. Additionally, the main issues regarding the development and implementation of such systems in healthcare are still elusive.
%So we aim to understand how distributed mechanisms behave compared to using all data in the healthcare space and if they are a suitable replacement for traditional machine-learning pipelines. The contributions of this paper are:
%\begin{myitemize}
%    \item Understand how to address the lack of data quality of real-world data regarding distributed model creation;
%    \item Evaluate a distributed model against its local counterparts;
%    \item Measure the prediction performance difference between a distributed model and a centralised one;
%    \item Identify the capabilities of a distributed model to track population changes on the local datasets;
%    \item Open a research path for using distributed models to predict several target variables in obstetrics clinical research.
%\end{myitemize}



As the use of \ac{ai} is increasing in the healthcare space \cite{deep_learning_increase_health}, increased demand for ethical usage of personal patient data is occurring as well \cite{ehtical_use_ml}. This has been happening both on the governmental side, with several regulations passed to protect citizens' data and personal information (such as \ac{gdpr} in the \ac{eu} \cite{gdpr_article} and \ac{hipaa} in the \ac{us} \cite{hippa}), and on the public side, with an increased concern with continuous data breaches across institutions \cite{abdulrahmanSurveyFederatedLearning2021}.  So,  we are now faced with a dilemma on a compromise between what is possible to do with the available data and what should be done regarding patient privacy \cite{swarm_learning}. This is the main reason why health institutions implement burdensome processes and methodologies for sharing patient data, often costing a great deal of time, money, and human resources, seldomly overtaking the ideal time frame for analysing such data.
Due to these privacy concerns, the traditional method for using data in healthcare is, nowadays, by focusing on data from a single institution in order to predict or infer something regarding those patients; this could be understood as local learning. This approach has some drawbacks, namely data quantity, data quality and possible class imbalance \cite{rajkomarMachineLearningMedicine2019}, never quite raising into its full potential for promoting best healthcare practices
\cite{federated_healthcare_informatics,usage_ai_healthcare,wangAIHealthState2019} with data sharing between institutions.
In order to overcome this issue, there are a few, more complex, systems that aggregate data from several institutions, so more robust algorithms could be trained. However, this globally centralised aggregation of data encompasses a very important data breach hazard. 

This is the setting where distributed learning could create a greater impact. A halfway point between local and centralised learning is where we train several models, one in each institution (or silo), and where the sole information that leaves the premises is a trained model or its metadata. A distributed model is built as the aggregation of all the local models, consequently aiming to create a model similar to one globally trained with all the data in a centralised server. However, the distributed model never contacted with any data, only the local models did. This provides the opportunity to create better models, improve data protection, reduce training time and cost and provide better scaling capabilities  \cite{jatainContemplativePerspectiveFederated2021}.

%including federated-learning approaches, where a central system orchestrates the operation \cite{federated_learning_intro} or a swarm/peer-to-peer framework where silos communicate with each other. However,
%There are already some implementations of distributed systems in the healthcare space, but we lack a robust understanding of how these models behave with real data, when compared with the classical models built with all the aggregated data. So we aim to understand how distributed mechanisms behave compared to using all data in the healthcare space and if they are a suitable replacement for traditional machine-learning pipelines. The contributions of this paper are:

While numerous multi-institutional initiatives have successfully established integrated data repositories for healthcare research, there remains an incomplete understanding of the performance and scalability of distributed systems when directly compared to traditional, centralised models. Specifically, the nuanced behaviors of these distributed frameworks under real-world data conditions—contrasted against classical models that utilize aggregated data—have yet to be fully delineated. This paper aims to critically evaluate the efficacy and suitability of distributed mechanisms within the healthcare domain, assessing their potential as viable alternatives to conventional machine-learning pipelines. The contributions of this paper include:

\begin{myitemize}
    \item Evaluate a distributed model against its local counterparts;
    \item Measure the prediction performance difference between a distributed model and a centralised one;
%    \item Identify the capabilities of a distributed model to track population changes on the local datasets;
    %\item Understand how to address the lack of data quality of real-world data regarding distributed model creation;

  %  \item Open a research path for using distributed models to predict several target variables in obstetrics clinical research.
\end{myitemize}

\subsection{Theoretical background and Related Work}
Distributed learning \cite{distributed} can be understood as training several models in a different setting and then aggregating them as a whole. There are two main branches of these approaches, distinguishable by the existence of a central orchestrator server: federated learning where such an entity exists, and peer-to-peer (or swarm) \cite{swarm_learning} learning where it does not. 
Even though distributed learning has been receiving a lot of attention recently, only some of its concepts have been focused on, mainly distributed-deep learning with a federated learning approach \cite{xuFederatedLearningHealthcare2021,leeFederatedLearningClinical2020}. These methods use the strength of neural networks and several algorithms like federated averaging to create distributed models capable of handling complex data like text, sound, or image \cite{prayitnoSystematicReviewFederated2021}. However, considering that there are great amounts of information, especially in healthcare, stored as tabular data \cite{alvarezsanchezTAQIHToolTabular2019,dimartinoExplainableAIClinical2022,payrovnaziriExplainableArtificialIntelligence2020} and that neural networks are often not the best tool for such data structures \cite{borisovDeepNeuralNetworks2022a}, there is a lack of knowledge in the traditional machine learning techniques in a distributed manner.
%Federated Learning was introduced in 2016 \cite{konecny_federated_2016,mcmahanFederatedLearningDeep2016} and it was called federated since "the learning task is solved by a loose federation of participating devices (which we refer to as clients) which are coordinated by a central server" \cite{konecny_federated_2016}. Federated learning has two main architectures: a) horizontal and b) vertical  \cite{yangFederatedMachineLearning2019b}. 
%Horizontal refers to having the same features in all silos, but different populations in each silo. Vertical refers to having different features across silos for the same population. Then we have other approaches that expand the concept of federated learning with previous machine learning and deep learning methodologies such as transfer learning, reinforcement learning \cite{liuSystematicLiteratureReview2020} and quantum machine-learning \cite{quantum-fed-ml}. 
%Federated learning can also be classified by the information flow. Model data can be shared only with the main central server, as in more traditional methods, but also being incremental, sharing data model sequentially between silos, with central server orchestration \cite{cyclic_distribution}. 

Nevertheless, there have been some health-related  distributed machine-learning projects successfully implemented, such as euroCAT  \cite{eurocat} which implemented an infrastructure across five clinics in three countries. \ac{svm} models were used to learn from the data distributed across the five clinics. Each clinic has a connector to the outside where only the model's parameters are passed to the central server which acts as a master deployer regarding the model training with the radiation oncology data.
Also, ukCAT \cite{ukcat} did similar work, with an added centralised database in the middle, but the training being done with a decentralized system.
%Other methods and approaches have been also evaluated such as the work of Brisimi et. al. \cite{brisimi_federated_2018}  for predicting hospitalisations or deep learning methods oriented to analysing medical imaging \cite{chang_distributed_2018}, evaluating histological samples \cite{pathology-fl} and finally, some preprints showing the impact of federated learning regarding COVID-19 prediction \cite{vaid_federated_2020}. For a sound review please redirect to Zerka et al. \cite{zerkaSystematicReviewPrivacyPreserving2020a}

Finally, a few works have explored the evaluation of models in a distributed manner, for example, comparing  centralised machine learning, distributed machine learning and federated learning on MNIST dataset \cite{performance_evaluation_1}. Also, works that evaluate federated learning on MNIST, MIMIC-III and PhysioNet ECG datasets, but not in comparison with other methods  \cite{performance_evaluation_2}. The work by Tuladhar and colleagues \cite{distributed} uses healthcare images and/or public and curated datasets.
As far as we know, this is the first time a distributed machine learning evaluation is done with real-world clinical data from several different data sources.
\subsection{Materials}
Clinical data was gathered from nine different Portuguese hospitals regarding obstetric information, pertaining to admissions from 2019 to 2020. This originated nine different files representing different sets of patients but with the same features associated to them. The software for collecting data was the same in every institution (although different versions existed across hospitals) - ObsCare. The data columns are the same in every hospital's database. Each hospital was considered a silo and summary statistics of the different silos are reported in the tables \ref{tab:distributed_materials_1} and \ref{tab:distributed_materials_2}. The data dictionary is in appendix \ref{appendix:data_dict}.
%TC:ignore

{\small
\begin{table}[!ht]

\caption[Silos overview.]{\label{tab:distributed_materials_1}Silos overview. categorical columns have a snippet of the most used category and a percentage. Continuous variables have a mean and standard deviation. Abbreviation meaning in the appendix \ref{appendix:data_dict}. The last row is the number of patients. * columns were used as target.}

\centering
% !TeX root = ../../thesis.tex
\newcolumntype{L}{>{\scriptsize}l}  % "small" can be changed to "scriptsize" or "footnotesize" for even smaller text

\begin{tabular}{LLLLLLL}
   \toprule
      Variable &               Silo 1 &               Silo 2 &               Silo 3 &               Silo 4 &               Silo 5 &                Agrr. \\
   \midrule
   \hspace*{2mm} N (total) &              8039 &                 8566 &                 4989 &                 2364 &                18177 &                80874 \\
   
   \textbf{Actual Type of Delivery C (\%)}& 10 (52.6) & 3 (51.6) & 3 (57.8) & 3 (61.8) & 9 (61.5) & 11 (52.9) \\
   
   Bishop Score C (\%)&  15 (98.5) & 15 (78.8) & 13 (97.4) & 16 (86.4) & 15 (97.4) & 16 (95.3) \\
   
   \textbf{Blood Group C (\%)}& 9 (39.9) & 10 (39.9) & 9 (39.3) & 11 (37.9) & 10 (40.9) & 14 (40.5) \\
   
   \textbf{Body Mass Index $\mu (\sigma)$ } & 25.2 (8.6) & 25.2 (6.2) & 25.0 (5.3) & 25.0 (8.9) & 24.9 (7.8) & 25.1 (7.0) \\
   
   Cervical Consistency C (\%) & 4 (98.6) & 4 (83.4) & 4 (99.3) & 4 (87.4) & 4 (97.5) & 4 (96.5) \\
   
   Cervical Position C (\%)&  4 (98.6) & 4 (83.3) & 4 (99.3) & 4 (87.5) & 4 (97.6) & 4 (96.6) \\
   
   \textbf{Delivery Type C (\%)}& 6 (43.4) & 6 (53.5) & 5 (44.4) & 7 (52.2) & 7 (49.3) & 8 (51.3) \\
   
   Dilatation C (\%)&5 (98.5) & 5 (83.1) & 5 (99.3) & 5 (87.2) & 5 (97.5) & 5 (96.5) \\
   
   Effacement C (\%)& 5 (98.6) & 5 (83.2) & 5 (99.3) & 5 (87.2) & 5 (97.5) & 5 (96.5) \\
   
   Fetal Station  C (\%)&  5 (98.6) & 5 (83.3) & 5 (99.3) & 5 (87.9) & 5 (97.5) & 5 (96.6) \\
   
   \textbf{Followed physician C (\%)} & 3 (99.2) & 4 (92.2) & 3 (99.1) & 3 (94.3) & 3 (99.0) & 4 (97.9) \\
   
   \textbf{\begin{minipage}{3.8cm}\setstretch{0.65}Followed physician hospital delivery C (\%)\vspace{1mm}\end{minipage}}&  2 (87.6) & 2 (75.8) & 2 (81.4) & 2 (52.2) & 2 (71.0) & 2 (69.0) \\
   
   \textbf{\begin{minipage}{3.8cm}\setstretch{0.65}Followed physician primary care C (\%)\vspace{1mm}\end{minipage}}&  2 (61.3) & 2 (52.8) & 2 (78.1) & 2 (50.4) & 2 (70.4) & 2 (67.6) \\
   
   \begin{minipage}{3.8cm}\setstretch{0.65}Followed physician private clinic C (\%)\vspace{1mm}\end{minipage}&  2 (81.8) & 2 (85.0) & 2 (80.6) & 2 (78.8) & 2 (73.3) & 2 (75.8) \\
   
   Gestational Diabetes C (\%)&2 (87.7) & 2 (90.0) & 2 (90.2) & 2 (90.8) & 2 (89.8) & 2 (89.5) \\
   
   
   Induced Delivery  C (\%)& 2 (97.8) & 2 (83.9) & 2 (93.3) & 2 (91.9) & 2 (98.5) & 2 (92.5) \\
   
   \textbf{Mother Age $\mu (\sigma)$ } & 31.1 (5.7) & 30.7 (5.6) & 31.1 (5.9) & 31.1 (6.3) & 31.3 (5.6) & 31.1 (5.6) \\
   
   
   \begin{minipage}{3.9cm}\setstretch{0.65}Nr Deliveries forceps C (\%)\end{minipage} & 4 (99.2) & 3 (83.3) & 4 (94.3) & 4 (95.8) & 3 (60.1) & 5 (82.6) \\
   
   
   \begin{minipage}{3.9cm}\setstretch{0.65}Nr Deliveries no assistance C (\%)\vspace{1mm}\end{minipage} & 10 (74.7) & 9 (60.3) & 9 (74.9) & 9 (67.3) & 11 (45.4) & 12 (60.3) \\
   
   \begin{minipage}{3.9cm}\setstretch{0.65}Nr Deliveries vacuum C (\%)\vspace{1mm}\end{minipage} &  5 (90.4) & 4 (79.9) & 4 (89.0) & 4 (93.1) & 5 (55.3) & 5 (77.4) \\
   
   Nr of C-sections C (\%) & 6 (87.9) & 6 (72.6) & 5 (86.1) & 5 (89.5) & 6 (62.1) & 6 (74.6) \\
   
   \textbf{Nr of Pregnancies C (\%)} &  11 (40.9) & 11 (43.1) & 13 (39.1) & 12 (38.7) & 16 (42.8) & 19 (42.1) \\
   
   \textbf{Nr of born babies C (\%)} & 10 (44.8) & 10 (41.4) & 10 (36.9) & 10 (42.0) & 12 (35.3) & 12 (38.8) \\
   
   \textbf{Nr of consultations $\mu (\sigma)$ } &  7.3 (4.7) & 7.0 (6.4) & 6.4 (3.9) & 5.5 (3.6) & 10.5 (5.1) & 8.4 (5.1) \\
   
   Pelvis Adequacy C (\%) &4 (95.4) & 4 (77.7) & 4 (90.1) & 3 (96.9) & 4 (81.2) & 4 (82.6) \\
   
   \textbf{Position Admission C (\%)}  &  5 (88.5) & 6 (78.0) & 6 (51.8) & 3 (95.9) & 6 (71.3) & 7 (73.1) \\
   
   
   \textbf{Position on Delivery C (\%)} &5 (91.5) & 5 (94.4) & 5 (94.7) & 5 (95.5) & 5 (94.3) & 5 (93.9) \\
   
   \textbf{Pregnancy Type C (\%)} &  7 (62.1) & 7 (90.5) & 7 (85.4) & 7 (63.0) & 7 (89.2) & 7 (85.4) \\
   
   \textbf{Robson Group C (\%)} & 11 (22.4) & 11 (20.1) & 10 (23.8) & 10 (80.5) & 11 (27.7) & 11 (24.4) \\
   
   \begin{minipage}{3.7cm}\setstretch{0.65}Rupture amniotic pocket before delivery C (\%)\end{minipage} &  2 (91.1) & 2 (93.6) & 2 (89.3) & 2 (91.6) & 2 (84.6) & 2 (88.5) \\
   
   Smoker C (\%) & 2 (84.4) & 2 (85.2) & 2 (87.2) & 2 (89.7) & 2 (87.9) & 2 (88.1) \\
   
   \textbf{Spontaneous Delivery C (\%)} &  2 (70.3) & 2 (74.7) & 2 (64.8) & 2 (64.3) & 2 (59.7) & 2 (64.9) \\
   
   \textbf{Weeks on Admission C (\%)} &    38.1 (3.5) &    38.8 (2.2) &    38.9 (1.6) &    38.8 (2.4) &    38.6 (2.1) &    38.7 (2.2) \\
   
   
   \textbf{Weeks on Delivery $\mu (\sigma)$ } &    38.5 (2.8) &    38.9 (2.0) &    39.1 (1.7) &    39.0 (2.3) &    38.9 (2.0) &    38.9 (2.0) \\
   
   Weight on Admission $\mu (\sigma)$  &   81.4 (14.9) &   79.5 (14.5) &   78.0 (15.2) &   79.6 (16.3) &   78.3 (14.2) &   78.8 (14.5) \\
   
   
   \textbf{\begin{minipage}{3.3cm}\setstretch{0.65}Weight start  of pregnancy $\mu (\sigma)$ \vspace{1mm}\end{minipage}} &   66.4 (14.4) &   66.1 (13.5) &   65.5 (14.1) &   65.5 (14.1) &   65.5 (14.4) &   66.0 (14.1) \\
   \bottomrule
   \end{tabular}
   
   
\end{table}
}
\newpage

{\small
\begin{table}[!ht]

\caption[Silos overview part 2.]{\label{tab:distributed_materials_2}Silos overview part 2. categorical columns have a snippet of the most used category and a percentage. Continuous variables have a mean and standard deviation. Abbreviation meaning in the appendix \ref{appendix:data_dict}. The last row is the number of patients. * columns were used as target.}
\centering
\begin{tabular}{l|lllll}
\toprule
   Column &               Silo 6 &               Silo 7 &               Silo 8 &               Silo 9 &                Agrr. \\
\midrule
       IA &    31.3 \textbf{5.2} &    31.4 \textbf{5.4} &    31.5 \textbf{5.6} &    30.1 \textbf{5.6} &    31.1 \textbf{5.6} \\
       GS & a,rh.. \textbf{42\%} & a,rh.. \textbf{39\%} & a,rh.. \textbf{40\%} & a,rh.. \textbf{42\%} & a,rh.. \textbf{40\%} \\
       PI &   65.6 \textbf{13.5} &   66.0 \textbf{13.7} &   65.6 \textbf{14.1} &   67.4 \textbf{14.6} &   66.0 \textbf{14.1} \\
      PAI &   77.7 \textbf{13.4} &   79.2 \textbf{14.7} &   76.7 \textbf{13.0} &   83.1 \textbf{15.2} &   78.8 \textbf{14.5} \\
      IMC &    24.9 \textbf{5.1} &    24.9 \textbf{7.0} &    24.8 \textbf{8.0} &    25.7 \textbf{5.6} &    25.1 \textbf{7.0} \\
      CIG &   Null \textbf{91\%} &   Null \textbf{91\%} &   Null \textbf{86\%} &   Null \textbf{90\%} &   Null \textbf{88\%} \\
    APARA &    1.0 \textbf{38\%} &   Null \textbf{43\%} &   Null \textbf{41\%} &   Null \textbf{43\%} &   Null \textbf{39\%} \\
   AGESTA &    1.0 \textbf{44\%} &      1 \textbf{43\%} &    1.0 \textbf{42\%} &    1.0 \textbf{40\%} &    1.0 \textbf{42\%} \\
       EA &   Null \textbf{59\%} &   Null \textbf{61\%} &   Null \textbf{69\%} &   Null \textbf{61\%} &   Null \textbf{60\%} \\
       VA &   Null \textbf{79\%} &   Null \textbf{82\%} &   Null \textbf{88\%} &   Null \textbf{82\%} &   Null \textbf{77\%} \\
       FA &   Null \textbf{82\%} &   Null \textbf{86\%} &   Null \textbf{94\%} &   Null \textbf{89\%} &   Null \textbf{83\%} \\
       CA &   Null \textbf{69\%} &   Null \textbf{75\%} &   Null \textbf{85\%} &   Null \textbf{78\%} &   Null \textbf{75\%} \\
       TG & espo.. \textbf{88\%} & espo.. \textbf{85\%} & espo.. \textbf{86\%} & espo.. \textbf{93\%} & espo.. \textbf{85\%} \\
        V &      s \textbf{97\%} &      s \textbf{99\%} &      s \textbf{98\%} &      s \textbf{99\%} &      s \textbf{98\%} \\
    NRCPN &     6.8 \textbf{4.0} &     7.7 \textbf{3.2} &     9.3 \textbf{4.5} &     8.9 \textbf{5.5} &     8.4 \textbf{5.1} \\
       VP &   Null \textbf{68\%} &   Null \textbf{74\%} &   Null \textbf{71\%} &   Null \textbf{78\%} &   Null \textbf{76\%} \\
      VCS &   Null \textbf{53\%} &      s \textbf{87\%} &      s \textbf{63\%} &      s \textbf{87\%} &      s \textbf{68\%} \\
      VNH &   Null \textbf{62\%} &      s \textbf{63\%} &      s \textbf{69\%} &      s \textbf{83\%} &      s \textbf{69\%} \\
        B &   Null \textbf{90\%} &   Null \textbf{53\%} &   Null \textbf{93\%} &   Null \textbf{82\%} &   Null \textbf{83\%} \\
       AA &   Null \textbf{84\%} & apr... \textbf{61\%} &   Null \textbf{89\%} &   Null \textbf{74\%} &   Null \textbf{73\%} \\
       BS &   Null \textbf{99\%} &   Null \textbf{98\%} &   Null \textbf{99\%} &   Null \textbf{95\%} &   Null \textbf{95\%} \\
       BC &  Null \textbf{100\%} &  Null \textbf{100\%} &  Null \textbf{100\%} &   Null \textbf{97\%} &   Null \textbf{97\%} \\
      BDE &  Null \textbf{100\%} &  Null \textbf{100\%} &  Null \textbf{100\%} &   Null \textbf{97\%} &   Null \textbf{97\%} \\
      BDI &  Null \textbf{100\%} &  Null \textbf{100\%} &   Null \textbf{99\%} &   Null \textbf{97\%} &   Null \textbf{96\%} \\
       BE &  Null \textbf{100\%} &  Null \textbf{100\%} &  Null \textbf{100\%} &   Null \textbf{97\%} &   Null \textbf{96\%} \\
       BP &  Null \textbf{100\%} &  Null \textbf{100\%} &  Null \textbf{100\%} &   Null \textbf{97\%} &   Null \textbf{97\%} \\
      IGA &    38.7 \textbf{1.8} &    39.0 \textbf{2.0} &    38.6 \textbf{2.1} &    38.8 \textbf{1.9} &    38.7 \textbf{2.2} \\
     TPEE &   Null \textbf{65\%} &   Null \textbf{64\%} &   Null \textbf{65\%} &   Null \textbf{63\%} &   Null \textbf{65\%} \\
     TPEI &   Null \textbf{92\%} &   Null \textbf{86\%} &   Null \textbf{87\%} &   Null \textbf{94\%} &   Null \textbf{93\%} \\
      RPM &   Null \textbf{85\%} &   Null \textbf{84\%} &   Null \textbf{90\%} &   Null \textbf{94\%} &   Null \textbf{88\%} \\
       DG &   Null \textbf{92\%} &   Null \textbf{88\%} &   Null \textbf{90\%} &   Null \textbf{87\%} &   Null \textbf{89\%} \\
       TP & part.. \textbf{54\%} & part.. \textbf{52\%} & part.. \textbf{48\%} & part.. \textbf{59\%} & part.. \textbf{51\%} \\
      ANP & cefá.. \textbf{93\%} & cefá.. \textbf{94\%} & cefá.. \textbf{95\%} & cefá.. \textbf{94\%} & cefá.. \textbf{94\%} \\
     TPNP & espo.. \textbf{64\%} &  Null \textbf{100\%} & espo.. \textbf{50\%} & espo.. \textbf{65\%} & espo.. \textbf{53\%} \\
      SGP &    38.8 \textbf{1.8} &    39.2 \textbf{1.7} &    38.7 \textbf{2.0} &    39.0 \textbf{1.6} &    38.9 \textbf{2.0} \\
       GR &      1 \textbf{27\%} &      1 \textbf{25\%} &      1 \textbf{21\%} &      3 \textbf{27\%} &      1 \textbf{24\%} \\
       \midrule
N (total) &                12002 &                 8258 &                 6693 &                11786 &                80874 \\
\bottomrule
\end{tabular}


\end{table}
}
%TC:endignore

\subsection{Methods}
% !TeX root = ../../thesis.tex

The section will cover the steps we took for evaluating the models. We first addressed the preprocessing of the data, then the training of the models and finally the evaluation of the models. The evaluation was done by comparing the performance of the distributed model with the local and centralised models. The performance was measured by the \ac{auroc}, \ac{auprc}, \ac{rmse} and \ac{mae}. The results were then compared using a 2-sample T-test.
\subsubsection{Preprocessing}

The initial dataset underwent preprocessing by eliminating attributes that were missing more than 90\% of their data across all storage units (or silo). We standardized the representation of missing values, which varied widely, including representations such as "-1" "missing" or simply blank spaces. For imputation, we utilized the mean for continuous variables (calculated within site) and introduced a special category (NULLIMP) for categorical variables. We converted all categories into numerical values based on a predefined mapping that covered all potential categories across the datasets. Although this approach introduces an ordinal relationship and potential bias is created among features, we disregarded this concern because the methodology was uniformly applied across all datasets intended for training local, distributed and centralised. These preprocessing tasks were executed once for each dataset and silo.

However, in the context of training classification models, it is crucial that all classes of the target variable are known at the time of training and are represented in each split of the cross-validation process. To address this, we employed \ac{smote} \cite{smote} to up-sampled low-frequency target classes. We established a threshold of n$<$25 for low-frequency variables to ensure that each cross-validation split contained at least two instances of the class—although a minimum of 10 instances (10 splits) might suffice, we opted for 25 to mitigate potential distribution issues and have at least two examples of the class in each split. Additionally, we created dummy rows for missing target classes by imputing the mean for continuous variables and the mode for categorical variables (calculated within site). The necessity for up-sampling and missing variable creation was evaluated and applied as needed for each training session and for each target, considering that each session's split could result in a training set lacking instances of low-frequency classes.

All procedures were coded in python 3.9.7 with the usage of the scikit-learn library \cite{scikit-learn} and mlxtend library \cite{mlxtend}.


\subsubsection{Model Training}
To avoid pitfalls of inductive bias from a certain learning strategy, we learned six different models (i) Decision Trees, (ii) Bayesian methods, (iii) a logistic regression model with Stochastic Gradient Descent, (iv) \ac{knn}, (v) AdaBoost and (vi) Multi-layer Perceptron. The decision was to create diversity in the models used, in order to assess if the training methodology could have an impact on distributed model creation.
The distributed model was an ensemble of models from each silo on a weighted soft-voting basis. The weights were defined by weighted averages of the scores each model obtained in the training set. Then the final result is obtained by creating a weighted average of the class predictions for classification and a weighted average for regression. A model like this can be implemented with peer-to-peer or federated approaches.
Nineteen features were used as target outcomes. These features were selected by filtering by the percentage of null values (below 50\%). This choice was related to maintaining a equilibrium between having a wide range of variables to test how the target variables affects the outcome and having target variables that did go through an harsh imputation mechanism. For categorical outcomes, thirteen were selected (AA - Position Admission; ANP - Position on Delivery; AGESTA - Nr of Pregnancies; APARA - Nr of born babies; GS - Blood Group; GR - Robson Group; TG -Pregnancy Type; TP - Delivery Type; TPEE - Spontaneous Delivery; TPNP - Actual Type of Delivery; V - Followed physician; VCS - Followed physician primary care; VNH - Followed physician hospital delivery;). For continuous variables, six were selected (IA - Mother Age; IGA - Weeks on Admission; IMC - BMI; NRCPN - Nr of consultations; PI - Weight start of pregnancy; SGP - Weeks on Delivery;). Details can be seen in tables \ref{tab:1} and \ref{tab:1.2}.
Local models were built with each silo's data. The centralised model was trained with a training dataset from all the silos combined. 


%TC:ignore
\subsubsection{Model Performance Evaluation}

All models were built for a certain outcome variable with a repeated cross-validation (2 times and 10 splits each) and then compared, over 10 stochastic runs, with evaluation being performed on a test set held out from each silo. By performing cross-validation twice, we aimed to generate a more robust estimation of the model’s performance metrics by averaging the results over two separate runs, each partitioning the data differently. This approach is particularly useful in scenarios where data is limited or highly variable, as it provides a clearer insight into the model's expected performance in unseen data scenarios. The metrics used for classification models were Weighted \ac{auroc} computed as One-versus-Rest, Weighted \ac{auprc}. The metrics for regression models were \ac{rmse} and \ac{mae}. The algorithm is shown in the algorithm \ref{alg:1}. This rendered over 1000 different combinations. When a variable was used as outcome to predict, all others were used as predictors.




\begin{algorithm}[hbtp]
\caption{Creation and evaluation of the 3 different models. We first preprocessed data. Then for each target, we created a distributed and centralised model. Then, over 10 repetitions per silo, we created a new train and test set and local model and tested the centralised, distributed and local on this test set.}
\label{alg:1}
Pre-process all silos (null standardization, imputation, encoding)\;
\For {target in target list}{ 
  \For{n in 10 repetitions}{

\For {silo in imputed silos}{
Train-Test Split (80:20)\;
 check for low frequency or nonexistent labels in train set \;
 train local model with hyper-parameter tuning with 2x10 repeated \ac{cv} \;
 define weights based on scores in the train set (weighted average for predicting the value) for the distributed model\;

  }

  Create distributed (ensemble of all models) model with weights\;
  predict local on the test set\;
  predict distributed on the test set\;
\vspace{3mm}


Create a centralised model with all the data with a 2x10 repeated \ac{cv} \;
Test the centralised model on the test set\;
}}



 \end{algorithm}
%TC:endignore

After all the data was collected, we used the standard independent 2-sample T-test to check if the differences were significant with a $\alpha$ of 0.05. First, we compared the overall performance of the distributed model vs their centralised and local counterpart.  We also compared every distributed model per algorithm and sequentially the centralised and correspondent local model across all algorithms and repetitions and outcome variables with 2-sample T-test as well.



\subsection{Results}
% !TeX root = ../../thesis.tex

Table \ref{tab:allvsall} shows the aggregated metrics for \ac{auroc}, \ac{auprc}, \ac{rmse} and \ac{mae} for distributed, centralised and local models predicting capabilities on each silo. The data refers to the mean of the metric values for all columns tested as targets for all methods and all silos. We also calculated the 95\% confidence interval for each model (local and distributed per silo) in order to assess how well the distributed model would work as opposed to the local one per silo. We also calculated the \textit{P} value for the means of the distributed vs centralised and distributed vs local.
%TC:ignore
%\setlength{\tabcolsep}{7pt} % Default value: 6pt
%\renewcommand{\arraystretch}{1.3} % Default value: 1

\begin{table}[h!] 
 \setlength{\tabcolsep}{7pt} % Default value: 6pt 
 \renewcommand{\arraystretch}{1.3} % Default value: 1
  \captionsetup{justification=centering} 
\centering
\caption[Metrics for centralised model, distributed model and local model]{Comparison of the distributed model with the centralised model and with the local model (Mean for all model and all columns). 2-sample T-test for the means was used as hypothesis test. Bold for \textit{P} value below 0.05. AUPRC and AUROC for categorical target variable and RMSE and MAE for continuous target variable.}

\label{tab:allvsall}
\begin{tabular}{llcccc}
\toprule
 &  & M & SD & 95\% CI & \textit{P}  \\
\midrule
\multirow{3}{*}{AUPRC}
 & distributed & 0.691 & 0.216 & (0.686, 0.696) & - \\
  & centralised & 0.706 & 0.225 & (0.701, 0.711) & \bfseries 1.10e-17 \\
 & local & 0.659 & 0.220 & (0.654, 0.665) & \bfseries 4.71e-05 \\
 \hline

\multirow{3}{*}{AUROC} 
 & distributed & 0.723 & 0.182 & (0.718, 0.727) & - \\
 & centralised & 0.729 & 0.180 & (0.725, 0.734) & \bfseries 2.98e-26 \\
 & local & 0.692 & 0.164 & (0.688, 0.695) & \bfseries 2.48e-02 \\

\hline

\multirow{3}{*}{MAE} 
 & distributed & 2.370 & 1.608 & (2.315, 2.425) & - \\
 & centralised & 2.365 & 1.923 & (2.298, 2.431) & \bfseries 2.23e-04 \\
 & local & 2.527 & 1.799 & (2.465, 2.589) & 9.01e-01 \\

\hline

\multirow{3}{*}{RMSE} 
 & distributed & 21.171 & 46.078 & (19.584, 22.757) & - \\
 & centralised & 19.839 & 28.645 & (18.853, 20.826) & \bfseries 2.92e-02 \\
 & local & 23.771 & 49.776 & (22.057, 25.485) & 1.63e-01 \\
\hline
\end{tabular}
\end{table}





%TC:endignore

Figure \ref{fig:heatmap-cat} shows the \ac{auroc} of each algorithm and silo on the Y axis and target variable and type of model on the X. The color bar refers to the value of the \ac{auroc}. Blue being lower values and red bigger values. The same type of graph was created for regression, where the Figure \ref{fig:heatmpa-int} shows the \ac{mae} for each silo and algorithm and target variable and type of model. 


%TC:ignore

\begin{figure}[h!]
\centering
\captionsetup{justification=centering}

\caption[Heatmap of classification algorithm and silo vs Target variable and model type.]{Heatmap of classification algorithm and silo vs Target variable and model type. Value is the \ac{auroc} mean of all 10 experiments. Y axis is the algorithm and silo. X axis is Target variable and Method. AA - Position Admission; ANP - Position on Delivery; AGESTA - Nr of Pregnancies; APARA - Nr of born babies; GS - Blood Group; GR - Robson Group; TG -Pregnancy Type; TP - Delivery Type; TPEE - Spontaneous Delivery; TPNP - Actual Type of Delivery; V - Followed physician; VCS - Followed physician primary care; VNH - Followed physician hospital delivery;}\label{fig:heatmap-cat} 
\includegraphics[scale=0.22]{figures/heatmap-class.png}
\end{figure}
%TC:endignore
%TC:ignore

\begin{figure}[htbp]
\centering
\captionsetup{justification=centering}

\caption[Heatmap of regression algorithm and silo vs Target variable and model type.]{Heatmap of regression algorithm and silo vs Target variable and model type. Value is the \ac{mae} mean of all 10 experiments. The y axis is the algorithm and silo. X axis is Target variable and Method. IA - Mother Age; IGA - Weeks on Admission; IMC - BMI; NRCPN - Nr of consultations; PI - Weight start of pregnancy; SGP - Weeks on Delivery;}\label{fig:heatmpa-int} 
\includegraphics[scale=0.22]{figures/heatmap-reg.png}
\end{figure}
%TC:endignore

\definecolor{Gray}{gray}{0.85} 
 \begin{table}[h] 
 \setlength{\tabcolsep}{6pt} % Default value: 6pt 
 \renewcommand{\arraystretch}{1.1} % Default value: 1
  \captionsetup{justification=centering} 
\centering
\caption[Model comparison: Distributed versus centralised and local for every test]{Model comparison: Distributed versus centralised and local for every test. Each cell is the total of distributed model when compared with centralised model (row) and local model (column) across different silos and outcome variable. ($>$ for better, = for non significance and $<$ for worse). The first example is 72 which means that 72 iterations of the distributed SGD was better than the centralised and local. SGD: Stochastic Gradient Descent, NN: Neural Network, KNN: K-Nearest neighbours, ADA: AdaBoost, NB: Naive Bayes, DT: Decision Tree. Comparison was done with 2-sample T-test with a $\alpha$ of 0.05. (\% in parentheses)}
\label{tab:hyp}
\newcolumntype{L}{>{\scriptsize}l}  % "small" can be changed to "scriptsize" or "footnotesize" for even smaller text
\newcolumntype{R}{>{\scriptsize}r}  % "small" can be changed to "scriptsize" or "footnotesize" for even smaller text

\begin{tabular}{LLRRRR}
\toprule


 &  & Distributed $>$ Local & Distributed = Local & Distributed $<$ Local & \textbf{Row Total} \\


\hline \multirow{3}{*}{SGD} &Distributed $>$ Centralised  & 72 (7.0) & 14 (1.4) & 9 (0.8) & \textbf{95 (9.3)} \\
 & Distributed =  Centralised& 14 (1.4) & 17 (1.7) & 6 (0.6) & \textbf{37 (3.6) } \\
 & Distributed $<$ Centralised  & 11 (1.1) & 11 (1.1) & 17 (1.7) & \textbf{39 (3.8)} \\
% \hline
% \multicolumn{2}{c|}{SGD Total} & 97 (9.4)&42 (4.1) &32 (18.7)& \textbf{171}  \\
\hline \multirow{3}{*}{NN} & Distributed $>$ Centralised & 44 (4.3) & 44 (4.3) & 7 (0.7) & \textbf{95 (9.3)} \\
 & Distributed =  Centralised& 2 (0.2) & 33 (3.2) & 2 (0.2) & \textbf{37 (3.6)} \\
 & Distributed $<$ Centralised & 0 (0) & 17 (1.7) & 22 (2.1) & \textbf{39 (3.8)} \\
% \hline

% \multicolumn{2}{c|}{NN Total} & 46&94 &31 & \textbf{171}  \\

\hline \multirow{3}{*}{KNN} & Distributed $>$ Centralised & 16 (1.6) & 0 (0) & 1 (0.1) & \textbf{17 (1.7)} \\
 & Distributed =  Centralised & 10 (1) & 2 (0.2) & 1 (0.1) & \textbf{13 (1.3)} \\
 & Distributed $<$ Centralised & 72 (7)  & 28 (2.7) & 41 (4) & \textbf{141 (13.7)} \\
% \hline

% \multicolumn{2}{c|}{KNN Total} & 97&30 &43 & \textbf{171}  \\

\hline \multirow{3}{*}{ADA} & Distributed $>$ Centralised & 64 (6.2) & 25 (2.4) & 22 (2.1) & \textbf{111 (10.8)} \\
 & Distributed =  Centralised& 5 (0.5) & 12 (1.2) & 10 (1) & \textbf{27(2.6)} \\
 & Distributed $<$ Centralised & 10 (1) & 6 (0.6) & 17 (1.7) & \textbf{33 (3.2)} \\
% \hline

% \multicolumn{2}{c|}{ADA Total} & 79&43 &49 & \textbf{171}  \\

\hline \multirow{3}{*}{NB} & Distributed $>$ Centralised & 51 (5) & 19 (1.9) & 34 (3.3) & \textbf{104 (10.1) } \\
 &  Distributed =  Centralised & 5 (0.5) & 19 (1.9) & 12 (1.2) & \textbf{36 (3.5)} \\
 & Distributed $<$ Centralised  & 3 (0.3) & 4 (0.4) & 24 (2.3) & \textbf{31 (3)} \\
% \hline

% \multicolumn{2}{c|}{NB Total} & 59&42 &70 & \textbf{171}  \\

\hline \multirow{3}{*}{ DT} & Distributed $>$ Centralised & 27 (2.6) & 0 (0) & 1 (0.1) & \textbf{28 (2.7)} \\
 & Distributed = Centralised & 8 (0.8) & 0 (0) & 0 (0)& \textbf{8 (0.8)} \\
 & Distributed $<$ Centralised & 97 (9.5) & 12 (1.2) & 26 (2.5) & \textbf{135 (13.2)} \\
% \hline

% \multicolumn{2}{c|}{DT Total} & 132&12 &27 & \textbf{171}  \\
 
 \hline
  \textbf{Total} &  & \textbf{511 (49.8)} & \textbf{263 (25.6)} & \textbf{252 (24.6)} & \textbf{1026 (100)}\\
 \bottomrule
\end{tabular}
\end{table}

















\subsection{Discussion}



The imputation process was done using the mean value (for continuous variables) or a new category (NULLIMP) for categorical variables. All categories were encoded as numbers using a previous mapping created based on all possible categories in all silos. Even though an ordinal relationship is created among features, we believe that since we are applying this methodology to all datasets, which will be the source for all tests (local, distributed and centralised), that fact may be ignored.
When training classification models, all of the target variable classes must be known at that moment and should be present in each split of the cross-validation. So, when assessing the training dataset, low-frequency target classes (n $<$ 25) were up-sampled with \ac{smote} \cite{smote} and missing target classes were addressed with dummy rows creation by the imputation of the mean for continuous variables and mode for categorical variables (per silo). These preprocessing mechanisms were applied in each run and for each target.
The distributed model was an ensemble of models from each silo on a weighted soft-voting basis, defining weights and thresholds based on the training set scores. 
The first thing that is noticeable is the high scores achieved in our analysis which show that all algorithms in all forms (local, distributed and centralised) have a good grasp on ranking data (negative on the bottom and positives on the top of a scale) for classification or predicting the value for regression. We notice that distributed models have performance similar to their centralised counterparts. $\sim$59\% of all of the distributed models had similar or better performance than the centralised models. This suggests that a distributed model can be used to reliably infer information and does not compromise prediction performance when compared with the gold standard (centralised) while increasing privacy for the data owners.\\
Overall, our results suggest that it is possible to implement a distributed model without significantly losing information. However, there are still issues to be addressed. This methodology presents hurdles regarding categorical class handling. Firstly, all classes should be known first-hand and should be given to each model even if that silo in particular has no cases of that class. Secondly, low-frequency classes are also an issue to be addressed, since training the model with cross-validation will raise problems because each split should have all classes present. Our approach relied on sample creation for low and non-existent target classes. However, this approach is adding information to the model that is not originally there. The way we chose for minimising this issue was by creating dummy variables with median and mode imputations based only on the information in the dataset. Nevertheless, non-existent classes are impossible to address without prior information. These class problems could be partially tackled in production by implementing data management and governance procedures, namely data dictionaries. Still on data preprocessing, we applied ordinal encoding to the variables which will create a natural hierarchy between variables. One solution for this is to create binary columns for each class in each column. This will remove the hierarchy between classes but increase variable numbers and training time considerably.\\
Moreover, like in most secondary usage of data, other issues are important to keep in mind, even in such a controlled environment as this one. Even though the software is the same in every hospital, the clinical service is the same and the underlying data models are the same, the version of the software is not the same across all hospitals. This difference alone can alter the way each column is populated, mainly through front-end changes or label modification, among other aspects. Additionally, each hospital has its own workflows in practice that can also alter the way data is collected; changing timings or steps in a certain workflow can dramatically change the data acquisition and the reality it represents. \\
Another issue to consider is the path adopted to build the distributed model. In this case, it was decided to develop an ensemble of models with voting. However, other methods could have been employed, like parameter averaging, that should be tested as well. In particular, the usage of more robust neural networks could be assessed as well. We chose not to test state-of-the-art neural networks since the data volume was low for that use case and several papers have already demonstrated that neural networks are not the most suitable tool for tabular data \cite{grinsztajnWhyTreebasedModels2022,borisovDeepNeuralNetworks2022}. We chose to add MLPerceptron as a baseline for comparison with the remaining algorithms. The results show us that the performance was below the other algorithms, but in this concrete case, the problem may reside in the architecture chosen and hyperparameters used in the Cross-validation. Despite this, a precise and thorough demonstration of this use case would be important to consider such scenarios. \\
Furthermore, the algorithm underlying the distributed model is of importance as well for its performance versus the centralised model. Figures \ref{fig:heatmap-cat} and \ref{fig:heatmpa-int} and table \ref{tab:hyp} show us that Decision trees and K-nearest neighbours implemented in a centralised manner are consistently better than the distributed counterpart. Even though this improvement may have a relationship to the target variable (i.e figure \ref{fig:heatmpa-int} for IA and IGA variables), it is still an important fact to take into account when implementing such architectures. The performance of the models is also interesting to catch differences in silos. See silo 6 for TPNP (figure \ref{fig:heatmap-cat}) where silo 6 consistently behaves differently than the rest.
As for implementation, such a mechanism could be implemented in at least two manners; with a central orchestrator or without. The first one would assume a central point that would make a request to each silo for a prediction and then create the final prediction with the weighted averaging of each one. The second one would not require any additional platform and each silo would communicate with each of the others and receive the prediction and would create the final with their own. This implementation step would of course take into account variables that we were out of scope such as the communication between silos. 
Regarding the prediction capability as a whole, we found that this data is suitable to apply machine-learning models in order to predict several clinical outcomes, with very good results for several target variables. 

%Another topic that can be explored is the fact that a lower performance of the ensemble in a certain silo and/or target could be actually useful to characterise the population of that silo.

\subsection{Conclusion}
% !TeX root = ../../thesis.tex

This research demonstrated the efficacy of distributed models using real-world data by comparing their performance with that of local models, which are trained with data from individual silos, and centralized models, which utilize data from all silos. The findings reveal that an ensemble of models, essentially a distributed model as investigated in this study, can capture the nuances of the data, achieving performance comparable to a model constructed with comprehensive data. Even though The performance of these models is influenced by factors such as the inherent characteristics of the target variables and the data distribution across different silos, we are now fairly confident that distributed learning is a step forward regarding data privacy without loss of predictive performance when compared with centralised and local models.
Considering the robust performance metrics observed, with AUROC/AUPRC scores exceeding 80\% and MAE maintained below 1, further investigation into distributed models is warranted. Specifically, we aim to develop distributed models for predicting clinical outcomes, such as delivery type or Robson Group classifications, which hold significant potential for real-world clinical application like reducing unnecessary Cesarean Sections or accelerating diagnosis. These findings underscore that distributed learning not only advances data privacy but also maintains high prediction accuracy, promising substantial benefits for clinical practices.



\section{Can Institutions Share Their Performance Metrics Without Hesitation of Retaliation?}\label{subsec:benchmark}
This section is based on the paper entitled "Benchmarking institutions' health outcomes with clustering methods". This paper was focused on the fact that many healthcare institutions harbor reservations about openly sharing production metrics. One predominant concern is the potential for retaliatory actions, be it from regulatory bodies, competitors, or the public. In this paper, we propose the application of a clustering methodology that allows institutions to compare performance metrics without disclosing the actual values. The method is based on clustering, which involves grouping health institutions' outcomes into a known number of clusters, allowing institutions to position themselves in a range of clusters without sharing the true means of their target data. The proposed method uses the K-means and K-modes clustering algorithms and was tested on data from real Electronic health records and public datasets. This approach provides a valid benchmark of hospital metrics and performances while protecting the privacy of participating institutions. 
\subsection{Introduction}
Health institutions play a critical role in providing essential healthcare services to communities and ensuring that they operate efficiently and effectively is crucial. Benchmarking is a process that allows hospitals to compare their performance against that of other institutions, which can help identify areas of strength and weakness \cite{suydamPatientSafetyData2007}. By analysing and evaluating performance metrics, such as patient outcomes, operational efficiency, and financial management, hospitals can identify best practices and make data-driven decisions to improve their overall performance. It can also help hospitals identify and implement innovative practices that can lead to better patient care and improved staff satisfaction \cite{hulsenSharingCaringData2020}.

However, despite the numerous benefits of benchmarking, some hospitals may be hesitant to participate due to concerns about revealing weaknesses or being perceived as inferior to their peers. The fear of being judged or penalized for poor performance can sometimes lead hospitals to avoid sharing data, making it difficult to accurately assess their performance and identify areas for improvement. Privacy issues and concerns turn this opportunity into an even less desirable path \cite{hulsenSharingCaringData2020}. To address these concerns, benchmarking initiatives often ensure the confidentiality and anonymity of data to encourage participation and foster trust among participating institutions. However, this is usually not enough. In 2019, as stated in the work of Villanueva et al., \cite{villanuevaCharacterizingBiomedicalDataSharing2019}, 26\% of data-sharing initiatives are based on the aggregation of data and 24\% are based on sharing data in closed consortia. Only 15\% were based on open or controlled access.

To address concerns around privacy and confidentiality, we propose a new method of benchmarking based on clustering. This method involves grouping health institutions' outcomes into a known number of clusters, providing health institutions with the capability of positioning themselves in a range of clusters, without ever sharing the true means of their target data.

This approach to benchmarking not only addresses concerns around privacy and confidentiality. It has the potential to encourage greater participation in benchmarking initiatives, as hospitals can be assured of the anonymity and confidentiality of their data. By creating a more secure and private environment for benchmarking, hospitals can feel more comfortable sharing their data and participating in initiatives that can ultimately improve patient care and operational efficiency.

In conclusion, benchmarking is a crucial tool for hospitals to improve their performance and provide better care for their patients. While concerns around privacy and confidentiality may exist, the clustering approach to benchmarking provides a more accurate assessment of hospital performance while protecting the privacy of participating institutions. By embracing benchmarking initiatives and leveraging new approaches to benchmarking, hospitals can continuously improve their operations and ensure they provide the highest quality of care possible.
In this paper we propose:
\begin{myitemize}
    \item study how to implement clustering mechanism for benchmark
    \item address preprocessing issues for the raw data
    \item highlight pain points to deployment in the real world.
\end{myitemize}
\subsection{Rationale and Related Work}
This work was initially suggested as a follow-up to a previous work of Rodrigues et al., \cite{rodriguesLocalAlgorithmApproximate2018a} where clustering is applied to streaming data sources. We then thought if a similar approach could be applied to healthcare in order to be able to compare data distributions without ever knowing their real values of them.
Clustering in healthcare is often used to create clusters of patients, taking into account a given set of characteristics. This is used to find possible groups of phenotype and be able to characterise populations given the centroids \cite{walkerUnsupervisedLearningTechniques2019,basileInformaticsMachineLearning2018}. It is also used as a method of detecting regularities and patterns in multi-omics data that reveal different molecular subtypes \cite{nicoraIntegratedMultiOmicsAnalyses2020,rappoportMultiomicMultiviewClustering2018}. It can also be used to create unsupervised models for facilitating the annotation of data for supervised models \cite{mcalpineUtilityUnsupervisedMachine2022}. 

K-means \cite{lloydLeastSquaresQuantization1982,steinley2007initializing,macqueen1967classification} is an unsupervised clustering algorithm used to group data points into K distinct clusters based on their similarity. It is widely used in \ac{ml}, data mining, and image segmentation. The algorithm works by randomly initializing K centroids (or cluster centres) and assigning each data point to the nearest centroid. Then, the centroids are moved to the mean of the points assigned to each cluster. This process is repeated until convergence, where the clusters no longer change.

The objective of K-means is to minimize the sum of squared distances between each data point and its assigned centroid, which is also called the within-cluster sum of squares (WCSS). The algorithm attempts to find the best K clusters that minimize the WCSS. However, choosing the right value of K can be challenging, and the algorithm may converge to a suboptimal solution. Therefore, K-means is often run multiple times with different initializations to find the best clustering solution. Despite its simplicity, K-means can be computationally expensive when dealing with large datasets, and it may not work well with non-linearly separable data or when the clusters have different shapes and sizes.

K-modes is another clustering algorithm similar to K-means, but it is designed to work with categorical data. Unlike K-means, which computes the mean of continuous variables, K-modes computes the mode (or the most frequent value) of categorical variables within each cluster. The algorithm works by randomly initializing K centroids and assigning each data point to the nearest centroid based on the number of matching categories. Then, the centroids are moved to the mode of the categories within each cluster. This process is repeated until convergence, where the clusters no longer change.

The objective of K-modes is to minimize the dissimilarity between the data points within each cluster, which is often measured by the Hamming distance, Jaccard distance, or other similarity measures. Like K-means, choosing the right value of K is critical, and the algorithm may converge to a suboptimal solution. Therefore, K-modes is often run multiple times with different initializations to find the best clustering solution. K-modes is particularly useful when dealing with data that have a large number of categorical variables or when the data contain missing values. However, like K-means, K-modes may not work well with non-linearly separable data or when the clusters have different shapes and sizes.

However, as far as we know, this is the first time clustering is tested for exchanging information privately.

\subsection{Materials \& Methods}

\subsubsection{Materials}
We used two types of data in this paper. One is simpler and available online from the \ac{uci} dataset library, namely, the heart disease dataset \cite{misc_heart_disease_45}. We made fairly simple preprocessing on that dataset, namely removing the "?" by filling with null and then imputing missing values by imputing the mean on continuous variables and mode on categorical ones. We then separated the data into 3 distinct silos at random to mimic different health institutions.

In order to use real data and address problems found in the wild, we used clinical data gathered from nine different Portuguese hospitals regarding obstetric information, pertaining to admissions from 2019 to 2020. This originated from nine different files representing different sets of patients but with the same features associated with them. The software for collecting data was the same in every institution (although different versions existed across hospitals) - ObsCare. The data columns are the same in every hospital's database. Each hospital was considered a silo for comparison.
\subsubsection{Method Overview}
We used Python 3.9 to implement the mock example of such an use-case. The clustering was done with \textit{scikit-learn} library \cite{scikit-learn}. The algorithm proposed is shown in algorithm \ref{alg:bench1}.

%TC:ignore
\begin{algorithm}[hbtp]
\caption{Benchmarking with clustering}
\label{alg:bench1}

\SetAlgoLined



\For {variable in silo}{ 
initialize centroids\;
%\begin{itemize}
%    \item real cluster obfuscated with noise
%    \item true centroids
%    \item add noise to data and then create centroids
%\end{itemize}
}
%\SetKwRepeat{Do}{do}{while}
\While{No convergence}{
\begin{itemize}
    \item Send centroids to other silos
\item Receive other silo's information and add own centroids
\item Calculate new centroids
\item calculate score
\end{itemize}
}



\end{algorithm}
%TC:endignore

The method for assessing convergence is based on clustering metrics: the \ac{ri}. This metric computes a similarity measure between two clusters by considering all pairs of samples and counting pairs that are assigned in the same or different clusters in the predicted and true clusters \cite{hubertComparingPartitions1985}. The raw RI score is: $RI = (number\; of\; agreeing\; pairs) / (number\; of\; pairs)$.
Furthermore, convergence must be obtained through several iterations to make sure it's stable, so a buffer period is also important. For the results section, we set the threshold as 0.9 and repetitions at 20.

In this paper, we propose to show how such an implementation could be done while addressing issues with data formats, types and preprocessing. So, we want to check if the encoding of categorical data affects the model and which method is better for encoding such variables. Additionally, we will try to understand if it is possible to create mechanisms for mixed data if categorical and continuous data must be used and evaluated separately and if so, through which mechanisms.
We will test (1) continuous variables alone, and (2) encoded categorical variables as ordinal. We will also test (3) K-modes  and (4) K-means with the proportion of each category for categorical data.
K-means was used as implemented in \textit{scikit-learn} \cite{scikit-learn} and K-modes, as implemented by J. de Vos \cite{devos2015}.


\subsection{Results}
As for results, the data from heart disease rendered the figure \ref{fig:cluster_free_3s}. In this, we focused on continuous variables only. For easier reading, the data is as shown in the table \ref{tab:datapoints}. We used data from the real world to test if everything would work similarly, rendering the image \ref{fig:cluster_mydata_9s}. We added a binary category to show how meaningless the value turn in order to get any information out of it.




%TC:ignore

\begin{figure}[htpb]
\centering
\captionsetup{justification=centering}
\caption[Clustering for 3 continuous variables with 3 silos]{Clustering for 3 continuous variables with 3 silos and true centroids (S2) and true means (S2) for example purposes; The values were normalized for visualization purposes with MinMax}
\includegraphics[scale=0.50]{figures/my_cluster_3.png}
\label{fig:cluster_free_3s} 
\end{figure}



%TC:endignore
\begin{table}[htbp]
\centering
 \setlength{\tabcolsep}{7pt} % Default value: 6pt 
 \renewcommand{\arraystretch}{1.35} % Default value: 1
  \captionsetup{justification=centering} 
\caption[Final Data points after convergence of clustering]{Final Data points after convergence; S1, S2 and S3 are the centroids obtained in each silo (S) after convergence; True centroids are the centroids of the true means of all silos (TC)}
\label{tab:datapoints}
\begin{tabular}{lccc}
\toprule
 & Age & trestbps & chol \\
\midrule
S1 & 46.3 , 61.1 & 121.1 , 148.9 & 218.9 , 300.8 \\
S2  & 45.8 , 61.0 & 120.7 , 149.9 & 220.9 , 304.0 \\
S3 & 45.5 , 61.0 & 120.5 , 149.6 & 216.1 , 297.4 \\
TC  & 45.6 , 60.8 & 121.0 , 149.6 & 215.8 , 297.9   \\

\bottomrule
\end{tabular}
\end{table}




%TC:ignore



\begin{figure}[H]
\centering
\captionsetup{justification=centering}
\caption[Clustering for 3 variables with 9 silos]{Clustering for 3 variables with 9 silos and true centroids of the true means (TC); 2 continuous and 1 categorical one hot encoded, The values were normalised for visualisation purposes with MinMax}\label{fig:cluster_mydata_9s} 
\includegraphics[scale=0.60]{figures/my_cluster_9.png}
\end{figure}
%TC:endignore

As before, the data is in table format in \ref{tab:datapoints_9}.

%%True Centroid & 24.9 , 25.3 & 66.6 , 65.6  & 0.24 , 0.29 \\

\begin{table}[htbp]
\centering
 \setlength{\tabcolsep}{7pt} % Default value: 6pt 
 \renewcommand{\arraystretch}{1.35} % Default value: 1
  \captionsetup{justification=centering} 
\caption{Final Data points after convergence and true centroids of the true means of each silo (TC)}
\label{tab:datapoints_9}
\begin{tabular}{lccc}
\toprule
 & \acs{bmi} & Initial Weight & Birth by Cesarian \\
\midrule
TC & 24.9 , 383.1 & 60.5 , 85.0  & 0 , 1 \\
S1 & 40.1 , 409.4 & 60.4 , 85.0 & 0.96 , -0.04 \\
S2 & 40.1 , 410.4 & 61.7 , 86.3 & 0.99 , -0.01 \\
S3 & 40.0 , 410.4 & 61.9 , 86.5 & 0.96 , -0.04 \\
S4 & 40.6 , 411.3 & 61.9 , 86.5 & 0.96 , -0.04 \\
S5 & 40.0 , 410.4 & 60.5 , 85.1 & 1.0 , 0.0 \\
S6 & 40.1 , 409.3 & 60.4 , 84.9 & 1.0 , 0.0 \\
S7 & 40.7 , 411.3 & 60.5 , 85.0 & 0.96 , -0.04 \\
S8 & 40.0 , 410.4 & 86.5 , 61.9 & 1.0 , 0.0 \\
S9 & 41.0 , 410.4 & 85.0 , 60.4 & 1.0 , 0.0 \\
\bottomrule
\end{tabular}
\end{table}




Then we experimented with categorical variables. Figure \ref{fig:cluster_3_cat} shows the convergence of the silos with proportion data and K-means with that and with K-modes.
\begin{figure}[ht]
\caption{Clustering for 3 variables with 3 silos - (A) categorical variables with  proportion with K-Means and (B)  Categorical with K-modes  }\label{fig:cluster_3_cat} 
  \subcaptionbox*{(A)}[.60\linewidth]{%
    \includegraphics[width=\linewidth]{figures/my_cluster_3_cat.png}%
  }%
  \hfill
  \subcaptionbox*{(B)}[.44\linewidth]{%
    \includegraphics[width=\linewidth]{figures/my_cluster_3_cat_kmodes.png}%
  }
\end{figure}


\subsection{Discussion}
As per the discussion, there are a few issues to be addressed. First as per data preprocessing. In order to cluster be obtained, the null data must be filled out. There are a few strategies to do so. One option is to eliminate records/rows with empty cells or impute data. Either is a possibility, with pros and cons but the capability of having a dataset where no null records are present across several features may be difficult to find in the wild, especially since there are often optional and conditional fields in most Electronic Health Records (EHR). So imputation becomes more interesting, since it enables the usage of the whole dataset, even if biases are introduced.
Mixed types of datasets are also an issue to be aware of. In this case, not only imputation but also encoding a categorical variable is a vital step to take in the preprocessing phase. There are usually two main methods of data encoding, ordinal encoding and binary encoding. The first one keeps a unique column as the original data but maps every category to an increasing natural number. This creates an ordering in the data, often a misrepresentation of reality, not only due to this hierarchy but only because it assumes the differences between ranks of the hierarchy are always the same (1). The second is related to expanding the number of columns into the number of categories and creating 0s and 1s for the category. In machine-learning terms, binary seems more suited to be applied, but for benchmarking purposes, both are below par in terms of interpretability. For categorical data, we found out that K-modes seem to fulfil the requirements in a better way, providing better interpretability and reasoning about the results. However, it should be noted that we applied K-modes in a multivariate fashion and K-means in a univariate fashion.
Given that no percentage is provided, only the mode of the data, we believe it is still hard to get any real insight from the centroids. However, K-modes provides less information, since it only shows the top two categories. Which, for example. binary targets, provide little to no information. However, for larger categorical sets, the information provided could be better. Moreover, the number of centroids pretended could be more important as well. Agreeing on only 1 centroid would render the mode of the data provided by all silos, which could be more interesting.
As for continuous data, the use of real data was insightful, since BMI had a few very big outliers around 300 and 400, which rendered centroids around that data. Even if not all silos had examples of these outliers, the ones that do have, pass that into the remaining. One possible workaround would be an addition of an extra cluster in order to catch possible outliers.
However, this should be addressed in detail and assess how outliers could subvert the data from the silos and how to work around that.

% should be addressed more in focus since it was outside of the scope of this paper
As for the next steps, a few issues could be addressed in depth. Regarding imputation, it could be interesting to understand how imputation, and which methods are more suitable to use for real-world scenarios. If the imputation of variables with a high null percentage influence significantly a centroid formation.
Communication could be important as well. Which action is to be taken when a silo is "down" and does not send information to the remaining. Cluster information should be addressed as well. They need to be agreed upon beforehand in the scope of this paper. But if it could be selected by each silo? Would that be feasible or a convergence could be achieved?
Finally, there is the question if there is the possibility of having leaks of true means across iterations by adversarial learning. At present time, we cannot be sure that the values are totally private, but then again, nothing is.


% abordar na questao a vantagem da privacidade mas que é possivel haver leak dos valores por adervisal learning.

\subsection{Conclusion}
We believe that this work helps create the foundation for exchanging data across healthcare institutions without revealing the true data points. It could be useful for benchmarking and promoting a higher adoption rate.
Even though there are still issues to be addressed, we think that the path is full of possibilities.




\begin{savequote}[75mm]
We never are definitely right, we can only be sure we are wrong.
\qauthor{Richard P. Feynman}
\end{savequote}
\chapter{Discussion} \label{chap:disc}

%frases mt curtas

Extracting knowledge from healthcare data is not easy. It relies on the availability of data, which is not always the case, and on the ability to extract knowledge from it. In this chapter, we discuss the main challenges we faced during the development of this thesis, and how we overcame them. We also discuss the limitations of our work, and how it can be improved in the future. Finally, we discuss the main contributions of this thesis, and how they can be used to improve the quality of healthcare.
The first problem is getting access to data. The data is not always available, and when it is, it is not always in the format we need. Ethics committees and \ac{dpo} requirements are put in place in order to guarantee the patient's privacy and security, but a lot of times at the cost of timely access to data. I consider that synthetic data can have a good impact on this work. While we can leave the legal processes be, we may use synthetic data with a heavy focus on security to develop and test our algorithms. This is a very promising area of research, and I believe it will be a game-changer in the future.
Parallel to this approach are distributed paradigms. Having a distributed approach to data analysis could be of great help. This would allow for the data to be analyzed in its original location in a more secure way and timely manner. If metrics and models could be built by local teams and shared across regions and/or countries to leverage the power of the many for single institutions could be groundbreaking. However, underlying both these approaches are data dictionaries and data governance tools. Having the correct functional/clinical description of data could be of great impact on the usage of data. Having already the variables defined as categorical, numerical and so on could be of great help. This is a very important aspect of data science, and it is often overlooked. Simple statistics of datasets could be useful as well. For example, the number of missing values, the number of unique values, the number of outliers, and so on. This would help the data scientist to understand the data better and to know what to expect from it. 

This issue also relates to the second big hurdle of knowledge extraction from healthcare data - quality. As discussed in section \ref{subsec:dq}, this is a very complex and sometimes elusive concept. In our case, this implied a lot of time spent with data preprocessing. We had to deal with missing values, outliers, and correctness in the context of the records, and data in different formats. We also had to link together different databases from different \acp{his} which brought to light new problems like the new dimensions of correctness of data. There is a common saying that sums this pretty well \textit{When we have one watch, we know the time, but when we have two, we may never know}. So if we had different information regarding the same variable in different systems, how to decide what is true?
Another aspect that is often overlooked is the relationship with the clinicians. We need to understand that they are the ones who will use the tools we develop, and they need to be involved in the process. We need to understand their needs and their workflow. We need to understand what they need and how they need it. We need to understand that they are not data scientists, and they do not have the time to learn how to use our tools. We need to make it easy for them to use our tools. Now healthcare is often explained in terms of clinical teams of different backgrounds. A similar concept could be beneficial for harvesting knowledge from data.
Thirdly, building software or tools based on this data is still an early subject that possibly requires a legal and technical framework. A legal is connected to the impact of such tools in healthcare. If drugs require such a long time to be approved in order to assess security, how can we approve a tool that can have a similar impact? A technical framework is connected to the fact that we are still in the early stages of a new health data science paradigm. We are still trying to understand how to use data, and how to extract knowledge from it. We are still trying to understand how to evaluate the performance of our tools.  We are still trying to understand how to evaluate the impact of our tools in healthcare in a timely manner in a way that is not biased and that is not too expensive. Imposing similar structures to drugs is ill-advised since it could possibly kill the innovation potential and the interest in providing such tools. And this is where a quality infrastructure could be of use. Seriously betting of biomedical informatics could render huge payoffs down the line. Having the human and material resources to build data infrastructures on local (healthcare institutions) and regional, or even country-wise or cross-country policies to use effective use healthcare data is essential. At the time of the writing of this thesis, examples like \ac{ehds} are very promising initiatives that could help to overcome the hurdles of data availability and quality. However cross-country initiatives will always be as good as the weakest link, so it is important to have a common framework and a common goal and to have the resources to achieve it. In concrete, having data pipelines, data governance and data interoperability tools, and data quality tools are essential. Having a common data dictionary and a common data format would also be of great help. This would allow for a more efficient use of data, and it would allow for the use of healthcare data to drive innovation.
Tightly connected with this is the possibility of having \ac{rwe} support clinical decisions live. Having data like the one produced in \ref{subsec:ipop} in real-time or with high update frequency could be leveraged in order to further support clinicians in making decisions based on data. However, we would require not only the premisses already discussed, like data quality and cross-colaboration clinics, but a trust-framework would also be necessary. In order to make the automatic dashboard and metrics reliable, transparency is key. Having explainability and transparency in the process of evidence production will be key to building trust and accountability.

%data quality é um sério problema - buscar papers ricardo e isso
%data dictionary é vital
%relacao com medicos é vital
%synthetic data é mirage - mas pode ser util por contexto - IPOP
%Metricas de avaliação sao fundamentais, nao conseguimos gerir o que nao conseguimos medir
%focar no impacto
%usar métodos alternativos - distribuido, p.ex.
%metodos de analise em tempo real de eficácia medicamentos - mt trabalho manual
%data engineering, preprocessing - é super dificil e de facto demora muito mais tempo.



The challenges of extracting knowledge from healthcare data are multi-faceted, as evident from the issues of data access, quality, and the complex relationship with clinicians. Another vital aspect is the integration of real-world evidence (RWE) into clinical decision-making processes. RWE, derived from data collected outside of controlled clinical trials, offers immense potential for informing healthcare decisions. However, its integration requires meticulous attention to data quality, governance, and transparency. As healthcare data becomes increasingly digitized and voluminous, the opportunity to leverage RWE in real-time or with high-frequency updates grows. This could significantly enhance the ability of clinicians to make data-driven decisions. However, for RWE to be effectively integrated, it necessitates not only robust data infrastructure but also a trust framework. Clinicians and patients alike must have confidence in the accuracy, reliability, and transparency of the data and the algorithms used. Building this trust involves ensuring that data processing and decision-making algorithms are transparent and explainable, fostering a sense of accountability and reliability in the system.

Furthermore, the evolution of healthcare data science underscores the need for a comprehensive legal and technical framework. The comparison to drug approval processes highlights the importance of stringent evaluation for healthcare tools, balancing safety and innovation. The legal framework should address the ethical implications and societal impact of these tools, while the technical framework should focus on performance evaluation, data extraction techniques, and impact assessment. Establishing such frameworks is crucial for navigating the complexities of health data science and for fostering an environment where innovation can thrive without compromising patient safety or data integrity. This approach also involves the creation of quality infrastructures, emphasizing biomedical informatics, and developing robust data infrastructures at various levels, from local healthcare institutions to regional and international collaborations.

Lastly, the future of healthcare data science depends heavily on cross-disciplinary collaboration and common frameworks. Initiatives like the European Health Data Space (EHDS) are steps in the right direction, promoting data availability and quality through collaborative efforts. However, the success of such initiatives relies on the strength of their weakest links, necessitating uniform standards, shared goals, and adequate resources across all participating entities. Concrete measures like establishing common data dictionaries, data formats, and interoperability tools are essential. These efforts will pave the way for more efficient data utilization, driving innovation in healthcare. Such an integrated approach, combining technical prowess with legal and ethical considerations, is vital for realizing the full potential of healthcare data in improving patient outcomes and advancing medical science. 
\begin{savequote}[75mm]
We never are definitely right, we can only be sure we are wrong.
\qauthor{Richard P. Feynman}
\end{savequote}
\chapter{Discussion} \label{chap:disc}

%frases mt curtas

Extracting knowledge from healthcare data is a complex and multifaceted challenge. This endeavor is contingent upon the availability and the interpretability of data. In this chapter, we delve into the primary challenges encountered in this thesis, elucidating how these obstacles were navigated. Additionally, we scrutinize the limitations inherent in the current methodologies for discovery, summarization, and application of knowledge derived from health data. Conclusively, this discourse highlights the seminal contributions of this thesis in enhancing healthcare quality through innovative data utilization strategies.

\section{Accessing Data}
The first problem is getting access to data. The data is not always available, and when it is, it is not always in the format we need. Ethics committees and \ac{dpo} requirements are put in place in order to guarantee the patient's privacy and security, but a lot of times at the cost of timely access to data. We consider that synthetic data can have a good impact on this work. While we can leave the legal processes be, we may use synthetic data with a heavy focus on security to develop and test our algorithms. This is a very promising area of research, and we believe it will be a game-changer in the future.
Parallel to this approach are distributed paradigms. Having a distributed approach to data analysis could be of great help. This would allow for the data to be analysed in its original location in a more secure way and timely manner. If metrics and models could be built by local teams and shared across regions and/or countries to leverage the power of the many for single institutions could be groundbreaking. However, underlying both these approaches are data dictionaries and data governance tools. Having the correct functional/clinical description of data could be of great impact on the usage of data. Having already the variables defined as categorical, numerical and so on could be of great help. This is a very important aspect of data science, and it is often overlooked. Simple statistics of datasets could be useful as well. For example, the number of missing values, the number of unique values, the number of outliers, and so on. This would help the data scientist to understand the data better and to know what to expect from it. 

\section{Data Quality}

The previous point relates to the second big hurdle of knowledge extraction from healthcare data - quality. As discussed in section \ref{subsec:dq}, this is a very complex and sometimes elusive concept. In our case, this implied a lot of time spent with data preprocessing. We had to deal with missing values, outliers, and correctness in the context of the records, and data in different formats. We also had to link together different databases from different \acp{his} which brought to light new problems like the new dimensions of correctness of data. There is a common saying that sums this pretty well \textit{When we have one watch, we know the time, but when we have two, we may never know}. So if we had different information regarding the same variable in different systems, how to decide what is true?
Another aspect that is often overlooked is the relationship with the clinicians. We need to understand that they are the ones who will use the tools we develop, and they need to be involved in the process. We need to understand their needs and their workflow. Furthermore, we need to understand what they need and how they need it. We need to understand that they are not data scientists, and they do not have the time to learn how to use our tools. We need to make it easy for them to use our tools. Now healthcare is often explained in terms of clinical teams of different backgrounds. A similar concept could be beneficial for harvesting knowledge from data.


\section{Building robust software to support AI}
Thirdly, building software or tools based on this data is still an early subject that possibly requires a legal and technical framework. A legal is connected to the impact of such tools in healthcare. If drugs require such a long time to be approved in order to assess security, how can we approve a tool that can have a similar impact? A technical framework is connected to the fact that we are still in the early stages of a new \ac{heads} paradigm. We are still trying to understand how to use data, and how to extract knowledge from it. We are still trying to understand how to evaluate the performance of our tools.  We are still trying to understand how to evaluate the impact of our tools in healthcare in a timely manner in a way that is not biased and that is not too expensive. Imposing similar structures to drugs is ill-advised since it could possibly kill the innovation potential and the interest in providing such tools. And this is where a quality infrastructure could be of use. Seriously betting of biomedical informatics could render huge payoffs down the line. Having the human and material resources to build data infrastructures on local (healthcare institutions) and regional, or even country-wise or cross-country policies to use effective use healthcare data is essential. At the time of the writing of this thesis, examples like \ac{ehds} are very promising initiatives that could help to overcome the hurdles of data availability and quality. However, cross-country initiatives will always be as good as the weakest link, so it is important to have a common framework and a common goal and to have the resources to achieve it. In concrete, having data pipelines, data governance and data interoperability tools, and data quality tools are essential. Having a common data dictionary and a common data format would also be of great help. This would allow for a more efficient use of data, and it would allow for the use of healthcare data to drive innovation.
Tightly connected with this is the possibility of having \ac{rwe} support clinical decisions live. Having data like the one produced in \ref{subsec:ipop} in real-time or with high update frequency could be leveraged in order to further support clinicians in making decisions based on data. However, we would require not only the premisses already discussed, like data quality and cross-collaboration clinics, but a trust-framework would also be necessary. In order to make the automatic dashboard and metrics reliable, transparency is key. Having explainability and transparency in the process of evidence production will be key to building trust and accountability.


\section{Evaluation of AI tools}
The challenges of extracting knowledge from healthcare data are multi-faceted, as evident from the issues of data access, quality, and the complex relationship with clinicians. Another vital aspect is the integration of real-world evidence (RWE) into clinical decision-making processes. RWE, derived from data collected outside of controlled clinical trials, offers immense potential for informing healthcare decisions. However, its integration requires meticulous attention to data quality, governance, and transparency. As healthcare data becomes increasingly digitized and voluminous, the opportunity to leverage RWE in real-time or with high-frequency updates grows. This could significantly enhance the ability of clinicians to make data-driven decisions. However, for RWE to be effectively integrated, it necessitates not only robust data infrastructure but also a trust framework. Clinicians and patients alike must have confidence in the accuracy, reliability, and transparency of the data and the algorithms used. Building this trust involves ensuring that data processing and decision-making algorithms are transparent and explainable, fostering a sense of accountability and reliability in the system.

Furthermore, the evolution of healthcare data science underscores the need for a comprehensive legal and technical framework. The comparison to drug approval processes highlights the importance of stringent evaluation for healthcare tools, balancing safety and innovation. The legal framework should address the ethical implications and societal impact of these tools, while the technical framework should focus on performance evaluation, data extraction techniques, and impact assessment. Establishing such frameworks is crucial for navigating the complexities of \ac{heads} and for fostering an environment where innovation can thrive without compromising patient safety or data integrity. This approach also involves the creation of quality infrastructures, emphasizing biomedical informatics, and developing robust data infrastructures at various levels, from local healthcare institutions to regional and international collaborations.

\section{Cross-disciplinary collaboration}
The future trajectory of healthcare data science is heavily reliant on cross-disciplinary collaboration and shared frameworks. Initiatives like the \ac{ehds} signify positive strides towards enhanced data availability and quality through collaborative efforts. However, the efficacy of such initiatives is contingent upon the uniformity of standards, shared objectives, and adequate resourcing among all participating entities. Essential measures include establishing common data dictionaries, formats, and interoperability tools. These collaborative endeavors are pivotal in streamlining data usage, thereby catalyzing innovation in healthcare. An integrative approach, melding technical expertise with legal and ethical considerations, is crucial for harnessing the full potential of healthcare data in improving patient outcomes and advancing medical science.


In conclusion, this thesis presents a comprehensive examination of the multifarious aspects involved in harnessing healthcare data for knowledge extraction. The identified challenges and proposed solutions underscore the intricate interplay between data accessibility, quality, and interdisciplinary collaboration. The synthesis of this research contributes significantly to the field, providing a nuanced understanding of the complexities involved in leveraging healthcare data. This work not only advances academic knowledge but also holds the potential to inform and transform practical applications in healthcare, ultimately aiming to enhance patient care and outcomes.

%data quality é um sério problema - buscar papers ricardo e isso
%data dictionary é vital
%relacao com medicos é vital
%synthetic data é mirage - mas pode ser util por contexto - IPOP
%Metricas de avaliação sao fundamentais, nao conseguimos gerir o que nao conseguimos medir
%focar no impacto
%usar métodos alternativos - distribuido, p.ex.
%metodos de analise em tempo real de eficácia medicamentos - mt trabalho manual
%data engineering, preprocessing - é super dificil e de facto demora muito mais tempo.

 


%% comment next 2 commands if numbered appendices are not used
\appendix
\chapter{} \label{ap1:loren}

\section{Data Dictionary}
\label{appendix:data_dict}
\begin{table}[H]
\renewcommand{\arraystretch}{0.58}
%\setlength{\tabcolsep}{14pt}
\begin{tabular}{ l   l }

\toprule
Acronym  &    Description \\
\midrule
IA &  Mother Age \\
GS  & Blood Group \\
PI &   Weight at the beginning of pregnancy \\
PAI &  Weight on Admission \\
IMC &   BMI \\
CIG &   If Smoker During Pregnancy \\
APARA  & Number of previously born babies\\
AGESTA  &   Number of Pregnancies   \\
EA &     Number of Previous Eutocic Deliveries with no assistance \\
VA &      Number of Previous Eutocic Deliveries with help of vacuum extraction \\
FA &     Number of Previous Eutocic Deliveries with help of forceps \\
CA &   Number of  Previous C-sections \\
TG &     Pregnancy Type (spontaneous, In vitro fertilisation...) \\
V &     If the pregnancy was accompanied by MD \\
NRCPN &     Number of prenatal consultations \\
VH &      If the pregnancy was followed by a MD in a hospital \\
VP &    If the pregnancy was followed by a MD in a private clinic \\
VCS &  If the pregnancy was followed by a MD in a primary care facility \\
VNH &    If the pregnancy was followed by a MD in the same hospital the delivery was made \\
B  & Pelvis Adequacy \\
AA & Baby's Position on Admission \\
BS &  Bishop Score \\
BC &   Bishop Score Cervical Consistency\\
BDE &  Bishop Score Fetal Station \\
BDI &  Bishop Score Dilatation \\
BE  &   Bishop Score Effacement \\
BP & Bishop Score Cervical Position \\
IGA  & Number of Weeks on Admission \\
TPEE  & If the delivery was spontaneous   \\
TPEI  &  If the delivery was induced  \\
RPM  & If there was a rupture of the amniotic pocket before delivery began \\
DG &  Gestational Diabetes \\
TP & Delivery Type \\
ANP  & Baby's Position on Delivery \\
TPNP  & Actual Type of Delivery\\
SGP  & Pregnancy Weeks on Delivery \\
GR  &   Robson Group \\
\bottomrule
\end{tabular}
\end{table}

%% questionarios
%% 
\chapter{} \label{ap2:loren}

\section{C-section assessment questionnaire}
\label{appendix:quest-obs}
%\includepdf[pages=-]{"appendices/questionário_final_obs.pdf"}
\includegraphics[scale=0.6]{appendices/Captura de ecrã 2023-07-27, às 18.20.57.png}
\section{Data quality questionnaire}
\label{appendix:quest-dq}
%\includepdf[pages=-]{appendices/questionário_final-dq.pdf}

\includegraphics[scale=0.7]{appendices/Captura de ecrã 2023-07-27, às 18.22.05.png}

%%----------------------------------------
%% Final materials
%%----------------------------------------

%% Bibliography
%% Comment the next command if BibTeX file not used
%% bibliography is in ``myrefs.bib''

%\PrintBib{references, full-thesis}
%\bibliographystyle{plainnat}
\printbibliography
%% Index
%% Uncomment next command if index is required
%% don't forget to run ``makeindex thesis'' command
%% \PrintIndex

\end{document}
