\chapter*{List of Publications}
% \chaptermark{REPRODUCIBILITY}



\textbf{Core Research Papers} \\
The 8 papers described below are the core structure of this thesis (4 were already published, and 4 are under review). The manuscripts are listed by order of appearance in the thesis.
\\
\begin{itemize}
    \item Coutinho-Almeida, J., Rodrigues, P., \& Cruz-Correia, R. (2021). GANs for Tabular Healthcare Data Generation: A Review on Utility and Privacy. In Discovery Science (pp. 282–291). Springer International Publishing.

    \item Coutinho-Almeida, J., Cruz-Correia, R., \& Rodrigues, P. (2022). Dataset Comparison Tool: Utility and Privacy. Stud Health Technol Inform, 294, 23–27.
    

    \item (in review) Using Machine Learning Models' feature importance to assess dataset similarity 


    \item  (in review) Development and Validation of a Data Quality Evaluation Tool in Obstetrics Real-World Data through \ac{hl7} \ac{fhir} interoperable Bayesian Networks and Expert Rules
    
    \item (accepted) Coutinho-Almeida, J., Cardoso, A., Cruz-Correia, R., \& Pereira-Rodrigues, P. (2024). Evaluating distributed-learning on real-world obstetrics data: Comparing distributed, centralized and local models Scientific Reports 
    
    \item (in review) Benchmarking institutions' health outcomes with clustering methods 

    
    \item (in review) Comparative Analysis of Palbociclib and Ribociclib: A real world data and Propensity Score-Adjusted Evaluation with endocrine therapy
    
    
    \item Coutinho-Almeida, J., Cardoso, A., Cruz-Correia, R., \& Pereira-Rodrigues, P. (2024). Fast Healthcare Interoperability Resources–Based Support System for Predicting Delivery Type: Model Development and Evaluation Study. JMIR Formative Research, 
    8, e54109. https://doi.org/10.2196/54109

\end{itemize}



\textbf{Other Publications and activities}\\
In addition, during the duration of this thesis conduction, the candidate was also the author and co-author of other papers. Although these studies were not part of the thesis core structure, they were important to improve the researcher's knowledge of the field and/or to present the results to the community. They are listed below:\\


Coutinho-Almeida, J., \& Cruz-Correia, R. (2022). Developing a Process Mining Tool Based on HL7. Procedia Computer Science, 196, 501–508.\\


Holmgren, A., Esdar, M., Hüsers, J., \& Coutinho-Almeida, J. (2023). Health Information Exchange: Understanding the Policy Landscape and Future of Data Interoperability. Yearbook of Medical Informatics, s-0043-1768719.\\



Costa, P., Almeida, J., Araujo, S., Alves, P., Cruz-Correia, R., Saranto, K., \& Mantas, J. (2023). Biomedical and Health Informatics Teaching in Portugal: Current Status. Heliyon, 9(3).\\

