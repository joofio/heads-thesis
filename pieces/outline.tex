\chapter*{Outline}

% \chaptermark{REPRODUCIBILITY}

%\doublespacing
The idea for this thesis first formed in my mind during a mental process of understanding how clinical knowledge could be improved in terms of quality, quantity, and speed of generation. The feeling was that new technology, especially the ones related to digital and informatics domain took years to be fully implemented in practice and harness the potential benefits they provided. I felt that healthcare, like other domains, had a serious gap between academia and industry. So the potential of all these discoveries was lost in "translation".
So how could we leverage this? \\
This thesis is organized as follows:\\
%improve to better suit papers
Chapter \ref{chap:intro} synthesizes the aim and specific objectives of this thesis.
Chapter \ref{chap:sota} presents a brief introduction to core concepts for the thesis, like \ac{kdd}, \ac{ebm}, and privacy and ethical concerns.\\
%Chapters \ref{chap:usecase}, 4 and 5 present the main results of the different studies developed to achieve this thesis’s three main objectives. When a chapter has more than one article with conclusions, a summary is made.
Chapter \ref{chap:usecase} corresponds to the papers published. The papers cover a wide range of the traditional \ac{kdd} steps so they are grouped around the phases they represent the most.\\

Chapter \ref{chap:disc} represents the overall discussion of all of the papers and experiments done in the thesis.\\

Chapter \ref{chap:conclusion} communicates the conclusion, limitations and future work.\\

Attachments include ethical permissions and supplementary data to some of the papers.
