\chapter*{Outline}

% \chaptermark{REPRODUCIBILITY}

%\doublespacing
The idea for this thesis first formed in our mind during a mental process of understanding how clinical knowledge could be improved in terms of quality, quantity, and speed of generation. The feeling was that new technology, especially the ones related to digital and informatics domain took years to be fully implemented in practice and harness the potential benefits they provided. I felt that healthcare, like other domains, had a serious gap between academia and industry. So, the potential of all these discoveries was lost in "translation".
So how could we leverage this? \\
This thesis is organized as follows:\\
%improve to better suit papers
Chapter \ref{chap:intro} synthesizes the aim and specific objectives of this thesis.
Chapter \ref{chap:sota} presents a brief introduction to core concepts for the thesis, such as \ac{kdd}, \ac{ebm} or privacy and ethical concerns.\\
%Chapters \ref{chap:usecase}, 4 and 5 present the main results of the different studies developed to achieve this thesis’s three main objectives. When a chapter has more than one article with conclusions, a summary is made.
Chapters \ref{chap:goal1}, \ref{chap:goal2} and \ref{chap:goal3} correspond to the papers published. The papers cover a wide range of the traditional \ac{kdd} steps, so they are grouped around the phases they represent the most.\\

Chapter \ref{chap:disc}  presents the different published studies, arranged according to their connection with the traditional \ac{kdd} steps.\\

Chapter \ref{chap:conclusion} communicates the conclusion, limitations, and future work.\\

Attachments include ethical permissions and supplementary data to some of the papers.

%This chapter will comprise the work done during this PhD. The works developed and corresponding papers were a search for improving data usage in several steps of the \ac{kdd} process. We can see some work dedicated to leveraging data acquisition or alternatives to it, like the works depicted in section \ref{subsec:distributed}, \ref{subsec:benchmark}, \ref{subsec:similarity}, \ref{subsec:tabular} and \ref{subsec:gans}. Others will focus more on how to use the data in order to make a difference in clinical practice like the section \ref{subsec:ipop}, \ref{subsec:obs} and \ref{subsec:dq}.
