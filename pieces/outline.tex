\chapter*{Outline}



This thesis is structured as follows:
%improve to better suit papers
Chapter \ref{chap:intro} synthesizes the aim and specific objectives of this thesis.
Chapter \ref{chap:sota} presents a brief introduction to core concepts for the thesis, such as \ac{kdd}, \ac{ebm} or privacy and ethical concerns.\\
%Chapters \ref{chap:usecase}, 4 and 5 present the main results of the different studies developed to achieve this thesis’s three main objectives. When a chapter has more than one article with conclusions, a summary is made.
In order to enable a better structure of the thesis, we split the different works across the data science methodology: \ac{kdd}. Focusing on the steps of data cleaning/generation, then data acquisition and analysis and finally on the usage and application of knowledge created from the data in order to have impact in real-world. With this, chapter \ref{chap:goal1} refers to work done on the generating  and preprocessing of data. Chapter \ref{chap:goal2} focus on data acquisition methods and chapter \ref{chap:goal3} focus on enabling decisions in healthcare practice based on data, through clinical research and/or \ac{cdss}. 

Chapter \ref{chap:goal1} delves into innovative methods for enhancing data quality, a critical component of the data preparation phase in \ac{kdd}. This chapter presents findings from research focused on the development and utilization of synthetic data generation and automatic data quality assessment methods. The use of \acp{gan} to create realistic, non-sensitive datasets exemplifies the synthesis of new data collection methodologies that expand data volume while protecting privacy. Furthermore, this chapter explores automatic tools designed to assess data quality, significantly reducing manual effort and enhancing the efficiency and accuracy of data analysis. Such advancements are crucial for ensuring the integrity and reliability of datasets, which form the foundation for effective data mining and subsequent knowledge discovery.

Chapter \ref{chap:goal2} assesses health data science methods under the constraints of limited data access, aligning with the data access and data mining phases of KDD. This chapter investigates the effectiveness of distributed data approaches and benchmarking strategies that enable data science techniques to operate efficiently without comprehensive data access. The focus on distributed data analysis across multiple locations supports privacy and security, particularly critical in sensitive health data scenarios. Additionally, benchmarking these methods provides a framework to evaluate their effectiveness, ensuring that the insights derived are both accurate and actionable despite limitations in data availability.

Chapter \ref{chap:goal3} explores the practical application of health data in real-time decision-making processes, particularly in clinical settings like drug evaluation and obstetrics. This chapter relates closely to the interpretation/evaluation phase of \ac{kdd}, where data insights are translated into actionable knowledge. By applying causality principles and \acl{ml} models, this research assesses the effectiveness of breast cancer drug treatments and develops \ac{cdss} for obstetrics. The innovative use of explainable \ac{ml} and \ac{iptw} showcases how advanced data analysis techniques can influence policy and decision-making in healthcare, demonstrating the real-world applicability of \ac{kdd} processes.


Chapter \ref{chap:disc} summarizes the findings of the works developed in this thesis and how they can be leveraged. \\

Chapter \ref{chap:conclusion} communicates the conclusion, limitations, and future work.\\

Attachments include supplementary data to some papers.

%This chapter will comprise the work done during this PhD. The works developed and corresponding papers were a search for improving data usage in several steps of the \ac{kdd} process. We can see some work dedicated to leveraging data acquisition or alternatives to it, like the works depicted in section \ref{subsec:distributed}, \ref{subsec:benchmark}, \ref{subsec:similarity}, \ref{subsec:tabular} and \ref{subsec:gans}. Others will focus more on how to use the data in order to make a difference in clinical practice like the section \ref{subsec:ipop}, \ref{subsec:obs} and \ref{subsec:dq}.
