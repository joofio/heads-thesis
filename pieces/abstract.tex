\chapter*{Abstract}


This thesis delves into the intricate process of extracting knowledge from healthcare data, a task fraught with challenges yet brimming with potential. Central to this investigation is the acknowledgment, inspired by Richard P. Feynman, that absolute certainty is elusive in scientific inquiry; instead, this journey is marked by continual learning and improvement. We confront various obstacles, including data accessibility, quality concerns, and the integration of real-world evidence into clinical practice. The main issues affecting this process can be explained as follows:

\begin{itemize}
    \item Van der Lei’s First Law of Medical Informatics: Data shall be used only for the purpose for which they were collected.
    \item Effectiveness of Routine Data Sources and Analytical Innovations: Examining the extent to which routine data sources and innovations in analytical methods alleviate the need for randomized clinical trials.
    \item Governance, Privacy, and Trust Issues: Addressing governance, privacy, and trust questions when routine health data are made available for research.

\end{itemize}

A portion of this work is dedicated to addressing data quality. The quality of healthcare data emerges as a complex and elusive concept, demanding extensive data preprocessing to manage missing values, outliers, and inconsistencies across different health information systems. The thesis emphasizes the criticality of clear functional and clinical data descriptions, advocating for comprehensive data dictionaries and governance tools to facilitate effective data utilization. On the one hand, we assessed how machine learning can help support the quality of health data. On the other hand, we explored synthetic data as a potential method for creating more data, while also offering a secure and legal avenue for data analysis, as well as algorithm development and testing. \\

We also explored the requirements of ethics committees and Data Protection Officers , which, while designed to safeguard patient privacy, often impede timely data access or even prevent access altogether. The thesis evaluates the application of distributed data analysis, allowing for secure, location-based data analysis, thereby enhancing the timeliness and security of the process. We assess the potential of distributed machine learning models to support decision-making and compare different institutions. \\

Furthermore, this work investigates how the seamless integration of real-world evidence into clinical practice can drive innovation and improve patient outcomes. We explore the hurdles and potential of AI-based clinical decision support systems and how observational data can be used to create knowledge and trust in real-world settings. Emphasizing the need for a trust framework that ensures transparency and explainability in evidence production, we argue that this is crucial for building clinician and patient trust in the data and decision-making processes. \\

With this work, we underscore the necessity of a collaborative approach with clinicians, who are the end-users of the developed tools. Understanding their needs and workflows is paramount, requiring user-friendly tools that clinicians can seamlessly integrate into their practice without needing extensive data science training. Additionally, we realized that information preprocessing is vital, as is the fundamental cataloguing of existing data in institutions for rigorous data analysis. We also understood that distributed methods can facilitate access to information and provide greater protection than traditional methods of data usage and analysis.

In conclusion, this thesis contributes to the field of healthcare data science by highlighting the multifaceted challenges and proposing innovative approaches for effective knowledge extraction from healthcare data. It underscores the importance of cross-disciplinary collaboration, robust data infrastructures, and a balanced legal and technical framework to harness the full potential of healthcare data, ultimately driving innovation and improving patient outcomes.



\vspace*{10mm}\noindent
\textbf{Keywords}: real-world data, synthetic data, distributed-learning, machine-learning, data quality




\chapter*{Resumo}
Esta tese debruça-se sobre o intrincado processo de extração de conhecimento a partir de dados de saúde, uma tarefa repleta de desafios mas também cheia de potencial. Central para esta investigação é a frase, inspirada por Richard P. Feynman, de que a certeza absoluta é ilusória na investigação científica; ao invés disso, esta jornada é marcada por aprendizagem e melhoria contínuas. Há vários obstáculos a ultrapassar, incluindo acessibilidade dos dados, assim como a sua qualidade  e a integração de evidência do mundo real na prática clínica. Os principais problemas que afetam este processo podem ser explicados da seguinte forma:

\begin{itemize}
    \item Primeira Lei de Informática Médica de Van der Lei: Os dados devem ser usados apenas para os fins para os quais foram recolhidos.
    \item Efetividade das Fontes de Dados de saúde primárias e inovações analíticas: Explorar até que ponto as fontes de dados primárias e as inovações em métodos estatísticos reduzem a necessidade de realizar ensaios clínicos randomizados.
    \item Questões de Governança, Privacidade e Confiança: Abordar questões de governança, privacidade e confiança quando os dados de saúde primários são disponibilizados para investigação.

\end{itemize}

Uma parte deste trabalho é dedicada às várias dimensões que compreendem o conceito de qualidade dos dados. Por um lado, avaliámos como a aprendizagem automática pode ajudar a melhorar a representatividade da realidade dos dados de saúde. Por outro lado, explorámos os dados sintéticos como um método potencial para criar maior volume de dados, assim como oferecendo uma via mais segura e legal para análise de dados, desenvolvimento e teste de algoritmos ou software. Sabemos que a qualidade dos dados de saúde é um conceito complexo e difícil de definir, exigindo um extenso pré-processamento para gerir valores ausentes, \textit{outliers} e inconsistências entre diferentes sistemas de informação de saúde. A tese enfatiza a importância de descrições claras de dados funcionais e clínicos, defendendo dicionários de dados abrangentes e ferramentas de governança para facilitar a utilização eficaz dos dados. \\

Também explorámos os requisitos dos comités de ética e dos Encarregados de Proteção de Dados, que, embora concebidos para proteger a privacidade dos pacientes, muitas vezes dificultam ou impedem o acesso atempado aos dados. A tese avalia a aplicação da análise distribuída de dados, permitindo uma análise de dados segura e circunscrita ao local onde os dados estão guardados, melhorando assim a celeridade e segurança do processo. Avaliámos o potencial dos modelos distribuídos de aprendizagem automática para apoiar a tomada de decisões e comparar diferentes instituições sem nunca retirar os dados do seu local de origem. \\

Além disso, este trabalho investiga como a integração da evidência do mundo real na prática clínica pode impulsionar a inovação e melhorar os \textit{outcomes} clínicos. Explorámos os obstáculos e o potencial dos sistemas de apoio à decisão clínica baseados em Inteligência Artificial e como os dados observacionais podem ser utilizados para criar conhecimento com confiança em ambiente real. Enfatizando a necessidade de uma estrutura robusta e resiliente que garanta transparência e explicabilidade na produção de evidência, argumentamos que isso é crucial para fomentar e manter a confiança dos profissionais de saúde e doentes nos dados e nos processos de tomada de decisão. \\

Com este trabalho, sublinhamos a necessidade de uma abordagem colaborativa com os clínicos, que são os utilizadores finais das ferramentas desenvolvidas. Compreender as suas necessidades e fluxos de trabalho é primordial, exigindo ferramentas fáceis de usar que os clínicos possam integrar perfeitamente na sua prática sem precisar de formação extensa em ciência de dados. Adicionalmente, percebemos que o processamento da informação é vital, assim como é fundamental a catalogação dos dados existentes nas instituições para uma análise rigorosa dos dados. Percebemos também que métodos distribuídos que podem facilitar e agilizar os acessos à informação e conseguem garantir maior proteção que os métodos tradicionais de uso e análise de dados.

Em conclusão, esta tese contribui para o campo da ciência de dados em saúde ao destacar os desafios multifacetados e propor abordagens inovadoras para a extração eficaz de conhecimento a partir de dados de saúde. Ela salienta a importância da colaboração interdisciplinar, infraestruturas robustas de dados e um enquadramento legal e técnico equilibrado, assim como o uso dos métodos analíticos e estatísticos mais adequados para aproveitar todo o potencial dos dados de saúde, impulsionando a inovação e melhorando os resultados dos pacientes.

\vspace*{10mm}\noindent
\textbf{Palavras-chave}: Dados de mundo real, Dados sintéticos, Aprendizagem distribuida, Qualidade de dados, Aprendizagem automática, Inteligência Artificial
