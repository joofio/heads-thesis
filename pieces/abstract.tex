\chapter*{Abstract}

This thesis delves into the intricate process of extracting knowledge from healthcare data, a task fraught with challenges yet brimming with potential. Central to our investigation is the acknowledgment, inspired by Richard P. Feynman, that absolute certainty is elusive in scientific inquiry; instead, our journey is marked by continual learning and improvement. We confront various obstacles, including data accessibility, quality concerns, and the integration of real-world evidence (RWE) into clinical practice, while also exploring innovative solutions like synthetic data and distributed data analysis paradigms.

A significant portion of our work is dedicated to addressing the dual challenges of data access and quality. The stringent requirements of ethics committees and Data Protection Officers (DPOs), designed to safeguard patient privacy, often impede timely data access. We propose synthetic data as a potential workaround, offering a secure and legal avenue for algorithm development and testing. Additionally, we underscore the importance of distributed data analysis, allowing for secure, location-based data analysis, thereby enhancing the timeliness and security of the process.

Quality of healthcare data emerges as a complex and elusive concept, demanding extensive data preprocessing to manage missing values, outliers, and inconsistencies across different health information systems (HIS). The thesis highlights the criticality of clear functional and clinical data descriptions, advocating for comprehensive data dictionaries and governance tools to facilitate effective data utilization.

Moreover, we emphasize the necessity of a collaborative approach with clinicians, who are the end-users of the developed tools. Understanding their needs and workflows is paramount, necessitating user-friendly tools that clinicians can seamlessly integrate into their practice without the need for extensive data science training.

Another pivotal aspect of this thesis is the exploration of a legal and technical framework for healthcare data science, akin to the rigorous approval processes for pharmaceuticals. Such a framework should balance safety and innovation, ensuring that new tools and methodologies are both effective and ethical. This approach is complemented by a strong focus on biomedical informatics and the development of robust data infrastructures, both locally and internationally, to enhance data availability and quality.

The thesis also explores the potential of RWE to support clinical decisions in real-time, emphasizing the need for a trust framework that ensures transparency and explainability in evidence production. This is crucial for building clinician and patient trust in the data and decision-making processes.

In conclusion, this thesis contributes to the field of healthcare data science by highlighting the multifaceted challenges and proposing innovative approaches for effective knowledge extraction from healthcare data. It underscores the importance of cross-disciplinary collaboration, robust data infrastructures, and a balanced legal and technical framework to harness the full potential of healthcare data, ultimately driving innovation and improving patient outcomes.

\vspace*{10mm}\noindent
\textbf{Keywords}: real-world data, \ac{heads}, distributed-learning, machine-learning, data quality

\chapter*{Resumo}

Esta tese mergulha no processo intrincado de extrair conhecimento dos dados de saúde, uma tarefa repleta de desafios, mas também de potencial. Central para nossa investigação é o reconhecimento, inspirado por Richard P. Feynman, de que a certeza absoluta é ilusória na pesquisa científica; em vez disso, nossa jornada é marcada por aprendizado contínuo e melhoria. Confrontamos vários obstáculos, incluindo acessibilidade dos dados, preocupações com a qualidade e a integração de evidências do mundo real (RWE) na prática clínica, explorando também soluções inovadoras como dados sintéticos e paradigmas de análise de dados distribuídos.

Uma parte significativa do nosso trabalho é dedicada a enfrentar os desafios duplos de acesso e qualidade dos dados. Os requisitos rigorosos dos comités de ética e dos Responsáveis pela Proteção de Dados (DPOs), projetados para salvaguardar a privacidade do paciente, muitas vezes impedem o acesso oportuno aos dados. Propomos dados sintéticos como uma solução potencial, oferecendo um caminho seguro e legal para o desenvolvimento e teste de algoritmos. Além disso, sublinhamos a importância da análise de dados distribuída, permitindo uma análise de dados segura e baseada na localização, aumentando assim a pontualidade e a segurança do processo.

A qualidade dos dados de saúde surge como um conceito complexo e elusivo, exigindo extenso pré-processamento de dados para gerir valores ausentes, outliers e inconsistências entre diferentes sistemas de informação de saúde (HIS). A tese destaca a criticidade de descrições claras de dados funcionais e clínicos, defendendo ferramentas abrangentes de dicionários de dados e governança para facilitar a utilização eficaz dos dados.

Além disso, enfatizamos a necessidade de uma abordagem colaborativa com os clínicos, que são os utilizadores finais das ferramentas desenvolvidas. Entender as suas necessidades e fluxos de trabalho é fundamental, necessitando de ferramentas de fácil utilização que os clínicos possam integrar sem problemas na sua prática, sem a necessidade de extensa formação em ciência de dados.

Outro aspecto fundamental desta tese é a exploração de um quadro legal e técnico para a ciência de dados de saúde, semelhante aos rigorosos processos de aprovação para produtos farmacêuticos. Tal quadro deve equilibrar segurança e inovação, garantindo que novas ferramentas e metodologias sejam eficazes e éticas. Esta abordagem é complementada por um forte enfoque na informática biomédica e no desenvolvimento de infraestruturas de dados robustas, tanto a nível local como internacional, para melhorar a disponibilidade e qualidade dos dados.

A tese também explora o potencial da RWE para apoiar decisões clínicas em tempo real, enfatizando a necessidade de um quadro de confiança que garanta transparência e explicabilidade na produção de evidências. Isso é crucial para construir confiança dos clínicos e pacientes na precisão, confiabilidade e transparência dos dados e dos algoritmos utilizados. A construção dessa confiança implica garantir que os processos de tratamento de dados e de tomada de decisão sejam transparentes e explicáveis, fomentando um sentido de responsabilidade e confiabilidade no sistema.

Em conclusão, esta tese contribui para o campo da ciência de dados de saúde, destacando os desafios multifacetados e propondo abordagens inovadoras para a extração eficaz de conhecimento dos dados de saúde. Sublinha a importância da colaboração interdisciplinar, infraestruturas de dados robustas e um quadro legal e técnico equilibrado para aproveitar todo o potencial dos dados de saúde, impulsionando a inovação e melhorando os resultados dos pacientes.

\vspace*{10mm}\noindent
\textbf{Keywords}: keyword1, Keyword2, keyword3
