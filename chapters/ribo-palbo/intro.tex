Currently, metastatic breast cancer is difficult to treat. Patients with \ac{hr+} and \ac{her2} breast cancer, the most common subtype, typically undergo \ac{et}. Therefore, new treatments can be very useful in improving quality of life, reducing toxicity, and decreasing scenarios of hormonal resistance.
Medications from the group of \ac{cdk46i} appear as a potential improvement in the therapeutic approach to advanced breast cancer. Within this group, there are palbociclib, ribociclib and abemaciclib. \ac{cdk46} are responsible for regulating the cell cycle at the transition between the G1 and S phases. In many neoplasms, this cycle is deregulated, and it promotes uncontrolled cell proliferation. It is then possible for these medications to have better effectiveness. These medications were approved by INFARMED, I.P. after an analysis of the therapeutic value they offer. This decision was made based on data provided by clinical trials done with these medications. The MONALEESA \cite{hortobagyiUpdatedResultsMONALEESA22018, slamonPhaseIIIRandomized2018, tripathyRibociclibEndocrineTherapy2018} studies were used for ribociclib, PALOMA \cite{vermaPalbociclibCombinationFulvestrant2016, rugoImpactPalbociclibLetrozole2018, finnCyclindependentKinaseInhibitor2015a} for palbociclib, and MONARCH \cite{goetzMONARCHAbemaciclibInitial2017, sledgeMONARCHAbemaciclibCombination2017} for abemaciclib.
These studies focused on testing the hypothesis of treating \ac{cdk46i} in combination with an aromatase inhibitor or fulvestrant as an alternative to the gold standard. In these research findings, it was determined that there was a notable enhancement in effectiveness, supporting their application in clinical practice.
However, this evaluation was based on clinical trials with very specific inclusion and exclusion criteria and in a highly controlled environment. It is then vital to study how these new molecules compare to current practice in terms of treatment effectiveness in a real-world setting. In the meticulously controlled setting of clinical trials, patient selection often skews towards relatively healthier individuals with fewer comorbidities. However, in real-world clinical practice, patients present a diverse range of health profiles, co-existing illnesses, and medication histories that may influence drug efficacy and safety. Real-world data, drawn from electronic health records, insurance claims databases, and patient registries, offers the advantage of reflecting a more heterogeneous patient population, thus potentially uncovering insights not readily apparent in clinical trial settings. Understanding the effectiveness and safety of \ac{cdk46i} in real-world conditions is crucial for tailoring more individualized treatment regimens, optimizing outcomes, and enhancing the quality of life for patients with \ac{hr+}, \ac{her2} breast cancer \cite{harbeckCDK4InhibitorsHR2021}. Nevertheless, observational studies have inherent limitations, such as confounding by indication, which can lead to biased estimates of treatment effects. To tackle this, there are causality-based assessments that can be employed in order to better estimate the causal effects of treatments.
Incorporating statistical techniques like \ac{iptw} can play an essential role in enhancing the quality of real-world evidence by accounting for treatment selection bias and balancing observed covariates between treatment groups. \ac{iptw}, grounded in the framework of causal inference, allows for the mimicking of a randomized control trial-like setting within observational studies. By assigning weights to individual patients based on their propensity scores—the likelihood of receiving a particular treatment given a set of observed characteristics—analyses can achieve a balance between different treatment arms, thereby reducing bias and confounding factors. Establishing causality, rather than mere association, is vital for the robust interpretation of real-world data. As we strive to understand the long-term impact, efficacy, and safety of \ac{cdk46i} in \ac{hr+}, \ac{her2} breast cancer, the rigorous application of \ac{iptw} and causal inference methods can substantially augment the validity of real-world findings, making them a more reliable basis for clinical decision-making \cite{austinIntroductionPropensityScore2011,austinUsePropensityScore2014}
So in this paper, we propose:
\begin{itemize}
    \item To compare the effectiveness of the \ac{cdk46i} drug class in terms of  \ac{pfs}  and  \ac{os};

    \item To assess the Hazard Ratio of using the \ac{cdk46i} drug class in terms of \ac{pfs} and \ac{os}.
    
    \item  To compare the effectiveness of \ac{cdk46i} in combination with letrozole or fulvestrant with the previous standard of care in terms of \ac{pfs} and \ac{os} in patients with \ac{hr+}, \ac{her2} advanced breast cancer with bone only metastasis.

    \item To assess the differences in effectiveness between the three \ac{cdk46i} in combination with letrozole or fulvestrant in terms of \ac{pfs} and \ac{os} with causality principles in mind, especially the counterfactual theory and \ac{iptw}.
    
\end{itemize}
