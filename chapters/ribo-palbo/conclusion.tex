In conclusion, our findings underscore the efficacy of \ac{cdk46i} in real-world settings. We can confidently affirm the impact of Ribociclib on \ac{pfs}. This assertion aligns with clinical trial outcomes and real-world data further substantiates these findings. However, we cannot do the same for \ac{os}. Our results indicate that Ribociclib combined with letrozole or fulvestrant when compared to both is not superior to these alternatives used alone. The same happens when comparing ribociclib combined with letrozole with letrozole alone. 
However, we cannot do so for Palbociclib. Palbociclib combined with fulvestrant or letrozole was not significantly better than letrozole or fulvestrant alone for \ac{pfs} nor \ac{os}. This is something interesting that we want to follow up with.
Delving deeper into the characteristics of the patient population, including safety profiles, economic implications, and quality of life metrics, would be insightful. Additionally, a thorough examination of biomarkers within the population could offer invaluable insights. Finally, extending the follow-up period would be beneficial as well. We intend to explore these facets in subsequent publications.
It’s imperative to note that our data is sourced from a singular institution, limiting the capability of generalization of our results to a broader population. Nonetheless, we posit that this study lays a foundational groundwork for future research in this domain. While our evidence is rooted in observational data, and we’ve made adjustments for known confounders, the potential for residual confounding remains. Although the use of propensity score matching enhances the comparative robustness between the groups, the presence of unmeasured confounders cannot be entirely ruled out. Furthermore, the small sample size of our study limits the statistical power of our findings. For next steps we aim to further analyse the clinical variables that have an impact on the outcome of the combination of CDK4/6 with fulvestrant or letrozole and these drugs used alone in order to infer pharmaeconomic implications and possible profiles of patient that would not benefit from this combination which would be vital for economic reasons and to apply in countries with low access to these drugs.