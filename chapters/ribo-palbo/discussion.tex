The aim of this study was to evaluate the real-world use of palbociclib and ribociclib in combination with\ac{et}for \ac{hr+}/\ac{her2} and compare this drug class with traditional \ac{et}. Few real-world evidence studies of palbociclib and ribociclib used in daily clinical practice have been published identifying clinical benefit, patient profile, and sequencing of treatment, with even less evidence for the Portuguese population.

When comparing with clinical trials, regarding patient profile, in our study, 51\% had visceral metastasis and 35\% had bone-only metastases compared with 49\% and 38\% in PALOMA-2, and 60\% and 25\% in PALOMA-3, respectively \cite{rugoImpactPalbociclibLetrozole2018,cristofanilliFulvestrantPalbociclibFulvestrant2016a}.
As for ribociclib and bone-only metastases, MONALEESA-7 \cite{tripathyRibociclibEndocrineTherapy2018} has 24\% and MONALEESA-2 has 40\% \cite{hortobagyiUpdatedResultsMONALEESA22018} and our study has 30\%. Regarding menopausal status, our study has 20\% premenopausal and 80\% postmenopausal. 



Of note, the range of median \ac{pfs} for first-line palbociclib was 15.5–25.5 months, which is shorter than 27.6 months observed in a post hoc analysis of the PALOMA-2 clinical trial with extended follow-up \cite{rugoImpactPalbociclibLetrozole2018}, but in line with RWE studies (13.3–20.2 months) \cite{harbeckCDK4InhibitorsHR2021}. When assessed with only letrozole as a combination, the median \ac{pfs} increased to 28.6 months [95\% CI 25.5-not reached]. Additionally, analyzing the postmenopausal women subgroup, palbociclib showed a median \ac{pfs} of 16.3 months [95\% CI 12.9 -20]. Furthering analysis of the postmenopausal and with letrozole, the median was 47.6 months [95\% 25.6-2–not reached].

As for ribociclib, median survival time was not reached whether in \ac{os} and \ac{pfs}. So we can at least say that the median \ac{pfs} is longer than 50 months. This is longer than the median \ac{pfs} of 23.8 months (95\% CI 19.2–not reached) reported in the MONALEESA-7 trial \cite{tripathyRibociclibEndocrineTherapy2018} and longer than  25.3 months (95\% CI 23.0–30.3) in the MONALEESA-2 trial \cite{hortobagyiUpdatedResultsMONALEESA22018}. Regarding the subgroup analysis of postmenopausal women, we noticed that the median was not reached for women treated with ribociclib and fulvestrant or letrozole (RIB-LT/FUL) and postmenopausal women treated with ribociclib in combination with letrozole (RIB-LT).

When directly comparing ribociclib and palbociclib without any adjustments, one might deduce that ribociclib is superior to palbociclib. However, after adjusting for confounding variables, there is no significant difference between the two inhibitors in terms of \ac{pfs} or \ac{os} as indicated in table 2. This observation is further corroborated by the lower plots in figure 1, where even a subgroup analysis of \ac{cdk46i} combined solely with letrozole reveals non-significant difference between the two.

In the first-line comparison, the analysis of \ac{os} outcomes reveals no substantial difference between \ac{et} alone and the combination of \ac{cdk46i} with \ac{et}, irrespective of whether the \ac{cdk46i} are administered with fulvestrant or letrozole  (PAL-LT/FUL vs LT/FUL; \textit{P}=0.57 | RIB-LT/FUL vs LT/FUL \textit{P}=0.069) or exclusively with letrozole (PAL-LT vs LT; p = 0.979 | RIB-LT vs LT; \textit{P}=0.179)(figure \ref{fig:grouped} left). With respect to \ac{pfs}, ribociclib demonstrates superior efficacy when compared its combination with any of the adjuvants to these adjuvants alone (RIB-LT/FUL vs LT/FUL; \ac{hr}=0.21) as well as when combined only with letrozole (RIB-LT vs LT; \ac{hr}=0.27). Additionally, palbociclib exhibits significant improvement in \ac{pfs} when combined with letrozole  (PAL-LT vs LT; \ac{hr}=0.65) (figure \ref{fig:grouped} right).
When comparing with propensity scores weighting, we found out that ribociclib is significantly better than palbociclib for \ac{pfs} and \ac{os}, providing a median \ac{os} of over 40 months and median \ac{pfs} of around 42 months. Adjusted for the weighted variables, Ribociclib is not significantly better for \ac{pfs}, but has a \textit{P} value of 0.013 for \ac{os} with an \ac{hr}of 0.48. However, the Cox regression adjusted for variables and weights are not significant, even when the \textit{P} value for \ac{pfs} is 0.05. This suggests that a more in depth analysis may be necessary.


