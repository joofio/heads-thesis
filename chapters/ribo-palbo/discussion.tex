The aim of this study was to evaluate the real-world use of palbociclib and ribociclib in combination with ET for HR+/HER2$-$ and compare this drug class with traditional endocrine therapy. Few real-world evidence studies of palbociclib and ribociclib used in daily clinical practice have been published identifying clinical benefit, patient profile, and sequencing of treatment, with even less evidence for the Portuguese population.

When comparing with clinical trials, regarding patient profile, in our study, 51\% had visceral metastasis and 35\% had bone-only metastases compared with 49\% and 38\% in PALOMA-2, and 60\% and 25\% in PALOMA-3, respectively \cite{rugoImpactPalbociclibLetrozole2018,cristofanilliFulvestrantPalbociclibFulvestrant2016a}.
As for ribociclib and bone-only metastases, MONALEESA-7 \cite{tripathyRibociclibEndocrineTherapy2018} has 24\% and MONALEESA-2 has 40\% \cite{hortobagyiUpdatedResultsMONALEESA22018} and our study has 30\%.



Of note, the range of median PFS for first-line palbociclib was 15.5–25.5 months, which is shorter than 27.6 months observed in a post hoc analysis of the PALOMA-2 clinical trial with extended follow-up \cite{rugoImpactPalbociclibLetrozole2018}, but in line with RWE studies (13.3–20.2 months) \cite{harbeckCDK4InhibitorsHR2021}. When assessed with only letrozol as a combination, the median PFS increased to 28.6 months [95\% CI 25.5-not reached].
As for ribociclib, median survival time was not reached whether in OS and PFS. So we can at least say that the median PFS is longer than 50 months. This is longer than the median progression-free survival of 23.8 months (95\% CI 19.2–not reached) reported in the MONALEESA-7 trial \cite{tripathyRibociclibEndocrineTherapy2018} and longer than  25.3 months (95\% CI 23.0–30.3) in the MONALEESA-2 trial \cite{hortobagyiUpdatedResultsMONALEESA22018}. 

Regarding the comparison between ET and CDK4/6i first line, we found out that neither OS and PFS have significant changes when compared ET to Palbociclib 1st line. We can see the values similar to clinical trials when comparing only the letrozol group (both combination and letrozol alone). For this subgroup, we have similar results to clinical trials, with palbociclib being significantly better, with an HR of around 0.5.

Ribociclib is significantly better for the PFS when compared with letrozol and fulvestrant and with letrozol alone, with an HR of around 0.29 for PFS and 0.28 for ribociclib.
This would imply that a combination with fulvestrant should be more effective when used with ribociclib and palbociclib.
To note, that despite their results, the values in table \ref*{tab:cox} suggest that when we adjust for the variables indicated, ribociclib is significantly better than palbociclib in terms of PFS with an HR of around 0.6.

When comparing with propensity scores weighting, we found out that ribociclib is significantly better than palbociclib for PFS. Our findings suggest that ribociclib could be a better approach for treating HR+, HE- metastic breast cancer, providing a median OS of over 40 months and median PFS of around 42 months. 


