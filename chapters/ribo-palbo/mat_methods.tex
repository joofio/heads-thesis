

\subsection{Study Design}

This retrospective study was designed in 2022. The study aimed to evaluate the clinical benefit and long-term survival of patients with HR+/HER2- that started treatment with CDK4/6 inhibitors plus endocrine therapy in different lines of treatment between the 14th of March 2017 and the 31st of December 2021. The follow-up period was set until June 2022. Inclusion criteria: women and men, Hormone receptor-positive and HER2 negative in the primary tumor or metastatic site after biopsy. Exclusion criteria: Patients that had only one ambulatory medication, and patients involved in clinical trials, diagnosed with other neoplasms or with active treatment during the study period. The control group was defined by a population of patients, that were treated with hormone therapy as first-line (due to bone metastases only) between 2015 and 13 of match 2017.
The evaluation of effectiveness will involve overall survival and progression-free analysis. We will compare the two different cyclin-dependent kinase inhibitors in terms of effectiveness in real-world patients and will also compare the effectiveness of this class combined with endocrine therapy against traditional endocrine therapy.


\subsection{Data collection}
All data were collected from medical and administrative records from baseline to last visit or death. The data was collected from Instituto Português de Oncologia – Porto (IPO-P). Table 1 shows a comparison between the groups.
Data included for population treated with CDK4/6 inhibitors plus endocrine therapy: demographic information, age at first diagnosis and age at the beginning of treatment, clinical characteristics and performance status by Eastern Cooperative Oncology Group scale (ECOG), treatment line and treatment schema - CDK4/6 inhibitor and endocrine therapy, stage of cancer, site of metastases (bone, soft tissue, visceral, central nervous system-CNS with or without another site).
Data included for the population treated with endocrine therapy as first-line: demographic information, age at first diagnosis and age at the beginning of treatment, clinical characteristics and performance status by Eastern Cooperative Oncology Group scale (ECOG), stage of the cancer.
For comparison purposes, we used palbociclib and ribociclib since we had a small number of patients treated with abemaciclib (12).

 
\begin{table}
\caption{Descriptive statistics of cyclin-dependent kinase inhibitors group and endocrine therapy group. The Drug/combination refers to the actual drug or the combination for CDK4/6}
\centering
\label{tab:stats_ipop_cdk}
\begin{tabular}{l|lllll}
\toprule
   Column &               Silo 6 &               Silo 7 &               Silo 8 &               Silo 9 &                Agrr. \\
\midrule
       IA &    31.3 \textbf{5.2} &    31.4 \textbf{5.4} &    31.5 \textbf{5.6} &    30.1 \textbf{5.6} &    31.1 \textbf{5.6} \\
       GS & a,rh.. \textbf{42\%} & a,rh.. \textbf{39\%} & a,rh.. \textbf{40\%} & a,rh.. \textbf{42\%} & a,rh.. \textbf{40\%} \\
       PI &   65.6 \textbf{13.5} &   66.0 \textbf{13.7} &   65.6 \textbf{14.1} &   67.4 \textbf{14.6} &   66.0 \textbf{14.1} \\
      PAI &   77.7 \textbf{13.4} &   79.2 \textbf{14.7} &   76.7 \textbf{13.0} &   83.1 \textbf{15.2} &   78.8 \textbf{14.5} \\
      IMC &    24.9 \textbf{5.1} &    24.9 \textbf{7.0} &    24.8 \textbf{8.0} &    25.7 \textbf{5.6} &    25.1 \textbf{7.0} \\
      CIG &   Null \textbf{91\%} &   Null \textbf{91\%} &   Null \textbf{86\%} &   Null \textbf{90\%} &   Null \textbf{88\%} \\
    APARA &    1.0 \textbf{38\%} &   Null \textbf{43\%} &   Null \textbf{41\%} &   Null \textbf{43\%} &   Null \textbf{39\%} \\
   AGESTA &    1.0 \textbf{44\%} &      1 \textbf{43\%} &    1.0 \textbf{42\%} &    1.0 \textbf{40\%} &    1.0 \textbf{42\%} \\
       EA &   Null \textbf{59\%} &   Null \textbf{61\%} &   Null \textbf{69\%} &   Null \textbf{61\%} &   Null \textbf{60\%} \\
       VA &   Null \textbf{79\%} &   Null \textbf{82\%} &   Null \textbf{88\%} &   Null \textbf{82\%} &   Null \textbf{77\%} \\
       FA &   Null \textbf{82\%} &   Null \textbf{86\%} &   Null \textbf{94\%} &   Null \textbf{89\%} &   Null \textbf{83\%} \\
       CA &   Null \textbf{69\%} &   Null \textbf{75\%} &   Null \textbf{85\%} &   Null \textbf{78\%} &   Null \textbf{75\%} \\
       TG & espo.. \textbf{88\%} & espo.. \textbf{85\%} & espo.. \textbf{86\%} & espo.. \textbf{93\%} & espo.. \textbf{85\%} \\
        V &      s \textbf{97\%} &      s \textbf{99\%} &      s \textbf{98\%} &      s \textbf{99\%} &      s \textbf{98\%} \\
    NRCPN &     6.8 \textbf{4.0} &     7.7 \textbf{3.2} &     9.3 \textbf{4.5} &     8.9 \textbf{5.5} &     8.4 \textbf{5.1} \\
       VP &   Null \textbf{68\%} &   Null \textbf{74\%} &   Null \textbf{71\%} &   Null \textbf{78\%} &   Null \textbf{76\%} \\
      VCS &   Null \textbf{53\%} &      s \textbf{87\%} &      s \textbf{63\%} &      s \textbf{87\%} &      s \textbf{68\%} \\
      VNH &   Null \textbf{62\%} &      s \textbf{63\%} &      s \textbf{69\%} &      s \textbf{83\%} &      s \textbf{69\%} \\
        B &   Null \textbf{90\%} &   Null \textbf{53\%} &   Null \textbf{93\%} &   Null \textbf{82\%} &   Null \textbf{83\%} \\
       AA &   Null \textbf{84\%} & apr... \textbf{61\%} &   Null \textbf{89\%} &   Null \textbf{74\%} &   Null \textbf{73\%} \\
       BS &   Null \textbf{99\%} &   Null \textbf{98\%} &   Null \textbf{99\%} &   Null \textbf{95\%} &   Null \textbf{95\%} \\
       BC &  Null \textbf{100\%} &  Null \textbf{100\%} &  Null \textbf{100\%} &   Null \textbf{97\%} &   Null \textbf{97\%} \\
      BDE &  Null \textbf{100\%} &  Null \textbf{100\%} &  Null \textbf{100\%} &   Null \textbf{97\%} &   Null \textbf{97\%} \\
      BDI &  Null \textbf{100\%} &  Null \textbf{100\%} &   Null \textbf{99\%} &   Null \textbf{97\%} &   Null \textbf{96\%} \\
       BE &  Null \textbf{100\%} &  Null \textbf{100\%} &  Null \textbf{100\%} &   Null \textbf{97\%} &   Null \textbf{96\%} \\
       BP &  Null \textbf{100\%} &  Null \textbf{100\%} &  Null \textbf{100\%} &   Null \textbf{97\%} &   Null \textbf{97\%} \\
      IGA &    38.7 \textbf{1.8} &    39.0 \textbf{2.0} &    38.6 \textbf{2.1} &    38.8 \textbf{1.9} &    38.7 \textbf{2.2} \\
     TPEE &   Null \textbf{65\%} &   Null \textbf{64\%} &   Null \textbf{65\%} &   Null \textbf{63\%} &   Null \textbf{65\%} \\
     TPEI &   Null \textbf{92\%} &   Null \textbf{86\%} &   Null \textbf{87\%} &   Null \textbf{94\%} &   Null \textbf{93\%} \\
      RPM &   Null \textbf{85\%} &   Null \textbf{84\%} &   Null \textbf{90\%} &   Null \textbf{94\%} &   Null \textbf{88\%} \\
       DG &   Null \textbf{92\%} &   Null \textbf{88\%} &   Null \textbf{90\%} &   Null \textbf{87\%} &   Null \textbf{89\%} \\
       TP & part.. \textbf{54\%} & part.. \textbf{52\%} & part.. \textbf{48\%} & part.. \textbf{59\%} & part.. \textbf{51\%} \\
      ANP & cefá.. \textbf{93\%} & cefá.. \textbf{94\%} & cefá.. \textbf{95\%} & cefá.. \textbf{94\%} & cefá.. \textbf{94\%} \\
     TPNP & espo.. \textbf{64\%} &  Null \textbf{100\%} & espo.. \textbf{50\%} & espo.. \textbf{65\%} & espo.. \textbf{53\%} \\
      SGP &    38.8 \textbf{1.8} &    39.2 \textbf{1.7} &    38.7 \textbf{2.0} &    39.0 \textbf{1.6} &    38.9 \textbf{2.0} \\
       GR &      1 \textbf{27\%} &      1 \textbf{25\%} &      1 \textbf{21\%} &      3 \textbf{27\%} &      1 \textbf{24\%} \\
       \midrule
N (total) &                12002 &                 8258 &                 6693 &                11786 &                80874 \\
\bottomrule
\end{tabular}

\end{table}




\subsection{Statistical Analysis}
%copied
R was used for statistical analysis. Demographic, clinical characteristics and side effects were analyzed using descriptive statistics (count, percentages and median/range). Kaplan–Meier test was used to determine the median PFS and OS in the entire population and subgroups. Log-rank test was used for comparisons of PFS and OS among different subgroups. Cox Regression was used to assess feature importance and impact. All statistical tests were two-sided, and the significance level was 0.05. The evaluation of the proportional hazards assumptions was done by Schoenfeld residues analysis.
We applied propensity score weights to achieve a more robust comparison between the two groups of CDK4\/6i. We used the existence of visceral metastases, treatment line, age at treatment start, and stage. We used the WeightIt package for R \cite{WeightIt}. We applied the weights to the Kaplan-Meier curves and to the Cox Regression. We applied the weights to get the ATE which is $E[Y_i(1)-Y_i(0)]$, the average effect of moving an entire population from untreated to treated, or from one drug to the other. Weights were used instead of matching since it is more suited for calculating ATE and the need to preserve the sample size since it is already small from the start. The formula for calculating the weights was through propensity score weighting with GLM. Multiple comparisons were done with the Benjamini-Hochberg (BH) method. 



%Pretende-se que os dados venham do Instituto Português de Oncologia – Porto (IPO-P). Pretende-se utilizar a base de dados do hospital dos últimos 5 anos.
%O estudo será registado e respeitará todos os requisitos éticos de aprovação a comissão de ética de cada instituição participante. Caso a instituição tenha um encarregado de Proteção de Dados (EPD), este será contactado a fim de dar seu parecer, e caso necessário, a sua opinião para melhorar possíveis pontos relacionados à segurança de dados.

%The goal is to identify the clinical or biological variables that have the most impact on a specific outcome. To achieve this, considering the statistical models found, we can model things in terms of time series or survival trees to group outcomes and the clusters found. With this, we can analyze the effectiveness of the clinical course, probabilities of recurrence, or survival rates by subgroup. These models and relationships can form the basis of a clinical decision support system and can be crucial for making better healthcare decisions.

%Some of the techniques used are unsupervised techniques such as k-means, DBSCAN, or hierarchical clustering.
%The goal is to identify the clinical or biological variables that have the most impact on a specific outcome. To achieve this, considering the statistical models found, we can model things in terms of time series or survival trees to group outcomes and the clusters found. With this, we can analyze the effectiveness of the clinical course, probabilities of recurrence, or survival rates by subgroup. These models and relationships can form the basis of a clinical decision support system and can be crucial for making better healthcare decisions.

%Some of the techniques used are unsupervised techniques such as k-means, DBSCAN, or hierarchical clustering.




% breast cancer cells have either estrogen (ER) or progesterone (PR) receptors or both. (HR +)