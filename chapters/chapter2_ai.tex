\ac{ai} has already been under public focus for a few years now, but its concept is still elusive, mainly due to the fact that the definition has been changing rapidly as well.
From the very beginning, the field of \ac{ai} was about not only understanding but also building intelligent entities \cite{DBLP:books/aw/RN2020}. Intelligent entities can be understood as machines that can act according to what is expected in a wide range of situations.
The first work of \ac{ai} could be credited to Warren McCulloch and Walter Pitts (1943) with the proposed model of artificial neurons. In the 50s, \ac{ai} could be associated with the works of Christopher Strachey, of two chess-playing programs.
In the 60s, the perceptrons could be indicated as state-of-the-art \ac{ai}. In the 80s, expert systems were providing advanced reasoning that the so-called weak methods of previous iterations could not compete with.
The 90s brought the probabilistic reasoning and \ac{ml} which led to more robust systems that went further than the boolean logic used so far. In the 2000s, big data and \ac{ml} got focused on. Big data was used as a matter symbolizing the increasing amounts of data in some industries \cite{dashBigDataHealthcare2019}, and \ac{ml} as \textit{ the study of computer algorithms that improve automatically through experience} \cite{mitchell1997machine}. This last definition is especially important since it is currently used as a synonym of \ac{ai} across several industries but have actual different meanings like discussed below. This era probably peaked around the IBM Watson victory in jeopardy, but with way fewer interesting results in healthcare \cite{swetlitzIBMWatsonSupercomputer2018}, and 2010s brought deep learning. 
Nowadays, \ac{ai} is a buzzword that is used to describe a wide range of systems, from the simplest to the most complex. It is clearly trending, reports on \ac{ai} show that papers regarding the subject have seen a 20-fold increase from 2010-2019.
%cite (20fold)
For defining \ac{ai}, we could use the definition provided by a group of experts the European Commission asked to write some guidelines on \ac{ai} \cite{DefinitionAIMain2019}
From this document, it is understood, that first, we need to address the difference between intelligence and rationality. Since the first is more subjective and even philosophical, the second is more pragmatic and related to the capability of choosing the best action to take in a certain scenario towards a certain goal. This is a more concrete concept and although it is not the same as intelligence, it should be a part of it \cite{DefinitionAIMain2019,DBLP:books/aw/RN2020}.
From these two concepts, we can go even deeper, and define that rationality can be achieved in an \ac{ai} system by perceiving the environment, reasoning with what is perceived, and acting on the environment. From these three elements, we can argue that reasoning is the core functionality, which is related to taking data, understanding it or interpreting it, and reasoning on this data through a model (numerical or symbolical) to reach the best action.

There is also the need to address the current distinction for \ac{ai} which is the narrow and general \ac{ai}. The first is the one that exists nowadays and it's an \ac{ai} that is not generic, it is focused on a specific task. The second is the one that is not yet achieved, and it is the one that is more generic and can be applied to several tasks. This is the one that is usually associated with the popular or common concept of \ac{ai} \cite{DefinitionAIMain2019,DBLP:books/aw/RN2020}. So, with this is mind, we reached the definition of \ac{ai} as:
\begin{quote}
    \textit{Artificial intelligence (AI) systems are software (and possibly also hardware) systems designed by humans that, given a complex goal, act in the physical or digital dimension by perceiving their environment through data acquisition, interpreting the collected structured or unstructured data, reasoning on the knowledge, or processing the information, derived from this data and deciding the best action(s) to take to achieve the given goal. \ac{ai} systems can either use symbolic rules or learn a numeric model, and they can also adapt their behaviour by analysing how the environment is affected by their previous actions. As a scientific discipline, \ac{ai} includes several approaches and techniques, such as machine learning (of which deep learning and reinforcement learning are specific examples), machine reasoning (which includes planning, scheduling, knowledge representation and reasoning, search, and optimization), and robotics (which includes control, perception, sensors and actuators, as well as the integration of all other techniques into cyber-physical systems).} \cite{DefinitionAIMain2019}
    \end{quote}

Of course, since the developments of this area have been so vast, this concept may become outdated very quickly. However, it is a good starting point to understand the concept of \ac{ai} and its implications.

%This is connected with the concept that \ac{ai} is tightly connected with the subjective nature of humans. If we feel that \ac{ai} is something related to what a human can do, it can be widely diverse from person to person.

%however, there are some definitions that could be interesting to explore in order to get the concept for the purpose of this thesis cleared up.

