% !TeX root = ../thesis.tex



\ac{ai} has already been under public focus for a few years now, but its concept is still elusive, mainly because the definition has been changing rapidly as well. From the very beginning, the field of \ac{ai} focused not only on understanding but also on building intelligent entities \cite{DBLP:books/aw/RN2020}. Intelligent entities can be understood as machines that can act according to what is expected in a wide range of situations.

The first work of \ac{ai} can be credited to Warren McCulloch and Walter Pitts (1943) with the proposed model of artificial neurons. In the 1950s, \ac{ai} could be associated with the works of Christopher Strachey and two chess-playing programs.

In the 1960s, perceptrons were considered state-of-the-art \ac{ai}. In the 1980s, expert systems provided advanced reasoning that the so-called weak methods of previous iterations could not compete with.

The 1990s brought probabilistic reasoning and \ac{ml}, which led to more robust systems that went beyond the Boolean logic used so far. In the 2000s, big data and \ac{ml} became the focus. Big data was used as a symbol of the increasing amounts of data in some industries \cite{dashBigDataHealthcare2019}, and \ac{ml} as \textit{the study of computer algorithms that improve automatically through experience} \cite{mitchell1997machine}. This last definition is especially important since it is currently used as a synonym for \ac{ai} across several industries but has a different meaning, as discussed below. This era probably peaked around IBM Watson's victory in Jeopardy but yielded far fewer interesting results in healthcare \cite{swetlitzIBMWatsonSupercomputer2018}, and the 2010s brought deep learning.
%TODO LLM
Nowadays, \ac{ai} is a buzzword that is used to describe a wide range of systems, from the simplest to the most complex. It is clearly trending, as reports on \ac{ai} show that papers regarding the subject have seen a 20-fold increase from 2010 to 2019.
%TODO cite (20fold)
To define \ac{ai}, we can refer to the description given by a team of specialists whom the European Commission tasked with developing guidelines on \ac{ai} \cite{DefinitionAIMain2019}. This document clarifies that initially, it is essential to distinguish between intelligence and rationality. Intelligence, being more subjective and philosophical, differs from rationality, which is pragmatic and linked to the ability to select the optimal course of action in a given situation to achieve a specific goal. While rationality is a more tangible concept and not identical to intelligence, it should be considered an integral component of it \cite{DefinitionAIMain2019,DBLP:books/aw/RN2020}.

From these two concepts, we can go even deeper and define that rationality can be achieved in an \ac{ai} system by perceiving the environment, reasoning with what is perceived, and acting on the environment. From these three elements, we can argue that reasoning is the core functionality, which is related to taking data, understanding it or interpreting it, and reasoning on this data through a model (numerical or symbolical) to reach the best action.

There is also the need to address the current distinction in \ac{ai}, which is the narrow and general \ac{ai}. The first is the one that exists nowadays, and it's an \ac{ai} that is not generic; it is focused on a specific task. The second is the one that is not yet achieved, and it is more generic and can be applied to several tasks. This is the one that is usually associated with the popular or common concept of \ac{ai} \cite{DefinitionAIMain2019,DBLP:books/aw/RN2020}. So, with this in mind, \ac{ai} can be defined as:

\begin{quote}
    \textit{Artificial intelligence systems are software (and possibly also hardware) systems designed by humans that, given a complex goal, act in the physical or digital dimension by perceiving their environment through data acquisition, interpreting the collected structured or unstructured data, reasoning on the knowledge, or processing the information, derived from this data and deciding the best action(s) to take to achieve the given goal. \ac{ai} systems can either use symbolic rules or learn a numeric model, and they can also adapt their behaviour by analysing how the environment is affected by their previous actions. As a scientific discipline, \ac{ai} includes several approaches and techniques, such as machine learning (of which deep learning and reinforcement learning are specific examples), machine reasoning (which includes planning, scheduling, knowledge representation and reasoning, search, and optimization), and robotics (which includes control, perception, sensors and actuators, as well as the integration of all other techniques into cyber-physical systems).} \cite{DefinitionAIMain2019}
    \end{quote}
    
Certainly, given the rapid advancements in this field, this concept might become outdated swiftly. Nonetheless, it serves as an effective starting point for grasping the fundamentals of \ac{ai} and its implications.

%This is connected with the concept that \ac{ai} is tightly connected with the subjective nature of humans. If we feel that \ac{ai} is something related to what a human can do, it can be widely diverse from person to person.

%however, there are some definitions that could be interesting to explore in order to get the concept for the purpose of this thesis cleared up.

