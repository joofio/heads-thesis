\begin{savequote}[85mm]
Nothing great in the world was accomplished 
without passion.
\qauthor{Friedrich Hegel}
\end{savequote}

\chapter{Case Studies}\label{chap:usecase}
This chapter will comprise the work done during this PhD. The works developed and corresponding papers were a search for improving data usage in several steps of the \ac{kdd} process. We can see some work dedicated to leveraging data acquisition or alternatives to it, like the works depicted in section \ref{subsec:distributed}, \ref{subsec:benchmark}, \ref{subsec:similarity}, \ref{subsec:tabular} and \ref{subsec:gans}. Others will focus more on how to use the data in order to make a difference in clinical practice like the section \ref{subsec:ipop}, \ref{subsec:obs} and \ref{subsec:dq}.



\section{Can GANs help create realistic datasets?}\label{subsec:gans}
This section is based on the paper entitled "GANs for Tabular Healthcare Data Generation: A Review on Utility and Privacy". It focuses on a review of the \ac{gan} framework for creating synthetic data for healthcare. Tries to compile the metrics used for comparing and assessing synthetic data in terms of utility - or how similar they are to the original data and privacy - how protective of the patient's data it is. 

% !TeX root = ../thesis.tex

\subsection{Introduction}

With the growing technological advances, the quantity of healthcare-related data produced around the world increased exponentially \cite{choi_generating_2017,henry_adoption_2016}.
Consequently, the potential for harvesting this data also increases. The value locked within
this data could help provide better healthcare with new information about diseases,
drugs, and preventive therapies. It can also help create better \acp{his}, meaning an overall better clinical practice \cite{ISI:000502534100049}. But for this to happen, data must reach capable hands at the right time.
But the release of clinical data has several barriers attached and rightly so. The leakage of patient’s privacy can break the confidence of the population in healthcare
professionals and institutions. Patient safety
and privacy should be kept at all costs. However, the current mechanisms for privacy maintenance are very long, bureaucratic, and time-consuming, nationally \cite{comissao_nacional_protecao_de_dados_principios_2015}, and internationally \cite{office_for_civil_rights_guidance_2013}. The current scenario and general methods for privacy safeguards are related to pseudo-anonymisation techniques.
The removal of certain attributes, identifier modification, code grouping, or discretization are some methodologies. But not even these are totally safe \cite{el_emam_systematic_2011}.
Synthetic data appear as an alternative for clinical data sharing, promising great data
utility with minimal privacy concerns. Synthetic data is data that is generated automatically through programmatic processes. This is especially impactful for the case at hand
since synthetic data has no explicit connection with the original data. There are several
mechanisms for data synthesis postulated by \cite{goncalves_generation_2020}, there are
process-driven methods and data-driven methods. Process-driven methods generate
data through pre-determined models inputted into the generator. Data-driven methods
produce new data based on inputted source data. With this, it is possible to create new
patient data that has no relation to reality while providing the same statistical relations
between variables. This provides the basis for quality clinical research on top of this
new data. Even though these techniques are still new and in rapid development, the
results seem interesting \cite{goncalves_generation_2020}, but not without questions and doubts
\cite{stadler_synthetic_2020}.
Creating a thorough survey based on the generation of synthetic data is seldom a simple task when compared to other surveys since synthetic data is present across several domains and has several uses, like software testing, assessing methods, or generating hypotheses. Moreover, synthesis has
the double meaning of summing up information and generating something, easily wielding hundreds of results per query. Finally, trying to filter
algorithms aimed at tabular data is also burdensome, since not always it is easy to discriminate input types. These factors make the survey interesting to focus on the state-of-the-art mechanisms of generating tabular data.

\subsection{Theoretical background}
First introduced in 2014, \acp{gan} \cite{goodfellow_generative_2014} have been under the scope and have been proven very good for generating complex data. Images, text, and video have been successfully generated with very good performances. %cite?
The original architecture is based on two artificial neural networks trained simultaneously in a competitive manner. One of them, the generator, has the objective of generating the most realistic possible data, while the second network – the discriminator, has the opposite aim of aiming to distinguish the realistic data from the synthetic
data the best it can. So, the elegance of this architecture is that each network tries to
make the other perform better every time. The \ac{gan} architecture is shown in \ref{fig:gan-arch}.
% w - \omega
% θ - \theta
% G - generator
% D - discriminator

\begin{figure}
\centering
%\includegraphics[width=\textwidth]{image.png}
\includegraphics[scale=0.75]{figures/image.png}

\caption{\ac{gan} framework} \label{fig:gan-arch}
\end{figure}
The generator is represented by $G_{\theta}$ where the parameter $\theta$ represents the weights of
the neural network. It takes as input, a Gaussian random variable, and outputs $G_{\theta}$(Z).
Distribution of $G_{\theta}$(Z) is denoted by $P_{\theta}$. The goal of the generator is to choose $\theta$ such that the output $G_{\theta}$(Z) has a distribution close to the real data. The discriminator is represented by $D_{\omega}$, parametrized by weights $\omega$. The goal of the discriminator is to assign 1 to the samples from the real distribution $P_{X}$ and 0 to the generated samples ($P_{\theta}$). So, \acp{gan} can be mathematically represented by a \textit{MinMax} game identified by:
\begin{equation}
\min_{G}\max_{D} \; E [log(D_{\omega}(X)) + log(1-D_{\omega}(G_{\theta}(Z))]
\end{equation}
So, $G$ must minimize this equation and $D$ must maximize it, each one tweaking the weights of its network ($\theta$ and $\omega$) to do so. This is the loss function on the initial \ac{gan} architecture. After the classification of $D$, the $G$ is trained again with the error signal from $D$ through backpropagation. This equation is the log of the probability of $D$ predicting that the real data is genuine and the log probability of $D$ classifying synthetic data as not genuine. The equation is essentially the same as minimising the \ac{jsd} \cite{goodfellow_generative_2014}:
\begin{equation}
\min_{G} JS(P_{x}||P_{\theta})
\end{equation}
Where the JS means the \acl{jsd} between the probability of the real data and the probability of the generated data. The JS divergence provides a measure of the distance between two probability distributions. Therefore, the minimization over $\theta$ means, choosing the $P_{\theta}$ that is closest to the target distribution $P_{X}$ in the JS divergence distance. Despite the significant results provided by \acp{gan} with continuous real values, categorical values still seem to be a problem for this approach \cite{kusner_gans_2016}, since it is not directly applicable for calculating the gradients of latent categorical variables in order to train these networks through backpropagation. This happens since the output of the generator, even though can be transformed into a multinomial distribution with a \textit{softmax} layer, sampling from it is not a differentiable operation, limiting the backpropagation process of the \ac{gan}.

%One of the most famous alternative GAN architectures is the \textit{Wasserstein}  GAN (WGAN)
%\cite{arjovsky_wasserstein_2017} . It improves GAN by using the \textit{Wasserstein} distance instead of the \textit{Jensen-shannon divergence}. It is represented by the
%following minimax game:
%\begin{equation}
%\min_{G}\max_{\omega \; \epsilon \;W} \;\; E[f_{\omega}(X)] - E[f_{\omega}(G_{\theta}(Z))] 
%\end{equation} 
%Where functions $f_{\omega} \;  \omega \; \epsilon \;W$  are all \textit{K-Lipschitz} (with respect to $x$) for some K.
%This was proposed because it helps prevent mode collapse, where the generator maps
%several inputs to the same output. An example of this is a generator that starts creating
%several images with the same patterns or colours. The fact that this loss function can be
%very customisable in distinct GAN implementation, applying different functions can
%lead to different outcomes and architectures.

%\iffalse
%\subsubsubsection{Differential privacy}
%The privacy model most used in generative models is differential privacy \cite{dwork_differential_2006}. It is a mathematical model for measuring privacy loss. The basis of it is a mechanism that adds noise to a dataset through randomisation. Applying differential privacy is a method for guaranteeing that adding or removing any record from the original data does not influence the generation output. The formal definition, as stated in \cite{dwork_differential_2006} is:
%\begin{equation}
%Pr[M(x)\; \epsilon \; S] \leq e^\varepsilon Pr[M(y)\; \epsilon \; S] + \delta
%\end{equation} 
%Where M is a randomised mechanism, $\varepsilon$ is the privacy budget (balance between privacy and utility). The $x$ and $y$ represent adjacent datasets that differ on only one record. So, we say that M gives $\varepsilon$-differential privacy if the probability of seeing an event S in the $x$ dataset is at most equal to $e^\varepsilon$ multiplied by the probability that we see S when the dataset is $y$. Variable $\delta$ is the probability of an uncontrolled breach. With this, a smaller $\varepsilon$ will yield better privacy but a less accurate response, so it can be used to tweak the utility and privacy of the dataset.
%\fi

\subsection{Methods}
This search was made between December 2020 and January 2021. It was made on “Web of Science”, IEEE, PubMed, Arxiv and finally GitHub. The terms searched were related to \acp{gan}, synthetic data generation, electronic health records, patient data, or tabular data. Applications of \acp{gan} to non-tabular data were filtered, like image, sound, video, or graphs. Time series and text data were also removed since the methodology for synthesizing this type of data has specific functions related to the nature of the data. The filter for date was after 2014 since \acp{gan} were introduced at that time. The queries used were similar to the one below, adapted for the search mechanics for each website.


\begin{scriptsize}

{\fontfamily{courier}\selectfont
("generation" OR "creation" OR "synthesis" OR "synthesizing" OR "generating" OR "creating") AND ("synthetic data" OR "synthetic patient" OR "synthetic electronic health record" OR "synthetic EHR" OR "realistic patient data" OR "realistic health record" OR ("synthetic" AND "privacy" AND "utility"))  AND ("GAN" OR "Generative Adversarial Network")}
\end{scriptsize}
%\begin{verbnobox}[\scriptsize]
%("generation" OR "creation" OR "synthesis" OR "synthesizing" OR
%"generating" OR "creating") AND ("synthetic data" OR
%"synthetic patient" OR "synthetic electronic health record" 
%OR "synthetic EHR" OR "realistic patient data" OR "realistic 
%health record" OR ("synthetic" AND "privacy"  AND "utility")) 
%AND ("GAN" OR "Generative Adversarial Network")
%\end{verbnobox}

From the total articles found (1165) with all the queries, 100 articles were chosen for full text and in the end, 22 papers with \ac{gan} implementations that were tested on tabular data were selected. 
%These implementations were studied and evaluated in terms of privacy concerns and utility performance and compared when possible. 
%For selecting the articles and helping with paper selection, RAYYAN QCRI online \cite{rayyan:14:cochrane} was used along with Mendeley Reference Manager for paper management.


\subsection{Results}
The selected papers ranged from 2017 to 2020. Being that 2 are from 2017, 4 from 2018, 8 from 2019 and 8 from 2020. All authors showed original \ac{gan} implementations, apart from 2 papers. Beaulieu-Jones et al. \cite{beaulieu-jones_privacy-preserving_2019} used a
\ac{gan} architecture that was originally published with usage on image datasets \cite{odena_conditional_2017}. Additionally, Vega-Marquez et al.  \cite{ISI:000490706700022} used an already known implementation of conditional \acp{gan} \cite{mirza_conditional_2014}. We classified papers regarding 3 metrics: utility, privacy and clinical. For utility, we looked for  methods for measuring the generated data's quality. As for privacy, we aimed for some mechanism for measuring the privacy loss of the new data. Concerning clinical metrics, any kind of evaluation from healthcare professionals was considered. This can be seen in table \ref{tab:review_gans}.


\begin{table}[btph]
\caption{Summary of the articles selected.}\label{tab:review_gans}
\centering
\begin{tabular}{l|llclc}
%\begin{tabular}{l|llclc}

\toprule
 ID & year  & Acronym & Article & Metric & Code \\

% & \multicolumn{1}{|c}{Year \hspace{2 mm}}  & \multicolumn{1}{|c|}{Acronym} & \multicolumn{1}{c}{\hspace{2 mm} Article \hspace{1 mm}} & \multicolumn{1}{|c|}{Metric} & \multicolumn{1}{c|}{\hspace{2 mm}Code \hspace{2 mm}}  \\

\midrule
1 & 2017 & medGAN  &\cite{choi_generating_2017}          & Utility, Privacy, Clinical \hspace{1 mm} &  \cite{medGANurl}  \\

2 & 2017 & POSTER & \cite{ISI:000440307700174} & Utility, Privacy  &   \cite{POSTERurl}\\

3 & 2018 & table-GAN & \cite{park_data_2018} & Utility, Privacy         &   \cite{table-GAN-url} \\

4 & 2018 & dp-GAN & \cite{xie_differentially_2018}& Utility, Privacy &   \cite{dp-GAN-url}  \\

5 & 2018 & mc-medGAN & \begin{tabular}[c]{@{}l@{}}\cite{camino_generating_2018}\end{tabular}    & Utility &   \cite{mcmedgan-url}\\

6 & 2018 & TGAN &  \cite{xu_synthesizing_2018}& Utility &  \cite{tgan-url}\\

7 & 2019 & PATE-GAN & \begin{tabular}[c]{@{}l@{}}\cite{jordon_pate-gan_2019}\end{tabular}    & Utility, Privacy   & --\\

8 & 2019 & SPRINT-GAN & \cite{beaulieu-jones_privacy-preserving_2019}           & Utility, Privacy, Clinical &  \cite{sprint-GAN-url} \\

9 & 2019 & GAN-based &     \cite{ISI:000524576200016} & Utility, Privacy & --\\

10 & 2019 & CTGAN &  \cite{xu_modeling_2019} & Utility & \cite{CTGAN-url}\\

11 & 2019 & WGAN-DP & \begin{tabular}[c]{@{}l@{}}\cite{brenninkmeijer_generation_2019}\end{tabular} & Utility, Privacy    & \cite{WGAN-DP-url} \\

12 & 2019 & PPGAN &  \cite{liu_ppgan_2019} & Utility, Privacy &  \cite{ppgan-url}\\

13 & 2019 & medBGAN &   \cite{ISI:000502534100049} & Utility & --\\

14 & 2019 & medWGAN& \cite{medwgan}& Utility &  \cite{medWGAN-url}  \\

15 & 2020 & ADS-GAN& \begin{tabular}[c]{@{}l@{}}\cite{ISI:000557358500024}\end{tabular} & Utility, Privacy& --\\

16 & 2020 & corGAN & \cite{2001.09346}& Utility, Privacy &  \cite{corGAN-url}\\
17 & 2020 & CGAN &  \cite{ISI:000490706700022}& Utility& --\\

18 & 2020 & DPAutoGAN&  \cite{tantipongpipat2020differentially}& Utility, Privacy & \cite{DPAutoGAN-url}\\

19 & 2020 & GAN Boosting& \cite{2007.11934} & Utility, Privacy &  \cite{postganurl}\\

20 & 2020 & RDP-CGAN&  \cite{rdp-gan}& Utility, Privacy &  \cite{rdp-cgan-url}\\

21 & 2020 & WCGAN-GP&  \cite{WCGAN-GP}& Utility, Privacy & --\\
22 & 2020 & SMOOTH-GAN &  \cite{smooth-gan} & Utility & \cite{smooth-gan-url}\\
\hline
\end{tabular}
\end{table}


The metrics the authors used are exhibited in table \ref{tab:results_review}. 



\begin{landscape}
\renewcommand{\arraystretch}{1.02} %add more space between cells (1.2 is a factor)
\begin{table}[htbp]
\caption{Metrics utilised for evaluation} \label{tab:results_review} 

\begin{tabular}{p{26mm} p{84mm} p{60mm}}
\hline
Acronym & Utility & Privacy \\
\hline
medGAN	& \begin{enumerate*}
    \item Bern.
    \item Pred F1
\end{enumerate*} & \begin{enumerate*}
\item Attrib. disc. \item Memb. inf. \item KNN	\end{enumerate*}  \\

%\arrayrulecolor{mygrey}\hline

POSTER &	\begin{enumerate*}
\item Pred Acc.
\item Corre. Mat. \item BD
\end{enumerate*} & DP\\
table-GAN &	\begin{enumerate*}
\item Cumul. Dist.
\item Pred F1$\vert$MRE
 \end{enumerate*} &	\begin{enumerate*} \item Eucl.
\item Member. inf.   \end{enumerate*}  \\

dp-GAN &	\begin{enumerate*} \item Pred AUC \item Bern. \end{enumerate*} &	DP	 \\

mc-medGAN & 	\begin{enumerate*} \item Pred F1$\vert$AUC \item Bern. \item ME F1$\vert$Acc\end{enumerate*} & -- 	\\

TGAN &	\begin{enumerate*} \item KNN \item NMI \item Pred F1 \end{enumerate*} & --\\

PATE-GAN & \begin{enumerate*} \item  Pred AUC$\vert$AUPRC \end{enumerate*}	& DP \\
SPRINT-GAN & \begin{enumerate*}	\item Pred AUC \item Corre. Mat. \end{enumerate*} &	DP \\

GAN-based &	  \begin{enumerate*} \item Pred Acc. \item Corre. Mat. \end{enumerate*} & \begin{enumerate*} \item Hit. Rate \item R. Linkage  \item Eucl. \end{enumerate*} \\

CTGAN &	\begin{enumerate*} \item Pred F1$\vert$R2$\vert$Acc. \end{enumerate*} &  --\\

WGAN-DP &	\begin{enumerate*} \item Corre. Mat.
 \item PCA \item  Pearson RMSE\newline\item Pred F1$\vert$RMSE$\vert$1-MAPE(F1)  \end{enumerate*}  & \begin{enumerate*} \item Eucl. \item Dupl. \item DP \end{enumerate*}	\\

PPGAN &	\begin{enumerate*} \item GS \end{enumerate*}	& DP \\

medBGAN	& \begin{enumerate*}  \item Assoc. Rul.
 \item  CCS Pred F1
\item KS
 \end{enumerate*}	& -- \\

medWGAN	& \begin{enumerate*} \item Assoc. Rul.\item  CCS Pred F1 \item KS \end{enumerate*} & -- \\

ADS-GAN &  \begin{enumerate*} \item $\chi^{2}$ \item JSD \item WD
\item t-test \item Pred AUROC\newline
\item Corre. Mat. \end{enumerate*}	& DP \\

CorGAN &	\begin{enumerate*} \item Pred F1 \item Bern. \end{enumerate*}
 &	Member. Inf. \\

CGAN &	\begin{enumerate*} \item Pearson
\item Spearman \item Pred F1$\vert$AUC$\vert$Acc \end{enumerate*}  &	-- \\

DPAutoGAN & \begin{enumerate*} \item Pred AUROC$\vert$R2  \item Bern. \end{enumerate*}	
 &	DP \\
GAN Boosting & \begin{enumerate*} \item pRMSE \item Pred AUROC$\vert$AUPRC$\vert$Acc. \end{enumerate*}	
	& DP \\
RDP-CGAN & \begin{enumerate*} \item Pred F1$\vert$AUROC$\vert$AUPRC \item MMD  \end{enumerate*}	
	& DP \\
WCGAN-GP & \begin{enumerate*} \item Corre. Mat.
 \item Pred F1  \end{enumerate*} & \begin{enumerate*} \item Dupl.
\item Eucl. \end{enumerate*} \\
SMOOTH-GAN & \begin{enumerate*} \item DW MAE \item Pearson  \item Pred AUROC$\vert$AUPRC \end{enumerate*}	
	& -- \\
\hline

\end{tabular}
\end{table}
\end{landscape}


Regarding privacy, 15 papers assessed it or included some kind of mechanism to improve data protection. The most common was including Differential Privacy (DP) in the generation process. Other mechanisms for measuring privacy loss were Membership Inference (Member. Inf.), Attributes Disclosure (Attrib. Disc.), Euclidean distance (Eucl.), record-linkage (R. Linkage) and \ac{knn}.
As for utility, all papers assessed it. There were 3 major areas of utility assessment: Dimension-wise (DW) probability, cross-testing, and distance metrics. The most basic one was dimension-wise probability, which is important for making sanity checks for the generated data, comparing the distributions of each column between real and synthetic. In this category, we can find Bernoulli (Bern.), cumulative distributions (Cumul. Dist.), Pearson correlation (Pearson) and Spearman correlation (Spearman), correlation coefficients (CCS), chi-squared test ($\chi^{2}$),  \ac{ks} or Correlation Matrices (Corre. Mat.).
Cross-testing was about training machine-learning algorithms with both datasets in order to compare the results. The key factor is generating a synthetic dataset based on the training set and then training models on the original training set and the generated dataset. Then the models are compared regarding their predictive capability on the (real) test set. This was a way of assessing if the generator models were capturing inter-variable relationships. The authors applied different metrics from AUC, F1, \ac{auprc}, Accuracy (Acc.) to \ac{mre}. Finally, there was also the application of distance metrics, for measuring the difference between column distribution in both datasets. \acl{jsd}, Wasserstein Distance (WD), Bhattacharyya Distance (BD) or Generate Scores (GS) that was a metric implemented by the authors of \cite{liu_ppgan_2019} that creates a metric based on the sum of the mean of \textit{kullblack-leibler}  distance of all columns. Other less used methods were \ac{pca}, propensity score mean squared error ratio (pMSE). NMI (Normalised Mutual Information), which is the ability to capture correlations between columns by computing the pairwise mutual information and MMD (Maximum Mean Discrepancy), which is similar to distance metrics were also used. Regarding datasets utilized, the most used was MIMIC-III \cite{mimiciii} (9 times). The papers used 27 different datasets, being 16 healthcare-related and 11 non-healthcare related. 
Finally, regarding clinical evaluation, only two papers assessed it, as it is possible to see in table \ref{tab:review_gans}. Both had a group of clinicians assessing a sample of both real and synthetic information and evaluating from 0 to 10, where 10 is the most realistic.
One major point preventing a larger comparison is that despite some papers using the same dataset and same methodologies, the presented values are different, making it difficult for a clear comparison of results. One example is a dimension-wise prediction with F1 score for MIMIC-III. CorGAN presents the mean difference between the two classifications (real on real and synthetic on synthetic), while medBGAN presents the correlation coefficients of the two, and medGAN only presents the visual comparisons. Regarding code availability, 16 papers had the code publicly available in some form. As of January 2021, papers pointed in table \ref{tab:review_gans} have public code.

%\begin{figure}[h]
%\centering
%\includegraphics[scale=0.50]{Paper_plot.png}
%\includegraphics[scale=0.54]{Rplot01.jpeg}

%\caption{Datasets used for measuring Utility and Privacy.} \label{fig2}
%\end{figure}



\subsection{Implications for future research}
From the work done in this paper, it is clear that synthetic data generation is a growing field. The increasing number of papers through the years as the growing quality in the mechanisms of generating data and assessing its quality is clear proof. 
It also became apparent that privacy and utility in synthetic data represent a delicate balance. The very same definition of differential privacy represents it. The compromise between privacy and utility is real and should be taken into account when creating privacy-demanding datasets.
Creating statistically good tabular datasets is already possible, but that task becomes increasingly difficult if privacy concerns are added. 
However, privacy is also a complex subject, and the context of the setting is important for privacy assessment, which explains the different approaches for evaluating privacy protection of synthetic data.
From this review, we believe that a proper evaluation of synthetic data generators in the healthcare setting with privacy concerns should at least include utility and privacy evaluations. For utility, we believe that evaluating column-wise is a nice first check but insufficient alone.
For table-wise, since there is no fundamental metric for assessing the inter-column correlations between mixed-type variables, cross-testing is the best next thing. Distance metrics are nice to have and seem to have the potential for creating a table-wise metric \cite{metrics}, so presenting them is important. Second, for privacy evaluation, we believe that Differential Privacy in itself is not a guarantee of protection for real patients. More research and depth should be employed when presenting results for such generators; record-linkage and attribute disclosure can provide extra guarantees.
Thirdly, a clinical evaluation should be done as well to understand if the synthetic patients are a reality in the clinical setting. Since the correlations could be correct but clinically (or biologically) they might not make sense. Finally, in the scope of this paper, only \acp{gan} were assessed, but there are more mechanisms for generating data, and could be interesting to assess how all of them perform on the same datasets. There are other methods for handling the mixed data types that regularly appear in clinical settings, like \acp{vae} Gaussian Mixtures, \ac{bn}, and imputation mechanisms, making them excellent candidates for this assessment.



\subsection{Conclusion}
In this paper, we had the opportunity to survey the current framework for generating tabular data using \acp{gan} and which ones were already tested in the healthcare setting. We summarised the utility and privacy metrics employed, and the datasets used to measure them. We analysed the code availability and made suggestions for further work on cataloguing, comparing, and assessing synthetic health data generators. A survey with a global benchmark of methodologies, despite being arduous, could yield great results for the community and take the aim of this paper further.

\section{Pulling the current metrics of assessing datasets}\label{subsec:tabular}
This section is based on the paper entitled "Dataset Comparison Tool: Utility and Privacy". This work followed the work on section \ref{subsec:gans}, where we compiled ways of assessing the utility of synthetic data. We understood that the mechanisms were far from consensual and a tool could be of use to merge all of this into a single file and report about data. Our purpose was to facilitate health data owners and legal responsible to understand how similar and protective a dataset was regarding the original one.


\subsubsection{Introduction}
Synthetic data can be defined as data that has no connection with a real-world phenomenon or event. It was not originated from a process in the real world, but rather a synthetic one. The idea is that synthetic data can have similar properties with real data, without needing to have an independent process for its generation.
Synthetic data has been used over the years for several usages, but in healthcare is still not very used. However, this scenario seems to be changing. It can be used for several use cases namely \cite{synthetic-data-usage}; i) Software testing, ii) educational purposes, iii) \ac{ml}, iv) regulatory, v) retention, vi) secondary and vii) enhanced privacy.

Software testing relates to using synthetic data to create use cases for software testing. This can be used for the development or pre-production stages for example. Often the data needed is not available on-demand and a synthetic generator of reliable data could be useful. Educational purposes relate to, at least, two different scenarios. One is for onboarding of employees \cite{synthetic-data-usage}, other is related to healthcare students for using health information systems and creating mechanisms for providing reliable data on-demand.
\ac{ml} is one of the areas where synthetic data has more widespread usage, where data augmentation through data synthesis is already common. It can be used for class imbalance, sample-size boosting or machine-learning algorithms testing. Regulatory purposes could be important as well, with the rise of \ac{ai} as medical device systems and synthetic data could be used to properly evaluate these systems under controlled environments. Retention can be an important case for synthetic data as well, since personal data must not be kept more than it would be necessary. Synthetic data generators can be of use, where the original data can be deleted and a generator kept for further usage, given that the privacy mechanisms are properly employed. Secondary uses relate to using synthetic data to share data with academia or industry. Simulacrum \cite{simulacrum} is a nice example of how the NHS uses these mechanisms to help scientists get a better grasp of data before having to fill documentation to query the real data. The same occurs for \ac{IKNL}, which has a synthetic version of the cancer registry for scientific purposes \cite{synthetic_2} and the \ac{MHRA} that uses synthetic data as well for its CPRD real-word evidence \cite{mhra_cprd}.

Finally, an aspect that is underlying to all these applications is the promise that synthetic data can be used to improve privacy. Even though specially tweaked data generators can be used to create more privacy-aware datasets, it will be inherently always at the cost of some utility \cite{stadler_synthetic_2020}. So, even though synthetic data it is not the silver bullet as primarily thought, synthetic data generation can be undeniably used to help create more private data for all the use cases seen above, at the cost of its utility. As for proper methods of evaluating security and utility, are, for now, open research questions. At the present time, it is still complicated to properly assess the utility of the generated data. We have qualitative and quantitative methods. Qualitative methods are related to plots, quantitative are related to some value that defines a evaluation metric. These quantitative metrics can be applied to equal columns from each data set, pair of column from each dataset or applied over the whole datasets. As for privacy metrics, the metrics rely on duplicates. Full duplicates or membership inference related. 

So in this paper, we developed a data pipeline for data analysis in order to create a report for providing several metrics for data utility and privacy.


\subsubsection{Methods}

The pipeline relies on python and latex for creating the document. It relies also on several packages that implemented methods for evaluating data, namely scipy \cite{scipy}, sdmetrics \cite{sdv} and scikit-learn \cite{scikit-learn} and mlxtend \cite{mlxtend}. Its basis is related to uploading 2 datasets, and a report in pdf is produced. Being that is based on programmatic code, it can be easily converted into API.
The report has a section for dataset description, columns removed due to high-null and brief variable overview. Then a null comparison is done by column and dataset. Following this is the utility subsection. Firstly by visual methodologies: heat maps for the correlation, bar plots for categorical, density plots for continuous and a pair plot for an overview. As for the quantitative utility evaluation, we divided it column-wise, pair-wise and table-wise. The first comprehends the \textit{Kolmogorov–Smirnov} test for continuous and $\chi^2$ test for categorical variables.  Distance metrics were also applied to categorical columns. First, they are transformed into distributions and then distance metrics are applied. The results is a descriptive overview of the distance metrics, having minimum value, average, max value and standard deviation. The distance metrics selected were \textit{Jensen-Shannon Divergence}, \textit{Wasserstein distance}, \textit{Kullback–Leibler divergence} and entropy.
As for pair-wise metrics, we used a discrete and continuous \textit{Kullblack-Leibler divergence}. In this, two pairs of continuous columns are compared using \textit{Kullback–Leibler divergence}. For this, they are put into bins for further application. The same is applied to categorical columns without binning.
As for table-wise metrics, first, we used likelihood metrics. We fitted several Gaussian Mixture models or Bayesian network models to the real data and then calculate the likelihood of the synthetic data belonging to the same distribution. The metrics are likelihood for Gaussian mixture and Bayesian models and log-likelihood for the Bayesian model as well.

\begin{center}
\begin{table}[htpb]
    \caption{Metrics Assessed}
    \label{tab:variables}
\begin{tabular}{@{}lll@{}}   \toprule
Metric                       & Method       & Context         \\\midrule
Bar Plot                     & visual       & utility         \\ 
KDE Plot                     & visual       & utility         \\ 
Heat-map                     & visual       & utility         \\ 
Pair-plot                    & visual       & utility         \\ 
KS test                      & column-quantitative & utility         \\ 
ChiSquared Test              & column-quantitative & utility         \\ 
Kullback–Leibler divergence  & column-quantitative & utility         \\ 
Jensen-Shannon Divergence    & column-quantitative & utility         \\ 
Wasserstein distance         & column-quantitative & utility         \\ 
Entropy                      & column-quantitative & utility         \\ 
DiscKLD                      & table-quantitative & utility         \\ 
ContinuousKLD                & table-quantitative & utility         \\ 
BNLikelihood                 & table-quantitative & utility         \\ 
BNLogLikelihood              & table-quantitative & utility         \\ 
GMLogLikelihood              & table-quantitative & utility         \\ 
Same dataset ratio           & table-quantitative & utility         \\ 
Support rules                & table-quantitative & utility         \\ 
Different dataset validation & table-quantitative & utility         \\ 
Duplicates                   & quantitative & privacy         \\ 
Duplicate minus 1            & quantitative & privacy         \\ 
Record Linkage             & quantitative & privacy         \\ 
Matrix distance              & quantitative & privacy/utility  \\ 
Cosine distance              & quantitative & privacy/utility \\ 
Euclidean distance           & quantitative & privacy/utility \\ \bottomrule

\end{tabular}
\end{table}
\end{center}


Then we used machine-learning models (linear regression and decision trees) to assess how similar models behave on both datasets. First, we tested on the same dataset in order to compare evaluation metrics. Then we cross-tested, meaning that the training set was drawn from one dataset and the test set was drawn from the second dataset. Finally, data privacy constraints duplicate evaluation, duplicate existence by removal of a single column and a record linkage approach. With the record linkage, we define a record linkage blocking ("age" in the example) and then try to match rows from the synthetic dataset to the real, with varying known attributes. Then matrix, euclidean and cosine distance was also calculated. Even though it is used for privacy evaluation, by definition, we could also use it for utility assessment. For proper assessment, the continuous and categorical variables should be indicated at the start of the code. The metrics are listed in the table \ref{tab:variables}.




%class GMLogLikelihood(SingleTableMetric):
%    """GaussianMixture Single Table metric.
 %   This metric fits multiple GaussianMixture models to the real data and then
  %  evaluates how likely it is that the synthetic data belongs to the same
%    distribution as the real data.


%class BNLikelihood(SingleTableMetric):
%    """BayesianNetwork Likelihood Single Table metric.
%    This metric fits a BayesianNetwork to the real data and then evaluates how
%    likely it is that the synthetic data belongs to the same distribution.
%    The output is the average probability across all the synthetic rows.


%class BNLogLikelihood(BNLikelihood):
%    """BayesianNetwork Log Likelihood Single Table metric.
%    This metric fits a BayesianNetwork to the real data and then evaluates how
%    likely it is that the synthetic data belongs to the same distribution.
%    The output is the average log probability across all the synthetic rows.
    
%CSTest: Chi-Squared test to compare the distributions of two categorical columns.
%KSTest: Kolmogorov-Smirnov test to compare the distributions of two numerical columns %using their empirical CDF.


%
%class ContinuousKLDivergence(ColumnPairsMetric):
%	 """Continuous Kullback–Leibler Divergence based metric.%
	
%	 This approximates the KL divergence by binning the continuous values
%	 to turn them into categorical values and then computing the relative
%	 entropy. Afterwards normalizes the value applying ``1 / (1 + KLD)``.




\subsubsection{Results}
A trial example of comparing data is available for data in the UCI repository, namely the heart disease dataset \cite{misc_heart_disease_45}. The synthetic data was created by using the synthpop package \cite{synthpop}. The variables evaluated are listed in table below. The code can be seen in https://github.com/joofio/dataset-comparasion-report. As an example. The image for visual analysis for categorical (figure \ref{fig:catgorical}) and continuous variables (figure \ref{fig:continuous}).


\begin{figure}[H]
    \centering
    \includegraphics[scale=0.23]{figures/cat.png}
    \caption{Categorical Variables plotted}
    \label{fig:catgorical}
\end{figure}

\begin{figure}[t]
    \centering
    \includegraphics[width=\textwidth]{figures/continuous_plot_0.png}
    \caption{Continuous Variables plotted}
    \label{fig:continuous}
\end{figure}



\subsubsection{Discussion \& Conclusion}

The data possible to create to evaluate similarities between two datasets is important not only for synthetic vs real datasets. For example in distributed learning, where different silos exist, with similar or even equal features, a method for evaluating the similarities can be useful for understanding how the populations are similar between them, trying to shed light on the most similar among them, or different in order to understand the differences in the silos or data acquisition inside them.
Furthermore, the differences can be assessed on a more granular level. The column-wise similarities can be different from the inter-columns similarities and this in itself, can be a metric of interest regarding the quality of the synthetic data and its generator.

With this work, we hope to help institutions and academics getting to access to a benchmark of the datasets provided in order to leverage synthetic data in the healthcare space. Finally, we hope this work helps other researchers reach an evaluation metric that could be a unique and clear response to the question of how similar two datasets are.




\section{Can we use machine learning feature to compare datasets?}\label{subsec:similarity}
This section is based on the paper entitled "Using Machine Learning Models' feature importance to assess dataset similarity". The reasoning behind this paper was the results of \ref*{subsec:gans}, where we felt that evaluation metrics for synthetic data could be improved. Better yet, we felt that the comparison of two datasets (that shared the same columns) could be done in a more robust way. Being that the current gold-standard was cross-validation which was not bound to any number range and the significance of the result could not be easily interpretable. We used the feature importance of several \ac{ml} models to compare datasets and concluded that it was a valid alternative to the traditional metrics.

\subsection{Introduction}
% !TeX root = ../../thesis.tex

In recent years, the use of \ac{ai} and \ac{ml} algorithms has gained increasing prominence in healthcare research and practice. One of the key requirements for the successful application of these methods is access to large, high-quality datasets. However, in many cases, the availability of such datasets can be limited due to issues around data privacy, security, and ethical concerns \cite{chingOpportunitiesObstaclesDeep2018a}. To address this challenge, synthetic data has emerged as a promising solution. Synthetic data refers to artificially generated data that closely mimic the statistical properties and patterns of real-world data \cite{mullerEvaluationSyntheticElectronic2022}.

Synthetic data has the potential to overcome many of the limitations associated with real-world data, such as the lack of sufficient data volume, noise, and privacy concerns. Even though there are still doubts if the privacy part is the silver bullet sometimes referred to \cite{stadlerSyntheticDataPrivacy2020}, the upsampling part is a standard use for years now. However, the quality of synthetic data generated by various techniques can vary significantly, and it is essential to assess the quality of synthetic data before its usage. In healthcare, the assessment of synthetic data is crucial to ensure that it can provide valid insights and inform decision-making processes.

The assessment of synthetic data in healthcare is essential for its successful use in various applications, such as developing predictive models, testing algorithms, and conducting clinical trials. The use of synthetic data can significantly enhance the efficiency and effectiveness of healthcare research and practice. However, it is crucial to ensure that the synthetic data used in these applications are of high quality and validated to provide reliable and valid insights. The evaluation of synthetic data quality involves comparing its statistical properties and patterns with those of the original data. We can assess how similar columns are to each other through several statistical tests, and then we can infer some inter-column properties with methods like cross-validation, where two datasets are split into train tests and cross-tested and then the ratio between the evaluation result of both datasets is used as a metric \cite{mullerEvaluationSyntheticElectronic2022,goncalvesGenerationEvaluationSynthetic2020a}. However, this methodology is a big proxy for such an inter-column relationship. Can we try to provide a better metric than this one to evaluate how similar are the inter-column relationship of two distinct datasets? In this paper, we suggest using feature importance values to create a more explainable and reasonable metric for inter-column relationships.

\subsection{Rationale and Related Work}
% !TeX root = ../../thesis.tex

Recently there has been a series of works related to assessing how synthetic data generators behave with data like the work of Emam et al. \cite{emamUtilityMetricsEvaluating2022} that especially focused on utility metrics for synthetic data generators. At the moment, comparing data is based on intra-columns and inter-columns relationship. The intracolumn relationship is assumed as something that compares equal columns between datasets, with highly known statistical methods like chi-squared or \ac{ks} like done in the works of \cite{combrinkComparingSyntheticTabular2022} among many others, acting more like sanity checks than anything else. 
Other known metrics are distance-based metrics like \ac{jsd}, Wasserstein Distance, Bhattacharyya Distance or Hellinger distance, which are based on the calculation of the distance between distributions like seen in the works of several teams \cite{ISI:000557358500024,choiGeneratingMultilabelDiscrete2017,Baowaly2019}.

However, regarding inter-column relationships, the metrics applied are often very different across papers. One example of trying to capture inter-column relationship is about the use of propensity score \cite{rosenbaumCentralRolePropensity1983,mullerEvaluationSyntheticElectronic2022} where a classifier is trained to the merged datasets, with the added variable of the original dataset (i.e., 1 for real and 0 for synthetic). The model is trained and the propensity Mean square error is the  mean squared difference of the estimated probability from the average prediction
Most recently, a unified metric appeared as the sum of other metrics known as described in the work of Chundawat et al., \cite{chundawatTabSynDexUniversalMetric2022}, known as TabSynDex. Other examples are likelihood of fitness like in the works of \cite{xuModelingTabularData2019b}, coverage support \cite{goncalvesGenerationEvaluationSynthetic2020a} or very specific metrics implemented for evaluating specific data generators.
However, the most used metric is cross-validation, which takes two datasets, one that is real and a second which is synthetic and we split both into train and test and train a machine learning model on the real data training set, then we test the model on both test sets. Then a ratio is created, rendering the actual value. This methodology, even if gold-standard at the moment for this type of study, has some liabilities since this value can be a bit erratic, and even above one since the evaluation metric could be better on the second dataset and we don't have a clear grasp of what that can represent in terms of dataset similarity. The image \ref{fig:1} represents this in detail. Several works used this metric as the comparing metric \cite{mullerEvaluationSyntheticElectronic2022}.

%TC:ignore
\begin{figure}[tbph]
\centering
\caption{Cross-Validation of datasets}\label{fig:1} 
\includegraphics[scale=0.35]{figures/imagem1.pdf}
\end{figure}
%TC:endignore




%Utility Metrics for Evaluating Synthetic Health Data Generation Methods: Validation Study



\subsection{Materials \& Methods}
\subsubsection{Method Overview}
% !TeX root = ../../thesis.tex

For this work, our goal is to test several metrics based on the ranking of feature importance of a trained model. Normalized Discounted Cumulative Gain (NDCG) \cite{wangTheoreticalAnalysisNDCG} which is the sum of the true scores ranked in the order induced by the predicted scores, after applying a logarithmic discount. Then divide by the best possible score to obtain a score between 0 and 1. It is calculated by
\[
\text{{NDGC}} = \frac{{\text{{DCG}}(P)}}{{\text{{IDCG}}(P)}}
\]
where $\text{{DCG}}(P)$ is the Discounted Cumulative Gain and $\text{{IDCG}}(P)$ is the Ideal Discounted Cumulative Gain. 

\begin{small}    
    \begin{table}[H]
        \footnotesize
            \caption{Descriptive statistics of datasets used. Mean (Standard Deviation) for continuous variables. Mode [nr categories] for categorical variables.}\label{tab:descrptive_feature}
                \begin{tabularx}{\textwidth}{lXll|lXll}
                    \toprule
                Dataset & Column                         & Statistic    & \% Nulls & Dataset & Column          & Statistic    & \% Nulls \\
                \midrule
                heart   & Age                            & 54.4 (9.0)   & 0.0      & liver   & gammagt         & 38.3 (39.3)  & 0.0      \\
                heart   & sex                            & 1.0 {[}2{]}  & 0.0      & liver   & drinks          & 3.5 (3.3)    & 0.0      \\
                heart   & cp                             & 4.0 {[}4{]}  & 0.0      & liver   & Selector        & 2 {[}2{]}    & 0.0      \\
                heart   & trestbps                       & 131.7 (17.6) & 0.0      & thyroid & Class           & 1 {[}3{]}    & 0.0      \\
                heart   & chol                           & 246.7 (51.8) & 0.0      & thyroid & T3              & 109.6 (13.1) & 0.0      \\
                heart   & fbs                            & 0.0 {[}2{]}  & 0.0      & thyroid & TST             & 9.8 (4.7)    & 0.0      \\
                heart   & restecg                        & 0.0 {[}3{]}  & 0.0      & thyroid & TSTRI           & 2.1 (1.4)    & 0.0      \\
                heart   & thalach                        & 149.6 (22.9) & 0.0      & thyroid & TSH             & 2.9 (6.1)    & 0.0      \\
                heart   & exang                          & 0.0 {[}2{]}  & 0.0      & thyroid & TMAX            & 4.2 (8.1)    & 0.0      \\
                heart   & oldpeak                        & 1.0 (1.2)    & 0.0      & tumour   & class           & 1 {[}21{]}   & 0.0      \\
                heart   & slope                          & 1.0 {[}3{]}  & 0.0      & tumour   & age             & 2 {[}3{]}    & 0.0      \\
                heart   & ca                             & 0.0 {[}4{]}  & 1.3      & tumour   & sex             & 2 {[}2{]}    & 0.3      \\
                heart   & thal                           & 3.0 {[}3{]}  & 0.7      & tumour   & histologic-type & 2 {[}3{]}    & 19.8     \\
                heart   & num                            & 0 {[}5{]}    & 0.0      & tumour   & degree-of-diffe & 3 {[}3{]}    & 45.7     \\
                breast  & Clump Thickness               & 4.4 (2.8)    & 0.0      & tumour   & bone            & 2 {[}2{]}    & 0.0      \\
                breast  & Uniformity of Cell Size     & 3.1 (3.1)    & 0.0      & tumour   & bone-marrow     & 2 {[}2{]}    & 0.0      \\
                breast  & Uniformity of Cell Shape    & 3.2 (3.0)    & 0.0      & tumour   & lung            & 2 {[}2{]}    & 0.0      \\
                breast  & Marginal Adhesion             & 2.8 (2.9)    & 0.0      & tumour   & pleura          & 2 {[}2{]}    & 0.0      \\
                breast  & Single Epithelial Cell Size & 3.2 (2.2)    & 0.0      & tumour   & peritoneum      & 2 {[}2{]}    & 0.0      \\
                breast  & Bare Nuclei                   & 3.5 (3.6)    & 2.3      & tumour   & liver           & 2 {[}2{]}    & 0.0      \\
                breast  & Bland Chromatin               & 3.4 (2.4)    & 0.0      & tumour   & brain           & 2 {[}2{]}    & 0.0      \\
                breast  & Normal Nucleoli               & 2.9 (3.1)    & 0.0      & tumour   & skin            & 2 {[}2{]}    & 0.3      \\
                breast  & Mitoses                        & 1.6 (1.7)    & 0.0      & tumour   & neck            & 2 {[}2{]}    & 0.0      \\
                breast  & Class                          & 2 {[}2{]}    & 0.0      & tumour   & supraclavicular & 2 {[}2{]}    & 0.0      \\
                liver   & mcv                            & 90.2 (4.4)   & 0.0      & tumour   & axillar         & 2 {[}2{]}    & 0.3      \\
                liver   & alkphos                        & 69.9 (18.3)  & 0.0      & tumour   & mediastinum     & 2 {[}2{]}    & 0.0      \\
                liver   & sgpt                           & 30.4 (19.5)  & 0.0      & tumour   & abdominal       & 2 {[}2{]}    & 0.0      \\
                liver   & sgot                           & 24.6 (10.1)  & 0.0      &         &                 &              &          \\
                  \bottomrule
                \end{tabularx}
    
            \end{table}
        \end{small}
Cohen's kappa coefficient \cite{doi:10.1177/001316446002000104}  is a statistic that is commonly used to assess the level of agreement between two or more raters or evaluators who are providing categorical ratings or rankings of a set of items. So, we want to use to assess if it could be of use to check how similar the ranking of the features is, using the numbers as categorical.
\[
\kappa = \frac{{P_o - P_e}}{{1 - P_e}}
\]

where \(P_o\) is the observed agreement between the two raters and \(P_e\) is the expected agreement between the two raters by chance.



We also intend to use the $R^2$ to check if the explainability changes across datasets.
\[
R^2 = 1 - \frac{{\sum_{i=1}^n (y_i - \hat{y}_i)^2}}{{\sum_{i=1}^n (y_i - \bar{y})^2}}
\]

where \(y_i\) are the observed values of the dependent variable, \(\hat{y}_i\) are the predicted values of the dependent variable, \(\bar{y}\) is the mean of the observed values of the dependent variable and \(n\) is the number of data points.

Then we intend to use ranking metrics, namely Kendall tau, weighted Kendall tau and \ac{rbo}.
Kendall tau is a measure of correlation that measures the similarity between two rankings. It is commonly used in statistics and data analysis to evaluate the agreement or disagreement between two sets of rankings.

The Kendall tau coefficient \cite{kendallTreatmentTiesRanking1945} is defined as the difference between the number of concordant and discordant pairs of observations, divided by the total number of pairs. A concordant pair is a pair of observations that have the same ranking order in both sets, while a discordant pair is a pair of observations that have opposite ranking orders. The Kendall tau coefficient ranges from -1 to 1, where -1 represents perfect negative correlation, 0 represents no correlation, and 1 represents perfect positive correlation. 
\[
\tau = \frac{{\text{{number of concordant pairs}} - \text{{number of discordant pairs}}}}{{\text{{total number of pairs}}}}
\]


Weighted Kendall tau  \cite{vignaWeightedCorrelationIndex2015} is an extension of Kendall tau that takes into account the importance or weight of each observation in the rankings. In some cases, some observations may be more important than others, and their positions in the ranking may have a greater impact on the overall correlation. Weighted Kendall tau assigns a weight to each observation, and the correlation is calculated based on the weighted concordant and discordant pairs.
\[
\tau_w = \frac{{\sum_{i<j} w_{ij} \cdot sgn(x_i - x_j)}}{{\sum_{i<j} w_{ij}}}
\]
where $w_{ij}$ is the weight associated with the pair $(x_i, x_j)$ and $sgn(\cdot)$ is the sign function.

The \ac{rbo} \cite{webberSimilarityMeasureIndefinite2010} is a measure used to compare the similarity of two ranked lists, especially when these lists are of different lengths or have only partial overlap. The \ac{rbo} value ranges from 0 (no overlap) to 1 (complete agreement). One of the key features of \ac{rbo} is that it gives more weight to the top-ranked items. The formula for \ac{rbo} at a given depth \( d \) is as follows:

\[
RBO_d = (1 - p) \sum_{k=1}^{d} \left[ p^{k-1} \cdot \frac{|S_{k} \cap T_{k}|}{k} \right]
\]

Where  \( S_{k} \) and \( T_{k} \) are the sets of elements in the top \( k \) positions of the two ranked lists \( S \) and \( T \) respectively, \( |S_{k} \cap T_{k}| \) is the size of the intersection of these top \( k \) elements, \( p \) is a persistence parameter (usually between 0 and 1) that determines the weight given to the rankings at different depths. A lower value of \( p \) gives more weight to the top-ranked items and \( d \) is the depth to which you are calculating the \ac{rbo}, which can be up to the length of the longest list. This formula calculates the \ac{rbo} up to a finite depth \( d \). The parameter \( p \) is crucial as it models the user's persistence in considering the rankings down the list. The higher the value of \( p \), the more the metric considers items further down the list. 


Finally, we intend to use text-distance metrics. The theory behind this experiment is to treat the ordered columns in a ranking manner and apply text-distance metrics to check the distance between the two. Levenshtein distance \cite{navarroGuidedTourApproximate2001} is the minimum number of single-character insertions, deletions, or substitutions required to transform one string into another. Damerau-Levenshtein distance \cite{navarroGuidedTourApproximate2001} is similar to Levenshtein distance but also includes the transposition of two adjacent characters as an allowable operation. The hamming distance \cite{6772729} is a measure of the difference between two strings of equal length, defined as the number of positions at which the corresponding symbols are different. Jaro-Winkler distance \cite{navarroGuidedTourApproximate2001} is a string similarity measure that takes into account the number of matching characters, the number of transpositions, and the length of common prefixes, with a higher weight given to the common prefix.




\begin{algorithm}[hbtp]
\small
\SetAlgoLined

\For {dataset in datasets list}{
    create two copies of dataset, one that remains the same and a second to be disturbed with permutation \\
\For {ML algorithms in ML algorithms}{
\For {i in number of columns to test}{
\For {rep in  10 repetitions}{ 
create second dataset by permutating values in i columns in the first dataset
\For {target in dataset columns}{
\begin{itemize}
    \item Train-Test Split (95:5) for both
    \item model fit to train for both
    \item get feature importance per column for both
    \item Create an ordered rank of features for both
    \item Create new metrics values by comparing the results from both
    \item make cross-classification to compare with feature importance metrics
    \item aggregate results per metric 
\end{itemize}

 }
 }}}}

 \caption{Testing similarity scores in tabular datasets. Dataset list is the 5 datasets used in this work. \Ac{ml} algorithms are the 6 algorithms used in this work. Number of columns to test is the number of columns in the dataset. 10 repetitions is the number of times the columns are permutated.}\label{alg:simil_1}
\end{algorithm}


Like seen in algorithm~\ref{alg:simil_1}, the \ac{cc} was performed several times and according to the image~\ref{fig:cross_classi1}, we trained the model on dataset1 and tested on dataset1 and 2 and compared the results. However, we applied this twice; 1) where dataset1 is the original dataset and the dataset2 is the permutated/synthetic (RS) and 2) where dataset1 is the permutated/synthetic and dataset2 is the real one or original (SR). Both are examples of \acl{cc}.

The algorithms chosen were decision trees with \textit{gini} entropy function for decision making on splits on classification and squared error for regression; random forests with 100 trees and \textit{gini} criterion for classification and squared error for regression; support vector machines with C-support Vector classification and Epsilon-Support Vector Regression, \ac{knn} with 5 neighbours and uniform weights, linear regression/logistic regression and gaussian naive bayes for classification and Bayesian ridge with 300 maximum iterations and $\alpha$ and $\lambda$ of $1e^{-6}$. All of these were used as implemented in the \textit{scikit-learn} package \cite{scikit-learn}. The hyperparameters chosen were the default ones. We felt that tuning was not necessary here to test our hypothesis, since it is based on the ratio of results.
The text distance metrics were implemented by the text-distance package \cite{orsiniumTextdistanceComputeDistance}. Kendall tau, weighted Kendall tau were used as implemented by \textit{scipy} \cite{virtanenSciPyFundamentalAlgorithms2020a} and \ac{rbo}, as implemented in \cite{chenRankbiasedOverlapRBO2023}.
The methods chosen for creating several synthetic datasets were the synthpop package \cite{synthpop} with "cart" method, which is rpart implementation of a CART model. We also used  the SDV package \cite{SDV} to leverage their implementation of the CTGAN and Gaussian Copula to create 2 more synthetic datasets to test different methodologies of synthetic data creation.







\subsubsection{Data used}
% !TeX root = ../../thesis.tex

We used 5 datasets from the \ac{uci} repository. The ones chosen were related to healthcare and were heart disease \cite{misc_heart_disease_45}, thyroid disease \cite{misc_thyroid_disease_102}, liver disorders \cite{misc_liver_disorders_60}, breast cancer \cite{misc_breast_cancer_wisconsin_diagnostic_17} and the primary tumour dataset \cite{misc_primary_tumor_83}. We made minimal preprocessing on the datasets, namely removing the missing variables by imputing the mean on continuous variables and mode on categorical.
We also created a synthetic dataset by applying the \textit{synthpop} package to this data \cite{synthpop}. With this package, all variables were synthesised using the "cart" method, which is rpart implementation of a CART model.


\subsection{Results}
% !TeX root = ../../thesis.tex

With the method described in the algorithm \ref{alg:simil_1}, we created a figure where the metrics are presented with increasingly different datasets: Figure \ref{fig:lineplot}.
%Then we compared the difference in the metric across iterations, rendering figure \ref{fig:boxplot}.

%TC:ignore
\begin{figure}[htbp]
\centering
\caption[Plot showing the variation of different metrics over increasingly changed datasets.]{Plot showing the decrease of the metric over increasingly changed datasets. The X axis represents the number of columns mutated. The Y axis represents the value of the metric and the hue represents the algorithm used to calculate the metric.}\label{fig:lineplot} 
\includegraphics[scale=0.37]{figures/multiple_datasets.png}
\end{figure}
%TC:endignore

The number of repetitions and how that impacts the variance of the scores is shown in Figure~\ref{fig:facet_plot}.


%TC:ignore
\begin{figure}[htbp]
    \centering
    \caption{Heatmap showing the variance of different repetitions for every metric and the number of different columns changed. X is the metric. Y is the number of repetitions and the number of columns. This was obtained by getting the variance of all values from all datasets.  }\label{fig:facet_plot} 
    \includegraphics[scale=0.60]{figures/heatmap-runs.png}
    \end{figure}
    %TC:endignore

As for the test for the synthetic and real dataset, the results are displayed in figure~\ref{fig:synth_result} and ~\ref{fig:synth_heat}. This is the metrics distribution for the comparison of the 5 mentioned datasets and the synthetic counterpart generated as stated in the methods section.
    %TC:ignore
\begin{figure}[htbp]
    \centering
    \caption{Distributions of the metrics results comparing 5 synthetic and real datasets across 3 different generation methods}\label{fig:synth_result} 
    \includegraphics[scale=0.60]{figures/synthetic_violin_swarm_colored_by_group_custom.png}
    \end{figure}
    
    \begin{figure}[htbp]
        \centering
        \caption{Values and comparison of the metrics results comparing 5 synthetic and real datasets across 3 different generation methods}\label{fig:synth_heat} 
        \includegraphics[scale=0.60]{figures/heatmap-synth.png}
        \end{figure}
    %TC:endignore
\subsection{Discussion}
% !TeX root = ../../thesis.tex

With the results found, we feel that are better alternatives to cross-validation. At least  Kendall tau, Weighted Kendall tau, and \ac{rbo} seem like alternatives to cross-validation.
Firstly, they seem to be directly connected to a difference in the dataset, secondly, they are a 0-1 metric and thirdly, variance across different iterations is also lower.
From these, the metrics based on ranking metrics seem to work best, where Kendall tau, Weighted Kendall tau and \ac{rbo} have better performance than the rest. As for the variance with the number of repetitions, we also see that the ranking-based metrics have good stability, while the cross-validation and text-based metric have higher variability with a low number of repetitions (even if low) - figure \ref{fig:facet_plot}. 
Text metrics also have a suitable performance, even though they have a drastic drop with only one column mutated (figure \ref{fig:lineplot}).


\subsection{Conclusion}
% !TeX root = ../../thesis.tex

Comparing two tabular datasets has been growing in demand in the past year mainly because of the increase in popularity of tabular data synthesis methods which have exhibited the potential in generating valuable synthetic data. However, due to the absence of a uniform metric, evaluating different methods has been inconsistent. This research proposes some alternatives for assessing synthetic tabular data's utility. \ac{rbo} seems to have the potential to capture inter-column relationships in a more consistent way than cross-classification. They could become a useful tool for comparing statistical methods of generating synthetic tabular data. Furthermore, this metric can aid in evaluating these generators' training, providing insights into improving synthetic data quality. The proposed metrics open up possibilities for future research to enhance tabular data synthesis methods and compare two datasets overall. Future research could be expanded in new comparison with others evaluation metrics, other datasets and other synthetic data generators.


\section{Data quality Metrics}\label{subsec:dq}
This section is based on the paper entitled "Development and Validation of a Data Quality Evaluation Tool in Obstetrics Real-World Data through HL7-FHIR interoperable Bayesian Networks and Expert Rules" This paper focuses on the fact that data quality is a major concern in healthcare. We developed a tool that could be used to assess the quality of data in a \ac{ehr} and provide a report on the quality of the data. We used a combination of \ac{bn} and expert rules to assess the quality of the data. Furthermore, we tested the tool on 9 real-world datasets of obstetrics \acp{ehr} and concluded that the tool was a valid alternative to the traditional methods of assessing data quality.
%\input{chapters/data-quality-paper}
\subsection{Introduction}
With the wide spreading of healthcare information systems across all contexts of healthcare practice, the production of health-related data has followed this incremental behaviour. The potential for using this data to create new clinical knowledge and push medicine further is tempting \cite{martin-sanchezBigDataMedicine2014}.
However, to correctly use the data stored in \acp{ehr}, the quality of the data must be robust enough to sustain the clinical decisions made based on this data. Data quality cannot be construed as a linear concept; it is intrinsically dependent on the context in which it is evaluated. The quality thresholds and dimensions required to classify the quality of the data depend on the purpose that we intend to use that very same data \cite{waljiElectronicHealthRecords2019}. These uses can be very distinct and have different impacts as well. For one, we can use data to support day-to-day decisions regarding individual patients’ care \cite{verheijPossibleSourcesBias2018}. These decisions can include ones based on recorded information to understand a patient’s history, clinical decision support systems based on this data, or even using the data to help support a more macro, public health-oriented decision. Another area is using information for management purposes. The data can be used by management bodies and regulatory authorities to extract metrics regarding the quality of care or reimbursement purposes. Thirdly, data can be used for research purposes, namely observational studies and, more recently, to support clinical trials through real-world evidence analysis \cite{coreyAssessingQualitySurgical2020,verheijPossibleSourcesBias2018,wengClinicalDataQuality2020}. 
So, all the \ac{ehr} data-based decisions can only be as good as the data supporting them. Several studies have already warned about the lack of data quality in \acp{ehr} and how this can be a significant hurdle to an accurate representation of the population and potentially lead to erroneous healthcare decisions \cite{reimerDataQualityAssessment2016a,joukesImpactElectronicPaperBased2019a,huserMultisiteEvaluationData2016,zhangUnderstandingDetectingDefects2020,kramerImpactDataQuality2021,gigantiImpactDataQuality2019}.

There are several steps in the data lifecycle that can be prone to error, from data generation, where the data is registered by healthcare professionals, passing by data processing, whether inside healthcare institutions or by software engineers aiming to reuse data, to data interpretation and reuse, where investigators
try to interpret the meaning of registered data\cite{wengClinicalDataQuality2020}.
So, with all of the data’s possible uses added to the several steps that can introduce errors throughout the data lifecycle, data quality frameworks and sequential implementations can have very distinct approaches and methodologies to assess data quality. Data quality tools for checking data being registered live to support day-to-day decisions will be significantly different from one whose only purpose is to provide quality checks for research purposes. So, methodologies to tackle these issues are necessary for guaranteeing the quality of healthcare practice and the knowledge derived from \ac{ehr} data. Consequently, in this paper, we propose:
\begin{myitemize}
    \item Create a tool for identifying data quality issues in obstetrics \acp{ehr};
    \item Enlighten on the issues that can appear with a full deployment of such a tool
    \item Suggestion of a creation of a single score for data quality for comparison of high-quality and low-quality records in a database.
    \item Assess how such a tool can work in early-stage real-world scenarios and how to work with obstetricians to improve data quality.
    \item Identify data quality issues on obstetrics data
\end{myitemize}


%Data quality is a crucial aspect of the healthcare industry, as it impacts the accuracy of diagnoses, treatment plans, and patient outcomes. The reliability and accuracy of healthcare data have far-reaching consequences, including financial implications, patient safety, and legal ramifications. Inaccurate or incomplete data can lead to incorrect diagnoses, inappropriate treatments, and ultimately harm to patients. Therefore, ensuring the quality of healthcare data is essential to providing effective and safe healthcare services.

%One of the main reasons why data quality is so critical in healthcare is that healthcare data is often used to make important decisions, such as treatment plans, patient management, and resource allocation. Inaccurate or incomplete data can lead to misdiagnosis, inappropriate treatment, and increased healthcare costs. Furthermore, inaccurate data can hinder research efforts and impede the development of new treatments and therapies.

%Another key aspect of data quality in healthcare is its role in patient safety. Accurate and reliable data is essential for ensuring patient safety, particularly in areas such as medication management, clinical decision-making, and adverse event reporting. Poor data quality can lead to medication errors, adverse drug reactions, and other types of harm to patients.

%Finally, data quality is also important for legal and regulatory compliance in healthcare. Accurate and complete data is required for compliance with regulations such as HIPAA, the Affordable Care Act, and other regulatory requirements. Poor data quality can result in legal and financial penalties, as well as reputational damage for healthcare organizations.

%Overall, data quality is a crucial aspect of healthcare, with far-reaching implications for patient safety, healthcare costs, and regulatory compliance. Ensuring the quality of healthcare data requires a comprehensive approach that includes data governance, data management, data quality assurance, and ongoing monitoring and improvement efforts. By prioritizing data quality, healthcare organizations can provide better patient care, improve outcomes, and reduce costs.





\subsection{Background and Related Work}
There is already a significant number of papers trying to define data quality assessment frameworks for \ac{ehr} data, all of them plausible and recommendable, already described in other papers \cite{bianAssessingPracticeData2020}. The literature has over 20 different methods, descriptions, and summaries of  different frameworks over the years. Some may be highlighted from the review from Weiskopf et al., \cite{weiskopfMethodsDimensionsElectronic2013}, where five data quality concepts were identified over 230 papers: Completeness, Correctness, Concordance, Plausibility and Currency. 



The work of Saez et al. defined a unified set of \ac{dq} dimensions: completeness, consistency, duplicity, correctness, timeliness, spatial stability, contextualization, predictive value, and reliability \cite{saezOrganizingDataQuality2012}. Then a review of Bian et al. \cite{bianAssessingPracticeData2020} expanded on the previous ones, categorizing data quality into 14 dimensions and mapping them to the previous most known definitions. These were: currency, correctness, plausibility, completeness, concordance, comparability, conformance, flexibility, relevance, usability, security, information loss, consistency, and interpretability.

Finally, the work of Khan et al. tried to harmonize data quality assessment frameworks, which simplified all previous concepts into three main categories: Conformance, Completeness and Plausibility and two assessment contexts: Verification and Validation \cite{kahnHarmonizedDataQuality2016a}.
Despite all of these comprehensive works, there is still no consensus regarding which one is best or which has taken the lead in usage. Moreover, looking at all of the descriptions related in the literature, a significant portion of concepts are overlapping, and sometimes hard to conceptualize such dimensions in practice.

As for implementations, there are already some available, such as the work from \cite{phanAutomatedDataCleaning2020} where a tool created by primary care in the Flanders was built to assess completeness and percentage of values within the normal range.
The work from Liaw et al. \cite{liawQualityAssessmentRealworld2021} already reviewed some data quality assessment tools, like tools from OHDSI \cite{hripcsakObservationalHealthData2015} or TAQIH \cite{alvarezsanchezTAQIHToolTabular2019}. 
Additionally, we found some others with similar purposes and characteristics like the work presented data dataquieR \cite{schmidtFacilitatingHarmonizedData2021}, an R language-based package that can assess several data quality dimensions in observational health research data. 
Also, the work from Razzaghi et al. developed a methodology for assessing data quality in clinical data \cite{razzaghiDevelopingSystematicApproach2022}, taking into account the semantics of data and their meanings within their context. Furthermore, the work from Rajan et al. \cite{rajanContentAgnosticComputable2019} presented a tool that can assess data quality and characterize health data repositories. Parallel to this, Kaspner et al. created a tool called DQAStats that enables the profiling and quality assessment of the MIRACUM database, being possible to integrate into other databases as well \cite{kapsnerLinkingConsortiumWideData2021a}.

Regarding data quality assessment as a whole, the works of \cite{estiriSemisupervisedEncodingOutlier2019}, focused on outlier detection in large-scale data repositories. The works of \cite{saezEHRtemporalVariabilityDelineatingTemporal2020} focused on the exploration and identification of dataset shifts, contributing to the broad examination and repurposing of large, longitudinal data sets. The works of García-de-Léon-Chocano \cite{saStandardizedDataQuality2017,garcia-de-leon-chocanoConstructionQualityassuredInfant2016,garci;a-de-leon-chocanoConstructionQualityassuredInfant2015} are the only ones focused on obstetrics data, but aimed to improve the process of generating high quality data repositories for research and best practices monitoring. These are similar and complementary works to this one. Finally, the work of \cite{springateREHRPackageManipulating2017} focused on the manipulation of \ac{ehr} data, including data quality assessment, data cleaning, and data extraction. However, these tools are not meant to be used at the production level, assessing data as it is being registered or outputs reports for human consumption and not a quantitative metric for metric comparison. Furthermore, none of these tools had standard-based interoperability in mind. Finally, we have not seen, until the moment of this paper, any implementation that used machine learning to evaluate the correctness of the value.

\subsection{Materials}
The data was gathered from 9 different Portuguese hospitals regarding obstetric information: data from the mother, several data points about the fetus and delivery mode. The data is from 2019 to 2020. The software for collecting data was the same in every institution, and the columns were the same, even though the version of each software differed across hospitals. Across the different hospitals, data rows ranged from 2364 to 18177. The sum of all rows is 73351 rows.  The data dictionary is in appendix \ref{appendix:data_dict}. This study received Institutional Review Board approval from all hospitals included in this study with the following references: Centro Hospitalar São João; 08/2021, Centro Hospitalar Baixo Vouga; 12-03-2021, Unidade Local de Saúde de Matosinhos; 39/CES/JAS, Hospital da Senhora da Oliveira; 85/2020, Centro Hospitalar Tâmega Sousa; 43/2020, Centro Hospitalar Vila Nova de Gaia/Espinho; 192/2020, Centro Hospitalar entre Douro e Vouga; CA-371/2020-0t\_MP/CC, Unidade Local de saúde do Alto Minho; 11/2021. All methods were carried out in accordance with relevant guidelines and regulations.
Data was anonymized before usage. 
For this purpose, we took the Khan harmonized framework since we understood it as simpler to communicate, we feel that the three main categories are indeed non-reducible, which makes sense from an organizational standpoint. Furthermore, the work done by Khan et al. with mapping to already existing frameworks could help compare this work with others who felt the need to use other frameworks. With this in mind, we will use three main categories, Completeness, Plausibility and Conformance. Completeness relates to missing data. Plausibility relates to how believable the values are. Conformance relates to the compliance of the data representation, like formatting, computational conformance and other data standards implemented. 

%TC:ignore
\begin{figure}[htbp]
\centering
\caption{Dimensions of data quality}\label{fig:categories} 
\includegraphics[scale=0.29]{figures/data-quality-v1.png}
\end{figure}
%TC:endignore
\subsection{Methods}
% !TeX root = ../../thesis.tex

For completeness, we used the inverse of the percentage of nulls in the training set. For plausibility, several methods were applied. The first was a Bayesian network. 

In our approach, Bayesian networks, which are probabilistic graphical models, play a pivotal role in predicting the plausibility of different elements. These networks are structured as directed acyclic graphs, where each node represents a variable and edges denote conditional dependencies among these variables \unskip~\cite{pearl1988probabilistic}. This structure allows the network to efficiently manage and represent the probabilistic relationships between multiple variables. The core strength of Bayesian networks in our context lies in their ability to predict the plausibility of various elements by analyzing these interdependencies. By integrating the conditional probabilities of variables and their dependencies, the network can infer the likelihood of certain outcomes or states, thereby assessing the plausibility of different columns in our dataset, when compared with the registered value. 

With this, we hope to capture the heterogeneous essence of the data, as well as possible outliers that are also plausible. We chose this model for its dual advantages: its capability to classify the plausibility of all columns within a single unified framework, and its interpretability, which allows for a clearer understanding of how each variable influences the overall plausibility prediction. The networks were created with the pgmpy package \unskip~\cite{pgmpy}. 

Secondly, we added the outlier-tree method\unskip~\cite{cortesExplainableOutlierDetection2020} which tries to integrate a decision tree that ''predicts'' the values of each column based on the values of each other column. In the process, every time separation is evaluated, it takes observations from each branch as a homogeneous cluster to search for outliers in the predicted 1-d distribution of the column. Outliers are determined according to confidence intervals in this 1-d distribution and need to have large gaps in order to be marked as outliers in the next observation. Because it looks for outliers in the branch of the decision tree, it knows the conditions that make it a rare observation relative to other observation types corresponding to the same conditions, and these conditions are always related to target variables (as predicted by them).  As such, it can only detect outliers described by decision tree logic, and unlike other methods such as isolation forests, it can not assign outlier points to each observation, or detect outliers that are generally rare, but will always provide human-readable justification when it recognizes outliers. Therefore, these methods not only identify anomalies based on a single column/variable but also consider the context of the data, providing a more nuanced understanding of what constitutes an outlier. This contextual awareness ensures that the outliers are not merely statistical deviations but are also substantively significant within the specific framework of the target variables. 

We added also elliptic envelope and Local Outlier Factor as complementary models to these two. Elliptic envelope is a method that assumes a Gaussian distribution of data, fitting an ellipse to the central data points to identify outliers. It works best with normally distributed data but is less effective in higher dimensions or non-normal distributions. Local Outlier Factor measures the local density deviation of a data point relative to its neighbors, identifying outliers without assuming a specific data distribution. It is versatile for different data structures but sensitive to parameter settings, like the number of neighbors. 

An Interquartile Range (IQR) based metric was also added as a supportive metric. This metric used the difference between Q1 and the triple of IQR  to define a lower threshold and Q3 + 3IQR to define an upper threshold. We only categorized as outlier the values that fell outside these margins. Finally, a rule system was implemented to leverage domain knowledge in the overall scoring. The system is based on great expectations package \unskip~\cite{GXProactiveCollaborative}. A set of 17 rules was defined by the team, focusing on impossible numbers or relationship between variables  or value format. The rules covered plausability and conformance. 

The Conformance-based were related to technical issues like the format of dates (date of birth like d/m/y), and conformance to the value set (i.e. Robson group, bishop scores, or delivery types). Plausibility rules were linked to expected values for BMI, weight, and gestational age (gestational age between 20 and 44). We also added plausibility for the relationship between columns, namely weight across different weeks of gestation (weight week 35 {\textgreater} weight week 25). We have also added a relationship of greatness between ultrasound weights more than 5 weeks apart. 

As for preprocessing, all null representations were standardized, we also removed features with high missing rates ({\textgreater} 80\% ). The imputation process was performed with the median for continuous and a new category (NULLIMP) for categorical variables.

For the usage of the Bayesian network in particular, the continuous variables were discretized into three bins defined by quantile. We defined three as the number of bins in order to reduce the number of states in each node of the network. The evaluation was done with cross-validation with 10 splits and two repetitions for each column as the target.

The API for serving the prediction models was developed with FastAPI. So, the methods applied in terms of the DQA framework shown in figure \ref{fig:categories} are described in the table \ref{tab:methods}.

\begin{table}[htpb]
\caption{Implemented Methods in the tool. The first column is the category or data quality dimension. The second is a subcategory of the first column if applicable and the third column is the actual method used to assess such a dimension.} \label{tab:methods}
\renewcommand{\arraystretch}{1.4}
\setlength{\tabcolsep}{10pt}

\begin{tabularx}{\textwidth}{ p{2cm} p{3.5cm} X }
\hline
 Category   & Subcategory           & Method   \\ \hline
Completeness     & N/A               & Score by the inverse percentage of missing in the train data         \\ 
Plausibility & Atemporal Plausibility & Bayesian model prediction based on the other values of row \\ 
Plausibility & Atemporal Plausibility         & Z-score for column value based on IQR train data       \\    
Plausibility & Atemporal Plausibility           & Elliptic Envelope                       \\ 
Plausibility & Atemporal Plausibility           & Local Outlier Factor                \\ 
Conformance & Value Conformance           & Manual Rule engine                           \\ 
Plausibility & Atemporal Plausibility           & Manual Rule engine                      \\ 
Plausibility & Atemporal Plausibility           & outlier-tree                      \\ 
Conformance & Value Conformance & Manual Rule engine\\
\hline
\end{tabularx}

\end{table}


For trying to compile all of these models into a single value, that could grasp the quality of the row or patient, a scoring method was created. The method of calculating the final score is stated in figure \ref{fig:scoring_method}. 
%TC:ignore
\begin{figure}[htbp]
    \centering
    \caption{Workflow and weights used for creating the final score and which elements are used to do so.}\label{fig:scoring_method} 
    \includegraphics[scale=0.29]{figures/score-method.png}
    \end{figure}
    %TC:endignoregrama

To assess the tool's usefulness, we implemented it in a production environment and collect metrics regarding the data being produced. Then we presented some rows (or patient's records) to selected obstetrics clinicians for them to assess how likely the information is to be suitable for usage and rank it according to the perceived quality of the record. This was done through questionnaire, presenting the data and asking the clinicians to rank them from 1-10 and to describe the most important feature for the decision. We then compared the results with the ones from the model to make sanity checks regarding the model's performance and adequacy. We used Kendal Tau and Average Spearman's Rank Correlation Coefficient. Kendall Tau is a non-parametric statistic used to measure the strength and direction of the association between two ordinal variables. It calculates the difference between the number of concordant and discordant pairs of observations, normalized to ensure a value between -1 (perfect disagreement) and 1 (perfect agreement). Spearman's rank correlation coefficient is a non-parametric measure that assesses the strength and direction of a monotonic relationship between two ranked variables. It is based on the ranked values of the variables rather than their raw data, producing a value between -1 (perfect inverse relationship) and 1 (perfect direct relationship). Finally, we tested the capability of the model to discriminate bad quality records from good quality records, testing various thresholds of rank of quality, taken into account physicians responses.
We wrote all the code in Python 3.10.6 with the usage of the scikit-learn library for preprocessing, and evaluation\unskip~\cite{scikit-learn}.

\subsection{Results}
A Bayesian network with structure and parameters learned from the training dataset reached an average of \ac{auroc} of 0.89. The results are in the table \ref{tab:result_auc}.

\begin{table}[htbp] 
 \caption{Validation Results (Column AUROC)} 
 \label{tab:result_auc} 

\renewcommand{\arraystretch}{1.2}
\setlength{\tabcolsep}{18pt}

\begin{tabularx}{\textwidth} { X X X X  }
\hline
IA & 0.674 & AA & 0.787 \\
PI & 0.873 & TP & 0.868 \\
IMC & 0.872 & A30 & 0.863 \\
NRCPN & 0.703 & GR & 0.936 \\
IGA & 0.955 & ANP & 1.000 \\
SGP & 0.962 & VCS & 0.730 \\
EPC30 & 0.946 & TG & 0.836 \\
APARA & 0.989 & TPEE & 0.842 \\
AGESTA & 0.931 & V & 0.987 \\
EA & 0.993 & VNH & 0.849 \\
VA & 0.962 & TPNP & 0.928 \\
FA & 0.962 & VP & 0.793 \\
CA & 0.998 &  \\
\hline
 \multicolumn{4}{c}{\textbf{Average}  \textbf{0.890}} \\

\hline
\end{tabularx}
\end{table}


The network is as represented in figure \ref{fig:network}.
%TC:ignore
\begin{figure}[htbp]
\centering
\caption{Network learned}\label{fig:network} 
\includegraphics[scale=0.68]{figures/network.png}
\end{figure}
%TC:endignore

As for the rules created, they were conformance based, like the format of dates, and conformance to the value set (i.e. Robson group, bishop scores, or delivery types). We also added plausibility rules, like expected values for BMI, weight and gestational age. We also added plausibility for the relationship between columns, namely weight across different weeks of gestation. We added a relationship of greatness between weights more than 5 weeks apart. 





\subsubsection{Deployment \& Validation} 
The purpose of this model is to be served as an API for usage within a healthcare institution and act as a supplementary decision support tool for obstetrics teams. Although a concrete, vendor-specific information model and health information system were initially used, our goal is to develop a more universal clinical decision support system. This system should be usable across all systems involved in birth and obstetrics departments. Therefore, we constructed it using the \ac{hl7} \ac{fhir}  R5 version standard. This approach simplifies the process of \ac{api} interaction.
Rather than utilizing a proprietary model for the data, we based our decision on the use of \ac{fhir} resources: Bundle and Observation. These resources handle the request and response through a customized operation named "\$quality\_check". Our intention is to publish the profiles of these objects to streamline API access via standardized mechanisms and data models. The current version of the profiles can be accessed at this URL: \url{https://joofio.github.io/obs-cdss-fhir/}. 


For validation, we deployed the tool in docker format in a hospital to gather new data. We gathered 3231 new cases and returned a score for quality as exemplified in figure \ref{fig:scores}. Being that the score is from 0 to 1, the average score was 0.12 and \acp{iqr} was 0.15. We also used the clinician from one of the hospitals that we get data from and asked this clinician to assess 10 records in terms of quality. We gathered the 10 records at random and asked the clinician to assess them in terms of quality. Our purpose was then to compare the rankings of each evaluator; the model and the clinician, in order to assess how similar they were.



%TC:ignore
\begin{figure}[htbp]
\centering
\caption{Scores}\label{fig:scores} 
\includegraphics[scale=0.78]{figures/Scoring.png}
\end{figure}
%TC:endignore

\subsection{Discussion}
% !TeX root = ../../thesis.tex
This work adds several pieces of information to the state of the art of data quality analysis. First we tried to map the output of an automatic assessment tool to the human perception of quality and the issues linked to doing so. Secondly, the fact that we applied explainable machine learning methods such as bayesian networks to leverage the potency of advanced data analysis without compromising interpretability and explainability. Furthermore, a single model was able to reach high performance metrics for almost all variables. Thirdly, the fact that interoperability standard such as FHIR can be adopted to facilitate the usage and information exchange of such tools. However, there are also shortcoming and challenges to address. The first is that data quality is still an elusive concept since it has a contextual dimension and the quality of the record depends on the usage of the information. For example, data aimed at primary usage and day-to-day healthcare decisions about a patient will have different requirements regarding the importance of some variable or completeness of information very different from data needed to create summary statistics for key performance indicators extraction. Moreover, the data is still very vendor-specific. Even though we used an interoperability standard, the semantic layer, more connected with terminology is still lacking. This is an issue to be addressed in order to improve the interoperability of the standard. Moreover, we do not know how the training done with this data is generalizable to other vendors. One opportunity arises of mapping all of this data to a widely used terminology like SNOMED CT or LOINC. Nevertheless, the usage of FHIR and the fact that the data is mapped to a standard terminology, makes it easier to use the data in other systems and to compare the results with other studies. Furthermore, being available freely and online makes it easier to understand how to map vendor-specific datasets to the model and use it in other contexts. Regarding the model, the usage of explainable methodologies like outlier-tree and transparent models like Bayesian networks are vital for clinical application. Since we use a single model to classify possible errors in the records, the ability to try to show clinicians why that value was tagged is of uttermost importance in order to get feedback and action from humans. From the experience gathered with the study, we believe that a weaker but transparent model could have more impact than better performant but opaque ones. If explainability and interpretability are important for any ML problem, this need only increases when we are dealing with such subjective concepts as data quality.

Regarding the clinical evaluation, we found that asking clinicians to purely assess the quality of a record in an EHR is not an easy task. We discovered that for a proper assessment, a context and objective must be defined in order to make the evaluation more objective and manageable. Moreover, the ranking methodology, though very useful for comparison with the model, presents challenges for clinicians who find it difficult to order 10 records when some appear to be of equal quality. This is a very important aspect to consider when designing an evaluation method for data quality. Perhaps a categorical evaluation of yes/no would be more effective than ordering several records. These reasons might explain the great variability between clinicians (figure \ref{fig:clinical-dq}) and between clinicians and the model (Spearman and Kendall tau). Despite that, our preliminary results are promising, demonstrating an AUROC curve for categorizing bad quality records as high as 88\% and low as 56\%. The highest value was achieved by classifying all record with a mean rank of 4 or above as bad quality and the others as good quality records. However, these results rely on very few samples, so more data and research are needed in this area since it is a very subjective decision, and it should take into account the context and the objective of the evaluation. For example, if the objective is research use, the weights given to each dimension can be a set. On the other hand, if the objective is to use the data for day-to-day clinical decisions, another set of weights could be used. 

For the next steps, a promising research direction would be identifying contexts for applying data quality checks like primary usage, research purposes, and aggregated analysis for decision-making among others. This could enhance targeting those contexts and understanding the importance of each variable for those use cases. Incorporating this approach into the tool to weigh the different variables according to the context would be beneficial.  Finally, gaining access to more data and clinician evaluation of records, although challenging, is important to thoroughly assess the performance of the tool.


\subsection{Conclusion}
This work is still an early draft of a production-ready tool. However, we feel the work done is already a valuable insight into how to use data quality frameworks and several statistical tools in order to assess ehr data quality. This is a fundamental process not only to guarantee the quality of data for primary usage on a day-to-day but also for securing quality for secondary analysis and usage.
We believe the fact that we created an interoperable tool that was trained on real obstetrics data from 9 different hospitals and has the ability to provide a single score for a clinical record can help institutions, academics, and ehr vendors implement data quality assessment tools in their own systems and institutions.

For the next steps, we would like to further evaluate the score and its relationship with clinical usefulness. This would also include a further assessment of a  threshold for the score for defining a record that would require human attention.





\section{Leveraging Distributed systems in healthcare: is it advisable?}\label{subsec:distributed}
This section is based on the paper entitled "Evaluating distributed-learning algorithms on real-world healthcare data". This paper was focused on the fact that access to healthcare data is often laboursome and time-consuming. So we evaluated the distributed paradigm to its gold-standard, the centralized paradigm. We used 9 real-world datasets of obstetrics \acp{ehr} and compared the performance of several \ac{ml} algorithms in both paradigms. We concluded that the distributed paradigm is a valid alternative to the centralized paradigm, with the added benefit of not requiring heavy data sharing.

\subsection{Introduction}
% !TeX root = ../../thesis.tex

%As the use of \ac{ai} is increasing in the healthcare space \cite{deep_learning_increase_health}, increased demand for ethical usage of personal patient data is occurring as well \cite{ehtical_use_ml}. This has been happening both on the governmental side, with several regulations passed to protect citizens' data and personal information (such as \ac{gdpr} in the \ac{eu} \cite{gdpr_article} and \ac{hipaa} in the \ac{us} \cite{hippa}), and on the public side, with an increased concern with continuous data breaches across institutions \cite{abdulrahmanSurveyFederatedLearning2021}. So,  we are now faced with a dilemma on a compromise between what is possible to do with the available data and what should be done regarding patient privacy \cite{swarm_learning}. This is the main reason why health institutions implement burdensome processes and methodologies for sharing patient data, often costing a great deal of time, money, and human resources, seldomly overtaking the ideal time frame for analysing such data.
%Due to these privacy concerns, the traditional method for using data in healthcare is, nowadays, by focusing on data from a single institution in order to predict or infer something regarding those patients; this could be understood as local learning. This approach has some drawbacks, namely data quantity, data quality and possible class imbalance \cite{rajkomarMachineLearningMedicine2019}, never quite raising into its full potential for promoting best healthcare practices
%\cite{federated_healthcare_informatics,usage_ai_healthcare,wangAIHealthState2019} with data sharing between institutions.
%In order to overcome this issue, there are a few, more complex, systems that aggregate data from several institutions, so more robust algorithms could be trained. However, this globally centralised aggregation of data encompasses a very important data breach hazard. 

%This is the setting where distributed learning could create a greater impact. A halfway point between local and centralised learning is where we train several models, one in each institution (or silo), and where the sole information that leaves the premises is a trained model or its metadata. A distributed model is built as the aggregation of all the local models, consequently aiming to create a model similar to one globally trained with all the data in a centralised server. However, the distributed model never contacted with any data, only the local models did. This provides the opportunity to create better models, improve data protection, reduce training time and cost and provide better scaling capabilities  \cite{jatainContemplativePerspectiveFederated2021}.

%including federated-learning approaches, where a central system orchestrates the operation \cite{federated_learning_intro} or a swarm/peer-to-peer framework where silos communicate with each other. However,
%There are already some implementations of distributed systems in the healthcare space, but we lack a robust understanding of how these models behave with real data, when compared with the classical models built with all the aggregated data. Additionally, the main issues regarding the development and implementation of such systems in healthcare are still elusive.
%So we aim to understand how distributed mechanisms behave compared to using all data in the healthcare space and if they are a suitable replacement for traditional machine-learning pipelines. The contributions of this paper are:
%\begin{myitemize}
%    \item Understand how to address the lack of data quality of real-world data regarding distributed model creation;
%    \item Evaluate a distributed model against its local counterparts;
%    \item Measure the prediction performance difference between a distributed model and a centralised one;
%    \item Identify the capabilities of a distributed model to track population changes on the local datasets;
%    \item Open a research path for using distributed models to predict several target variables in obstetrics clinical research.
%\end{myitemize}



As the use of \ac{ai} is increasing in the healthcare space \cite{deep_learning_increase_health}, increased demand for ethical usage of personal patient data is occurring as well \cite{ehtical_use_ml}. This has been happening both on the governmental side, with several regulations passed to protect citizens' data and personal information (such as \ac{gdpr} in the \ac{eu} \cite{gdpr_article} and \ac{hipaa} in the \ac{us} \cite{hippa}), and on the public side, with an increased concern with continuous data breaches across institutions \cite{abdulrahmanSurveyFederatedLearning2021}.  So,  we are now faced with a dilemma on a compromise between what is possible to do with the available data and what should be done regarding patient privacy \cite{swarm_learning}. This is the main reason why health institutions implement burdensome processes and methodologies for sharing patient data, often costing a great deal of time, money, and human resources, seldomly overtaking the ideal time frame for analysing such data.
Due to these privacy concerns, the traditional method for using data in healthcare is, nowadays, by focusing on data from a single institution in order to predict or infer something regarding those patients; this could be understood as local learning. This approach has some drawbacks, namely data quantity, data quality and possible class imbalance \cite{rajkomarMachineLearningMedicine2019}, never quite raising into its full potential for promoting best healthcare practices
\cite{federated_healthcare_informatics,usage_ai_healthcare,wangAIHealthState2019} with data sharing between institutions.
In order to overcome this issue, there are a few, more complex, systems that aggregate data from several institutions, so more robust algorithms could be trained. However, this globally centralised aggregation of data encompasses a very important data breach hazard. 

This is the setting where distributed learning could create a greater impact. A halfway point between local and centralised learning is where we train several models, one in each institution (or silo), and where the sole information that leaves the premises is a trained model or its metadata. A distributed model is built as the aggregation of all the local models, consequently aiming to create a model similar to one globally trained with all the data in a centralised server. However, the distributed model never contacted with any data, only the local models did. This provides the opportunity to create better models, improve data protection, reduce training time and cost and provide better scaling capabilities  \cite{jatainContemplativePerspectiveFederated2021}.

%including federated-learning approaches, where a central system orchestrates the operation \cite{federated_learning_intro} or a swarm/peer-to-peer framework where silos communicate with each other. However,
%There are already some implementations of distributed systems in the healthcare space, but we lack a robust understanding of how these models behave with real data, when compared with the classical models built with all the aggregated data. So we aim to understand how distributed mechanisms behave compared to using all data in the healthcare space and if they are a suitable replacement for traditional machine-learning pipelines. The contributions of this paper are:

While numerous multi-institutional initiatives have successfully established integrated data repositories for healthcare research, there remains an incomplete understanding of the performance and scalability of distributed systems when directly compared to traditional, centralised models. Specifically, the nuanced behaviors of these distributed frameworks under real-world data conditions—contrasted against classical models that utilize aggregated data—have yet to be fully delineated. This paper aims to critically evaluate the efficacy and suitability of distributed mechanisms within the healthcare domain, assessing their potential as viable alternatives to conventional machine-learning pipelines. The contributions of this paper include:

\begin{myitemize}
    \item Evaluate a distributed model against its local counterparts;
    \item Measure the prediction performance difference between a distributed model and a centralised one;
%    \item Identify the capabilities of a distributed model to track population changes on the local datasets;
    %\item Understand how to address the lack of data quality of real-world data regarding distributed model creation;

  %  \item Open a research path for using distributed models to predict several target variables in obstetrics clinical research.
\end{myitemize}

\subsection{Theoretical background and Related Work}
Distributed learning \cite{distributed} can be understood as training several models in a different setting and then aggregating them as a whole. There are two main branches of these approaches, distinguishable by the existence of a central orchestrator server: federated learning where such an entity exists, and peer-to-peer (or swarm) \cite{swarm_learning} learning where it does not. 
Even though distributed learning has been receiving a lot of attention recently, only some of its concepts have been focused on, mainly distributed-deep learning with a federated learning approach \cite{xuFederatedLearningHealthcare2021,leeFederatedLearningClinical2020}. These methods use the strength of neural networks and several algorithms like federated averaging to create distributed models capable of handling complex data like text, sound, or image \cite{prayitnoSystematicReviewFederated2021}. However, considering that there are great amounts of information, especially in healthcare, stored as tabular data \cite{alvarezsanchezTAQIHToolTabular2019,dimartinoExplainableAIClinical2022,payrovnaziriExplainableArtificialIntelligence2020} and that neural networks are often not the best tool for such data structures \cite{borisovDeepNeuralNetworks2022a}, there is a lack of knowledge in the traditional machine learning techniques in a distributed manner.
%Federated Learning was introduced in 2016 \cite{konecny_federated_2016,mcmahanFederatedLearningDeep2016} and it was called federated since "the learning task is solved by a loose federation of participating devices (which we refer to as clients) which are coordinated by a central server" \cite{konecny_federated_2016}. Federated learning has two main architectures: a) horizontal and b) vertical  \cite{yangFederatedMachineLearning2019b}. 
%Horizontal refers to having the same features in all silos, but different populations in each silo. Vertical refers to having different features across silos for the same population. Then we have other approaches that expand the concept of federated learning with previous machine learning and deep learning methodologies such as transfer learning, reinforcement learning \cite{liuSystematicLiteratureReview2020} and quantum machine-learning \cite{quantum-fed-ml}. 
%Federated learning can also be classified by the information flow. Model data can be shared only with the main central server, as in more traditional methods, but also being incremental, sharing data model sequentially between silos, with central server orchestration \cite{cyclic_distribution}. 

Nevertheless, there have been some health-related  distributed machine-learning projects successfully implemented, such as euroCAT  \cite{eurocat} which implemented an infrastructure across five clinics in three countries. \ac{svm} models were used to learn from the data distributed across the five clinics. Each clinic has a connector to the outside where only the model's parameters are passed to the central server which acts as a master deployer regarding the model training with the radiation oncology data.
Also, ukCAT \cite{ukcat} did similar work, with an added centralised database in the middle, but the training being done with a decentralized system.
%Other methods and approaches have been also evaluated such as the work of Brisimi et. al. \cite{brisimi_federated_2018}  for predicting hospitalisations or deep learning methods oriented to analysing medical imaging \cite{chang_distributed_2018}, evaluating histological samples \cite{pathology-fl} and finally, some preprints showing the impact of federated learning regarding COVID-19 prediction \cite{vaid_federated_2020}. For a sound review please redirect to Zerka et al. \cite{zerkaSystematicReviewPrivacyPreserving2020a}

Finally, a few works have explored the evaluation of models in a distributed manner, for example, comparing  centralised machine learning, distributed machine learning and federated learning on MNIST dataset \cite{performance_evaluation_1}. Also, works that evaluate federated learning on MNIST, MIMIC-III and PhysioNet ECG datasets, but not in comparison with other methods  \cite{performance_evaluation_2}. The work by Tuladhar and colleagues \cite{distributed} uses healthcare images and/or public and curated datasets.
As far as we know, this is the first time a distributed machine learning evaluation is done with real-world clinical data from several different data sources.
\subsection{Materials}
Clinical data was gathered from nine different Portuguese hospitals regarding obstetric information, pertaining to admissions from 2019 to 2020. This originated nine different files representing different sets of patients but with the same features associated to them. The software for collecting data was the same in every institution (although different versions existed across hospitals) - ObsCare. The data columns are the same in every hospital's database. Each hospital was considered a silo and summary statistics of the different silos are reported in the tables \ref{tab:distributed_materials_1} and \ref{tab:distributed_materials_2}. The data dictionary is in appendix \ref{appendix:data_dict}.
%TC:ignore

{\small
\begin{table}[!ht]

\caption[Silos overview.]{\label{tab:distributed_materials_1}Silos overview. categorical columns have a snippet of the most used category and a percentage. Continuous variables have a mean and standard deviation. Abbreviation meaning in the appendix \ref{appendix:data_dict}. The last row is the number of patients. * columns were used as target.}

\centering
% !TeX root = ../../thesis.tex
\newcolumntype{L}{>{\scriptsize}l}  % "small" can be changed to "scriptsize" or "footnotesize" for even smaller text

\begin{tabular}{LLLLLLL}
   \toprule
      Variable &               Silo 1 &               Silo 2 &               Silo 3 &               Silo 4 &               Silo 5 &                Agrr. \\
   \midrule
   \hspace*{2mm} N (total) &              8039 &                 8566 &                 4989 &                 2364 &                18177 &                80874 \\
   
   \textbf{Actual Type of Delivery C (\%)}& 10 (52.6) & 3 (51.6) & 3 (57.8) & 3 (61.8) & 9 (61.5) & 11 (52.9) \\
   
   Bishop Score C (\%)&  15 (98.5) & 15 (78.8) & 13 (97.4) & 16 (86.4) & 15 (97.4) & 16 (95.3) \\
   
   \textbf{Blood Group C (\%)}& 9 (39.9) & 10 (39.9) & 9 (39.3) & 11 (37.9) & 10 (40.9) & 14 (40.5) \\
   
   \textbf{Body Mass Index $\mu (\sigma)$ } & 25.2 (8.6) & 25.2 (6.2) & 25.0 (5.3) & 25.0 (8.9) & 24.9 (7.8) & 25.1 (7.0) \\
   
   Cervical Consistency C (\%) & 4 (98.6) & 4 (83.4) & 4 (99.3) & 4 (87.4) & 4 (97.5) & 4 (96.5) \\
   
   Cervical Position C (\%)&  4 (98.6) & 4 (83.3) & 4 (99.3) & 4 (87.5) & 4 (97.6) & 4 (96.6) \\
   
   \textbf{Delivery Type C (\%)}& 6 (43.4) & 6 (53.5) & 5 (44.4) & 7 (52.2) & 7 (49.3) & 8 (51.3) \\
   
   Dilatation C (\%)&5 (98.5) & 5 (83.1) & 5 (99.3) & 5 (87.2) & 5 (97.5) & 5 (96.5) \\
   
   Effacement C (\%)& 5 (98.6) & 5 (83.2) & 5 (99.3) & 5 (87.2) & 5 (97.5) & 5 (96.5) \\
   
   Fetal Station  C (\%)&  5 (98.6) & 5 (83.3) & 5 (99.3) & 5 (87.9) & 5 (97.5) & 5 (96.6) \\
   
   \textbf{Followed physician C (\%)} & 3 (99.2) & 4 (92.2) & 3 (99.1) & 3 (94.3) & 3 (99.0) & 4 (97.9) \\
   
   \textbf{\begin{minipage}{3.8cm}\setstretch{0.65}Followed physician hospital delivery C (\%)\vspace{1mm}\end{minipage}}&  2 (87.6) & 2 (75.8) & 2 (81.4) & 2 (52.2) & 2 (71.0) & 2 (69.0) \\
   
   \textbf{\begin{minipage}{3.8cm}\setstretch{0.65}Followed physician primary care C (\%)\vspace{1mm}\end{minipage}}&  2 (61.3) & 2 (52.8) & 2 (78.1) & 2 (50.4) & 2 (70.4) & 2 (67.6) \\
   
   \begin{minipage}{3.8cm}\setstretch{0.65}Followed physician private clinic C (\%)\vspace{1mm}\end{minipage}&  2 (81.8) & 2 (85.0) & 2 (80.6) & 2 (78.8) & 2 (73.3) & 2 (75.8) \\
   
   Gestational Diabetes C (\%)&2 (87.7) & 2 (90.0) & 2 (90.2) & 2 (90.8) & 2 (89.8) & 2 (89.5) \\
   
   
   Induced Delivery  C (\%)& 2 (97.8) & 2 (83.9) & 2 (93.3) & 2 (91.9) & 2 (98.5) & 2 (92.5) \\
   
   \textbf{Mother Age $\mu (\sigma)$ } & 31.1 (5.7) & 30.7 (5.6) & 31.1 (5.9) & 31.1 (6.3) & 31.3 (5.6) & 31.1 (5.6) \\
   
   
   \begin{minipage}{3.9cm}\setstretch{0.65}Nr Deliveries forceps C (\%)\end{minipage} & 4 (99.2) & 3 (83.3) & 4 (94.3) & 4 (95.8) & 3 (60.1) & 5 (82.6) \\
   
   
   \begin{minipage}{3.9cm}\setstretch{0.65}Nr Deliveries no assistance C (\%)\vspace{1mm}\end{minipage} & 10 (74.7) & 9 (60.3) & 9 (74.9) & 9 (67.3) & 11 (45.4) & 12 (60.3) \\
   
   \begin{minipage}{3.9cm}\setstretch{0.65}Nr Deliveries vacuum C (\%)\vspace{1mm}\end{minipage} &  5 (90.4) & 4 (79.9) & 4 (89.0) & 4 (93.1) & 5 (55.3) & 5 (77.4) \\
   
   Nr of C-sections C (\%) & 6 (87.9) & 6 (72.6) & 5 (86.1) & 5 (89.5) & 6 (62.1) & 6 (74.6) \\
   
   \textbf{Nr of Pregnancies C (\%)} &  11 (40.9) & 11 (43.1) & 13 (39.1) & 12 (38.7) & 16 (42.8) & 19 (42.1) \\
   
   \textbf{Nr of born babies C (\%)} & 10 (44.8) & 10 (41.4) & 10 (36.9) & 10 (42.0) & 12 (35.3) & 12 (38.8) \\
   
   \textbf{Nr of consultations $\mu (\sigma)$ } &  7.3 (4.7) & 7.0 (6.4) & 6.4 (3.9) & 5.5 (3.6) & 10.5 (5.1) & 8.4 (5.1) \\
   
   Pelvis Adequacy C (\%) &4 (95.4) & 4 (77.7) & 4 (90.1) & 3 (96.9) & 4 (81.2) & 4 (82.6) \\
   
   \textbf{Position Admission C (\%)}  &  5 (88.5) & 6 (78.0) & 6 (51.8) & 3 (95.9) & 6 (71.3) & 7 (73.1) \\
   
   
   \textbf{Position on Delivery C (\%)} &5 (91.5) & 5 (94.4) & 5 (94.7) & 5 (95.5) & 5 (94.3) & 5 (93.9) \\
   
   \textbf{Pregnancy Type C (\%)} &  7 (62.1) & 7 (90.5) & 7 (85.4) & 7 (63.0) & 7 (89.2) & 7 (85.4) \\
   
   \textbf{Robson Group C (\%)} & 11 (22.4) & 11 (20.1) & 10 (23.8) & 10 (80.5) & 11 (27.7) & 11 (24.4) \\
   
   \begin{minipage}{3.7cm}\setstretch{0.65}Rupture amniotic pocket before delivery C (\%)\end{minipage} &  2 (91.1) & 2 (93.6) & 2 (89.3) & 2 (91.6) & 2 (84.6) & 2 (88.5) \\
   
   Smoker C (\%) & 2 (84.4) & 2 (85.2) & 2 (87.2) & 2 (89.7) & 2 (87.9) & 2 (88.1) \\
   
   \textbf{Spontaneous Delivery C (\%)} &  2 (70.3) & 2 (74.7) & 2 (64.8) & 2 (64.3) & 2 (59.7) & 2 (64.9) \\
   
   \textbf{Weeks on Admission C (\%)} &    38.1 (3.5) &    38.8 (2.2) &    38.9 (1.6) &    38.8 (2.4) &    38.6 (2.1) &    38.7 (2.2) \\
   
   
   \textbf{Weeks on Delivery $\mu (\sigma)$ } &    38.5 (2.8) &    38.9 (2.0) &    39.1 (1.7) &    39.0 (2.3) &    38.9 (2.0) &    38.9 (2.0) \\
   
   Weight on Admission $\mu (\sigma)$  &   81.4 (14.9) &   79.5 (14.5) &   78.0 (15.2) &   79.6 (16.3) &   78.3 (14.2) &   78.8 (14.5) \\
   
   
   \textbf{\begin{minipage}{3.3cm}\setstretch{0.65}Weight start  of pregnancy $\mu (\sigma)$ \vspace{1mm}\end{minipage}} &   66.4 (14.4) &   66.1 (13.5) &   65.5 (14.1) &   65.5 (14.1) &   65.5 (14.4) &   66.0 (14.1) \\
   \bottomrule
   \end{tabular}
   
   
\end{table}
}
\newpage

{\small
\begin{table}[!ht]

\caption[Silos overview part 2.]{\label{tab:distributed_materials_2}Silos overview part 2. categorical columns have a snippet of the most used category and a percentage. Continuous variables have a mean and standard deviation. Abbreviation meaning in the appendix \ref{appendix:data_dict}. The last row is the number of patients. * columns were used as target.}
\centering
\begin{tabular}{l|lllll}
\toprule
   Column &               Silo 6 &               Silo 7 &               Silo 8 &               Silo 9 &                Agrr. \\
\midrule
       IA &    31.3 \textbf{5.2} &    31.4 \textbf{5.4} &    31.5 \textbf{5.6} &    30.1 \textbf{5.6} &    31.1 \textbf{5.6} \\
       GS & a,rh.. \textbf{42\%} & a,rh.. \textbf{39\%} & a,rh.. \textbf{40\%} & a,rh.. \textbf{42\%} & a,rh.. \textbf{40\%} \\
       PI &   65.6 \textbf{13.5} &   66.0 \textbf{13.7} &   65.6 \textbf{14.1} &   67.4 \textbf{14.6} &   66.0 \textbf{14.1} \\
      PAI &   77.7 \textbf{13.4} &   79.2 \textbf{14.7} &   76.7 \textbf{13.0} &   83.1 \textbf{15.2} &   78.8 \textbf{14.5} \\
      IMC &    24.9 \textbf{5.1} &    24.9 \textbf{7.0} &    24.8 \textbf{8.0} &    25.7 \textbf{5.6} &    25.1 \textbf{7.0} \\
      CIG &   Null \textbf{91\%} &   Null \textbf{91\%} &   Null \textbf{86\%} &   Null \textbf{90\%} &   Null \textbf{88\%} \\
    APARA &    1.0 \textbf{38\%} &   Null \textbf{43\%} &   Null \textbf{41\%} &   Null \textbf{43\%} &   Null \textbf{39\%} \\
   AGESTA &    1.0 \textbf{44\%} &      1 \textbf{43\%} &    1.0 \textbf{42\%} &    1.0 \textbf{40\%} &    1.0 \textbf{42\%} \\
       EA &   Null \textbf{59\%} &   Null \textbf{61\%} &   Null \textbf{69\%} &   Null \textbf{61\%} &   Null \textbf{60\%} \\
       VA &   Null \textbf{79\%} &   Null \textbf{82\%} &   Null \textbf{88\%} &   Null \textbf{82\%} &   Null \textbf{77\%} \\
       FA &   Null \textbf{82\%} &   Null \textbf{86\%} &   Null \textbf{94\%} &   Null \textbf{89\%} &   Null \textbf{83\%} \\
       CA &   Null \textbf{69\%} &   Null \textbf{75\%} &   Null \textbf{85\%} &   Null \textbf{78\%} &   Null \textbf{75\%} \\
       TG & espo.. \textbf{88\%} & espo.. \textbf{85\%} & espo.. \textbf{86\%} & espo.. \textbf{93\%} & espo.. \textbf{85\%} \\
        V &      s \textbf{97\%} &      s \textbf{99\%} &      s \textbf{98\%} &      s \textbf{99\%} &      s \textbf{98\%} \\
    NRCPN &     6.8 \textbf{4.0} &     7.7 \textbf{3.2} &     9.3 \textbf{4.5} &     8.9 \textbf{5.5} &     8.4 \textbf{5.1} \\
       VP &   Null \textbf{68\%} &   Null \textbf{74\%} &   Null \textbf{71\%} &   Null \textbf{78\%} &   Null \textbf{76\%} \\
      VCS &   Null \textbf{53\%} &      s \textbf{87\%} &      s \textbf{63\%} &      s \textbf{87\%} &      s \textbf{68\%} \\
      VNH &   Null \textbf{62\%} &      s \textbf{63\%} &      s \textbf{69\%} &      s \textbf{83\%} &      s \textbf{69\%} \\
        B &   Null \textbf{90\%} &   Null \textbf{53\%} &   Null \textbf{93\%} &   Null \textbf{82\%} &   Null \textbf{83\%} \\
       AA &   Null \textbf{84\%} & apr... \textbf{61\%} &   Null \textbf{89\%} &   Null \textbf{74\%} &   Null \textbf{73\%} \\
       BS &   Null \textbf{99\%} &   Null \textbf{98\%} &   Null \textbf{99\%} &   Null \textbf{95\%} &   Null \textbf{95\%} \\
       BC &  Null \textbf{100\%} &  Null \textbf{100\%} &  Null \textbf{100\%} &   Null \textbf{97\%} &   Null \textbf{97\%} \\
      BDE &  Null \textbf{100\%} &  Null \textbf{100\%} &  Null \textbf{100\%} &   Null \textbf{97\%} &   Null \textbf{97\%} \\
      BDI &  Null \textbf{100\%} &  Null \textbf{100\%} &   Null \textbf{99\%} &   Null \textbf{97\%} &   Null \textbf{96\%} \\
       BE &  Null \textbf{100\%} &  Null \textbf{100\%} &  Null \textbf{100\%} &   Null \textbf{97\%} &   Null \textbf{96\%} \\
       BP &  Null \textbf{100\%} &  Null \textbf{100\%} &  Null \textbf{100\%} &   Null \textbf{97\%} &   Null \textbf{97\%} \\
      IGA &    38.7 \textbf{1.8} &    39.0 \textbf{2.0} &    38.6 \textbf{2.1} &    38.8 \textbf{1.9} &    38.7 \textbf{2.2} \\
     TPEE &   Null \textbf{65\%} &   Null \textbf{64\%} &   Null \textbf{65\%} &   Null \textbf{63\%} &   Null \textbf{65\%} \\
     TPEI &   Null \textbf{92\%} &   Null \textbf{86\%} &   Null \textbf{87\%} &   Null \textbf{94\%} &   Null \textbf{93\%} \\
      RPM &   Null \textbf{85\%} &   Null \textbf{84\%} &   Null \textbf{90\%} &   Null \textbf{94\%} &   Null \textbf{88\%} \\
       DG &   Null \textbf{92\%} &   Null \textbf{88\%} &   Null \textbf{90\%} &   Null \textbf{87\%} &   Null \textbf{89\%} \\
       TP & part.. \textbf{54\%} & part.. \textbf{52\%} & part.. \textbf{48\%} & part.. \textbf{59\%} & part.. \textbf{51\%} \\
      ANP & cefá.. \textbf{93\%} & cefá.. \textbf{94\%} & cefá.. \textbf{95\%} & cefá.. \textbf{94\%} & cefá.. \textbf{94\%} \\
     TPNP & espo.. \textbf{64\%} &  Null \textbf{100\%} & espo.. \textbf{50\%} & espo.. \textbf{65\%} & espo.. \textbf{53\%} \\
      SGP &    38.8 \textbf{1.8} &    39.2 \textbf{1.7} &    38.7 \textbf{2.0} &    39.0 \textbf{1.6} &    38.9 \textbf{2.0} \\
       GR &      1 \textbf{27\%} &      1 \textbf{25\%} &      1 \textbf{21\%} &      3 \textbf{27\%} &      1 \textbf{24\%} \\
       \midrule
N (total) &                12002 &                 8258 &                 6693 &                11786 &                80874 \\
\bottomrule
\end{tabular}


\end{table}
}
%TC:endignore

\subsection{Methods}
% !TeX root = ../../thesis.tex

The section will cover the steps we took for evaluating the models. We first addressed the preprocessing of the data, then the training of the models and finally the evaluation of the models. The evaluation was done by comparing the performance of the distributed model with the local and centralised models. The performance was measured by the \ac{auroc}, \ac{auprc}, \ac{rmse} and \ac{mae}. The results were then compared using a 2-sample T-test.
\subsubsection{Preprocessing}

The initial dataset underwent preprocessing by eliminating attributes that were missing more than 90\% of their data across all storage units (or silo). We standardized the representation of missing values, which varied widely, including representations such as "-1" "missing" or simply blank spaces. For imputation, we utilized the mean for continuous variables (calculated within site) and introduced a special category (NULLIMP) for categorical variables. We converted all categories into numerical values based on a predefined mapping that covered all potential categories across the datasets. Although this approach introduces an ordinal relationship and potential bias is created among features, we disregarded this concern because the methodology was uniformly applied across all datasets intended for training local, distributed and centralised. These preprocessing tasks were executed once for each dataset and silo.

However, in the context of training classification models, it is crucial that all classes of the target variable are known at the time of training and are represented in each split of the cross-validation process. To address this, we employed \ac{smote} \cite{smote} to up-sampled low-frequency target classes. We established a threshold of n$<$25 for low-frequency variables to ensure that each cross-validation split contained at least two instances of the class—although a minimum of 10 instances (10 splits) might suffice, we opted for 25 to mitigate potential distribution issues and have at least two examples of the class in each split. Additionally, we created dummy rows for missing target classes by imputing the mean for continuous variables and the mode for categorical variables (calculated within site). The necessity for up-sampling and missing variable creation was evaluated and applied as needed for each training session and for each target, considering that each session's split could result in a training set lacking instances of low-frequency classes.

All procedures were coded in python 3.9.7 with the usage of the scikit-learn library \cite{scikit-learn} and mlxtend library \cite{mlxtend}.


\subsubsection{Model Training}
To avoid pitfalls of inductive bias from a certain learning strategy, we learned six different models (i) Decision Trees, (ii) Bayesian methods, (iii) a logistic regression model with Stochastic Gradient Descent, (iv) \ac{knn}, (v) AdaBoost and (vi) Multi-layer Perceptron. The decision was to create diversity in the models used, in order to assess if the training methodology could have an impact on distributed model creation.
The distributed model was an ensemble of models from each silo on a weighted soft-voting basis. The weights were defined by weighted averages of the scores each model obtained in the training set. Then the final result is obtained by creating a weighted average of the class predictions for classification and a weighted average for regression. A model like this can be implemented with peer-to-peer or federated approaches.
Nineteen features were used as target outcomes. These features were selected by filtering by the percentage of null values (below 50\%). This choice was related to maintaining a equilibrium between having a wide range of variables to test how the target variables affects the outcome and having target variables that did go through an harsh imputation mechanism. For categorical outcomes, thirteen were selected (AA - Position Admission; ANP - Position on Delivery; AGESTA - Nr of Pregnancies; APARA - Nr of born babies; GS - Blood Group; GR - Robson Group; TG -Pregnancy Type; TP - Delivery Type; TPEE - Spontaneous Delivery; TPNP - Actual Type of Delivery; V - Followed physician; VCS - Followed physician primary care; VNH - Followed physician hospital delivery;). For continuous variables, six were selected (IA - Mother Age; IGA - Weeks on Admission; IMC - BMI; NRCPN - Nr of consultations; PI - Weight start of pregnancy; SGP - Weeks on Delivery;). Details can be seen in tables \ref{tab:1} and \ref{tab:1.2}.
Local models were built with each silo's data. The centralised model was trained with a training dataset from all the silos combined. 


%TC:ignore
\subsubsection{Model Performance Evaluation}

All models were built for a certain outcome variable with a repeated cross-validation (2 times and 10 splits each) and then compared, over 10 stochastic runs, with evaluation being performed on a test set held out from each silo. By performing cross-validation twice, we aimed to generate a more robust estimation of the model’s performance metrics by averaging the results over two separate runs, each partitioning the data differently. This approach is particularly useful in scenarios where data is limited or highly variable, as it provides a clearer insight into the model's expected performance in unseen data scenarios. The metrics used for classification models were Weighted \ac{auroc} computed as One-versus-Rest, Weighted \ac{auprc}. The metrics for regression models were \ac{rmse} and \ac{mae}. The algorithm is shown in the algorithm \ref{alg:1}. This rendered over 1000 different combinations. When a variable was used as outcome to predict, all others were used as predictors.




\begin{algorithm}[hbtp]
\caption{Creation and evaluation of the 3 different models. We first preprocessed data. Then for each target, we created a distributed and centralised model. Then, over 10 repetitions per silo, we created a new train and test set and local model and tested the centralised, distributed and local on this test set.}
\label{alg:1}
Pre-process all silos (null standardization, imputation, encoding)\;
\For {target in target list}{ 
  \For{n in 10 repetitions}{

\For {silo in imputed silos}{
Train-Test Split (80:20)\;
 check for low frequency or nonexistent labels in train set \;
 train local model with hyper-parameter tuning with 2x10 repeated \ac{cv} \;
 define weights based on scores in the train set (weighted average for predicting the value) for the distributed model\;

  }

  Create distributed (ensemble of all models) model with weights\;
  predict local on the test set\;
  predict distributed on the test set\;
\vspace{3mm}


Create a centralised model with all the data with a 2x10 repeated \ac{cv} \;
Test the centralised model on the test set\;
}}



 \end{algorithm}
%TC:endignore

After all the data was collected, we used the standard independent 2-sample T-test to check if the differences were significant with a $\alpha$ of 0.05. First, we compared the overall performance of the distributed model vs their centralised and local counterpart.  We also compared every distributed model per algorithm and sequentially the centralised and correspondent local model across all algorithms and repetitions and outcome variables with 2-sample T-test as well.



\subsection{Results}
% !TeX root = ../../thesis.tex

Table \ref{tab:allvsall} shows the aggregated metrics for \ac{auroc}, \ac{auprc}, \ac{rmse} and \ac{mae} for distributed, centralised and local models predicting capabilities on each silo. The data refers to the mean of the metric values for all columns tested as targets for all methods and all silos. We also calculated the 95\% confidence interval for each model (local and distributed per silo) in order to assess how well the distributed model would work as opposed to the local one per silo. We also calculated the \textit{P} value for the means of the distributed vs centralised and distributed vs local.
%TC:ignore
%\setlength{\tabcolsep}{7pt} % Default value: 6pt
%\renewcommand{\arraystretch}{1.3} % Default value: 1

\begin{table}[h!] 
 \setlength{\tabcolsep}{7pt} % Default value: 6pt 
 \renewcommand{\arraystretch}{1.3} % Default value: 1
  \captionsetup{justification=centering} 
\centering
\caption[Metrics for centralised model, distributed model and local model]{Comparison of the distributed model with the centralised model and with the local model (Mean for all model and all columns). 2-sample T-test for the means was used as hypothesis test. Bold for \textit{P} value below 0.05. AUPRC and AUROC for categorical target variable and RMSE and MAE for continuous target variable.}

\label{tab:allvsall}
\begin{tabular}{llcccc}
\toprule
 &  & M & SD & 95\% CI & \textit{P}  \\
\midrule
\multirow{3}{*}{AUPRC}
 & distributed & 0.691 & 0.216 & (0.686, 0.696) & - \\
  & centralised & 0.706 & 0.225 & (0.701, 0.711) & \bfseries 1.10e-17 \\
 & local & 0.659 & 0.220 & (0.654, 0.665) & \bfseries 4.71e-05 \\
 \hline

\multirow{3}{*}{AUROC} 
 & distributed & 0.723 & 0.182 & (0.718, 0.727) & - \\
 & centralised & 0.729 & 0.180 & (0.725, 0.734) & \bfseries 2.98e-26 \\
 & local & 0.692 & 0.164 & (0.688, 0.695) & \bfseries 2.48e-02 \\

\hline

\multirow{3}{*}{MAE} 
 & distributed & 2.370 & 1.608 & (2.315, 2.425) & - \\
 & centralised & 2.365 & 1.923 & (2.298, 2.431) & \bfseries 2.23e-04 \\
 & local & 2.527 & 1.799 & (2.465, 2.589) & 9.01e-01 \\

\hline

\multirow{3}{*}{RMSE} 
 & distributed & 21.171 & 46.078 & (19.584, 22.757) & - \\
 & centralised & 19.839 & 28.645 & (18.853, 20.826) & \bfseries 2.92e-02 \\
 & local & 23.771 & 49.776 & (22.057, 25.485) & 1.63e-01 \\
\hline
\end{tabular}
\end{table}





%TC:endignore

Figure \ref{fig:heatmap-cat} shows the \ac{auroc} of each algorithm and silo on the Y axis and target variable and type of model on the X. The color bar refers to the value of the \ac{auroc}. Blue being lower values and red bigger values. The same type of graph was created for regression, where the Figure \ref{fig:heatmpa-int} shows the \ac{mae} for each silo and algorithm and target variable and type of model. 


%TC:ignore

\begin{figure}[h!]
\centering
\captionsetup{justification=centering}

\caption[Heatmap of classification algorithm and silo vs Target variable and model type.]{Heatmap of classification algorithm and silo vs Target variable and model type. Value is the \ac{auroc} mean of all 10 experiments. Y axis is the algorithm and silo. X axis is Target variable and Method. AA - Position Admission; ANP - Position on Delivery; AGESTA - Nr of Pregnancies; APARA - Nr of born babies; GS - Blood Group; GR - Robson Group; TG -Pregnancy Type; TP - Delivery Type; TPEE - Spontaneous Delivery; TPNP - Actual Type of Delivery; V - Followed physician; VCS - Followed physician primary care; VNH - Followed physician hospital delivery;}\label{fig:heatmap-cat} 
\includegraphics[scale=0.22]{figures/heatmap-class.png}
\end{figure}
%TC:endignore
%TC:ignore

\begin{figure}[htbp]
\centering
\captionsetup{justification=centering}

\caption[Heatmap of regression algorithm and silo vs Target variable and model type.]{Heatmap of regression algorithm and silo vs Target variable and model type. Value is the \ac{mae} mean of all 10 experiments. The y axis is the algorithm and silo. X axis is Target variable and Method. IA - Mother Age; IGA - Weeks on Admission; IMC - BMI; NRCPN - Nr of consultations; PI - Weight start of pregnancy; SGP - Weeks on Delivery;}\label{fig:heatmpa-int} 
\includegraphics[scale=0.22]{figures/heatmap-reg.png}
\end{figure}
%TC:endignore

\definecolor{Gray}{gray}{0.85} 
 \begin{table}[h] 
 \setlength{\tabcolsep}{6pt} % Default value: 6pt 
 \renewcommand{\arraystretch}{1.1} % Default value: 1
  \captionsetup{justification=centering} 
\centering
\caption[Model comparison: Distributed versus centralised and local for every test]{Model comparison: Distributed versus centralised and local for every test. Each cell is the total of distributed model when compared with centralised model (row) and local model (column) across different silos and outcome variable. ($>$ for better, = for non significance and $<$ for worse). The first example is 72 which means that 72 iterations of the distributed SGD was better than the centralised and local. SGD: Stochastic Gradient Descent, NN: Neural Network, KNN: K-Nearest neighbours, ADA: AdaBoost, NB: Naive Bayes, DT: Decision Tree. Comparison was done with 2-sample T-test with a $\alpha$ of 0.05. (\% in parentheses)}
\label{tab:hyp}
\newcolumntype{L}{>{\scriptsize}l}  % "small" can be changed to "scriptsize" or "footnotesize" for even smaller text
\newcolumntype{R}{>{\scriptsize}r}  % "small" can be changed to "scriptsize" or "footnotesize" for even smaller text

\begin{tabular}{LLRRRR}
\toprule


 &  & Distributed $>$ Local & Distributed = Local & Distributed $<$ Local & \textbf{Row Total} \\


\hline \multirow{3}{*}{SGD} &Distributed $>$ Centralised  & 72 (7.0) & 14 (1.4) & 9 (0.8) & \textbf{95 (9.3)} \\
 & Distributed =  Centralised& 14 (1.4) & 17 (1.7) & 6 (0.6) & \textbf{37 (3.6) } \\
 & Distributed $<$ Centralised  & 11 (1.1) & 11 (1.1) & 17 (1.7) & \textbf{39 (3.8)} \\
% \hline
% \multicolumn{2}{c|}{SGD Total} & 97 (9.4)&42 (4.1) &32 (18.7)& \textbf{171}  \\
\hline \multirow{3}{*}{NN} & Distributed $>$ Centralised & 44 (4.3) & 44 (4.3) & 7 (0.7) & \textbf{95 (9.3)} \\
 & Distributed =  Centralised& 2 (0.2) & 33 (3.2) & 2 (0.2) & \textbf{37 (3.6)} \\
 & Distributed $<$ Centralised & 0 (0) & 17 (1.7) & 22 (2.1) & \textbf{39 (3.8)} \\
% \hline

% \multicolumn{2}{c|}{NN Total} & 46&94 &31 & \textbf{171}  \\

\hline \multirow{3}{*}{KNN} & Distributed $>$ Centralised & 16 (1.6) & 0 (0) & 1 (0.1) & \textbf{17 (1.7)} \\
 & Distributed =  Centralised & 10 (1) & 2 (0.2) & 1 (0.1) & \textbf{13 (1.3)} \\
 & Distributed $<$ Centralised & 72 (7)  & 28 (2.7) & 41 (4) & \textbf{141 (13.7)} \\
% \hline

% \multicolumn{2}{c|}{KNN Total} & 97&30 &43 & \textbf{171}  \\

\hline \multirow{3}{*}{ADA} & Distributed $>$ Centralised & 64 (6.2) & 25 (2.4) & 22 (2.1) & \textbf{111 (10.8)} \\
 & Distributed =  Centralised& 5 (0.5) & 12 (1.2) & 10 (1) & \textbf{27(2.6)} \\
 & Distributed $<$ Centralised & 10 (1) & 6 (0.6) & 17 (1.7) & \textbf{33 (3.2)} \\
% \hline

% \multicolumn{2}{c|}{ADA Total} & 79&43 &49 & \textbf{171}  \\

\hline \multirow{3}{*}{NB} & Distributed $>$ Centralised & 51 (5) & 19 (1.9) & 34 (3.3) & \textbf{104 (10.1) } \\
 &  Distributed =  Centralised & 5 (0.5) & 19 (1.9) & 12 (1.2) & \textbf{36 (3.5)} \\
 & Distributed $<$ Centralised  & 3 (0.3) & 4 (0.4) & 24 (2.3) & \textbf{31 (3)} \\
% \hline

% \multicolumn{2}{c|}{NB Total} & 59&42 &70 & \textbf{171}  \\

\hline \multirow{3}{*}{ DT} & Distributed $>$ Centralised & 27 (2.6) & 0 (0) & 1 (0.1) & \textbf{28 (2.7)} \\
 & Distributed = Centralised & 8 (0.8) & 0 (0) & 0 (0)& \textbf{8 (0.8)} \\
 & Distributed $<$ Centralised & 97 (9.5) & 12 (1.2) & 26 (2.5) & \textbf{135 (13.2)} \\
% \hline

% \multicolumn{2}{c|}{DT Total} & 132&12 &27 & \textbf{171}  \\
 
 \hline
  \textbf{Total} &  & \textbf{511 (49.8)} & \textbf{263 (25.6)} & \textbf{252 (24.6)} & \textbf{1026 (100)}\\
 \bottomrule
\end{tabular}
\end{table}

















\subsection{Discussion}



The imputation process was done using the mean value (for continuous variables) or a new category (NULLIMP) for categorical variables. All categories were encoded as numbers using a previous mapping created based on all possible categories in all silos. Even though an ordinal relationship is created among features, we believe that since we are applying this methodology to all datasets, which will be the source for all tests (local, distributed and centralised), that fact may be ignored.
When training classification models, all of the target variable classes must be known at that moment and should be present in each split of the cross-validation. So, when assessing the training dataset, low-frequency target classes (n $<$ 25) were up-sampled with \ac{smote} \cite{smote} and missing target classes were addressed with dummy rows creation by the imputation of the mean for continuous variables and mode for categorical variables (per silo). These preprocessing mechanisms were applied in each run and for each target.
The distributed model was an ensemble of models from each silo on a weighted soft-voting basis, defining weights and thresholds based on the training set scores. 
The first thing that is noticeable is the high scores achieved in our analysis which show that all algorithms in all forms (local, distributed and centralised) have a good grasp on ranking data (negative on the bottom and positives on the top of a scale) for classification or predicting the value for regression. We notice that distributed models have performance similar to their centralised counterparts. $\sim$59\% of all of the distributed models had similar or better performance than the centralised models. This suggests that a distributed model can be used to reliably infer information and does not compromise prediction performance when compared with the gold standard (centralised) while increasing privacy for the data owners.\\
Overall, our results suggest that it is possible to implement a distributed model without significantly losing information. However, there are still issues to be addressed. This methodology presents hurdles regarding categorical class handling. Firstly, all classes should be known first-hand and should be given to each model even if that silo in particular has no cases of that class. Secondly, low-frequency classes are also an issue to be addressed, since training the model with cross-validation will raise problems because each split should have all classes present. Our approach relied on sample creation for low and non-existent target classes. However, this approach is adding information to the model that is not originally there. The way we chose for minimising this issue was by creating dummy variables with median and mode imputations based only on the information in the dataset. Nevertheless, non-existent classes are impossible to address without prior information. These class problems could be partially tackled in production by implementing data management and governance procedures, namely data dictionaries. Still on data preprocessing, we applied ordinal encoding to the variables which will create a natural hierarchy between variables. One solution for this is to create binary columns for each class in each column. This will remove the hierarchy between classes but increase variable numbers and training time considerably.\\
Moreover, like in most secondary usage of data, other issues are important to keep in mind, even in such a controlled environment as this one. Even though the software is the same in every hospital, the clinical service is the same and the underlying data models are the same, the version of the software is not the same across all hospitals. This difference alone can alter the way each column is populated, mainly through front-end changes or label modification, among other aspects. Additionally, each hospital has its own workflows in practice that can also alter the way data is collected; changing timings or steps in a certain workflow can dramatically change the data acquisition and the reality it represents. \\
Another issue to consider is the path adopted to build the distributed model. In this case, it was decided to develop an ensemble of models with voting. However, other methods could have been employed, like parameter averaging, that should be tested as well. In particular, the usage of more robust neural networks could be assessed as well. We chose not to test state-of-the-art neural networks since the data volume was low for that use case and several papers have already demonstrated that neural networks are not the most suitable tool for tabular data \cite{grinsztajnWhyTreebasedModels2022,borisovDeepNeuralNetworks2022}. We chose to add MLPerceptron as a baseline for comparison with the remaining algorithms. The results show us that the performance was below the other algorithms, but in this concrete case, the problem may reside in the architecture chosen and hyperparameters used in the Cross-validation. Despite this, a precise and thorough demonstration of this use case would be important to consider such scenarios. \\
Furthermore, the algorithm underlying the distributed model is of importance as well for its performance versus the centralised model. Figures \ref{fig:heatmap-cat} and \ref{fig:heatmpa-int} and table \ref{tab:hyp} show us that Decision trees and K-nearest neighbours implemented in a centralised manner are consistently better than the distributed counterpart. Even though this improvement may have a relationship to the target variable (i.e figure \ref{fig:heatmpa-int} for IA and IGA variables), it is still an important fact to take into account when implementing such architectures. The performance of the models is also interesting to catch differences in silos. See silo 6 for TPNP (figure \ref{fig:heatmap-cat}) where silo 6 consistently behaves differently than the rest.
As for implementation, such a mechanism could be implemented in at least two manners; with a central orchestrator or without. The first one would assume a central point that would make a request to each silo for a prediction and then create the final prediction with the weighted averaging of each one. The second one would not require any additional platform and each silo would communicate with each of the others and receive the prediction and would create the final with their own. This implementation step would of course take into account variables that we were out of scope such as the communication between silos. 
Regarding the prediction capability as a whole, we found that this data is suitable to apply machine-learning models in order to predict several clinical outcomes, with very good results for several target variables. 

%Another topic that can be explored is the fact that a lower performance of the ensemble in a certain silo and/or target could be actually useful to characterise the population of that silo.

\subsection{Conclusion}
% !TeX root = ../../thesis.tex

This research demonstrated the efficacy of distributed models using real-world data by comparing their performance with that of local models, which are trained with data from individual silos, and centralized models, which utilize data from all silos. The findings reveal that an ensemble of models, essentially a distributed model as investigated in this study, can capture the nuances of the data, achieving performance comparable to a model constructed with comprehensive data. Even though The performance of these models is influenced by factors such as the inherent characteristics of the target variables and the data distribution across different silos, we are now fairly confident that distributed learning is a step forward regarding data privacy without loss of predictive performance when compared with centralised and local models.
Considering the robust performance metrics observed, with AUROC/AUPRC scores exceeding 80\% and MAE maintained below 1, further investigation into distributed models is warranted. Specifically, we aim to develop distributed models for predicting clinical outcomes, such as delivery type or Robson Group classifications, which hold significant potential for real-world clinical application like reducing unnecessary Cesarean Sections or accelerating diagnosis. These findings underscore that distributed learning not only advances data privacy but also maintains high prediction accuracy, promising substantial benefits for clinical practices.



\section{Can Institutions share their performance metrics without hesitation of retaliation?}\label{subsec:benchmark}
This section is based on the paper entitled "Benchmarking institutions' health outcomes with clustering methods". This paper was focused on the fact that many healthcare institutions harbor reservations about openly sharing production metrics. One predominant concern is the potential for retaliatory actions, be it from regulatory bodies, competitors, or the public. In this paper, we propose the application of a clustering methodology that allows institutions to compare performance metrics without disclosing the actual values. The method is based on clustering, which involves grouping health institutions' outcomes into a known number of clusters, allowing institutions to position themselves in a range of clusters without sharing the true means of their target data. The proposed method uses the K-means and K-modes clustering algorithms and was tested on data from real Electronic health records and public datasets. This approach provides a valid benchmark of hospital metrics and performances while protecting the privacy of participating institutions. 
\subsection{Introduction}
Health institutions play a critical role in providing essential healthcare services to communities and ensuring that they operate efficiently and effectively is crucial. Benchmarking is a process that allows hospitals to compare their performance against that of other institutions, which can help identify areas of strength and weakness \cite{suydamPatientSafetyData2007}. By analysing and evaluating performance metrics, such as patient outcomes, operational efficiency, and financial management, hospitals can identify best practices and make data-driven decisions to improve their overall performance. It can also help hospitals identify and implement innovative practices that can lead to better patient care and improved staff satisfaction \cite{hulsenSharingCaringData2020}.

However, despite the numerous benefits of benchmarking, some hospitals may be hesitant to participate due to concerns about revealing weaknesses or being perceived as inferior to their peers. The fear of being judged or penalized for poor performance can sometimes lead hospitals to avoid sharing data, making it difficult to accurately assess their performance and identify areas for improvement. Privacy issues and concerns turn this opportunity into an even less desirable path \cite{hulsenSharingCaringData2020}. To address these concerns, benchmarking initiatives often ensure the confidentiality and anonymity of data to encourage participation and foster trust among participating institutions. However, this is usually not enough. In 2019, as stated in the work of Villanueva et al., \cite{villanuevaCharacterizingBiomedicalDataSharing2019}, 26\% of data-sharing initiatives are based on the aggregation of data and 24\% are based on sharing data in closed consortia. Only 15\% were based on open or controlled access.

To address concerns around privacy and confidentiality, we propose a new method of benchmarking based on clustering. This method involves grouping health institutions' outcomes into a known number of clusters, providing health institutions with the capability of positioning themselves in a range of clusters, without ever sharing the true means of their target data.

This approach to benchmarking not only addresses concerns around privacy and confidentiality. It has the potential to encourage greater participation in benchmarking initiatives, as hospitals can be assured of the anonymity and confidentiality of their data. By creating a more secure and private environment for benchmarking, hospitals can feel more comfortable sharing their data and participating in initiatives that can ultimately improve patient care and operational efficiency.

In conclusion, benchmarking is a crucial tool for hospitals to improve their performance and provide better care for their patients. While concerns around privacy and confidentiality may exist, the clustering approach to benchmarking provides a more accurate assessment of hospital performance while protecting the privacy of participating institutions. By embracing benchmarking initiatives and leveraging new approaches to benchmarking, hospitals can continuously improve their operations and ensure they provide the highest quality of care possible.
In this paper we propose:
\begin{myitemize}
    \item study how to implement clustering mechanism for benchmark
    \item address preprocessing issues for the raw data
    \item highlight pain points to deployment in the real world.
\end{myitemize}
\subsection{Rationale and Related Work}
This work was initially suggested as a follow-up to a previous work of Rodrigues et al., \cite{rodriguesLocalAlgorithmApproximate2018a} where clustering is applied to streaming data sources. We then thought if a similar approach could be applied to healthcare in order to be able to compare data distributions without ever knowing their real values of them.
Clustering in healthcare is often used to create clusters of patients, taking into account a given set of characteristics. This is used to find possible groups of phenotype and be able to characterise populations given the centroids \cite{walkerUnsupervisedLearningTechniques2019,basileInformaticsMachineLearning2018}. It is also used as a method of detecting regularities and patterns in multi-omics data that reveal different molecular subtypes \cite{nicoraIntegratedMultiOmicsAnalyses2020,rappoportMultiomicMultiviewClustering2018}. It can also be used to create unsupervised models for facilitating the annotation of data for supervised models \cite{mcalpineUtilityUnsupervisedMachine2022}. 

K-means \cite{lloydLeastSquaresQuantization1982,steinley2007initializing,macqueen1967classification} is an unsupervised clustering algorithm used to group data points into K distinct clusters based on their similarity. It is widely used in \ac{ml}, data mining, and image segmentation. The algorithm works by randomly initializing K centroids (or cluster centres) and assigning each data point to the nearest centroid. Then, the centroids are moved to the mean of the points assigned to each cluster. This process is repeated until convergence, where the clusters no longer change.

The objective of K-means is to minimize the sum of squared distances between each data point and its assigned centroid, which is also called the within-cluster sum of squares (WCSS). The algorithm attempts to find the best K clusters that minimize the WCSS. However, choosing the right value of K can be challenging, and the algorithm may converge to a suboptimal solution. Therefore, K-means is often run multiple times with different initializations to find the best clustering solution. Despite its simplicity, K-means can be computationally expensive when dealing with large datasets, and it may not work well with non-linearly separable data or when the clusters have different shapes and sizes.

K-modes is another clustering algorithm similar to K-means, but it is designed to work with categorical data. Unlike K-means, which computes the mean of continuous variables, K-modes computes the mode (or the most frequent value) of categorical variables within each cluster. The algorithm works by randomly initializing K centroids and assigning each data point to the nearest centroid based on the number of matching categories. Then, the centroids are moved to the mode of the categories within each cluster. This process is repeated until convergence, where the clusters no longer change.

The objective of K-modes is to minimize the dissimilarity between the data points within each cluster, which is often measured by the Hamming distance, Jaccard distance, or other similarity measures. Like K-means, choosing the right value of K is critical, and the algorithm may converge to a suboptimal solution. Therefore, K-modes is often run multiple times with different initializations to find the best clustering solution. K-modes is particularly useful when dealing with data that have a large number of categorical variables or when the data contain missing values. However, like K-means, K-modes may not work well with non-linearly separable data or when the clusters have different shapes and sizes.

However, as far as we know, this is the first time clustering is tested for exchanging information privately.

\subsection{Materials \& Methods}
\subsubsection{Method Overview}
We used Python 3.9 to implement the mock example of such an use-case. The clustering was done with \textit{scikit-learn} library \cite{scikit-learn}. The algorithm proposed is shown in algorithm \ref{alg:bench1}.

%TC:ignore
\begin{algorithm}[hbtp]
\caption{Benchmarking with clustering}
\label{alg:bench1}

\SetAlgoLined



\For {variable in silo}{ 
initialize centroids\;
%\begin{itemize}
%    \item real cluster obfuscated with noise
%    \item true centroids
%    \item add noise to data and then create centroids
%\end{itemize}
}
%\SetKwRepeat{Do}{do}{while}
\While{No convergence}{
\begin{itemize}
    \item Send centroids to other silos
\item Receive other silo's information and add own centroids
\item Calculate new centroids
\item calculate score
\end{itemize}
}



\end{algorithm}
%TC:endignore

The method for assessing convergence is based on clustering metrics: the \ac{ri}. This metric computes a similarity measure between two clusters by considering all pairs of samples and counting pairs that are assigned in the same or different clusters in the predicted and true clusters \cite{hubertComparingPartitions1985}. The raw RI score is: $RI = (number\; of\; agreeing\; pairs) / (number\; of\; pairs)$.
Furthermore, convergence must be obtained through several iterations to make sure it's stable, so a buffer period is also important. For the results section, we set the threshold as 0.9 and repetitions at 20.

In this paper, we propose to show how such an implementation could be done while addressing issues with data formats, types and preprocessing. So, we want to check if the encoding of categorical data affects the model and which method is better for encoding such variables. Additionally, we will try to understand if it is possible to create mechanisms for mixed data if categorical and continuous data must be used and evaluated separately and if so, through which mechanisms.
We will test (1) continuous variables alone, and (2) encoded categorical variables as ordinal. We will also test (3) K-modes  and (4) K-means with the proportion of each category for categorical data.
K-means was used as implemented in \textit{scikit-learn} \cite{scikit-learn} and K-modes, as implemented by J. de Vos \cite{devos2015}.

\subsubsection{Data used}
We used two types of data in this paper. One is simpler and available online from the \ac{uci} dataset library, namely, the heart disease dataset \cite{misc_heart_disease_45}. We made fairly simple preprocessing on that dataset, namely removing the "?" by filling with null and then imputing missing values by imputing the mean on continuous variables and mode on categorical ones. We then separated the data into 3 distinct silos at random to mimic different health institutions.

In order to use real data and address problems found in the wild, we used clinical data gathered from nine different Portuguese hospitals regarding obstetric information, pertaining to admissions from 2019 to 2020. This originated from nine different files representing different sets of patients but with the same features associated with them. The software for collecting data was the same in every institution (although different versions existed across hospitals) - ObsCare. The data columns are the same in every hospital's database. Each hospital was considered a silo for comparison.
\subsection{Results}
As for results, the data from heart disease rendered the figure \ref{fig:cluster_free_3s}. In this, we focused on continuous variables only. For easier reading, the data is as shown in the table \ref{tab:datapoints}. We used data from the real world to test if everything would work similarly, rendering the image \ref{fig:cluster_mydata_9s}. We added a binary category to show how meaningless the value turn in order to get any information out of it.




%TC:ignore

\begin{figure}[htpb]
\centering
\captionsetup{justification=centering}
\caption[Clustering for 3 continuous variables with 3 silos]{Clustering for 3 continuous variables with 3 silos and true centroids (S2) and true means (S2) for example purposes; The values were normalized for visualization purposes with MinMax}
\includegraphics[scale=0.50]{figures/my_cluster_3.png}
\label{fig:cluster_free_3s} 
\end{figure}



%TC:endignore
\begin{table}[htbp]
\centering
 \setlength{\tabcolsep}{7pt} % Default value: 6pt 
 \renewcommand{\arraystretch}{1.35} % Default value: 1
  \captionsetup{justification=centering} 
\caption[Final Data points after convergence of clustering]{Final Data points after convergence; S1, S2 and S3 are the centroids obtained in each silo (S) after convergence; True centroids are the centroids of the true means of all silos (TC)}
\label{tab:datapoints}
\begin{tabular}{lccc}
\toprule
 & Age & trestbps & chol \\
\midrule
S1 & 46.3 , 61.1 & 121.1 , 148.9 & 218.9 , 300.8 \\
S2  & 45.8 , 61.0 & 120.7 , 149.9 & 220.9 , 304.0 \\
S3 & 45.5 , 61.0 & 120.5 , 149.6 & 216.1 , 297.4 \\
TC  & 45.6 , 60.8 & 121.0 , 149.6 & 215.8 , 297.9   \\

\bottomrule
\end{tabular}
\end{table}




%TC:ignore



\begin{figure}[H]
\centering
\captionsetup{justification=centering}
\caption[Clustering for 3 variables with 9 silos]{Clustering for 3 variables with 9 silos and true centroids of the true means (TC); 2 continuous and 1 categorical one hot encoded, The values were normalised for visualisation purposes with MinMax}\label{fig:cluster_mydata_9s} 
\includegraphics[scale=0.60]{figures/my_cluster_9.png}
\end{figure}
%TC:endignore

As before, the data is in table format in \ref{tab:datapoints_9}.

%%True Centroid & 24.9 , 25.3 & 66.6 , 65.6  & 0.24 , 0.29 \\

\begin{table}[htbp]
\centering
 \setlength{\tabcolsep}{7pt} % Default value: 6pt 
 \renewcommand{\arraystretch}{1.35} % Default value: 1
  \captionsetup{justification=centering} 
\caption{Final Data points after convergence and true centroids of the true means of each silo (TC)}
\label{tab:datapoints_9}
\begin{tabular}{lccc}
\toprule
 & \acs{bmi} & Initial Weight & Birth by Cesarian \\
\midrule
TC & 24.9 , 383.1 & 60.5 , 85.0  & 0 , 1 \\
S1 & 40.1 , 409.4 & 60.4 , 85.0 & 0.96 , -0.04 \\
S2 & 40.1 , 410.4 & 61.7 , 86.3 & 0.99 , -0.01 \\
S3 & 40.0 , 410.4 & 61.9 , 86.5 & 0.96 , -0.04 \\
S4 & 40.6 , 411.3 & 61.9 , 86.5 & 0.96 , -0.04 \\
S5 & 40.0 , 410.4 & 60.5 , 85.1 & 1.0 , 0.0 \\
S6 & 40.1 , 409.3 & 60.4 , 84.9 & 1.0 , 0.0 \\
S7 & 40.7 , 411.3 & 60.5 , 85.0 & 0.96 , -0.04 \\
S8 & 40.0 , 410.4 & 86.5 , 61.9 & 1.0 , 0.0 \\
S9 & 41.0 , 410.4 & 85.0 , 60.4 & 1.0 , 0.0 \\
\bottomrule
\end{tabular}
\end{table}




Then we experimented with categorical variables. Figure \ref{fig:cluster_3_cat} shows the convergence of the silos with proportion data and K-means with that and with K-modes.
\begin{figure}[ht]
\caption{Clustering for 3 variables with 3 silos - (A) categorical variables with  proportion with K-Means and (B)  Categorical with K-modes  }\label{fig:cluster_3_cat} 
  \subcaptionbox*{(A)}[.60\linewidth]{%
    \includegraphics[width=\linewidth]{figures/my_cluster_3_cat.png}%
  }%
  \hfill
  \subcaptionbox*{(B)}[.44\linewidth]{%
    \includegraphics[width=\linewidth]{figures/my_cluster_3_cat_kmodes.png}%
  }
\end{figure}


\subsection{Discussion}
As per the discussion, there are a few issues to be addressed. First as per data preprocessing. In order to cluster be obtained, the null data must be filled out. There are a few strategies to do so. One option is to eliminate records/rows with empty cells or impute data. Either is a possibility, with pros and cons but the capability of having a dataset where no null records are present across several features may be difficult to find in the wild, especially since there are often optional and conditional fields in most Electronic Health Records (EHR). So imputation becomes more interesting, since it enables the usage of the whole dataset, even if biases are introduced.
Mixed types of datasets are also an issue to be aware of. In this case, not only imputation but also encoding a categorical variable is a vital step to take in the preprocessing phase. There are usually two main methods of data encoding, ordinal encoding and binary encoding. The first one keeps a unique column as the original data but maps every category to an increasing natural number. This creates an ordering in the data, often a misrepresentation of reality, not only due to this hierarchy but only because it assumes the differences between ranks of the hierarchy are always the same (1). The second is related to expanding the number of columns into the number of categories and creating 0s and 1s for the category. In machine-learning terms, binary seems more suited to be applied, but for benchmarking purposes, both are below par in terms of interpretability. For categorical data, we found out that K-modes seem to fulfil the requirements in a better way, providing better interpretability and reasoning about the results. However, it should be noted that we applied K-modes in a multivariate fashion and K-means in a univariate fashion.
Given that no percentage is provided, only the mode of the data, we believe it is still hard to get any real insight from the centroids. However, K-modes provides less information, since it only shows the top two categories. Which, for example. binary targets, provide little to no information. However, for larger categorical sets, the information provided could be better. Moreover, the number of centroids pretended could be more important as well. Agreeing on only 1 centroid would render the mode of the data provided by all silos, which could be more interesting.
As for continuous data, the use of real data was insightful, since BMI had a few very big outliers around 300 and 400, which rendered centroids around that data. Even if not all silos had examples of these outliers, the ones that do have, pass that into the remaining. One possible workaround would be an addition of an extra cluster in order to catch possible outliers.
However, this should be addressed in detail and assess how outliers could subvert the data from the silos and how to work around that.

% should be addressed more in focus since it was outside of the scope of this paper
As for the next steps, a few issues could be addressed in depth. Regarding imputation, it could be interesting to understand how imputation, and which methods are more suitable to use for real-world scenarios. If the imputation of variables with a high null percentage influence significantly a centroid formation.
Communication could be important as well. Which action is to be taken when a silo is "down" and does not send information to the remaining. Cluster information should be addressed as well. They need to be agreed upon beforehand in the scope of this paper. But if it could be selected by each silo? Would that be feasible or a convergence could be achieved?
Finally, there is the question if there is the possibility of having leaks of true means across iterations by adversarial learning. At present time, we cannot be sure that the values are totally private, but then again, nothing is.


% abordar na questao a vantagem da privacidade mas que é possivel haver leak dos valores por adervisal learning.

\subsection{Conclusion}
We believe that this work helps create the foundation for exchanging data across healthcare institutions without revealing the true data points. It could be useful for benchmarking and promoting a higher adoption rate.
Even though there are still issues to be addressed, we think that the path is full of possibilities.




\section{Leveraging data to assess treatment efficacy}\label{subsec:ipop}
This section is based on the paper entitled "Comparative Analysis of Palbociclib and Ribociclib: A real world data and Propensity Score-Adjusted Evaluation with endocrine therapy". This was a method of applying the knowledge of causality and transparent \ac{ml} models in order to assess the real-world effect of two drugs for breast cancer. We started with traditional analysis and then moved to a more complex approach, using \ac{iptw} methods in order to further compare treatments.


\subsection{Introduction}
% !TeX root = ../../thesis.tex

In recent years, the use of \ac{ai} and \ac{ml} algorithms has gained increasing prominence in healthcare research and practice. One of the key requirements for the successful application of these methods is access to large, high-quality datasets. However, in many cases, the availability of such datasets can be limited due to issues around data privacy, security, and ethical concerns \cite{chingOpportunitiesObstaclesDeep2018a}. To address this challenge, synthetic data has emerged as a promising solution. Synthetic data refers to artificially generated data that closely mimic the statistical properties and patterns of real-world data \cite{mullerEvaluationSyntheticElectronic2022}.

Synthetic data has the potential to overcome many of the limitations associated with real-world data, such as the lack of sufficient data volume, noise, and privacy concerns. Even though there are still doubts if the privacy part is the silver bullet sometimes referred to \cite{stadlerSyntheticDataPrivacy2020}, the upsampling part is a standard use for years now. However, the quality of synthetic data generated by various techniques can vary significantly, and it is essential to assess the quality of synthetic data before its usage. In healthcare, the assessment of synthetic data is crucial to ensure that it can provide valid insights and inform decision-making processes.

The assessment of synthetic data in healthcare is essential for its successful use in various applications, such as developing predictive models, testing algorithms, and conducting clinical trials. The use of synthetic data can significantly enhance the efficiency and effectiveness of healthcare research and practice. However, it is crucial to ensure that the synthetic data used in these applications are of high quality and validated to provide reliable and valid insights. The evaluation of synthetic data quality involves comparing its statistical properties and patterns with those of the original data. We can assess how similar columns are to each other through several statistical tests, and then we can infer some inter-column properties with methods like cross-validation, where two datasets are split into train tests and cross-tested and then the ratio between the evaluation result of both datasets is used as a metric \cite{mullerEvaluationSyntheticElectronic2022,goncalvesGenerationEvaluationSynthetic2020a}. However, this methodology is a big proxy for such an inter-column relationship. Can we try to provide a better metric than this one to evaluate how similar are the inter-column relationship of two distinct datasets? In this paper, we suggest using feature importance values to create a more explainable and reasonable metric for inter-column relationships.

\subsection{Materials \& Methods}


\subsection{Study Design}

This retrospective study was designed in 2022. The study aimed to evaluate the clinical benefit and long-term survival of patients with HR+/HER2- that started treatment with CDK4/6 inhibitors plus endocrine therapy in different lines of treatment between the 14th of March 2017 and the 31st of December 2021. The follow-up period was set until June 2022. Inclusion criteria: women and men, Hormone receptor-positive and HER2 negative in the primary tumor or metastatic site after biopsy. Exclusion criteria: Patients that had only one ambulatory medication, and patients involved in clinical trials, diagnosed with other neoplasms or with active treatment during the study period. The control group was defined by a population of patients, that were treated with hormone therapy as first-line (due to bone metastases only) between 2015 and 13 of match 2017.
The evaluation of effectiveness will involve overall survival and progression-free analysis. We will compare the two different cyclin-dependent kinase inhibitors in terms of effectiveness in real-world patients and will also compare the effectiveness of this class combined with endocrine therapy against traditional endocrine therapy.


\subsection{Data collection}
All data were collected from medical and administrative records from baseline to last visit or death. The data was collected from Instituto Português de Oncologia – Porto (IPO-P). Table 1 shows a comparison between the groups.
Data included for population treated with CDK4/6 inhibitors plus endocrine therapy: demographic information, age at first diagnosis and age at the beginning of treatment, clinical characteristics and performance status by Eastern Cooperative Oncology Group scale (ECOG), treatment line and treatment schema - CDK4/6 inhibitor and endocrine therapy, stage of cancer, site of metastases (bone, soft tissue, visceral, central nervous system-CNS with or without another site).
Data included for the population treated with endocrine therapy as first-line: demographic information, age at first diagnosis and age at the beginning of treatment, clinical characteristics and performance status by Eastern Cooperative Oncology Group scale (ECOG), stage of the cancer.
For comparison purposes, we used palbociclib and ribociclib since we had a small number of patients treated with abemaciclib (12).

 
\begin{table}
\caption{Descriptive statistics of cyclin-dependent kinase inhibitors group and endocrine therapy group. The Drug/combination refers to the actual drug or the combination for CDK4/6}
\centering
\label{tab:stats_ipop_cdk}
\begin{tabular}{l|lllll}
\toprule
   Column &               Silo 6 &               Silo 7 &               Silo 8 &               Silo 9 &                Agrr. \\
\midrule
       IA &    31.3 \textbf{5.2} &    31.4 \textbf{5.4} &    31.5 \textbf{5.6} &    30.1 \textbf{5.6} &    31.1 \textbf{5.6} \\
       GS & a,rh.. \textbf{42\%} & a,rh.. \textbf{39\%} & a,rh.. \textbf{40\%} & a,rh.. \textbf{42\%} & a,rh.. \textbf{40\%} \\
       PI &   65.6 \textbf{13.5} &   66.0 \textbf{13.7} &   65.6 \textbf{14.1} &   67.4 \textbf{14.6} &   66.0 \textbf{14.1} \\
      PAI &   77.7 \textbf{13.4} &   79.2 \textbf{14.7} &   76.7 \textbf{13.0} &   83.1 \textbf{15.2} &   78.8 \textbf{14.5} \\
      IMC &    24.9 \textbf{5.1} &    24.9 \textbf{7.0} &    24.8 \textbf{8.0} &    25.7 \textbf{5.6} &    25.1 \textbf{7.0} \\
      CIG &   Null \textbf{91\%} &   Null \textbf{91\%} &   Null \textbf{86\%} &   Null \textbf{90\%} &   Null \textbf{88\%} \\
    APARA &    1.0 \textbf{38\%} &   Null \textbf{43\%} &   Null \textbf{41\%} &   Null \textbf{43\%} &   Null \textbf{39\%} \\
   AGESTA &    1.0 \textbf{44\%} &      1 \textbf{43\%} &    1.0 \textbf{42\%} &    1.0 \textbf{40\%} &    1.0 \textbf{42\%} \\
       EA &   Null \textbf{59\%} &   Null \textbf{61\%} &   Null \textbf{69\%} &   Null \textbf{61\%} &   Null \textbf{60\%} \\
       VA &   Null \textbf{79\%} &   Null \textbf{82\%} &   Null \textbf{88\%} &   Null \textbf{82\%} &   Null \textbf{77\%} \\
       FA &   Null \textbf{82\%} &   Null \textbf{86\%} &   Null \textbf{94\%} &   Null \textbf{89\%} &   Null \textbf{83\%} \\
       CA &   Null \textbf{69\%} &   Null \textbf{75\%} &   Null \textbf{85\%} &   Null \textbf{78\%} &   Null \textbf{75\%} \\
       TG & espo.. \textbf{88\%} & espo.. \textbf{85\%} & espo.. \textbf{86\%} & espo.. \textbf{93\%} & espo.. \textbf{85\%} \\
        V &      s \textbf{97\%} &      s \textbf{99\%} &      s \textbf{98\%} &      s \textbf{99\%} &      s \textbf{98\%} \\
    NRCPN &     6.8 \textbf{4.0} &     7.7 \textbf{3.2} &     9.3 \textbf{4.5} &     8.9 \textbf{5.5} &     8.4 \textbf{5.1} \\
       VP &   Null \textbf{68\%} &   Null \textbf{74\%} &   Null \textbf{71\%} &   Null \textbf{78\%} &   Null \textbf{76\%} \\
      VCS &   Null \textbf{53\%} &      s \textbf{87\%} &      s \textbf{63\%} &      s \textbf{87\%} &      s \textbf{68\%} \\
      VNH &   Null \textbf{62\%} &      s \textbf{63\%} &      s \textbf{69\%} &      s \textbf{83\%} &      s \textbf{69\%} \\
        B &   Null \textbf{90\%} &   Null \textbf{53\%} &   Null \textbf{93\%} &   Null \textbf{82\%} &   Null \textbf{83\%} \\
       AA &   Null \textbf{84\%} & apr... \textbf{61\%} &   Null \textbf{89\%} &   Null \textbf{74\%} &   Null \textbf{73\%} \\
       BS &   Null \textbf{99\%} &   Null \textbf{98\%} &   Null \textbf{99\%} &   Null \textbf{95\%} &   Null \textbf{95\%} \\
       BC &  Null \textbf{100\%} &  Null \textbf{100\%} &  Null \textbf{100\%} &   Null \textbf{97\%} &   Null \textbf{97\%} \\
      BDE &  Null \textbf{100\%} &  Null \textbf{100\%} &  Null \textbf{100\%} &   Null \textbf{97\%} &   Null \textbf{97\%} \\
      BDI &  Null \textbf{100\%} &  Null \textbf{100\%} &   Null \textbf{99\%} &   Null \textbf{97\%} &   Null \textbf{96\%} \\
       BE &  Null \textbf{100\%} &  Null \textbf{100\%} &  Null \textbf{100\%} &   Null \textbf{97\%} &   Null \textbf{96\%} \\
       BP &  Null \textbf{100\%} &  Null \textbf{100\%} &  Null \textbf{100\%} &   Null \textbf{97\%} &   Null \textbf{97\%} \\
      IGA &    38.7 \textbf{1.8} &    39.0 \textbf{2.0} &    38.6 \textbf{2.1} &    38.8 \textbf{1.9} &    38.7 \textbf{2.2} \\
     TPEE &   Null \textbf{65\%} &   Null \textbf{64\%} &   Null \textbf{65\%} &   Null \textbf{63\%} &   Null \textbf{65\%} \\
     TPEI &   Null \textbf{92\%} &   Null \textbf{86\%} &   Null \textbf{87\%} &   Null \textbf{94\%} &   Null \textbf{93\%} \\
      RPM &   Null \textbf{85\%} &   Null \textbf{84\%} &   Null \textbf{90\%} &   Null \textbf{94\%} &   Null \textbf{88\%} \\
       DG &   Null \textbf{92\%} &   Null \textbf{88\%} &   Null \textbf{90\%} &   Null \textbf{87\%} &   Null \textbf{89\%} \\
       TP & part.. \textbf{54\%} & part.. \textbf{52\%} & part.. \textbf{48\%} & part.. \textbf{59\%} & part.. \textbf{51\%} \\
      ANP & cefá.. \textbf{93\%} & cefá.. \textbf{94\%} & cefá.. \textbf{95\%} & cefá.. \textbf{94\%} & cefá.. \textbf{94\%} \\
     TPNP & espo.. \textbf{64\%} &  Null \textbf{100\%} & espo.. \textbf{50\%} & espo.. \textbf{65\%} & espo.. \textbf{53\%} \\
      SGP &    38.8 \textbf{1.8} &    39.2 \textbf{1.7} &    38.7 \textbf{2.0} &    39.0 \textbf{1.6} &    38.9 \textbf{2.0} \\
       GR &      1 \textbf{27\%} &      1 \textbf{25\%} &      1 \textbf{21\%} &      3 \textbf{27\%} &      1 \textbf{24\%} \\
       \midrule
N (total) &                12002 &                 8258 &                 6693 &                11786 &                80874 \\
\bottomrule
\end{tabular}

\end{table}




\subsection{Statistical Analysis}
%copied
R was used for statistical analysis. Demographic, clinical characteristics and side effects were analyzed using descriptive statistics (count, percentages and median/range). Kaplan–Meier test was used to determine the median PFS and OS in the entire population and subgroups. Log-rank test was used for comparisons of PFS and OS among different subgroups. Cox Regression was used to assess feature importance and impact. All statistical tests were two-sided, and the significance level was 0.05. The evaluation of the proportional hazards assumptions was done by Schoenfeld residues analysis.
We applied propensity score weights to achieve a more robust comparison between the two groups of CDK4\/6i. We used the existence of visceral metastases, treatment line, age at treatment start, and stage. We used the WeightIt package for R \cite{WeightIt}. We applied the weights to the Kaplan-Meier curves and to the Cox Regression. We applied the weights to get the ATE which is $E[Y_i(1)-Y_i(0)]$, the average effect of moving an entire population from untreated to treated, or from one drug to the other. Weights were used instead of matching since it is more suited for calculating ATE and the need to preserve the sample size since it is already small from the start. The formula for calculating the weights was through propensity score weighting with GLM. Multiple comparisons were done with the Benjamini-Hochberg (BH) method. 



%Pretende-se que os dados venham do Instituto Português de Oncologia – Porto (IPO-P). Pretende-se utilizar a base de dados do hospital dos últimos 5 anos.
%O estudo será registado e respeitará todos os requisitos éticos de aprovação a comissão de ética de cada instituição participante. Caso a instituição tenha um encarregado de Proteção de Dados (EPD), este será contactado a fim de dar seu parecer, e caso necessário, a sua opinião para melhorar possíveis pontos relacionados à segurança de dados.

%The goal is to identify the clinical or biological variables that have the most impact on a specific outcome. To achieve this, considering the statistical models found, we can model things in terms of time series or survival trees to group outcomes and the clusters found. With this, we can analyze the effectiveness of the clinical course, probabilities of recurrence, or survival rates by subgroup. These models and relationships can form the basis of a clinical decision support system and can be crucial for making better healthcare decisions.

%Some of the techniques used are unsupervised techniques such as k-means, DBSCAN, or hierarchical clustering.
%The goal is to identify the clinical or biological variables that have the most impact on a specific outcome. To achieve this, considering the statistical models found, we can model things in terms of time series or survival trees to group outcomes and the clusters found. With this, we can analyze the effectiveness of the clinical course, probabilities of recurrence, or survival rates by subgroup. These models and relationships can form the basis of a clinical decision support system and can be crucial for making better healthcare decisions.

%Some of the techniques used are unsupervised techniques such as k-means, DBSCAN, or hierarchical clustering.




% breast cancer cells have either estrogen (ER) or progesterone (PR) receptors or both. (HR +)
\subsection{Results}
The median \ac{os} in the entire population treated with \ac{cdk46i} was 46 months (95\% CI 39.4–55.6). Median \ac{pfs} was 20.1 months (95\% CI 18.3–24.2). Following this, we compared Palbociclib and ribociclib only as first-line treatments. We found that regarding \ac{os}, there is no significant difference between the two, but ribociclib is significantly better in terms of \ac{pfs} (\textit{P} value $\le$ 0.001) (Figure \ref{fig:interest}). Additionally, we compared the same \ac{cdk46i} with letrozole as a combination only (PAL-LT and RIB-LT). Regarding this scenario, we found out that both were similar in terms of \ac{os} and \ac{pfs}.


\begin{figure}[ht]
  \caption{Survival curves for Palbociclib and Ribociclib (1st line) - \ac{pfs} and \ac{os}}\label{fig:interest} 
  \includegraphics[scale=0.45]{figures/interest_curve_both.jpeg}%

\end{figure}


We then compared both with a cox regression, where \ac{os} shows no significant difference between palbociclib and ribociclib when adjusted to the stage, visceral metastases, age, treatment line, combination and \ac{ecog}. The proportional hazards' assumption was confirmed with \textit{P} values all over 0.10.
\begin{table}[ht]
  \centering
  \caption[Cox Regression with palbociclib and Ribociclib - \acs{pfs} and \acs{os}]{Cox Regression with palbociclib and Ribociclib - \ac{pfs} and \ac{os}}\label{tab:cox} 
  \includegraphics[scale=0.20]{figures/cox_both.png}%

\end{table}

When comparing \ac{et} with \ac{cdk46i} as first-line treatment (figure \ref{fig:grouped}). For this study we only compared patients with bone only metastasis. When comparing both \ac{cdk46i} combined with Fulvestrant or letrozole, we see that  Ribociclib (RIB+LT/FUL) is significantly better for \ac{pfs} (\textit{P} value $\le$ 0.001 \ac{hr}=0.21) but not \ac{os}. For Palbociclib as the first line with Fulvestrant or letrozole (PAL+LT/FUL), we see that there is no significant difference in terms of \ac{pfs} and \ac{os} (\textit{P}=0.57 and 0.51). We also applied the same analysis but comparing only the letrozole combination with letrozole alone (PAL-LT/RIB-LT vs LT). We found that both ribociclib and palbociclib are significantly better in terms of \ac{pfs} (\ac{hr} 0.65 for palbociclib and 0.27 for ribociclib) but not \ac{os}.
\begin{figure}[ht]
  \centering

  \caption[Survival curves (\ac{os} and \ac{pfs}) comparing \ac{et} to \ac{cdk46i} combined with fulvestrant or letrozole as 1st line.]{Survival curves (\ac{os} and \ac{pfs}) comparing \ac{et} to \ac{cdk46i} combined with fulvestrant or letrozole as 1st line. First row is \ac{cdk46i} combined fulvestrant or letrozole vs fulvestrant or letrozole. Second row is \ac{cdk46i} combined with letrozole vs letrozole alone. \textit{P} values shown as pairwise vs. ET. }\label{fig:grouped} 
  \includegraphics[scale=0.42]{figures/grouped_curve_both.jpeg}%

\end{figure}

When comparing palbociclib and ribociclib adjusted for \ac{ate} weights, we found a different scenario from previous assessments. There is a significant difference between the two in terms of \ac{os} (figure \ref{fig:propensity}). The weights were calculated as stated in the methods section.


\begin{figure}[ht]
  \centering

  \caption{Comparison of palbociclib and ribociclib survival curves adjusted for propensity scores}\label{fig:propensity} 
  \includegraphics[scale=0.42]{figures/propensity_score_both.jpeg}%

\end{figure}

The Cox regression adjusted for the variables and with the weights applied to render an \ac{hr}=0.55 [95\% CI 0.28-1.09;\textit{P}=0.086] for \ac{os}. The \ac{hr}for \ac{pfs} is 0.56 [95\% CI 0.32-1;\textit{P}=0.05].
\subsection{Discussion}
The aim of this study was to evaluate the real-world use of palbociclib and ribociclib in combination with\ac{et}for \ac{hr+}/\ac{her2} and compare this drug class with traditional \ac{et}. Few real-world evidence studies of palbociclib and ribociclib used in daily clinical practice have been published identifying clinical benefit, patient profile, and sequencing of treatment, with even less evidence for the Portuguese population.

When comparing with clinical trials, regarding patient profile, in our study, 51\% had visceral metastasis and 35\% had bone-only metastases compared with 49\% and 38\% in PALOMA-2, and 60\% and 25\% in PALOMA-3, respectively \cite{rugoImpactPalbociclibLetrozole2018,cristofanilliFulvestrantPalbociclibFulvestrant2016a}.
As for ribociclib and bone-only metastases, MONALEESA-7 \cite{tripathyRibociclibEndocrineTherapy2018} has 24\% and MONALEESA-2 has 40\% \cite{hortobagyiUpdatedResultsMONALEESA22018} and our study has 30\%. Regarding menopausal status, our study has 20\% premenopausal and 80\% postmenopausal. 



Of note, the range of median \ac{pfs} for first-line palbociclib was 15.5–25.5 months, which is shorter than 27.6 months observed in a post hoc analysis of the PALOMA-2 clinical trial with extended follow-up \cite{rugoImpactPalbociclibLetrozole2018}, but in line with RWE studies (13.3–20.2 months) \cite{harbeckCDK4InhibitorsHR2021}. When assessed with only letrozole as a combination, the median \ac{pfs} increased to 28.6 months [95\% CI 25.5-not reached]. Additionally, analyzing the postmenopausal women subgroup, palbociclib showed a median \ac{pfs} of 16.3 months [95\% CI 12.9 -20]. Furthering analysis of the postmenopausal and with letrozole, the median was 47.6 months [95\% 25.6-2–not reached].

As for ribociclib, median survival time was not reached whether in \ac{os} and \ac{pfs}. So we can at least say that the median \ac{pfs} is longer than 50 months. This is longer than the median \ac{pfs} of 23.8 months (95\% CI 19.2–not reached) reported in the MONALEESA-7 trial \cite{tripathyRibociclibEndocrineTherapy2018} and longer than  25.3 months (95\% CI 23.0–30.3) in the MONALEESA-2 trial \cite{hortobagyiUpdatedResultsMONALEESA22018}. Regarding the subgroup analysis of postmenopausal women, we noticed that the median was not reached for women treated with ribociclib and fulvestrant or letrozole (RIB-LT/FUL) and postmenopausal women treated with ribociclib in combination with letrozole (RIB-LT).

When directly comparing ribociclib and palbociclib without any adjustments, one might deduce that ribociclib is superior to palbociclib. However, after adjusting for confounding variables, there is no significant difference between the two inhibitors in terms of \ac{pfs} or \ac{os} as indicated in table 2. This observation is further corroborated by the lower plots in figure 1, where even a subgroup analysis of \ac{cdk46i} combined solely with letrozole reveals non-significant difference between the two.

In the first-line comparison, the analysis of \ac{os} outcomes reveals no substantial difference between \ac{et} alone and the combination of \ac{cdk46i} with \ac{et}, irrespective of whether the \ac{cdk46i} are administered with fulvestrant or letrozole  (PAL-LT/FUL vs LT/FUL; \textit{P}=0.57 | RIB-LT/FUL vs LT/FUL \textit{P}=0.069) or exclusively with letrozole (PAL-LT vs LT; p = 0.979 | RIB-LT vs LT; \textit{P}=0.179)(figure \ref{fig:grouped} left). With respect to \ac{pfs}, ribociclib demonstrates superior efficacy when compared its combination with any of the adjuvants to these adjuvants alone (RIB-LT/FUL vs LT/FUL; \ac{hr}=0.21) as well as when combined only with letrozole (RIB-LT vs LT; \ac{hr}=0.27). Additionally, palbociclib exhibits significant improvement in \ac{pfs} when combined with letrozole  (PAL-LT vs LT; \ac{hr}=0.65) (figure \ref{fig:grouped} right).
When comparing with propensity scores weighting, we found out that ribociclib is significantly better than palbociclib for \ac{pfs} and \ac{os}, providing a median \ac{os} of over 40 months and median \ac{pfs} of around 42 months. Adjusted for the weighted variables, Ribociclib is not significantly better for \ac{pfs}, but has a \textit{P} value of 0.013 for \ac{os} with an \ac{hr}of 0.48. However, the Cox regression adjusted for variables and weights are not significant, even when the \textit{P} value for \ac{pfs} is 0.05. This suggests that a more in depth analysis may be necessary.



\subsection{Conclusion}
In conclusion, our findings underscore the efficacy of \ac{cdk46i} in real-world settings. We can confidently affirm the impact of Ribociclib on \ac{pfs}. This assertion aligns with clinical trial outcomes and real-world data further substantiates these findings. However, we cannot do the same for \ac{os}. Our results indicate that Ribociclib combined with letrozole or fulvestrant when compared to both is not superior to these alternatives used alone. The same happens when comparing ribocilib combined with letrozole with letrozole alone. 
However we cannot do so for Palbociclib. Palbociclib combined with fulvestrant or letrozole was not significantly better than letrozole or fulvestrant alone for \ac{pfs} nor \ac{os}. This is something interesting that we want to follow up with.
Delving deeper into the characteristics of the patient population, including safety profiles, economic implications, and quality of life metrics, would be insightful. Additionally, a thorough examination of biomarkers within the population could offer invaluable insights. Finnaly, extending the follow-up period would be benefitial as well. We intend to explore these facets in subsequent publications.
It’s imperative to note that our data is sourced from a singular institution, limiting the capability of generalization of our results to a broader population. Nonetheless, we posit that this study lays a foundational groundwork for future research in this domain. While our evidence is rooted in observational data, and we’ve made adjustments for known confounders, the potential for residual confounding remains. Although the use of propensity score matching enhances the comparative robustness between the groups, the presence of unmeasured confounders cannot be entirely ruled out. Furthermore, the small sample size of our study limits the statistical power of our findings. For next steps we aim to further analyse the clinical variables that have an impact on the outcome of the combination of CDK4/6 with fulvestrant or letrozole and these drugs used alone in order to infer pharmaeconomic implications and possible profiles of patient that would not benefit from this combination which would be vital for economical reasons and to apply in countries with low access to these drugs.


\section{Leveraging data to create Clinical Decision Support Systems}\label{subsec:obs}
This section is based on the paper entitled "Machine-learning in Obstetrics: FHIR-based Support System for predicting delivery type". This work was in part a result of the work in section \ref{subsec:distributed}. While testing for distributed mechanisms, we kind of felt that some evaluation metrics were inspiring to pursue this further. We built a \ac{cdss} system that is interoperable and aims to provide support for subpar evaluation of a \ac{cs}.

\subsection{Introduction}
The ability to provide care to both women and newborns during delivery is one of the most important aspects of healthcare and is often used as a metric to assess healthcare as a whole across different countries.
\acp{cs} are one of the most important aspects of delivering babies since it has a considerable impact on the mother's health and well-being. Despite this type of procedure increasing over the last few years, it is still illusive the reasons behind such events. Reports from 2016 suggest that this increment is a global phenomenon, being that from 1990 to 2014, this type of delivery almost increases by 3-fold from 6.7\% to 19.1\% \cite{betranIncreasingTrendCaesarean2016,chenNonClinicalInterventions2018}. Some of these impacts, being more prone to investigation in the last years, including the risk of infection, haemorrhage, organ injury and complications related to the use of anaesthesia or blood transfusion \cite{caesereanrisk1,caesereanrisk2}.
There is also a higher risk of complications in subsequent pregnancies like uterine rupture, abnormal placental implantation and the need for hysterectomy \cite{caesereanrisk3,caesereanrisk4}. As for the infant, \acp{cs} include the risk of respiratory problems, asthma and obesity in childhood \cite{caesereanrisk3}.
Facing this, in 2015, World Health Organisation released a statement regarding \acp{cs} rates. Even when other complications could not be totally assessed, it was concluded that \ac{cs} rates higher than 10\% were not associated with a reduction in maternal or newborn mortality \cite{worldhealthorganizationhumanreproductionprogramme10april2015WHOStatementCaesarean2015}.

Since there is no evidence that this type of procedure is beneficial for women or babies when there is no clear need for it, the focus on filtering such cases is important \cite{chenNonClinicalInterventions2018}.
Moreover, particularly in Portugal, \acp{cs} are used as a way of financing healthcare institutions. This was implemented as a strategy of decreasing \ac{cs}s across the country. A committee was created especially with the purpose of reducing the percentage of \acp{cs} nationally. One of the actions taken along this creation was the reduction of government funding for hospitals with rates of \acp{cs} above 25\%.
In 2020, the number of \acp{cs} in Portugal is about 36.3\%. Almost at the all-time high of 36.9\% in 2009 \cite{pordatacesarianas}.
So, lowering the proportion of \ac{cs} can provide health and financial benefits to institutions and populations alike. With this in mind, we developed a  machine-learning algorithm-based support system to assist clinical teams to detect cases of potentially unnecessary \acp{cs} for analysis. So in this paper, we propose:
\begin{myitemize}
    \item help to provide a method of bringing to the discussion of clinical staff possible less than optimal care regarding deliveries;
    \item elaborates on how clinical decision support systems can be developed using interoperability standards;
    \item understand, based on the gathered data, which are the more impacting features for predicting delivery type outcome;
    \item open a research path regarding the evaluation of this type of clinical decision support system prior to the delivery;
    \item Perform a concise economical analysis to assess the potential financial impact of implementing the proposed clinical decision support tool.

\end{myitemize}
\subsection{Rationale and Related Work}
% !TeX root = ../../thesis.tex

Regarding the related work, several teams already tackled the potential of predicting the delivery type before birth. We found studies related to predicting a successful vaginal birth after a previous C-section, such as the work of Lipschuetz et al., \cite{lipschuetzPredictionVaginalBirth2020}  where a gradient boosting method was used to predict such an event using prenatal data to do so. Grobman et al., \cite{grobman_development_2007} performed a similar study with a multivariable logistic regression model. Different modalities of data were also used to predict delivery type. Fergus et al. \cite{fergusClassificationCaesareanSection2017} introduces a method of predicting de- livery type using the fetal heart rate signals. Similarly, the work from Saleem et al. \cite{saleemStrategyClassificationVaginal2019a} proposed a method for predicting delivery type using interactions between the fetal heart rate and maternal uterine contraction. Finally, there are also studies that focus on predicting the delivery mode like the work of Ullah et al. \cite{ullah_reliable_2021} where a boosting algorithm was used in order to predict delivery mode with enriched datasets. In addition, Gimovsky et al. \cite{gimovskyBenchmarkingCesareanDelivery} introduced decision trees to predict \ac{cs} by physician group with 0.73 \ac{auroc}. The works of \cite{rossiRiskCalculatorPredict2020b} resulted in a seven-variable model with 0.78 \ac{auroc} and the works of \cite{guedaliaRealtimeDataAnalysis2020} resulted in a model with 0.82 \ac{auroc}, reaching 0.93 with a first cervical examination. Finally, the works of Meyer et al. \cite{meyerImplementationMachineLearning2020} focused around selecting suitable for a trial of labor after cesarean with \ac{auprc} around 0.351. However, to the best of our knowledge, there was no model tested in clinical practice, with an interoperable format of communication like \ac{fhir}, which tried to not only predict delivery type but also provide support about possibly worn deliveries and none with simulation about financial implication, making our paper a potential novelty on different dimensions.
\subsection{Materials}
Data was collected from nine different public Portuguese hospitals across the country, focusing on obstetric information, encompassing maternal data, various fetal data points, and the method of delivery in a retrospective manner. The data is from all patients that had information registered in the obstetrics EHR and had a registered outcome of the pregnancy from 2019 to 2020. Despite differing software versions across hospitals, each institution used identical EHR software, ensuring the columns remained consistent.

\subsection{Methods}
We wrote all of the code in Python 3.9.7 with the usage of the \textit{scikit-learn} library \cite{scikit-learn}. All null representations were standardized. Data was prepossessed with the removal of features with high missing rates ($gt$ 90\% overall). All missing value representations were standardized. The imputation process was done using the \ac{knn} imputation method (for continuous variables) or a new category (NULLIMP) for categorical variables. 
For this purpose, the Birth Type was reduced to binary. All assisted birth were merged into vaginal birth and \ac{cs} remained as the other class. Procedures and diagnosis were also used and were encoded as binary features, we took the time to analyse each one of them in order to avoid leakage since there were procedures obviously related to \acp{cs} and vaginal deliveries.
Feature creation was done through the free-text variable relating to the medication prescribed. Features were collected from it and converted into \ac{atc} Classification Group level 4, which stands for chemical subgroups. We also created some new features from data in the dataset, namely new categories related to the labour and condition of the baby.
Also, a few data quality issues were addressed, like impossible values that were transformed into null. In this category, the main issues were \ac{bmi}/Weight and gestational age.
Finally, only a few columns were selected. We used a mixture of surveying the literature and the feature with greater correlation with the outcome.
The models tested were Logistic Regression, Decision Tree, Random Forest, 3 different Boosting methods (as implemented by \ac{xgboost}, \ac{lightgbm} and \textit{scikit-learn}) and a linear model based on Stochastic Gradient Descent.
The evaluation was done with repeated stratified cross-validation with 10 splits and 2 repetitions.
The API for serving the prediction model was developed with FastAPI.
%For trying to provide an explainable model, Shapley Additive exPlanations (SHAP) \cite{shapleyvalues} values were introduced in order to supply a more robust prediction.

Finally, a clinical evaluation was carried out with questionnaires sent to several obstetrics specialists in order to assess the validity and possible impact of the model.

\subsection{Results}
\subsubsection{Descriptive Statistics}
The number of samples varied across the hospitals, ranging from 2364 to 18177. Distributions of the selected variables are presented in table \ref{tab:obs_material_1}. The sum of all samples totals 73351.

\begin{table}[htbp]
  \centering
  \caption[Distribution of features used for prediction of delivery Type]{Distribution of features used for prediction, Mean and Standard Deviation (SD) for continuous variables. Mode and percentage for categorical variables. Number of samples is 73351.}
  \label{tab:obs_material_1}
   \renewcommand{\arraystretch}{1.02} % Adjust the vertical spacing
  \setlength{\tabcolsep}{12pt} % Adjust the horizontal spacing
    \begin{tabular}{m{15em}cc}
    \toprule
        Variable & M (SD) & Mode [\%] \\ 
        \hline
        Mother Age & 31.0 (5.6) & ~ \\ 
        Weight pre-pregnancy & 65.8 (13.9) & ~ \\ 
        Weight on admission & 78.6 (14.2) & ~ \\ 
        \ac{bmi} & 25.0 (5.4) & ~ \\ 
        Previous eutocic delivery & 0.4 (0.7) & ~ \\ 
        Previous vacuum-assisted delivery & 0.1 (0.3) & ~ \\ 
        Previous forceps & 0.0 (0.1) & ~ \\ 
        Previous \ac{cs} & 0.1 (0.4) & ~ \\ 
        Fetal presentation on admission & ~ & cephalic [26.323\%] \\ 
        Bishop score & 5.5 (3.0) & ~ \\ 
        Gestational age on admission & 38.9 (1.9) & ~ \\ 
        Premature rupture of the membrane & ~ & No [87.991\%] \\ 
        Chronic hypertension & ~ & No [97.676\%] \\ 
        Gestational hypertension & ~ & No [97.749\%] \\ 
        Preeclampsia & ~ & No [98.299\%] \\ 
        Gestational diabetes & ~ & No [89.811\%] \\ 
        Gestational diabetes treated with diet & ~ & No [94.285\%] \\ 
        Gestational diabetes treated with insulin & ~ & No [98.083\%] \\ 
        Gestational diabetes treated with oral antidiabetic drugs & ~ & No [97.797\%] \\ 
        Maternal Diabetes & ~ & No [99.509\%] \\ 
        Type 1 Diabetes & ~ & No [99.816\%] \\ 
        Type 2 Diabetes & ~ & No [99.843\%] \\ 
        Presentation at birth & ~ & Vertex presentation [94.000\%] \\ 
        Delivery & ~ & Spontaneous [53.864\%] \\ 
        Gestational age on birth & 39.0 (1.8) & ~ \\ 
        Smoking during pregnancy & ~ & No [88.442\%] \\ 
        Alcohol consumption during pregnancy & ~ & No [98.65\%] \\ 
        Consumed drugs during pregnancy & ~ & No [99.825\%] \\ 
        Nr of pregnancies (with current) & 1.9 (1.1) & ~ \\ 
        Pregnancy type & ~ & Spontaneous [85.417\%] \\ 
        Surveillance & ~ & yes [97.699\%] \\ 
        Hospital surveillance & ~ & yes [67.807\%] \\ 
        Pelvis Adequacy & ~ & Adequate [17.512\%] \\ 
        Consistency of the cervix & 1.6 (0.6) & ~ \\ 
        Fetal station & 0.8 (0.8) & ~ \\ 
        Dilation of the cervix & 1.3 (0.8) & ~ \\ 
        Effacement of the cervix & 1.2 (1.2) & ~ \\ 
        Position of the cervix & 0.6 (0.7) & ~ \\ 
        Haematologic disease & ~ & No [95.674\%] \\ 
        Respiratory disease & ~ & No [95.605\%] \\ 
        Cerebral disease & ~ & No [98.793\%] \\ 
        Cardiac disease & ~ & No [92.967\%] \\ 
        Neuroaxis techniques & ~ & 1 [69.5\%] \\ 
        Number of children  & 0.6 (0.8)  & ~\\ 
        \bottomrule
    \end{tabular}


\end{table}
The outcome variable had the following distribution as stated in table \ref{tab:delivery_methods}

\begin{table}[htbp]
  \centering
  \caption{Distribution of Delivery Methods}
  \label{tab:delivery_methods}
  \renewcommand{\arraystretch}{1.5} % Adjust the vertical spacing
  \setlength{\tabcolsep}{12pt} % Adjust the horizontal spacing
  \begin{tabular}{lc}
    \hline
    \textbf{Type of delivery} & \textbf{Frequency (\%)} \\
\hline
    \ac{cs} & 19 803 \, (27\%) \\

    Vaginal & 38 189 \, (52\%) \\

    Instrumental delivery & 15 359 \, (21\%) \\
    \hline
  \end{tabular}
\end{table}
%Vaginal      0.520634
%Cesariana    0.269976
%auxiliado    0.209390



\subsubsection{The Model}
The AUROC is presented in table \ref{tab:performancemetricsauc} for the best hyper-parameters found for each algorithm in the training data. All models used the variables indicated in table \ref{tab:obs_material_1}.
\begin{table}[htbp]
  \centering
  \caption[Performance Metrics in the training set]{Repeated Cross-validation (10x2) results in the training set with mean AUROC and 95\% Confidence Interval (CI) for best hyper-parameters found for each algorithm. Wilcoxon Test for comparing with the best performing algorithm.}
  \label{tab:performancemetricsauc}
  \renewcommand{\arraystretch}{1.5} % Adjust the vertical spacing
  \setlength{\tabcolsep}{12pt} % Adjust the horizontal spacing
  \begin{tabular}{lccc}
    \hline
    \textbf{Metric} & \textbf{AUROC} & \textbf{CI 95\%} \textit{\textbf{P value}} \\
    \hline
    XGBoost & 0.8809 & 0.8799, 0.882 & - \\  
    Decision Tree & 0.8337 & 0.8324, 0.8349 & $le$ 0.001\\
    Logistic Regression & 0.8716 & 0.8706, 0.8726 & $le$ 0.001\\
    AdaBoost & 0.8753 & 0.874, 0.8766 & $le$ 0.001\\ 
    lightgbm & 0.8805 & 0.8793, 0.8817 &  0.003\\ 
    Stochastic Gradient Descent & 0.8704 & 0.8694, 0.8713& $le$ 0.001\\ 
    Random Forest & 0.8752 & 0.8743, 0.8762& $le$ 0.001 \\  
    \hline
  \end{tabular}
\end{table}
While XGBoost was the best-performing algorithm, we selected LightGBM  \cite{lightgbm} because of its speed and lower memory requirements, which we believe are better suited for deployment in a low-hardware environment. The threshold selected for deploying the model was 0.7457 which rendered the metrics in the test set, as shown in table \ref{tab:performancemetricsthreshold}.

\begin{table}[htbp]
  \centering
\caption{Performance Metrics in the test set with chosen threshold}
\label{tab:performancemetricsthreshold}
\renewcommand{\arraystretch}{1.5} % Adjust the vertical spacing
\setlength{\tabcolsep}{12pt} % Adjust the horizontal spacing
\begin{tabular}{lc}
    \hline
    \textbf{Metric} & \textbf{Value} \\
    \hline
    Accuracy & 0.8052 \\
    Sensitivity & 0.8223 \\
    Precision & 0.9023 \\
    F1 Score & 0.8605 \\
    \hline
  \end{tabular}
\end{table}




\subsubsection{Deployment}
The purpose of this model is to serve as an API for usage within a healthcare institution and to act as a supplementary clinical decision support tool for obstetrics teams. For this to happen, a health information system must make the requests to the API. Even though a concrete, vendor-specific information model and input health information system were used, we hope to create a more interoperable clinical decision support system that can be used by every system that acts on birth and obstetrics departments. Therefore, we built it around the HL7 FHIR standard (R5 version) to simplify the method of interacting with the API. This decision, opposed as to using a proprietary model for the data, sits upon the usage of FHIR resources: Bundle and Observation for request and returning the result as a message through a custom operation called "\$predict". It is intended to publish the profiles of these objects in order to facilitate access to the API using standardized mechanisms and data models. The current build of the profiles can be seen in the published FHIR Implementation Guide where the current specifications are described in detail \url{https://joofio.github.io/obs-cdss-fhir/}. The process is illustrated in figure \ref{fig:deploy}. We deployed this model in production in a single hospital without a user interface, collecting only the data and predictions for later discussion and analysis. We collected 3231 requests. During this period, 123 (3.8\%) alarms were triggered. From this, we tried to understand the level of certainty for the decision and check the difference from the threshold of these alarms. The distance to the threshold for 73 was lower than 0.1 and was bigger than 0.1 for 50 (1.55\%) cases.


%TC:ignore

\begin{figure}[htbp]
\centering
\captionsetup{justification=centering}
\caption{Deployment and decision mechanism of the model}\label{fig:deploy} 
\includegraphics[scale=0.60]{figures/obs-model.jpg}
\end{figure}
%TC:endignore

\subsubsection{Clinical Evaluation}
The median scores given by each clinician are presented in figure 2. We also predicted the result using our model as stated in figure 2. The model misclassified only one record (4). As for the analysis of missing features for the responders, they were divided into 3 categories: 1) Existent in the dataset but not included in the model, 2) Non-existent in the dataset and 3) existent in the dataset and included but that particular information was not filled for the patient assessed. This rendered a total of 62 \% non-existent and 38 \% existent but no information was provided at that moment. No feature mentioned existed in the dataset but had not been included in the model. From the non-existent, 38 \% were new clinical assessments, 38\% were linked to information from previous births, 15\% connected in more in-depth information about provided information (i.e, motive for induction) and 11\% were related to the mother's choice (if she wanted a C-section). As for feature importance, from the 60 answers, we got 55 \% with labor being the most important factor. 15 \% answered the number of previous vaginal births, 8 \% the evolution of weight and another 8\% the number of previous C-sections. The remaining 14\% were various features, from BMI, neuroaxis techniques, gestational age and weight of the mother. Of all of these, 90 \% were in the top 10 features of the model.



%TC:ignore

\begin{figure}[htbp]
\centering
\captionsetup{justification=centering}
\caption[Obstetrics questionnaires data]{Validation data. The colour represents the actual birth type. The boxplot represents the median and \ac{iqr} of the reviewers and the X represent each patient case. Contains 6 Vaginal births and 4 \acp{cs}. * represents wrong predictions of the model. (ID: 4)}\label{fig:clinical} 
\includegraphics[scale=0.60]{figures/clinical_assessment.png}
\end{figure}
%TC:endignore




\subsubsection{Potential Financial Impact}
The financial support provided to public hospitals in Portugal is partially tied to the rate of C-sections. To assess the potential impact of this mechanism on Portuguese public hospitals, we conducted a simulation. We got data for every public hospital for the last 12 months and applied a 3.8\% reduction (the rate of warnings triggered in the new dataset) and recalculated the rate of C-sections. The increase in support was calculated by the state-mandated rate as shown in table \ref{tab:corrections}. With this new rate, we observed that implementing our tool would result in financial benefits for 30\% (11 hospitals) of the public hospitals. Specifically, five hospitals would begin receiving support instead of no support at all. Three hospitals would experience a doubling of their financial benefit, while two hospitals would see a 50\% increase. Furthermore, one hospital would receive an additional one third of financial support. If we assumed that only half of the warnings found in the new data were actually true (1.9\%) we found that only 6 hospitals would be benefited. 3 from 0 to 0.25, 2 from 0.25 to 0.50 and 1 from 0.50 to 0.75.


\begin{table}[htbp]
  \centering
  \caption[Ruleset for financial support indexed to \acsp{cs}.]{Ruleset for state-provided financial support indexed to \acp{cs}. X is the current payment of a \ac{cs} inpatient episode. Adapted from \cite{acssTermosReferenciaPara2023}}
  \label{tab:corrections}
  \renewcommand{\arraystretch}{1.2} % Adjust the vertical spacing
  \setlength{\tabcolsep}{12pt} % Adjust the horizontal spacing
  \begin{tabular}{lc}
      \hline
Rate of \acp{cs}   & Support \\
    \hline
\textless 25\%       & x       \\
{[}25\%, 26.4\%{]}   & 0.75 x   \\
{[}26.5\%, 27.9\%{]} & 0.5 x    \\
{[}28\%, 29.4\%{]}   & 0.25 x   \\
\textgreater{}29.5\% & 0      \\
    \hline
\end{tabular}
\end{table}
\subsection{Discussion}
% !TeX root = ../../thesis.tex

The first thing to address about this model is the number of biases that we introduced in the model by choice. We joined all vaginal delivery types into a single category (assisted and non-assisted) which introduces a bias since these delivery modes are indeed different. Secondly, the fact that we want to predict if the delivery type was wrongly chosen, mainly for the case of a C-section that did not need to be so, is also a bias. We used this approach because the initially collected data did not have the representation of such events. So the biases of possibly wrong delivery types were present in the training data. We attempted to minimize this issue by selecting a threshold that gave the model higher sensitivity than specificity so that only large probabilities would trigger an alarm for human consideration. Parallel to this, we are starting to gather labeled cases, with the help of clinicians in order to create a better training dataset. Furthermore, since the data was collected from different hospitals, differences in the data input can also occur. Even though the health information system is the same, the processes that originate the data and are being used for secondary purposes could introduce several biases in the data. This is an issue that was accepted from the start regarding the mechanism of data collection and model training. Despite this, we reached a model with a very high AUROC (~88\%, 95\%CI [0.8795, 0.8815]), which is encouraging and versus the state of the art. Moreover, assuming that more data is provided and proper labeling is done regarding the outcome variable (like a clinical evaluation of needless C-sections) is added as well, a better model could be developed. Regarding the preliminary clinical evaluation, it was only possible to get an overview of the possible comparison due to the number of responders. Despite that, the results are encouraging, since the model seems to behave better than humans with the data provided. However, this is a biased vision, since clinicians in the real world have access to more data and information than the model has. It is encouraging, but caution is advised before more tests and evaluations are done. As for the deployment, future work could be the improvement of the API in order to map all variables to an ontology like snomed CT or similar, making it easier for every system and person to access it and get a suggestion of the delivery type. Finally, we believe the assessment can be improved. A more robust clinical assessment is necessary as well as a thorough analysis of the impact of the tool in the real world, since we need to create the bridge between the results of the model and how clinical decisions are affected by it. A full cost-effectiveness analysis is also necessary to understand the real world impact of the model. One interesting result is the fact that 38\% of the answers regarding the most important data element missing from the patient record refers to data that is being collected but was missing for that specific patient, raising an important question about data input methodology, interoperability and quality. If we cannot have access to data when it matters most, it can become meaningless. Missing data is a problem of biomedical data as a whole. However, when specifically targeted at machine learning usage of this data for predicting something, we did not find any works comparing them with clinicians. However, we did find reportings of similar missing values in obstetrics data \cite{venkateshMachineLearningStatistical2020} and we also found works of similar nature using machine learning models with a robust handling of missing data such as XGBoost \cite{bitarMachineLearningAlgorithm2023} to counter this problem. This indicates that our model has the potential to counter the missing data problem as well since LightGBM can also handle missing data natively.

\subsection{Conclusion}
We believe that we developed a fairly robust system for alarming for possibly wrong \acp{cs}, that could have a positive impact in real-world practice. However, there are issues to tackle before doing so. There is a need for further evaluating the impact of such a system on clinical practice decisions. There could be a very wide range of reasons that could lead to a possibly sub-par decision regarding delivery type. From the mother's decision to a lack of information at the decisive moment. This system is not meant to create hurdles regarding practice or to point out defective decisions putting certain professionals under the spotlight.
All the underlying assumptions and prejudices about having autonomous systems providing support for practice must be taken into account. Nevertheless, the metrics and results so far are definitely encouraging for having a positive impact on health and economic outcomes.






