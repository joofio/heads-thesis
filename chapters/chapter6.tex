\begin{savequote}[75mm]
An expert is a person who has made all the mistakes that can be made in a very narrow field.
\qauthor{Niels Bohr}
\end{savequote}
\chapter{Limitations, future work and conclusions} \label{chap:conclusion}

Regarding limitations, all the projects done in this thesis focus on different aspects of the process of extracting knowledge from healthcare data. Also, they are heavily reliant on specific use cases that are not necessarily generalizable. For example, the \ref{subsec:ipop} project is focused on a specific disease, and the \ref{subsec:obs} and \ref{subsec:distributed} projects are focused on a specific type of data, and the \ref{subsec:dq} project is focused on a specific type of data and a specific clinical specialty. This means that the results of these projects are not necessarily generalizable to other diseases or other types of data. However, the methods used in these projects are generalizable, and they can be used in other projects. For example, the \ref{subsec:ipop} methods can be used to predict in real-time, and the \ref{subsec:obs} and \ref{subsec:distributed} models can be used to analyse other types of data. The \ref{subsec:similarity} method can be used to analyse any type of dataset and incorporated into data pipelines.

In future work, I think the groundwork is laid for actually providing assistance to healthcare teams. However, actually deploying real-world \acp{cdss} is seldom an easy task and requires time, money and patience. This is why that part, the actual deployment of the tools, is left for future work. However, we did many tests in the real world and included clinicians in most of our work, so we are confident that the tools are ready to be deployed and create an impact.



For this work to be complete, I had to gather knowledge from different areas. From biology and chemistry, for the healthcare part of it, process design to understand and formalize processes, and math and statistics for machine learning and \ac{eda}, interoperability and standards for getting data together, ethics and privacy to gather data with guarantees to the patient's privacy and for creating ethical-aware models. Had to dwell into terminologies and healthcare codification and semantics to interpret data and also get acquainted with some clinical specialties like obstetrics and oncology. Had to collect evidence and make the bridge between \acp{rct} and observational data or \ac{rwd} so study design was also needed to bridge the gap.
Maybe this is one of the main issues with this domain, where a different set of skills is required to do everything. The alternative would be a team of different people and honestly, the most successful projects I have seen are the ones that have a team of different people.
Finally, the 

%focus on impact and application in real world. 