Clinical data was gathered from nine different Portuguese hospitals regarding obstetric information, pertaining to admissions from 2019 to 2020. This originated nine different files representing different sets of patients but with the same features associated to them. The software for collecting data was the same in every institution (although different versions existed across hospitals) - ObsCare. The data columns are the same in every hospital's database. Each hospital was considered a silo and summary statistics of the different silos are reported in the tables \ref{tab:distributed_materials_1} and \ref{tab:distributed_materials_2}. The data dictionary is in appendix \ref{appendix:data_dict}.
%TC:ignore

{\small
\begin{table}[!ht]

\caption[Silos overview.]{\label{tab:distributed_materials_1 part 1}Silos overview. categorical columns have a snippet of the most used category and a percentage. Continuous variables have a mean and standard deviation. Abbreviation meaning in the appendix \ref{appendix:data_dict}. The last row is the number of patients. * columns were used as target.}

\centering
% !TeX root = ../../thesis.tex
\newcolumntype{L}{>{\scriptsize}l}  % "small" can be changed to "scriptsize" or "footnotesize" for even smaller text

\begin{tabular}{LLLLLLL}
   \toprule
      Variable &               Silo 1 &               Silo 2 &               Silo 3 &               Silo 4 &               Silo 5 &                Total \\
   \midrule
   \hspace*{2mm} N (total) &              8039 &                 8566 &                 4989 &                 2364 &                18177 &                80874 \\
   
   \textbf{Actual Type of Delivery C (\%)}& 10 (52.6) & 3 (51.6) & 3 (57.8) & 3 (61.8) & 9 (61.5) & 11 (52.9) \\
   
   Bishop Score C (\%)&  15 (98.5) & 15 (78.8) & 13 (97.4) & 16 (86.4) & 15 (97.4) & 16 (95.3) \\
   
   \textbf{Blood Group C (\%)}& 9 (39.9) & 10 (39.9) & 9 (39.3) & 11 (37.9) & 10 (40.9) & 14 (40.5) \\
   
   \textbf{Body Mass Index $\mu (\sigma)$ } & 25.2 (8.6) & 25.2 (6.2) & 25.0 (5.3) & 25.0 (8.9) & 24.9 (7.8) & 25.1 (7.0) \\
   
   Cervical Consistency C (\%) & 4 (98.6) & 4 (83.4) & 4 (99.3) & 4 (87.4) & 4 (97.5) & 4 (96.5) \\
   
   Cervical Position C (\%)&  4 (98.6) & 4 (83.3) & 4 (99.3) & 4 (87.5) & 4 (97.6) & 4 (96.6) \\
   
   \textbf{Delivery Type C (\%)}& 6 (43.4) & 6 (53.5) & 5 (44.4) & 7 (52.2) & 7 (49.3) & 8 (51.3) \\
   
   Dilatation C (\%)&5 (98.5) & 5 (83.1) & 5 (99.3) & 5 (87.2) & 5 (97.5) & 5 (96.5) \\
   
   Effacement C (\%)& 5 (98.6) & 5 (83.2) & 5 (99.3) & 5 (87.2) & 5 (97.5) & 5 (96.5) \\
   
   Fetal Station  C (\%)&  5 (98.6) & 5 (83.3) & 5 (99.3) & 5 (87.9) & 5 (97.5) & 5 (96.6) \\
   
   \textbf{Followed physician C (\%)} & 3 (99.2) & 4 (92.2) & 3 (99.1) & 3 (94.3) & 3 (99.0) & 4 (97.9) \\
   
   \textbf{\begin{minipage}{3.8cm}\setstretch{0.65}Followed physician hospital delivery C (\%)\vspace{1mm}\end{minipage}}&  2 (87.6) & 2 (75.8) & 2 (81.4) & 2 (52.2) & 2 (71.0) & 2 (69.0) \\
   
   \textbf{\begin{minipage}{3.8cm}\setstretch{0.65}Followed physician primary care C (\%)\vspace{1mm}\end{minipage}}&  2 (61.3) & 2 (52.8) & 2 (78.1) & 2 (50.4) & 2 (70.4) & 2 (67.6) \\
   
   \begin{minipage}{3.8cm}\setstretch{0.65}Followed physician private clinic C (\%)\vspace{1mm}\end{minipage}&  2 (81.8) & 2 (85.0) & 2 (80.6) & 2 (78.8) & 2 (73.3) & 2 (75.8) \\
   
   Gestational Diabetes C (\%)&2 (87.7) & 2 (90.0) & 2 (90.2) & 2 (90.8) & 2 (89.8) & 2 (89.5) \\
   
   
   Induced Delivery  C (\%)& 2 (97.8) & 2 (83.9) & 2 (93.3) & 2 (91.9) & 2 (98.5) & 2 (92.5) \\
   
   \textbf{Mother Age $\mu (\sigma)$ } & 31.1 (5.7) & 30.7 (5.6) & 31.1 (5.9) & 31.1 (6.3) & 31.3 (5.6) & 31.1 (5.6) \\
   
   
   \begin{minipage}{3.9cm}\setstretch{0.65}Nr Deliveries forceps C (\%)\end{minipage} & 4 (99.2) & 3 (83.3) & 4 (94.3) & 4 (95.8) & 3 (60.1) & 5 (82.6) \\
   
   
   \begin{minipage}{3.9cm}\setstretch{0.65}Nr Deliveries no assistance C (\%)\vspace{1mm}\end{minipage} & 10 (74.7) & 9 (60.3) & 9 (74.9) & 9 (67.3) & 11 (45.4) & 12 (60.3) \\
   
   \begin{minipage}{3.9cm}\setstretch{0.65}Nr Deliveries vacuum C (\%)\vspace{1mm}\end{minipage} &  5 (90.4) & 4 (79.9) & 4 (89.0) & 4 (93.1) & 5 (55.3) & 5 (77.4) \\
   
   Nr of C-sections C (\%) & 6 (87.9) & 6 (72.6) & 5 (86.1) & 5 (89.5) & 6 (62.1) & 6 (74.6) \\
   
   \textbf{Nr of Pregnancies C (\%)} &  11 (40.9) & 11 (43.1) & 13 (39.1) & 12 (38.7) & 16 (42.8) & 19 (42.1) \\
   
   \textbf{Nr of born babies C (\%)} & 10 (44.8) & 10 (41.4) & 10 (36.9) & 10 (42.0) & 12 (35.3) & 12 (38.8) \\
   
   \textbf{Nr of consultations $\mu (\sigma)$ } &  7.3 (4.7) & 7.0 (6.4) & 6.4 (3.9) & 5.5 (3.6) & 10.5 (5.1) & 8.4 (5.1) \\
   
   Pelvis Adequacy C (\%) &4 (95.4) & 4 (77.7) & 4 (90.1) & 3 (96.9) & 4 (81.2) & 4 (82.6) \\
   
   \textbf{Position Admission C (\%)}  &  5 (88.5) & 6 (78.0) & 6 (51.8) & 3 (95.9) & 6 (71.3) & 7 (73.1) \\
   
   
   \textbf{Position on Delivery C (\%)} &5 (91.5) & 5 (94.4) & 5 (94.7) & 5 (95.5) & 5 (94.3) & 5 (93.9) \\
   
   \textbf{Pregnancy Type C (\%)} &  7 (62.1) & 7 (90.5) & 7 (85.4) & 7 (63.0) & 7 (89.2) & 7 (85.4) \\
   
   \textbf{Robson Group C (\%)} & 11 (22.4) & 11 (20.1) & 10 (23.8) & 10 (80.5) & 11 (27.7) & 11 (24.4) \\
   
   \begin{minipage}{3.7cm}\setstretch{0.65}Rupture amniotic pocket before delivery C (\%)\end{minipage} &  2 (91.1) & 2 (93.6) & 2 (89.3) & 2 (91.6) & 2 (84.6) & 2 (88.5) \\
   
   Smoker C (\%) & 2 (84.4) & 2 (85.2) & 2 (87.2) & 2 (89.7) & 2 (87.9) & 2 (88.1) \\
   
   \textbf{Spontaneous Delivery C (\%)} &  2 (70.3) & 2 (74.7) & 2 (64.8) & 2 (64.3) & 2 (59.7) & 2 (64.9) \\
   
   \textbf{Weeks on Admission C (\%)} &    38.1 (3.5) &    38.8 (2.2) &    38.9 (1.6) &    38.8 (2.4) &    38.6 (2.1) &    38.7 (2.2) \\
   
   
   \textbf{Weeks on Delivery $\mu (\sigma)$ } &    38.5 (2.8) &    38.9 (2.0) &    39.1 (1.7) &    39.0 (2.3) &    38.9 (2.0) &    38.9 (2.0) \\
   
   Weight on Admission $\mu (\sigma)$  &   81.4 (14.9) &   79.5 (14.5) &   78.0 (15.2) &   79.6 (16.3) &   78.3 (14.2) &   78.8 (14.5) \\
   
   
   \textbf{\begin{minipage}{3.3cm}\setstretch{0.65}Weight start  of pregnancy $\mu (\sigma)$ \vspace{1mm}\end{minipage}} &   66.4 (14.4) &   66.1 (13.5) &   65.5 (14.1) &   65.5 (14.1) &   65.5 (14.4) &   66.0 (14.1) \\
   \bottomrule
   \end{tabular}
   
   
\end{table}
}
\newpage

{\small
\begin{table}[!ht]

\caption[Silos overview part 2.]{\label{tab:distributed_materials_2}Silos overview part 2. categorical columns have a snippet of the most used category and a percentage. Continuous variables have a mean and standard deviation. Abbreviation meaning in the appendix \ref{appendix:data_dict}. The last row is the number of patients. * columns were used as target.}
\centering
% !TeX root = ../../thesis.tex

\begin{tabular}{llllll}
   \toprule
   Variable &               Silo 6 &               Silo 7 &               Silo 8 &               Silo 9 &                Agrr. \\
   \midrule
   
   \hspace*{2mm}  N (total) &                12002 &                 8258 &                 6693 &                11786 &                80874 \\
   
   \textbf{Actual Type of Delivery C (\%)} & 10 (63.8) & 0 (100) & 10 (50.1) & 9 (64.6) & 11 (52.9) \\
   Bishop Score C (\%) & 14 (99.3) & 15 (97.9) & 14 (99.2) & 15 (95.0) & 16 (95.3) \\
   \textbf{Blood Group C (\%)} & 13 (41.6) & 10 (39.2) & 10 (40.1) & 10 (41.7) & 14 (40.4) \\
   \textbf{Body Mass Index $\mu (\sigma)$ } & 24.9 (5.1) & 24.9 (7.0) & 24.8 (8.0) & 25.7 (5.6) & 25.1 (7.0) \\
   Cervical Consistency C (\%)  & 4 (99.5) & 4 (99.7) & 4 (99.5) & 4 (96.9) & 4 (96.5) \\
   Cervical Position C (\%) & 4 (99.5) & 4 (99.7) & 4 (99.5) & 4 (96.9) & 4 (96.5) \\
   \textbf{Delivery Type C (\%)} & 6 (54.3) & 5 (52.1) & 5 (47.8) & 5 (59.0) & 8 (51.3) \\
   Dilatation C (\%) & 5 (99.5) & 5 (99.7) & 5 (99.5) & 5 (96.9) & 5 (96.5) \\
   Effacement C (\%)  & 5 (99.5) & 5 (99.7) & 5 (99.5) & 5 (96.9) & 5 (96.5) \\
   Fetal Station C (\%) & 5 (99.5) & 5 (99.7) & 5 (99.5) & 5 (96.9) & 5 (96.6) \\
   \textbf{Followed physician C (\%)} & 3 (96.9) & 3 (99.4) & 3 (97.8) & 3 (99.2) & 4 (97.8) \\
   \textbf{\begin{minipage}{3.8cm}\setstretch{0.65}Followed physician hospital delivery C (\%)\vspace{1mm}\end{minipage}}& 2 (62.1) & 2 (63.2) & 2 (69.4) & 2 (83.1) & 2 (69.0) \\
   \textbf{\begin{minipage}{3.8cm}\setstretch{0.65}Followed physician primary care C (\%)\vspace{1mm}\end{minipage}} & 2 (53.1) & 2 (86.7) & 2 (63.1) & 2 (87.3) & 2 (67.6) \\
   \begin{minipage}{3.9cm}\setstretch{0.65}Followed physician private clinic C (\%)\vspace{1mm}\end{minipage} & 2 (68.2) & 2 (73.5) & 2 (71.0) & 2 (78.1) & 2 (75.8) \\
   Gestational Diabetes C (\%)  & 2 (92.2) & 2 (88.2) & 2 (89.9) & 2 (86.8) & 2 (89.5) \\
   Induced Delivery C (\%) & 2 (91.9) & 2 (85.9) & 2 (87.4) & 2 (93.9) & 2 (92.5) \\
   \textbf{Mother Age $\mu (\sigma)$ }  & 31.3 (5.2) & 31.4 (5.4) & 31.5 (5.6) & 30.1 (5.6) & 31.1 (5.6) \\
   
   \begin{minipage}{3.9cm}\setstretch{0.65}Nr Deliveries forceps C (\%)\vspace{1mm}\end{minipage} & 4 (82.0) & 4 (86.0) & 3 (94.0) & 4 (89.3) & 5 (82.6) \\
   \begin{minipage}{3.9cm}\setstretch{0.65}Nr Deliveries no assistance C (\%)\vspace{1mm}\end{minipage}  & 8 (58.8) & 9 (61.2) & 10 (68.9) & 9 (61.5) & 12 (60.3) \\
   \begin{minipage}{3.9cm}\setstretch{0.65}Nr Deliveries vacuum C (\%)\vspace{1mm}\end{minipage}& 4 (78.9) & 4 (81.6) & 4 (88.0) & 5 (82.3) & 5 (77.4) \\
   Nr of C-sections C (\%) & 6 (69.1) & 6 (74.5) & 5 (85.5) & 6 (77.8) & 6 (74.6) \\
   \textbf{Nr of Pregnancies C (\%)} & 13 (44.2) & 9 (42.9) & 11 (42.0) & 13 (40.2) & 19 (42.1) \\
   \textbf{Nr of born babies C (\%)} & 9 (38.4) & 9 (42.6) & 10 (41.2) & 10 (43.2) & 12 (38.8) \\
   \textbf{Nr of consultations $\mu (\sigma)$ } & 6.8 (4.0) & 7.7 (3.2) & 9.3 (4.5) & 8.9 (5.5) & 8.4 (5.1) \\
   Pelvis Adequacy  C (\%)  & 4 (89.6) & 4 (52.9) & 3 (93.1) & 4 (81.5) & 4 (82.6) \\
   \textbf{Position Admission C (\%)} & 6 (84.5) & 7 (61.3) & 5 (89.2) & 4 (74.2) & 7 (73.1) \\
   \textbf{Position on Delivery C (\%)} & 5 (93.0) & 5 (93.6) & 5 (94.8) & 5 (94.2) & 5 (93.9) \\
   
   \textbf{Pregnancy Type C (\%)} & 7 (88.0) & 7 (85.4) & 7 (86.0) & 7 (92.9) & 7 (85.4) \\
   \textbf{Robson Group C (\%)}& 11 (27.2) & 11 (24.7) & 11 (21.4) & 11 (26.7) & 11 (24.4) \\
   \begin{minipage}{3.7cm}\setstretch{0.65}Rupture amniotic pocket before delivery C (\%)\vspace{1mm}\end{minipage} & 2 (85.0) & 2 (84.4) & 2 (89.9) & 2 (93.8) & 2 (88.5) \\
   Smoker C (\%) & 2 (91.0) & 2 (90.7) & 2 (85.5) & 2 (89.9) & 2 (88.1) \\
   \textbf{Spontaneous Delivery C (\%)} & 2 (64.9) & 2 (64.0) & 2 (64.7) & 2 (62.9) & 2 (64.9) \\
   Weeks on Admission $\mu (\sigma)$ & 38.7 (1.8) & 39.0 (2.0) & 38.6 (2.1) & 38.8 (1.9) & 38.7 (2.2) \\
   \textbf{Weeks on Delivery $\mu (\sigma)$ }  & 38.8 (1.8) & 39.2 (1.7) & 38.7 (2.0) & 39.0 (1.6) & 38.9 (2.0) \\
   \textbf{Weeks on Admission $\mu (\sigma)$ } & 77.7 (13.4) & 79.2 (14.7) & 76.7 (13.0) & 83.1 (15.2) & 78.8 (14.5) \\
   \textbf{\begin{minipage}{3.3cm}\setstretch{0.65}Weight start of pregnancy $\mu (\sigma)$ \vspace{1mm}\end{minipage}}  & 65.6 (13.5) & 66.0 (13.7) & 65.6 (14.1) & 67.4 (14.6) & 66.0 (14.1) \\
   
   
   
   \bottomrule
   \end{tabular}
   

\end{table}
}
%TC:endignore
