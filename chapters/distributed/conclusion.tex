% !TeX root = ../../thesis.tex

This research demonstrated the efficacy of distributed models using real-world data by comparing their performance with that of local models, which are trained with data from individual silos, and centralised models, which utilize data from all silos. The findings reveal that an ensemble of models, essentially a distributed model as investigated in this study, can capture the nuances of the data, achieving performance comparable to a model constructed with comprehensive data. Even though The performance of these models is influenced by factors such as the inherent characteristics of the target variables and the data distribution across different silos, we are now fairly confident that distributed learning is a step forward regarding data privacy without loss of predictive performance when compared with centralised and local models.
Considering the robust performance metrics observed, with AUROC/AUPRC scores exceeding 80\% and MAE maintained below 1, further investigation into distributed models is warranted. Specifically, we aim to develop distributed models for predicting clinical outcomes, such as delivery type or Robson Group classifications, which hold significant potential for real-world clinical application like reducing unnecessary Cesarean Sections or accelerating diagnosis. These findings underscore that distributed learning not only advances data privacy but also maintains high prediction accuracy, promising substantial benefits for clinical practices.
