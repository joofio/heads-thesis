% !TeX root = ../thesis.tex

In 1996, David Sackett and colleagues defined \ac{ebm} as \textit{the conscientious, explicit, and judicious use of current best evidence in making decisions about the care of individual patients} \cite{sackettEvidenceBasedMedicine1996}. Despite having historical antecedents dating back to at least the 19th century, the term "evidence-based medicine" was first coined by a team at McMaster University in Canada in the 1980s \cite{thomaBriefHistoryEvidenceBased2015}. This was a time when clinical decision-making was mostly based on untested observations and physicians' experience, leading to variability in treatment strategies. The birth of \ac{ebm} marked a pivotal moment in medical history, aiming to standardize patient care and improve outcomes. So, \ac{ebm} is still a relatively recent concept in healthcare, which entails integrating the best available research evidence with clinical experience and patient values to make decisions about patient care. \Ac{ebm} can be defined into three major pillars:

\begin{myitemize}
    \item Best available evidence
    \item Clinical expertise
    \item Patient values, expectations, and/or wishes.
\end{myitemize}

Clinical expertise refers to the acumen and discernment gained from hands-on clinical experiences and consistent practice. This expertise manifests notably in enhanced diagnostic abilities and in the considerate recognition of a patient's unique circumstances, rights, and wishes when making care decisions. The term "Best available evidence" pertains to pertinent clinical studies, often stemming from epidemiological investigations. This is linked with the ability (and willingness) to challenge current diagnostic methods and treatments, introducing alternatives that are more robust, precise, effective, and safer. 

Without experience, clinical practices blindly follow the best available evidence, which is not always the best option for the patient, since sometimes it may be inapplicable to a specific scenario. Without evidence, clinical practice becomes stagnant and unable to evolve \cite{sackettEvidenceBasedMedicine1996}.

The main concept of \ac{ebm} is the hierarchy of evidence, which classifies different types of research studies based on their methodological quality and applicability to patients. At the top of this hierarchy are \acp{rct} and systematic reviews of \acp{rct}, which are considered to provide the most robust evidence. Observational studies, case series, and expert opinions are further down the hierarchy due to their inherent limitations (figure \ref{fig:ebm}). \ac{ebm} advocates for the application of the highest level of evidence available in clinical decision-making.

\begin{figure}
    \centering
    %\includegraphics[width=\textwidth]{image.png}
    \includegraphics[scale=0.55]{figures/ebm.png}
    
    \caption{\acl{ebm} diagram adapted from \cite{greenhalghHowReadPaper2019}} \label{fig:ebm}
    \end{figure}

Historically, medical decisions leaned heavily on anecdotal observations and the prevailing beliefs of seasoned practitioners. To underscore the dangers of relying solely on such expert opinions, Sackett frequently recounted the circumstances surrounding George Washington's unfortunate end. Despite being in good health at the age of 68, Washington developed epiglottitis. Rather than opting for a tracheostomy, a treatment method known since ancient Greek times, his physicians, guided by the prevailing expert opinion, chose bloodletting as the course of action. Tragically, this decision led to Washington's likely preventable death, highlighting the critical importance of grounding medical decisions in robust evidence.

Nonetheless, \ac{ebm} faces several criticisms. The primary critique is that \ac{ebm} merely reflects the core practice of medicine as it is already widely implemented. However, data indicate a different reality \cite{sackettEvidenceBasedMedicine1996}.
    
The second criticism targets the seemingly insurmountable challenge of staying current with the vast and ever-growing body of medical literature. While there are instances of clinicians successfully managing this feat, the argument still highlights a crucial issue: how to cope with the burgeoning overflow of evidence in modern times. It raises pertinent questions about not only keeping up with the literature but also ensuring its application in clinical practice. The utility of evidence lies in its application, making this a vital consideration. As we will explore in subsequent sections, this is where \ac{kdd} and \ac{ai} may offer significant contributions.
    
