Health institutions play a critical role in providing essential healthcare services to communities and ensuring that they operate efficiently and effectively is crucial. Benchmarking is a process that allows hospitals to compare their performance against that of other institutions, which can help identify areas of strength and weakness \cite{suydamPatientSafetyData2007}. By analyzing and evaluating performance metrics, such as patient outcomes, operational efficiency, and financial management, hospitals can identify best practices and make data-driven decisions to improve their overall performance. It can also help hospitals identify and implement innovative practices that can lead to better patient care and improved staff satisfaction \cite{hulsenSharingCaringData2020}.

However, despite the numerous benefits of benchmarking, some hospitals may be hesitant to participate due to concerns about revealing weaknesses or being perceived as inferior to their peers. The fear of being judged or penalized for poor performance can sometimes lead hospitals to avoid sharing data, making it difficult to accurately assess their performance and identify areas for improvement. Privacy issues and concerns turn this opportunity into an even less desirable path \cite{hulsenSharingCaringData2020}. To address these concerns, benchmarking initiatives often ensure the confidentiality and anonymity of data to encourage participation and foster trust among participating institutions. However, this is usually not enough. In 2019, as stated in the work of Villanueva et. al., \cite{villanuevaCharacterizingBiomedicalDataSharing2019}, 26\% of data-sharing initiatives are based on the aggregation of data and 24\% are based on sharing data in closed consortia. Only 15\% were based on open or controlled access.

To address concerns around privacy and confidentiality, we propose a new method of benchmarking based on clustering. This method involves grouping health institutions' outcomes into a known number of clusters, providing health institutions with the capability of positioning themselves in a range of clusters, without ever sharing the true means of their target data.

This approach to benchmarking not only addresses concerns around privacy and confidentiality. It has the potential to encourage greater participation in benchmarking initiatives, as hospitals can be assured of the anonymity and confidentiality of their data. By creating a more secure and private environment for benchmarking, hospitals can feel more comfortable sharing their data and participating in initiatives that can ultimately improve patient care and operational efficiency.

In conclusion, benchmarking is a crucial tool for hospitals to improve their performance and provide better care for their patients. While concerns around privacy and confidentiality may exist, the clustering approach to benchmarking provides a more accurate assessment of hospital performance while protecting the privacy of participating institutions. By embracing benchmarking initiatives and leveraging new approaches to benchmarking, hospitals can continuously improve their operations and ensure they provide the highest quality of care possible.
In this paper we propose:
\begin{myitemize}
    \item study how to implement clustering mechanism for benchmark
    \item address preprocessing issues for the raw data
    \item highlight pain points to deployment in the real world.
\end{myitemize}