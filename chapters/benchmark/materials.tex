We used two types of data in this paper. One is simpler and available online from the \ac{uci} dataset library, namely, the heart disease dataset \cite{misc_heart_disease_45}. We made fairly simple preprocessing on that dataset, namely removing the "?" by filling with null and then imputing missing values by imputing the mean on continuous variables and mode on categorical ones. We then separated the data into 3 distinct silos at random to mimic different health institutions.

In order to use real data and address problems found in the wild, we used clinical data gathered from nine different Portuguese hospitals regarding obstetric information, pertaining to admissions from 2019 to 2020. This originated from nine different files representing different sets of patients but with the same features associated with them. The software for collecting data was the same in every institution (although different versions existed across hospitals) - ObsCare. The data columns are the same in every hospital's database. Each hospital was considered a silo for comparison.