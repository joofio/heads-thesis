As per the discussion, there are a few issues to be addressed. First as per data preprocessing. In order to cluster be obtained, the null data must be filled out. There are a few strategies to do so. One option is to eliminate records/rows with empty cells or impute data. Either is a possibility, with pros and cons but the capability of having a dataset where no null records are present across several features may be difficult to find in the wild, especially since there are often optional and conditional fields in most \ac{ehr}. So imputation becomes more interesting, since it enables the usage of the whole dataset, even if biases are introduced.
Mixed types of datasets are also an issue to be aware of. In this case, not only imputation but also encoding a categorical variable is a vital step to take in the preprocessing phase. There are usually two main methods of data encoding, ordinal encoding and binary encoding. The first one keeps a unique column as the original data but maps every category to an increasing natural number. This creates an ordering in the data, often a misrepresentation of reality, not only due to this hierarchy but only because it assumes the differences between ranks of the hierarchy are always the same (1). The second is related to expanding the number of columns into the number of categories and creating 0s and 1s for the category. In machine-learning terms, binary seems more suited to be applied, but for benchmarking purposes, both are below par in terms of interpretability. For categorical data, we found out that K-modes seem to fulfil the requirements in a better way, providing better interpretability and reasoning about the results. However, it should be noted that we applied K-modes in a multivariate fashion and K-means in a univariate fashion.
Given that no percentage is provided, only the mode of the data, we believe it is still hard to get any real insight from the centroids. However, K-modes provides less information, since it only shows the top two categories. Which, for example. binary targets, provide little to no information. However, for larger categorical sets, the information provided could be better. Moreover, the number of centroids pretended could be more important as well. Agreeing on only 1 centroid would render the mode of the data provided by all silos, which could be more interesting.
As for continuous data, the use of real data was insightful, since \ac{bmi} had a few very big outliers around 300 and 400, which rendered centroids around that data. Even if not all silos had examples of these outliers, the ones that do have, pass that into the remaining. One possible workaround would be an addition of an extra cluster in order to catch possible outliers.
However, this should be addressed in detail and assess how outliers could subvert the data from the silos and how to work around that.

% should be addressed more in focus since it was outside of the scope of this paper
As for the next steps, a few issues could be addressed in depth. Regarding imputation, it could be interesting to understand how imputation, and which methods are more suitable to use for real-world scenarios. If the imputation of variables with a high null percentage influence significantly a centroid formation.
Communication could be important as well. Which action is to be taken when a silo is "down" and does not send information to the remaining. Cluster information should be addressed as well. They need to be agreed upon beforehand in the scope of this paper. But if it could be selected by each silo? Would that be feasible or a convergence could be achieved?
Finally, there is the question if there is the possibility of having leaks of true means across iterations by adversarial learning. At present time, we cannot be sure that the values are totally private, but then again, nothing is.


% abordar na questao a vantagem da privacidade mas que é possivel haver leak dos valores por adervisal learning.
