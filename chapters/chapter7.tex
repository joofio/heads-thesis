\begin{savequote}[75mm]
An expert is a person who has made all the mistakes that can be made in a very narrow field.
\qauthor{Niels Bohr}
\end{savequote}
\chapter{Limitations, future work and conclusions} \label{chap:conclusion}

In contemplating the limitations of this thesis, it's pivotal to recognize that each project, while rich in insights, is inherently constrained by its unique focus and context. These projects, though methodologically robust, were tailored to specific use cases and therefore may not offer broad generalizability. For instance, the project delineated in \ref{subsec:ipop} concentrates on a particular disease, while the projects in \ref{subsec:obs} and \ref{subsec:distributed} are oriented around a specific data type. Similarly, the project in \ref{subsec:dq} is circumscribed to a certain data type within a specific clinical specialty. This specificity implies that the outcomes of these projects might not be directly extrapolatable to other diseases or data types. However, it is crucial to note that the methodologies employed are versatile. For example, the real-time prediction techniques used in \ref{subsec:ipop} and the data analysis models in \ref{subsec:obs} and \ref{subsec:distributed} can be adapted for other contexts. Additionally, the \ref{subsec:similarity} method offers a versatile approach for analyzing diverse datasets and can be seamlessly integrated into various data pipelines.

Looking towards future endeavors, the foundation established in this thesis paves the way for practical aid to healthcare teams. The deployment of real-world \ac{cdss} is, however, a complex undertaking requiring substantial investment in terms of time, finances, and perseverance. Consequently, this aspect of tool deployment remains an avenue for future exploration. Nevertheless, extensive testing in real-world settings and the inclusion of clinicians in the developmental process imbue confidence in the readiness and potential impact of these tools upon deployment.


The comprehensiveness of this work necessitated a confluence of knowledge from disparate domains. This included insights from biology and chemistry for understanding healthcare nuances, process design for process formalization, mathematics and statistics for \ac{ml} and Exploratory Data Analysis , and interoperability standards for data amalgamation. Furthermore, ethical and privacy considerations were paramount for ensuring patient confidentiality and developing ethically sound models. Delving into healthcare terminologies, codifications, and semantics was essential for data interpretation, along with familiarity with clinical specialties like obstetrics and oncology. Bridging the gap between \acp{rct} and observational or Real-World Data required adeptness in study design.

This multidisciplinary nature of the field underscores a key challenge: the necessity for a diverse skill set or, alternatively, a collaborative team with varied expertise. Our observations underscore that the most successful projects in this domain are those that embrace such interdisciplinary collaboration. Therefore, the future trajectory of this field may well hinge on fostering such diverse, collaborative environments, enriching the scope and impact of healthcare data science.

%focus on impact and application in real world. 