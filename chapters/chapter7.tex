% !TeX root = ../thesis.tex

%\begin{savequote}[75mm]
%An expert is a person who has made all the mistakes that can be made in a very narrow field.
%\qauthor{Niels Bohr}

\begin{savequote}[75mm]
I may not have gone where I intended to go,
but I think I have ended up where I needed to be.
\qauthor{Douglas Adams}

\end{savequote}

\chapter{Conclusion} \label{chap:conclusion}
\initial{O}n a more personal note, I want to conclude this thesis, taking a step back and contemplate the work done so far. What I could have been done better and what I have learned throughout this journey.
But I also want to look ahead and point out possible directions in order to take the aim of this thesis further and create a substantial impact in the real world.

\section{Looking Back}
Reflecting on these last five years, my initial impression is one of shame regarding the earlier works in this thesis. Although it bothered me at first, I now believe it is a sign of growth and maturity. I want to believe that it shows how much I have learned and developed throughout these 5 years.

I have also discovered the value of collaboration and the importance of having a diverse team. The most successful projects are those that embrace interdisciplinary collaboration. Thus, the future trajectory of this field may well hinge on fostering such diverse, collaborative environments, enriching the scope and impact of healthcare data science.

Even though I've always liked the saying, 'If you want to go fast, go alone. If you want to go far, go together,' I never fully understood its importance until now. I've always tried to be a 'one-man show,' but there are limits to this approach, whether due to time or knowledge." \\

In contemplating the limitations of this thesis, it's pivotal to recognize that each project is inherently constrained by its unique focus and context. These projects were tailored to specific use cases and therefore may not offer broad generalizability. For instance, the work done in section \ref{subsec:obs} and \ref{subsec:distributed} are oriented around specific data types and formats and \ac{ml} models. Similarly, the project in \ref{subsec:dq} is circumscribed to a certain data type within a specific clinical specialty. This specificity implies that the outcomes of these projects might not be directly extrapolatable to other diseases or data types. However, it is crucial to note that the methodologies employed are versatile. For example, the real-time prediction techniques used in \ref{subsec:obs} and the data analysis models in \ref{subsec:ipop} and \ref{subsec:distributed} can be adapted for other contexts. Additionally, the \ref{subsec:similarity} method offers a versatile approach for analyzing diverse datasets and can be seamlessly integrated into various data pipelines.
But I cannot help to feel that the bigger limitation of this thesis is the lack of real-world deployment. While I have tried to simulate real-world scenarios, the lack of real-world deployment is a limitation. The deployment of real-world \ac{cdss} is, however, a complex undertaking requiring substantial investment in terms of time, finances, and perseverance. And of course, on top of this is the fact that we can bring problems to institutions and clinical workflow. Consequently, this aspect of tool deployment remains an avenue for future exploration. Nevertheless, extensive testing in real-world settings and the inclusion of clinicians in the developmental process imbue confidence in the readiness and potential impact of these tools upon deployment.



\section{Looking Ahead}
%focus on impact and application in real world. 
%TODO correct 

Looking towards future endeavors, the foundation established in this thesis paves the way for practical aid to healthcare teams. The deployment of real-world \ac{cdss} is, however, a complex undertaking requiring substantial investment in terms of time, finances, and perseverance. Consequently, this aspect of tool deployment remains an avenue for future exploration. Nevertheless, extensive testing in real-world settings and the inclusion of clinicians in the developmental process imbue confidence in the readiness and potential impact of these tools upon deployment.


The comprehensiveness of this work necessitated a confluence of knowledge from disparate domains. This included insights from biology and chemistry for understanding healthcare nuances, process design for process formalization, mathematics and statistics for \ac{ml} and Exploratory Data Analysis, and interoperability standards for data amalgamation. Furthermore, ethical and privacy considerations were paramount for ensuring patient confidentiality and developing ethically sound models. Delving into healthcare terminologies, codifications, and semantics was essential for data interpretation, along with familiarity with clinical specialties like obstetrics and oncology. Bridging the gap between \acp{rct} and observational or Real-World Data required adeptness in study design.

This multidisciplinary nature of the field underscores a key challenge: the necessity for a diverse skill set or, alternatively, a collaborative team with varied expertise. Our observations underscore that the most successful projects in this domain are those that embrace such interdisciplinary collaboration. Therefore, the future trajectory of this field may well hinge on fostering such diverse, collaborative environments, enriching the scope and impact of healthcare data science.
