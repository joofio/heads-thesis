\begin{savequote}[75mm]
If we knew what it was we were doing, it would not be called research, would it?
\qauthor{Albert Einstein}
\end{savequote}
\chapter{Introduction} \label{chap:intro}

\section{Rationale}
Healthcare practice revolves a lot around technology. Technology in the basic and rustic sense of the application of knowledge. It is an application science where we use our knowledge of biology, physics, chemistry, and math and apply those concepts in order to create treatments, diagnoses, procedures, etc.
However, in the last 20-30 years, technology, in the broader sense of digital and informatics technology started gaining traction in the healthcare space \cite{adler-milsteinHITECHActDrove2017}. Once a paper-based industry was now being digitalized. This digitalization is still ongoing \cite{abul-husnPersonalizedMedicinePower2019}. Firstly because of the rapid advancements of digital infrastructure and assets, and secondly due to the cost and willingness needed to do so \cite{kruseUseElectronicHealth2018,palabindalaAdoptionElectronicHealth2016}. But it is a fact that is happening and will probably continue to do so in the forthcoming years. Consequently, data has been increasing at a rapid pace as well. 
%artigos dados

So we are now faced with huge amounts of data, with the potential of describing clinical practice and outcomes from the whole world with a granularity impossible until now. But how can we materialize this potential?
being that the current gold-standard of evidence creation are \acp{rct} which can vary on quality and cost: whether time or resources. A \ac{rct} may cost no less than 20 million euros to run, according to a report submited to the \ac{us} Department of Health and Human Services \cite{sertkayaaylinEXAMINATIONCLINICALTRIAL2014}. Costing as much as 100 million \ac{us} dollars. This is indeed a very steep price to get the information we need to inovate.
Other mechanism, usually supported by these \acp{rct} are systematic reviews and meta-analysis, preconizated by \ac{ebm} which are estimated to cost around 140 thousands dollars each \cite{michelsonSignificantCostSystematic2019}.
So can this data, stored in \acp{ehr} could be leveraged to create:
\begin{myitemize}
    \item faster feedback to clinical practice
    \item better and faster methodologies for evidence synthesis
    \item confront and complement \acp{rct} with \ac{rwd}.
\end{myitemize}
But how could this be done? Was it a lack of technical skills? Still lack of data? Were the privacy and ethical barriers put in place to protect patients hindering clinical research? \\
Spoiler alert: I believe it is all of these and more. And so here we are.

%cost of systematics review
%https://www.ncbi.nlm.nih.gov/pmc/articles/PMC6722281/

%cost of trials
%https://www.sofpromed.com/how-much-does-a-clinical-trial-cost
\section{Research Objectives}
%%rework
This thesis has three main goals:
\begin{itemize}

    \item Goal 1: Synthesis of evidence (Chapter 3) %- sections \ref{sec:}
    \begin{itemize}
        \item  To gather and analyze existing evidence on risk and diagnostic factors for obstructive sleep apnea
(OSA) that posteriorly can be collected from Portuguese care centers, as for all available types of models to predict OSA.
    \end{itemize}


    \item Goal 2: Classification and Prediction (Chapter 4)
        \begin{itemize}
        \item  To propose diagnostic Bayesian network models based on factors identified, focusing on
interpretability.
    \end{itemize}


    \item Goal 3: Implementation (Chapter 5)
            \begin{itemize}
        \item  To develop and validate a decision support system, based on the proposed Bayesian network
model, to be used in Portuguese health system.
    \end{itemize}




\end{itemize}