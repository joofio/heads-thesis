\begin{savequote}[75mm]
If we knew what it was we were doing, it would not be called research, would it?
\qauthor{Albert Einstein}
\end{savequote}
\chapter{Introduction} \label{chap:intro}

\section{Rationale}
%%TODO nao gosto disto...
Healthcare practice revolves a lot around technology. Technology in the definitional sense of referring to \textit{" methods, systems, and devices which are the result of scientific knowledge being used for practical purposes"}. Healthcare and medicine are in fact applied sciences where we use our knowledge of biology, physics, chemistry, and math and apply those concepts in order to create treatments, diagnoses, procedures, etc.
However, in the last 20-30 years, computer science and informatics started gaining traction in the healthcare space \cite{adler-milsteinHITECHActDrove2017}. Once a paper-based industry is now being digitalized, but this process is prone to be never-ending  \cite{abul-husnPersonalizedMedicinePower2019} due to two main reasons. Firstly because of the rapid advancements of digital infrastructure and assets, that quickly deprecate current infrastructure and secondly due to the cost and willingness needed to implement or update said infrastructure \cite{kruseUseElectronicHealth2018,palabindalaAdoptionElectronicHealth2016}. However, this process is ongoing and will probably continue to increase in the forthcoming years. Consequently, \acp{ehr} and data have been increasing at a rapid pace as well.

Currently, the gold standard of evidence creation is \acp{rct} which can vary on quality, time, and resources. A \ac{rct} may cost no less than 20 million euros to run, according to a report submitted to the \ac{us} Department of Health and Human Services \cite{sertkayaaylinEXAMINATIONCLINICALTRIAL2014}. Costing as much as 100 million \ac{us} dollars. This is indeed a very steep price to get the information we need to innovate.
Another mechanism, usually supported by these \acp{rct} are systematic reviews and meta-analysis, precipitated by \ac{ebm} which are estimated to cost around 140 thousand dollars each \cite{michelsonSignificantCostSystematic2019}. But more than that is the time that it takes to create and publish a good paper of evidence synthesis, often keeping it hard to keep up with the pace of innovation.

So, could these huge amounts of clinical data generated by \acp{ehr} and \acp{his} have the potential to help with this matters? Could we use modern data analysis techniques to:
\begin{myitemize}
    \item faster feedback to clinical practice
    \item better and faster methodologies for evidence synthesis
    \item confront and complement \acp{rct} with \ac{rwd}.
\end{myitemize}

In theory, it makes total sense; let's use huge amounts of data, that mimic or describe actual care and outcomes from real people in order to create evidence and systems to further improve care for real people.


However, it is still inconclusive if \acp{cdss} based on this data have a beneficial impact on clinical and economic outcomes. Several studies still find it inconclusive \cite{muhiyaddinImpactClinicalDecision2020,kilsdonkFactorsInfluencingImplementation2017,muhiyaddinImpactClinicalDecision2020} to say the least.
Additionally, it is a popular assumption that 87\% of data science projects never get into production \cite{Why87Data2019}. This is particularly true for healthcare since we have several aspects that further hinder these changes, like data quality, privacy, ethics and Human-Computer Interaction.

%cost of systematics review
%https://www.ncbi.nlm.nih.gov/pmc/articles/PMC6722281/

%cost of trials
%https://www.sofpromed.com/how-much-does-a-clinical-trial-cost


%%% falar de perfil hibrido e saber de saude e de machine leraning e estatistica sera complicado


%%falar da pouca adoção de ML e CDSS
So, with this introduction, we have there is still a long way to go to harvest all the potential healthcare data has to offer. And so my research objectives are focused on powering up this adoption. What can be done to improve these chances? What can we bring to the table to enhance the rate of success?

%%%TODO finish

\section{Research Objectives}
%%rework
This thesis has three main goals:


\begin{itemize}
    \item Goal 1: Research methods for improving Data Quality. Whether through synthetic data generation to enlarge data volume and protect privacy or by creating automatic data quality assessments. Will be covered in sections \ref{subsec:gans}, \ref{subsec:tabular}, \ref{subsec:similarity} and \ref{subsec:dq}

    \item Goal 2: Assess alternative ways of usage of data without having access to all of it. This will be covered in sections \ref{subsec:distributed} and \ref{subsec:benchmark}.

    \item Goal 3: Difficulties and steps resulting from attempts to convert data into decisions and policies, whether through \ac{ml} or traditional statistics. This will be covered in sections \ref{subsec:ipop} and \ref{subsec:obs}.
\end{itemize}