We believe we have developed a robust system capable of detecting potentially incorrect C-section decisions, which could positively impact real-world medical practice. However, before implementation, several challenges must be addressed, particularly the need for further evaluation of the system's impact on clinical decision-making and the reasons underlying sub-optimal delivery type decisions. C-sections may be performed for various reasons, from a mother's preference to a decision made by the obstetrics team. This system is not designed to impede medical practice or to highlight flawed decisions, potentially scrutinizing specific professionals. Such caution is necessary when implementing systems like these. While having a high AUROC is beneficial, the real-world impact is another consideration. The assumptions and biases associated with autonomous systems supporting clinical practice must be carefully considered. Nonetheless, the metrics and results we have achieved so far are promising for positively influencing health and economic outcomes.