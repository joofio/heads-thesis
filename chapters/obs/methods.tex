We wrote all of the code in Python 3.9.7 with the usage of the \textit{scikit-learn} library \cite{scikit-learn}. All null representations were standardized. Data was prepossessed with the removal of features with high missing rates ($gt$ 90\% overall). All missing value representations were standardized. The imputation process was done using the \ac{knn} imputation method (for continuous variables) or a new category (NULLIMP) for categorical variables. 
For this purpose, the Birth Type was reduced to binary. All assisted birth were merged into vaginal birth and \ac{cs} remained as the other class. Procedures and diagnosis were also used and were encoded as binary features, we took the time to analyse each one of them in order to avoid leakage since there were procedures obviously related to \acp{cs} and vaginal deliveries.
Feature creation was done through the free-text variable relating to the medication prescribed. Features were collected from it and converted into \ac{atc} Classification Group level 4, which stands for chemical subgroups. We also created some new features from data in the dataset, namely new categories related to the labour and condition of the baby.
Also, a few data quality issues were addressed, like impossible values that were transformed into null. In this category, the main issues were \ac{bmi}/Weight and gestational age.
Finally, only a few columns were selected. We used a mixture of surveying the literature and the feature with greater correlation with the outcome.
The models tested were Logistic Regression, Decision Tree, Random Forest, 3 different Boosting methods (as implemented by \ac{xgboost}, \ac{lightgbm} and \textit{scikit-learn}) and a linear model based on Stochastic Gradient Descent.
The evaluation was done with repeated stratified cross-validation with 10 splits and 2 repetitions.
The API for serving the prediction model was developed with FastAPI.
%For trying to provide an explainable model, Shapley Additive exPlanations (SHAP) \cite{shapleyvalues} values were introduced in order to supply a more robust prediction.

Finally, a clinical evaluation was carried out with questionnaires sent to several obstetrics specialists in order to assess the validity and possible impact of the model.
