The first thing to address about this model is the number of biases that we introduced in the model by choice. We joined all vaginal delivery types into a single category (assisted and non-assisted) which introduces a  bias since these delivery modes are indeed different. Secondly, the fact that we want to predict if the delivery type was wrongly chosen, mainly for the case of a \ac{cs} that did not need to be so, is also a bias. We used this approach because the initially collected data did not have the representation of such events. So the biases of possibly wrong delivery types were present in the training data. We tried to go around this factor by selecting a threshold that gave the model higher sensitivity than specificity so that only large probabilities would trigger an alarm for human consideration. parallel to this, we are starting to gather labelled cases, with the help of clinicians in order to create a better training dataset.
Furthermore, since the data was collected from different hospitals,  differences in the data input can also occur. Even though the health information system is the same, the processes that originate the data and are being used for secondary purposes could introduce several biases in the data. This is an issue that was accepted from the start regarding the mechanism of data collection and model training.
Regarding the clinical evaluation, it was only possible to get a glimpse of several things due to the number of responders and the actual model still being implemented at the moment. Despite that, the results are encouraging, since the model seems to behave better than humans with the data provided. However, this is a biased vision, since clinicians in the real world, can have access to more data and information than the model has. It is encouraging, but caution is advised before more testing and evaluation are performed.
As for the deployment, further work could be the improvement of the API in order to map all variables to an ontology, making it easier for every system and person to access it and get a suggestion of the delivery type.
Finally, as for the next steps, we feel the assessment could still be improved, as for the model, taken that more data is provided and labelling is added as well. One issue interesting as well is the fact that 38\% of the answers regarding the most important data element missing from the patient record is data that is being collected but was missing for that patient in specific, raises an important question about data input and data interoperability. If we cannot have access to data when it matters most, it can become meaningless.


