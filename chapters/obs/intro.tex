% !TeX root = ../../thesis.tex

The ability to provide care to both women and newborns during delivery is one of the most important aspects of healthcare and is often used as a metric to assess healthcare as a whole across different countries. \acp{cs} are one of the most important aspects of delivering babies since it has a considerable impact on the mother's health and well-being. Despite the increased prevalence of this procedure over the last few years, the reasons behind this trend still remain unclear. Reports suggest that this increment is a global phenomenon, with the rate of \acp{cs} almost tripling from 6.7\% to 19.1\% between 1990 and 2014 \cite{betranIncreasingTrendCaesarean2016,chenNonClinicalInterventions2018}. Research on the impacts of \acp{cs} has focused on the risk of infection, haemorrhage, organ injury, and complications related to anaesthesia or blood transfusion \cite{caesereanrisk1,caesereanrisk2}.
There is also a higher risk of complications in subsequent pregnancies, such as uterine rupture, abnormal placental implantation, and the need for hysterectomy \cite{caesereanrisk3,caesereanrisk4}. As for the infant, \acp{cs} can lead to respiratory problems, asthma, and childhood obesity \cite{caesereanrisk3}.
In light of this, in 2015, \ac{who} stated that \acp{cs} rates higher than 10\% were not associated with a reduction in maternal or newborn mortality, even though other complications could not be fully assessed \cite{worldhealthorganizationhumanreproductionprogramme10april2015WHOStatementCaesarean2015}. In contrast, there is no evidence of the benefits of this procedure for women or babies when there is no clear medical need; therefore, it is paramount to focus on identifying and reducing such cases \cite{chenNonClinicalInterventions2018}.  It was estimated that in 2018, there were 8.8 million unnecessary C-sections \cite{hoxhaCaesareanSectionsHealth2021}.
It was with this in mind that a committee was established in Portugal with the specific purpose of decreasing the percentage of \acp{cs} nationwide. One of the policies resulting from this committee's work was the reduction of government funding per inpatient \ac{cs} episode for hospitals with rates of C-sections above 25\%; as of 2020, the number of \acp{cs} in Portugal stands at approximately 36.3\%, nearing the all-time high of 36.9\% in 2009 \cite{pordatacesarianas}. Furthermore, studies have shown that several countries could benefit from similar policies \cite{hoxhaCaesareanSectionsHealth2021}.
A quantitative analysis estimated that a reduction in \acp{cs} could save millions of dollars \cite{callanderFinancingMaternityEarly2020} worldwide. Therefore, lowering the proportion of \acp{cs} can yield health and financial benefits for both institutions and patients alike. With these considerations in mind, we developed a machine-learning algorithm-based support system to assist clinical teams in identifying cases of potentially unnecessary \acp{cs}. As such, in this paper, we propose to:


\begin{myitemize}
    \item elaborate on how clinical decision support systems for \acp{cs} can be developed using interoperability standards;
    \item understand, based on the data collected, which features have the most significant impact on predicting delivery type;
    \item conduct a concise economic analysis to assess the potential financial impact of implementing the proposed clinical decision support tool;
    \item  compare the system's output with clinicians' responses.

\end{myitemize}