The ability to provide care to both women and newborns during delivery is one of the most important aspects of healthcare and is often used as a metric to assess healthcare as a whole across different countries.
\acp{cs} are one of the most important aspects of delivering babies since it has a considerable impact on the mother's health and well-being. Despite this type of procedure increasing over the last few years, it is still illusive the reasons behind such events. Reports from 2016 suggest that this increment is a global phenomenon, being that from 1990 to 2014, this type of delivery almost increases by 3-fold from 6.7\% to 19.1\% \cite{betranIncreasingTrendCaesarean2016,chenNonClinicalInterventions2018}. Some of these impacts, being more prone to investigation in the last years, including the risk of infection, haemorrhage, organ injury and complications related to the use of anaesthesia or blood transfusion \cite{caesereanrisk1,caesereanrisk2}.
There is also a higher risk of complications in subsequent pregnancies like uterine rupture, abnormal placental implantation and the need for hysterectomy \cite{caesereanrisk3,caesereanrisk4}. As for the infant, \acp{cs} include the risk of respiratory problems, asthma and obesity in childhood \cite{caesereanrisk3}.
Facing this, in 2015, World Health Organisation released a statement regarding \acp{cs} rates. Even when other complications could not be totally assessed, it was concluded that \ac{cs} rates higher than 10\% were not associated with a reduction in maternal or newborn mortality \cite{worldhealthorganizationhumanreproductionprogramme10april2015WHOStatementCaesarean2015}.

Since there is no evidence that this type of procedure is beneficial for women or babies when there is no clear need for it, the focus on filtering such cases is important \cite{chenNonClinicalInterventions2018}.
Moreover, particularly in Portugal, \acp{cs} are used as a way of financing healthcare institutions. This was implemented as a strategy of decreasing \ac{cs}s across the country. A committee was created especially with the purpose of reducing the percentage of \acp{cs} nationally. One of the actions taken along this creation was the reduction of government funding for hospitals with rates of \acp{cs} above 25\%.
In 2020, the number of \acp{cs} in Portugal is about 36.3\%. Almost at the all-time high of 36.9\% in 2009 \cite{pordatacesarianas}.
So, lowering the proportion of \ac{cs} can provide health and financial benefits to institutions and populations alike. With this in mind, we developed a  machine-learning algorithm-based support system to assist clinical teams to detect cases of potentially unnecessary \acp{cs} for analysis. So in this paper, we propose:
\begin{myitemize}
    \item help to provide a method of bringing to the discussion of clinical staff possible less than optimal care regarding deliveries;
    \item elaborates on how clinical decision support systems can be developed using interoperability standards;
    \item understand, based on the gathered data, which are the more impacting features for predicting delivery type outcome;
    \item open a research path regarding the evaluation of this type of clinical decision support system prior to the delivery;
    \item Perform a concise economical analysis to assess the potential financial impact of implementing the proposed clinical decision support tool.

\end{myitemize}