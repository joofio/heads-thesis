Regarding the related work, several teams already tackled the potential of predicting the delivery type before birth. However, we believe that creating such a system may have a huge impact on the clinical team, patient and family regarding expectations. For such a system to be possible to enter clinical practice safely, as we hope this one does, several tests and evaluations should be done before going live.
On the other hand, we are aiming for a \textit{post-partum} analysis in order to signalise potential sub-optimal decisions so the clinical teams can evaluate the case afterwards, and hopefully, learn about what could have been done better.
Works and studies on this matter have been done before, related to second opinions in the healthcare practice regarding the decision of the \ac{cs} \cite{mandatorysecondopinion} and the implementation of clinical guidelines help as well \cite{reducingcaeresan}. We hope to provide support to help teams go in this direction.

Nevertheless, in the literature, there are works related to predicting a successful vaginal birth after a previous \ac{cs}, like the work of Lipschuetz et al., \cite{lipschuetzPredictionVaginalBirth2020} where a gradient boosting method was used to predict such event using prenatal data to do so. Grobman et al., \cite{grobman_development_2007} did similar work with a multivariable logistic regression model.
There was also the usage of different modalities of data for predicting delivery type. The work of Fergus et al. \cite{fergusClassificationCaesareanSection2017} introduces a method of predicting delivery type using the fetal heart rate signals. Similarly, the work from Saleem et al. \cite{saleemStrategyClassificationVaginal2019a} proposes a method of predicting delivery type using interactions between fetal heart rate and maternal uterine contraction.
Finally, there are also works that focus on predicting delivery mode before childbirth like the work of Ullah et al. \cite{ullah_reliable_2021} where a boosting algorithm was used in order to predict delivery mode with enriched datasets. Also the work of Gimovsky et al. \cite{gimovskyBenchmarkingCesareanDelivery} where decision trees were introduced  to predict \acp{cs} by physician group. \\
However, as far as we know, there was no model tested (even in a controlled setting) in clinical practice, with no interoperable format of communication like \ac{fhir} or employed to \textit{post-partum} setting and finally, none with real-world data related with Portuguese hospital, making that our paper could be a novelty under very different dimensions.


%---predict c-section

