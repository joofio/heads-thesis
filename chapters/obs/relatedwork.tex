% !TeX root = ../../thesis.tex

Regarding the related work, several teams already tackled the potential of predicting the delivery type before birth. We found studies related to predicting a successful vaginal birth after a previous C-section, such as the work of Lipschuetz et al., \cite{lipschuetzPredictionVaginalBirth2020}  where a gradient boosting method was used to predict such an event using prenatal data to do so. Grobman et al., \cite{grobman_development_2007} performed a similar study with a multivariable logistic regression model. Different modalities of data were also used to predict delivery type. Fergus et al. \cite{fergusClassificationCaesareanSection2017} introduces a method of predicting de- livery type using the fetal heart rate signals. Similarly, the work from Saleem et al. \cite{saleemStrategyClassificationVaginal2019a} proposed a method for predicting delivery type using interactions between the fetal heart rate and maternal uterine contraction. Finally, there are also studies that focus on predicting the delivery mode like the work of Ullah et al. \cite{ullah_reliable_2021} where a boosting algorithm was used in order to predict delivery mode with enriched datasets. In addition, Gimovsky et al. \cite{gimovskyBenchmarkingCesareanDelivery} introduced decision trees to predict \ac{cs} by physician group with 0.73 \ac{auroc}. The works of \cite{rossiRiskCalculatorPredict2020b} resulted in a seven-variable model with 0.78 \ac{auroc} and the works of \cite{guedaliaRealtimeDataAnalysis2020} resulted in a model with 0.82 \ac{auroc}, reaching 0.93 with a first cervical examination. Finally, the works of Meyer et al. \cite{meyerImplementationMachineLearning2020} focused around selecting suitable for a trial of labour after caesarean with \ac{auprc} around 0.351. However, to the best of our knowledge, there was no model tested in clinical practice, with an interoperable format of communication like \ac{fhir}, which tried to not only predict delivery type but also provide support about possibly worn deliveries and none with simulation about financial implication, making our paper a potential novelty on different dimensions.