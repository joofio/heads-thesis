\begin{savequote}[85mm]
    I don’t want to insist on it, Dave, but I am incapable of making an error.
    \qauthor{HAL 9000}
    \end{savequote}
    
\chapter{Identify Attempts to Convert Health Data into Decisions and Policies}\label{chap:goal3}
This chapter focuses on the practical application of health data in influencing decisions and policies, particularly in the realms of drug evaluation and obstetrics, as detailed in sections \ref{subsec:ipop} and \ref{subsec:obs}.

Section \ref{subsec:ipop} represents an innovative application of causality principles and transparent \ac{ml} models to assess the real-world effectiveness of two groups of breast cancer drugs. Beginning with traditional analysis methods, the study progressively adopts more complex techniques, including \ac{iptw} to enhance the comparative assessment of these treatments. This approach exemplifies how health data, analyzed through advanced methodologies, can influence drug policy and treatment choices in clinical settings.

Section \ref{subsec:obs} is an extension of research in distributed data mechanisms (referenced in section \ref{subsec:distributed}), uses machine learning to develop a \ac{cdss} designed to assist in evaluating \ac{cs}. The system's interoperability and its focus on supporting subpar evaluation of \ac{cs} demonstrate the utilization of health data in creating tools that aid in decision-making processes in obstetrics. This section underscores the importance of applying health data analytics in real-time clinical environments to inform decisions and shape obstetrical policies.



\section{How Can We Leverage Data to Assess Treatment Efficacy?}\label{subsec:ipop}
This section is based on the paper entitled "Comparative Analysis of Palbociclib and Ribociclib: A real world data and Propensity Score-Adjusted Evaluation with endocrine therapy". This was a method of applying the knowledge of causality and transparent \ac{ml} models in order to assess the real-world effect of two drugs for breast cancer. We started with traditional analysis and then moved to a more complex approach, using \ac{iptw} methods in order to further compare treatments.
    
    
\subsection{Introduction}
    % !TeX root = ../../thesis.tex

In recent years, the use of \ac{ai} and \ac{ml} algorithms has gained increasing prominence in healthcare research and practice. One of the key requirements for the successful application of these methods is access to large, high-quality datasets. However, in many cases, the availability of such datasets can be limited due to issues around data privacy, security, and ethical concerns \cite{chingOpportunitiesObstaclesDeep2018a}. To address this challenge, synthetic data has emerged as a promising solution. Synthetic data refers to artificially generated data that closely mimic the statistical properties and patterns of real-world data \cite{mullerEvaluationSyntheticElectronic2022}.

Synthetic data has the potential to overcome many of the limitations associated with real-world data, such as the lack of sufficient data volume, noise, and privacy concerns. Even though there are still doubts if the privacy part is the silver bullet sometimes referred to \cite{stadlerSyntheticDataPrivacy2020}, the upsampling part is a standard use for years now. However, the quality of synthetic data generated by various techniques can vary significantly, and it is essential to assess the quality of synthetic data before its usage. In healthcare, the assessment of synthetic data is crucial to ensure that it can provide valid insights and inform decision-making processes.

The assessment of synthetic data in healthcare is essential for its successful use in various applications, such as developing predictive models, testing algorithms, and conducting clinical trials. The use of synthetic data can significantly enhance the efficiency and effectiveness of healthcare research and practice. However, it is crucial to ensure that the synthetic data used in these applications are of high quality and validated to provide reliable and valid insights. The evaluation of synthetic data quality involves comparing its statistical properties and patterns with those of the original data. We can assess how similar columns are to each other through several statistical tests, and then we can infer some inter-column properties with methods like cross-validation, where two datasets are split into train tests and cross-tested and then the ratio between the evaluation result of both datasets is used as a metric \cite{mullerEvaluationSyntheticElectronic2022,goncalvesGenerationEvaluationSynthetic2020a}. However, this methodology is a big proxy for such an inter-column relationship. Can we try to provide a better metric than this one to evaluate how similar are the inter-column relationship of two distinct datasets? In this paper, we suggest using feature importance values to create a more explainable and reasonable metric for inter-column relationships.

    \subsection{Materials \& Methods}
    

\subsection{Study Design}

This retrospective study was designed in 2022. The study aimed to evaluate the clinical benefit and long-term survival of patients with HR+/HER2- that started treatment with CDK4/6 inhibitors plus endocrine therapy in different lines of treatment between the 14th of March 2017 and the 31st of December 2021. The follow-up period was set until June 2022. Inclusion criteria: women and men, Hormone receptor-positive and HER2 negative in the primary tumor or metastatic site after biopsy. Exclusion criteria: Patients that had only one ambulatory medication, and patients involved in clinical trials, diagnosed with other neoplasms or with active treatment during the study period. The control group was defined by a population of patients, that were treated with hormone therapy as first-line (due to bone metastases only) between 2015 and 13 of match 2017.
The evaluation of effectiveness will involve overall survival and progression-free analysis. We will compare the two different cyclin-dependent kinase inhibitors in terms of effectiveness in real-world patients and will also compare the effectiveness of this class combined with endocrine therapy against traditional endocrine therapy.


\subsection{Data collection}
All data were collected from medical and administrative records from baseline to last visit or death. The data was collected from Instituto Português de Oncologia – Porto (IPO-P). Table 1 shows a comparison between the groups.
Data included for population treated with CDK4/6 inhibitors plus endocrine therapy: demographic information, age at first diagnosis and age at the beginning of treatment, clinical characteristics and performance status by Eastern Cooperative Oncology Group scale (ECOG), treatment line and treatment schema - CDK4/6 inhibitor and endocrine therapy, stage of cancer, site of metastases (bone, soft tissue, visceral, central nervous system-CNS with or without another site).
Data included for the population treated with endocrine therapy as first-line: demographic information, age at first diagnosis and age at the beginning of treatment, clinical characteristics and performance status by Eastern Cooperative Oncology Group scale (ECOG), stage of the cancer.
For comparison purposes, we used palbociclib and ribociclib since we had a small number of patients treated with abemaciclib (12).

 
\begin{table}
\caption{Descriptive statistics of cyclin-dependent kinase inhibitors group and endocrine therapy group. The Drug/combination refers to the actual drug or the combination for CDK4/6}
\centering
\label{tab:stats_ipop_cdk}
\begin{tabular}{l|lllll}
\toprule
   Column &               Silo 6 &               Silo 7 &               Silo 8 &               Silo 9 &                Agrr. \\
\midrule
       IA &    31.3 \textbf{5.2} &    31.4 \textbf{5.4} &    31.5 \textbf{5.6} &    30.1 \textbf{5.6} &    31.1 \textbf{5.6} \\
       GS & a,rh.. \textbf{42\%} & a,rh.. \textbf{39\%} & a,rh.. \textbf{40\%} & a,rh.. \textbf{42\%} & a,rh.. \textbf{40\%} \\
       PI &   65.6 \textbf{13.5} &   66.0 \textbf{13.7} &   65.6 \textbf{14.1} &   67.4 \textbf{14.6} &   66.0 \textbf{14.1} \\
      PAI &   77.7 \textbf{13.4} &   79.2 \textbf{14.7} &   76.7 \textbf{13.0} &   83.1 \textbf{15.2} &   78.8 \textbf{14.5} \\
      IMC &    24.9 \textbf{5.1} &    24.9 \textbf{7.0} &    24.8 \textbf{8.0} &    25.7 \textbf{5.6} &    25.1 \textbf{7.0} \\
      CIG &   Null \textbf{91\%} &   Null \textbf{91\%} &   Null \textbf{86\%} &   Null \textbf{90\%} &   Null \textbf{88\%} \\
    APARA &    1.0 \textbf{38\%} &   Null \textbf{43\%} &   Null \textbf{41\%} &   Null \textbf{43\%} &   Null \textbf{39\%} \\
   AGESTA &    1.0 \textbf{44\%} &      1 \textbf{43\%} &    1.0 \textbf{42\%} &    1.0 \textbf{40\%} &    1.0 \textbf{42\%} \\
       EA &   Null \textbf{59\%} &   Null \textbf{61\%} &   Null \textbf{69\%} &   Null \textbf{61\%} &   Null \textbf{60\%} \\
       VA &   Null \textbf{79\%} &   Null \textbf{82\%} &   Null \textbf{88\%} &   Null \textbf{82\%} &   Null \textbf{77\%} \\
       FA &   Null \textbf{82\%} &   Null \textbf{86\%} &   Null \textbf{94\%} &   Null \textbf{89\%} &   Null \textbf{83\%} \\
       CA &   Null \textbf{69\%} &   Null \textbf{75\%} &   Null \textbf{85\%} &   Null \textbf{78\%} &   Null \textbf{75\%} \\
       TG & espo.. \textbf{88\%} & espo.. \textbf{85\%} & espo.. \textbf{86\%} & espo.. \textbf{93\%} & espo.. \textbf{85\%} \\
        V &      s \textbf{97\%} &      s \textbf{99\%} &      s \textbf{98\%} &      s \textbf{99\%} &      s \textbf{98\%} \\
    NRCPN &     6.8 \textbf{4.0} &     7.7 \textbf{3.2} &     9.3 \textbf{4.5} &     8.9 \textbf{5.5} &     8.4 \textbf{5.1} \\
       VP &   Null \textbf{68\%} &   Null \textbf{74\%} &   Null \textbf{71\%} &   Null \textbf{78\%} &   Null \textbf{76\%} \\
      VCS &   Null \textbf{53\%} &      s \textbf{87\%} &      s \textbf{63\%} &      s \textbf{87\%} &      s \textbf{68\%} \\
      VNH &   Null \textbf{62\%} &      s \textbf{63\%} &      s \textbf{69\%} &      s \textbf{83\%} &      s \textbf{69\%} \\
        B &   Null \textbf{90\%} &   Null \textbf{53\%} &   Null \textbf{93\%} &   Null \textbf{82\%} &   Null \textbf{83\%} \\
       AA &   Null \textbf{84\%} & apr... \textbf{61\%} &   Null \textbf{89\%} &   Null \textbf{74\%} &   Null \textbf{73\%} \\
       BS &   Null \textbf{99\%} &   Null \textbf{98\%} &   Null \textbf{99\%} &   Null \textbf{95\%} &   Null \textbf{95\%} \\
       BC &  Null \textbf{100\%} &  Null \textbf{100\%} &  Null \textbf{100\%} &   Null \textbf{97\%} &   Null \textbf{97\%} \\
      BDE &  Null \textbf{100\%} &  Null \textbf{100\%} &  Null \textbf{100\%} &   Null \textbf{97\%} &   Null \textbf{97\%} \\
      BDI &  Null \textbf{100\%} &  Null \textbf{100\%} &   Null \textbf{99\%} &   Null \textbf{97\%} &   Null \textbf{96\%} \\
       BE &  Null \textbf{100\%} &  Null \textbf{100\%} &  Null \textbf{100\%} &   Null \textbf{97\%} &   Null \textbf{96\%} \\
       BP &  Null \textbf{100\%} &  Null \textbf{100\%} &  Null \textbf{100\%} &   Null \textbf{97\%} &   Null \textbf{97\%} \\
      IGA &    38.7 \textbf{1.8} &    39.0 \textbf{2.0} &    38.6 \textbf{2.1} &    38.8 \textbf{1.9} &    38.7 \textbf{2.2} \\
     TPEE &   Null \textbf{65\%} &   Null \textbf{64\%} &   Null \textbf{65\%} &   Null \textbf{63\%} &   Null \textbf{65\%} \\
     TPEI &   Null \textbf{92\%} &   Null \textbf{86\%} &   Null \textbf{87\%} &   Null \textbf{94\%} &   Null \textbf{93\%} \\
      RPM &   Null \textbf{85\%} &   Null \textbf{84\%} &   Null \textbf{90\%} &   Null \textbf{94\%} &   Null \textbf{88\%} \\
       DG &   Null \textbf{92\%} &   Null \textbf{88\%} &   Null \textbf{90\%} &   Null \textbf{87\%} &   Null \textbf{89\%} \\
       TP & part.. \textbf{54\%} & part.. \textbf{52\%} & part.. \textbf{48\%} & part.. \textbf{59\%} & part.. \textbf{51\%} \\
      ANP & cefá.. \textbf{93\%} & cefá.. \textbf{94\%} & cefá.. \textbf{95\%} & cefá.. \textbf{94\%} & cefá.. \textbf{94\%} \\
     TPNP & espo.. \textbf{64\%} &  Null \textbf{100\%} & espo.. \textbf{50\%} & espo.. \textbf{65\%} & espo.. \textbf{53\%} \\
      SGP &    38.8 \textbf{1.8} &    39.2 \textbf{1.7} &    38.7 \textbf{2.0} &    39.0 \textbf{1.6} &    38.9 \textbf{2.0} \\
       GR &      1 \textbf{27\%} &      1 \textbf{25\%} &      1 \textbf{21\%} &      3 \textbf{27\%} &      1 \textbf{24\%} \\
       \midrule
N (total) &                12002 &                 8258 &                 6693 &                11786 &                80874 \\
\bottomrule
\end{tabular}

\end{table}




\subsection{Statistical Analysis}
%copied
R was used for statistical analysis. Demographic, clinical characteristics and side effects were analyzed using descriptive statistics (count, percentages and median/range). Kaplan–Meier test was used to determine the median PFS and OS in the entire population and subgroups. Log-rank test was used for comparisons of PFS and OS among different subgroups. Cox Regression was used to assess feature importance and impact. All statistical tests were two-sided, and the significance level was 0.05. The evaluation of the proportional hazards assumptions was done by Schoenfeld residues analysis.
We applied propensity score weights to achieve a more robust comparison between the two groups of CDK4\/6i. We used the existence of visceral metastases, treatment line, age at treatment start, and stage. We used the WeightIt package for R \cite{WeightIt}. We applied the weights to the Kaplan-Meier curves and to the Cox Regression. We applied the weights to get the ATE which is $E[Y_i(1)-Y_i(0)]$, the average effect of moving an entire population from untreated to treated, or from one drug to the other. Weights were used instead of matching since it is more suited for calculating ATE and the need to preserve the sample size since it is already small from the start. The formula for calculating the weights was through propensity score weighting with GLM. Multiple comparisons were done with the Benjamini-Hochberg (BH) method. 



%Pretende-se que os dados venham do Instituto Português de Oncologia – Porto (IPO-P). Pretende-se utilizar a base de dados do hospital dos últimos 5 anos.
%O estudo será registado e respeitará todos os requisitos éticos de aprovação a comissão de ética de cada instituição participante. Caso a instituição tenha um encarregado de Proteção de Dados (EPD), este será contactado a fim de dar seu parecer, e caso necessário, a sua opinião para melhorar possíveis pontos relacionados à segurança de dados.

%The goal is to identify the clinical or biological variables that have the most impact on a specific outcome. To achieve this, considering the statistical models found, we can model things in terms of time series or survival trees to group outcomes and the clusters found. With this, we can analyze the effectiveness of the clinical course, probabilities of recurrence, or survival rates by subgroup. These models and relationships can form the basis of a clinical decision support system and can be crucial for making better healthcare decisions.

%Some of the techniques used are unsupervised techniques such as k-means, DBSCAN, or hierarchical clustering.
%The goal is to identify the clinical or biological variables that have the most impact on a specific outcome. To achieve this, considering the statistical models found, we can model things in terms of time series or survival trees to group outcomes and the clusters found. With this, we can analyze the effectiveness of the clinical course, probabilities of recurrence, or survival rates by subgroup. These models and relationships can form the basis of a clinical decision support system and can be crucial for making better healthcare decisions.

%Some of the techniques used are unsupervised techniques such as k-means, DBSCAN, or hierarchical clustering.




% breast cancer cells have either estrogen (ER) or progesterone (PR) receptors or both. (HR +)
    \subsection{Results}
    The median \ac{os} in the entire population treated with \ac{cdk46i} was 46 months (95\% CI 39.4–55.6). Median \ac{pfs} was 20.1 months (95\% CI 18.3–24.2). Following this, we compared Palbociclib and ribociclib only as first-line treatments. We found that regarding \ac{os}, there is no significant difference between the two, but ribociclib is significantly better in terms of \ac{pfs} (\textit{P} value $\le$ 0.001) (Figure \ref{fig:interest}). Additionally, we compared the same \ac{cdk46i} with letrozole as a combination only (PAL-LT and RIB-LT). Regarding this scenario, we found out that both were similar in terms of \ac{os} and \ac{pfs}.


\begin{figure}[ht]
  \caption{Survival curves for Palbociclib and Ribociclib (1st line) - \ac{pfs} and \ac{os}}\label{fig:interest} 
  \includegraphics[scale=0.45]{figures/interest_curve_both.jpeg}%

\end{figure}


We then compared both with a cox regression, where \ac{os} shows no significant difference between palbociclib and ribociclib when adjusted to the stage, visceral metastases, age, treatment line, combination and \ac{ecog}. The proportional hazards' assumption was confirmed with \textit{P} values all over 0.10.
\begin{table}[ht]
  \centering
  \caption[Cox Regression with palbociclib and Ribociclib - \acs{pfs} and \acs{os}]{Cox Regression with palbociclib and Ribociclib - \ac{pfs} and \ac{os}}\label{tab:cox} 
  \includegraphics[scale=0.20]{figures/cox_both.png}%

\end{table}

When comparing \ac{et} with \ac{cdk46i} as first-line treatment (figure \ref{fig:grouped}). For this study we only compared patients with bone only metastasis. When comparing both \ac{cdk46i} combined with Fulvestrant or letrozole, we see that  Ribociclib (RIB+LT/FUL) is significantly better for \ac{pfs} (\textit{P} value $\le$ 0.001 \ac{hr}=0.21) but not \ac{os}. For Palbociclib as the first line with Fulvestrant or letrozole (PAL+LT/FUL), we see that there is no significant difference in terms of \ac{pfs} and \ac{os} (\textit{P}=0.57 and 0.51). We also applied the same analysis but comparing only the letrozole combination with letrozole alone (PAL-LT/RIB-LT vs LT). We found that both ribociclib and palbociclib are significantly better in terms of \ac{pfs} (\ac{hr} 0.65 for palbociclib and 0.27 for ribociclib) but not \ac{os}.
\begin{figure}[ht]
  \centering

  \caption[Survival curves (\ac{os} and \ac{pfs}) comparing \ac{et} to \ac{cdk46i} combined with fulvestrant or letrozole as 1st line.]{Survival curves (\ac{os} and \ac{pfs}) comparing \ac{et} to \ac{cdk46i} combined with fulvestrant or letrozole as 1st line. First row is \ac{cdk46i} combined fulvestrant or letrozole vs fulvestrant or letrozole. Second row is \ac{cdk46i} combined with letrozole vs letrozole alone. \textit{P} values shown as pairwise vs. ET. }\label{fig:grouped} 
  \includegraphics[scale=0.42]{figures/grouped_curve_both.jpeg}%

\end{figure}

When comparing palbociclib and ribociclib adjusted for \ac{ate} weights, we found a different scenario from previous assessments. There is a significant difference between the two in terms of \ac{os} (figure \ref{fig:propensity}). The weights were calculated as stated in the methods section.


\begin{figure}[ht]
  \centering

  \caption{Comparison of palbociclib and ribociclib survival curves adjusted for propensity scores}\label{fig:propensity} 
  \includegraphics[scale=0.42]{figures/propensity_score_both.jpeg}%

\end{figure}

The Cox regression adjusted for the variables and with the weights applied to render an \ac{hr}=0.55 [95\% CI 0.28-1.09;\textit{P}=0.086] for \ac{os}. The \ac{hr}for \ac{pfs} is 0.56 [95\% CI 0.32-1;\textit{P}=0.05].
    \subsection{Discussion}
    The aim of this study was to evaluate the real-world use of palbociclib and ribociclib in combination with\ac{et}for \ac{hr+}/\ac{her2} and compare this drug class with traditional \ac{et}. Few real-world evidence studies of palbociclib and ribociclib used in daily clinical practice have been published identifying clinical benefit, patient profile, and sequencing of treatment, with even less evidence for the Portuguese population.

When comparing with clinical trials, regarding patient profile, in our study, 51\% had visceral metastasis and 35\% had bone-only metastases compared with 49\% and 38\% in PALOMA-2, and 60\% and 25\% in PALOMA-3, respectively \cite{rugoImpactPalbociclibLetrozole2018,cristofanilliFulvestrantPalbociclibFulvestrant2016a}.
As for ribociclib and bone-only metastases, MONALEESA-7 \cite{tripathyRibociclibEndocrineTherapy2018} has 24\% and MONALEESA-2 has 40\% \cite{hortobagyiUpdatedResultsMONALEESA22018} and our study has 30\%. Regarding menopausal status, our study has 20\% premenopausal and 80\% postmenopausal. 



Of note, the range of median \ac{pfs} for first-line palbociclib was 15.5–25.5 months, which is shorter than 27.6 months observed in a post hoc analysis of the PALOMA-2 clinical trial with extended follow-up \cite{rugoImpactPalbociclibLetrozole2018}, but in line with RWE studies (13.3–20.2 months) \cite{harbeckCDK4InhibitorsHR2021}. When assessed with only letrozole as a combination, the median \ac{pfs} increased to 28.6 months [95\% CI 25.5-not reached]. Additionally, analyzing the postmenopausal women subgroup, palbociclib showed a median \ac{pfs} of 16.3 months [95\% CI 12.9 -20]. Furthering analysis of the postmenopausal and with letrozole, the median was 47.6 months [95\% 25.6-2–not reached].

As for ribociclib, median survival time was not reached whether in \ac{os} and \ac{pfs}. So we can at least say that the median \ac{pfs} is longer than 50 months. This is longer than the median \ac{pfs} of 23.8 months (95\% CI 19.2–not reached) reported in the MONALEESA-7 trial \cite{tripathyRibociclibEndocrineTherapy2018} and longer than  25.3 months (95\% CI 23.0–30.3) in the MONALEESA-2 trial \cite{hortobagyiUpdatedResultsMONALEESA22018}. Regarding the subgroup analysis of postmenopausal women, we noticed that the median was not reached for women treated with ribociclib and fulvestrant or letrozole (RIB-LT/FUL) and postmenopausal women treated with ribociclib in combination with letrozole (RIB-LT).

When directly comparing ribociclib and palbociclib without any adjustments, one might deduce that ribociclib is superior to palbociclib. However, after adjusting for confounding variables, there is no significant difference between the two inhibitors in terms of \ac{pfs} or \ac{os} as indicated in table 2. This observation is further corroborated by the lower plots in figure 1, where even a subgroup analysis of \ac{cdk46i} combined solely with letrozole reveals non-significant difference between the two.

In the first-line comparison, the analysis of \ac{os} outcomes reveals no substantial difference between \ac{et} alone and the combination of \ac{cdk46i} with \ac{et}, irrespective of whether the \ac{cdk46i} are administered with fulvestrant or letrozole  (PAL-LT/FUL vs LT/FUL; \textit{P}=0.57 | RIB-LT/FUL vs LT/FUL \textit{P}=0.069) or exclusively with letrozole (PAL-LT vs LT; p = 0.979 | RIB-LT vs LT; \textit{P}=0.179)(figure \ref{fig:grouped} left). With respect to \ac{pfs}, ribociclib demonstrates superior efficacy when compared its combination with any of the adjuvants to these adjuvants alone (RIB-LT/FUL vs LT/FUL; \ac{hr}=0.21) as well as when combined only with letrozole (RIB-LT vs LT; \ac{hr}=0.27). Additionally, palbociclib exhibits significant improvement in \ac{pfs} when combined with letrozole  (PAL-LT vs LT; \ac{hr}=0.65) (figure \ref{fig:grouped} right).
When comparing with propensity scores weighting, we found out that ribociclib is significantly better than palbociclib for \ac{pfs} and \ac{os}, providing a median \ac{os} of over 40 months and median \ac{pfs} of around 42 months. Adjusted for the weighted variables, Ribociclib is not significantly better for \ac{pfs}, but has a \textit{P} value of 0.013 for \ac{os} with an \ac{hr}of 0.48. However, the Cox regression adjusted for variables and weights are not significant, even when the \textit{P} value for \ac{pfs} is 0.05. This suggests that a more in depth analysis may be necessary.



    \subsection{Conclusion}
    In conclusion, our findings underscore the efficacy of \ac{cdk46i} in real-world settings. We can confidently affirm the impact of Ribociclib on \ac{pfs}. This assertion aligns with clinical trial outcomes and real-world data further substantiates these findings. However, we cannot do the same for \ac{os}. Our results indicate that Ribociclib combined with letrozole or fulvestrant when compared to both is not superior to these alternatives used alone. The same happens when comparing ribocilib combined with letrozole with letrozole alone. 
However we cannot do so for Palbociclib. Palbociclib combined with fulvestrant or letrozole was not significantly better than letrozole or fulvestrant alone for \ac{pfs} nor \ac{os}. This is something interesting that we want to follow up with.
Delving deeper into the characteristics of the patient population, including safety profiles, economic implications, and quality of life metrics, would be insightful. Additionally, a thorough examination of biomarkers within the population could offer invaluable insights. Finnaly, extending the follow-up period would be benefitial as well. We intend to explore these facets in subsequent publications.
It’s imperative to note that our data is sourced from a singular institution, limiting the capability of generalization of our results to a broader population. Nonetheless, we posit that this study lays a foundational groundwork for future research in this domain. While our evidence is rooted in observational data, and we’ve made adjustments for known confounders, the potential for residual confounding remains. Although the use of propensity score matching enhances the comparative robustness between the groups, the presence of unmeasured confounders cannot be entirely ruled out. Furthermore, the small sample size of our study limits the statistical power of our findings. For next steps we aim to further analyse the clinical variables that have an impact on the outcome of the combination of CDK4/6 with fulvestrant or letrozole and these drugs used alone in order to infer pharmaeconomic implications and possible profiles of patient that would not benefit from this combination which would be vital for economical reasons and to apply in countries with low access to these drugs.
    
    
    \section{How Can We Leverage Data To Create Clinical Decision Support Systems?}\label{subsec:obs}
    This section is based on the paper entitled "Machine-learning in Obstetrics: FHIR-based Support System for predicting delivery type". This work was in part a result of the work in section \ref{subsec:distributed}. While testing for distributed mechanisms, we kind of felt that some evaluation metrics were inspiring to pursue this further. We built a \ac{cdss} system that is interoperable and aims to provide support for subpar evaluation of a \ac{cs}.
    
    \subsection{Introduction}
    The ability to provide care to both women and newborns during delivery is one of the most important aspects of healthcare and is often used as a metric to assess healthcare as a whole across different countries.
\acp{cs} are one of the most important aspects of delivering babies since it has a considerable impact on the mother's health and well-being. Despite this type of procedure increasing over the last few years, it is still illusive the reasons behind such events. Reports from 2016 suggest that this increment is a global phenomenon, being that from 1990 to 2014, this type of delivery almost increases by 3-fold from 6.7\% to 19.1\% \cite{betranIncreasingTrendCaesarean2016,chenNonClinicalInterventions2018}. Some of these impacts, being more prone to investigation in the last years, including the risk of infection, haemorrhage, organ injury and complications related to the use of anaesthesia or blood transfusion \cite{caesereanrisk1,caesereanrisk2}.
There is also a higher risk of complications in subsequent pregnancies like uterine rupture, abnormal placental implantation and the need for hysterectomy \cite{caesereanrisk3,caesereanrisk4}. As for the infant, \acp{cs} include the risk of respiratory problems, asthma and obesity in childhood \cite{caesereanrisk3}.
Facing this, in 2015, World Health Organisation released a statement regarding \acp{cs} rates. Even when other complications could not be totally assessed, it was concluded that \ac{cs} rates higher than 10\% were not associated with a reduction in maternal or newborn mortality \cite{worldhealthorganizationhumanreproductionprogramme10april2015WHOStatementCaesarean2015}.

Since there is no evidence that this type of procedure is beneficial for women or babies when there is no clear need for it, the focus on filtering such cases is important \cite{chenNonClinicalInterventions2018}.
Moreover, particularly in Portugal, \acp{cs} are used as a way of financing healthcare institutions. This was implemented as a strategy of decreasing \ac{cs}s across the country. A committee was created especially with the purpose of reducing the percentage of \acp{cs} nationally. One of the actions taken along this creation was the reduction of government funding for hospitals with rates of \acp{cs} above 25\%.
In 2020, the number of \acp{cs} in Portugal is about 36.3\%. Almost at the all-time high of 36.9\% in 2009 \cite{pordatacesarianas}.
So, lowering the proportion of \ac{cs} can provide health and financial benefits to institutions and populations alike. With this in mind, we developed a  machine-learning algorithm-based support system to assist clinical teams to detect cases of potentially unnecessary \acp{cs} for analysis. So in this paper, we propose:
\begin{myitemize}
    \item help to provide a method of bringing to the discussion of clinical staff possible less than optimal care regarding deliveries;
    \item elaborates on how clinical decision support systems can be developed using interoperability standards;
    \item understand, based on the gathered data, which are the more impacting features for predicting delivery type outcome;
    \item open a research path regarding the evaluation of this type of clinical decision support system prior to the delivery;
    \item Perform a concise economical analysis to assess the potential financial impact of implementing the proposed clinical decision support tool.

\end{myitemize}
    \subsection{Rationale and Related Work}
    % !TeX root = ../../thesis.tex

Regarding the related work, several teams already tackled the potential of predicting the delivery type before birth. We found studies related to predicting a successful vaginal birth after a previous C-section, such as the work of Lipschuetz et al., \cite{lipschuetzPredictionVaginalBirth2020}  where a gradient boosting method was used to predict such an event using prenatal data to do so. Grobman et al., \cite{grobman_development_2007} performed a similar study with a multivariable logistic regression model. Different modalities of data were also used to predict delivery type. Fergus et al. \cite{fergusClassificationCaesareanSection2017} introduces a method of predicting de- livery type using the fetal heart rate signals. Similarly, the work from Saleem et al. \cite{saleemStrategyClassificationVaginal2019a} proposed a method for predicting delivery type using interactions between the fetal heart rate and maternal uterine contraction. Finally, there are also studies that focus on predicting the delivery mode like the work of Ullah et al. \cite{ullah_reliable_2021} where a boosting algorithm was used in order to predict delivery mode with enriched datasets. In addition, Gimovsky et al. \cite{gimovskyBenchmarkingCesareanDelivery} introduced decision trees to predict \ac{cs} by physician group with 0.73 \ac{auroc}. The works of \cite{rossiRiskCalculatorPredict2020b} resulted in a seven-variable model with 0.78 \ac{auroc} and the works of \cite{guedaliaRealtimeDataAnalysis2020} resulted in a model with 0.82 \ac{auroc}, reaching 0.93 with a first cervical examination. Finally, the works of Meyer et al. \cite{meyerImplementationMachineLearning2020} focused around selecting suitable for a trial of labor after cesarean with \ac{auprc} around 0.351. However, to the best of our knowledge, there was no model tested in clinical practice, with an interoperable format of communication like \ac{fhir}, which tried to not only predict delivery type but also provide support about possibly worn deliveries and none with simulation about financial implication, making our paper a potential novelty on different dimensions.
    \subsection{Methods}
    \subsubsection{Materials}
    Data was collected from nine different public Portuguese hospitals across the country, focusing on obstetric information, encompassing maternal data, various fetal data points, and the method of delivery in a retrospective manner. The data is from all patients that had information registered in the obstetrics EHR and had a registered outcome of the pregnancy from 2019 to 2020. Despite differing software versions across hospitals, each institution used identical EHR software, ensuring the columns remained consistent.

        
    We wrote all of the code in Python 3.9.7 with the usage of the \textit{scikit-learn} library \cite{scikit-learn}. All null representations were standardized. Data was prepossessed with the removal of features with high missing rates ($gt$ 90\% overall). All missing value representations were standardized. The imputation process was done using the \ac{knn} imputation method (for continuous variables) or a new category (NULLIMP) for categorical variables. 
For this purpose, the Birth Type was reduced to binary. All assisted birth were merged into vaginal birth and \ac{cs} remained as the other class. Procedures and diagnosis were also used and were encoded as binary features, we took the time to analyse each one of them in order to avoid leakage since there were procedures obviously related to \acp{cs} and vaginal deliveries.
Feature creation was done through the free-text variable relating to the medication prescribed. Features were collected from it and converted into \ac{atc} Classification Group level 4, which stands for chemical subgroups. We also created some new features from data in the dataset, namely new categories related to the labour and condition of the baby.
Also, a few data quality issues were addressed, like impossible values that were transformed into null. In this category, the main issues were \ac{bmi}/Weight and gestational age.
Finally, only a few columns were selected. We used a mixture of surveying the literature and the feature with greater correlation with the outcome.
The models tested were Logistic Regression, Decision Tree, Random Forest, 3 different Boosting methods (as implemented by \ac{xgboost}, \ac{lightgbm} and \textit{scikit-learn}) and a linear model based on Stochastic Gradient Descent.
The evaluation was done with repeated stratified cross-validation with 10 splits and 2 repetitions.
The API for serving the prediction model was developed with FastAPI.
%For trying to provide an explainable model, Shapley Additive exPlanations (SHAP) \cite{shapleyvalues} values were introduced in order to supply a more robust prediction.

Finally, a clinical evaluation was carried out with questionnaires sent to several obstetrics specialists in order to assess the validity and possible impact of the model.

    \subsection{Results}
    \subsubsection{Descriptive Statistics}
The number of samples varied across the hospitals, ranging from 2364 to 18177. Distributions of the selected variables are presented in table \ref{tab:obs_material_1}. The sum of all samples totals 73351.

\begin{table}[htbp]
  \centering
  \caption[Distribution of features used for prediction of delivery Type]{Distribution of features used for prediction, Mean and Standard Deviation (SD) for continuous variables. Mode and percentage for categorical variables. Number of samples is 73351.}
  \label{tab:obs_material_1}
   \renewcommand{\arraystretch}{1.02} % Adjust the vertical spacing
  \setlength{\tabcolsep}{12pt} % Adjust the horizontal spacing
    \begin{tabular}{m{15em}cc}
    \toprule
        Variable & M (SD) & Mode [\%] \\ 
        \hline
        Mother Age & 31.0 (5.6) & ~ \\ 
        Weight pre-pregnancy & 65.8 (13.9) & ~ \\ 
        Weight on admission & 78.6 (14.2) & ~ \\ 
        \ac{bmi} & 25.0 (5.4) & ~ \\ 
        Previous eutocic delivery & 0.4 (0.7) & ~ \\ 
        Previous vacuum-assisted delivery & 0.1 (0.3) & ~ \\ 
        Previous forceps & 0.0 (0.1) & ~ \\ 
        Previous \ac{cs} & 0.1 (0.4) & ~ \\ 
        Fetal presentation on admission & ~ & cephalic [26.323\%] \\ 
        Bishop score & 5.5 (3.0) & ~ \\ 
        Gestational age on admission & 38.9 (1.9) & ~ \\ 
        Premature rupture of the membrane & ~ & No [87.991\%] \\ 
        Chronic hypertension & ~ & No [97.676\%] \\ 
        Gestational hypertension & ~ & No [97.749\%] \\ 
        Preeclampsia & ~ & No [98.299\%] \\ 
        Gestational diabetes & ~ & No [89.811\%] \\ 
        Gestational diabetes treated with diet & ~ & No [94.285\%] \\ 
        Gestational diabetes treated with insulin & ~ & No [98.083\%] \\ 
        Gestational diabetes treated with oral antidiabetic drugs & ~ & No [97.797\%] \\ 
        Maternal Diabetes & ~ & No [99.509\%] \\ 
        Type 1 Diabetes & ~ & No [99.816\%] \\ 
        Type 2 Diabetes & ~ & No [99.843\%] \\ 
        Presentation at birth & ~ & Vertex presentation [94.000\%] \\ 
        Delivery & ~ & Spontaneous [53.864\%] \\ 
        Gestational age on birth & 39.0 (1.8) & ~ \\ 
        Smoking during pregnancy & ~ & No [88.442\%] \\ 
        Alcohol consumption during pregnancy & ~ & No [98.65\%] \\ 
        Consumed drugs during pregnancy & ~ & No [99.825\%] \\ 
        Nr of pregnancies (with current) & 1.9 (1.1) & ~ \\ 
        Pregnancy type & ~ & Spontaneous [85.417\%] \\ 
        Surveillance & ~ & yes [97.699\%] \\ 
        Hospital surveillance & ~ & yes [67.807\%] \\ 
        Pelvis Adequacy & ~ & Adequate [17.512\%] \\ 
        Consistency of the cervix & 1.6 (0.6) & ~ \\ 
        Fetal station & 0.8 (0.8) & ~ \\ 
        Dilation of the cervix & 1.3 (0.8) & ~ \\ 
        Effacement of the cervix & 1.2 (1.2) & ~ \\ 
        Position of the cervix & 0.6 (0.7) & ~ \\ 
        Haematologic disease & ~ & No [95.674\%] \\ 
        Respiratory disease & ~ & No [95.605\%] \\ 
        Cerebral disease & ~ & No [98.793\%] \\ 
        Cardiac disease & ~ & No [92.967\%] \\ 
        Neuroaxis techniques & ~ & 1 [69.5\%] \\ 
        Number of children  & 0.6 (0.8)  & ~\\ 
        \bottomrule
    \end{tabular}


\end{table}
The outcome variable had the following distribution as stated in table \ref{tab:delivery_methods}

\begin{table}[htbp]
  \centering
  \caption{Distribution of Delivery Methods}
  \label{tab:delivery_methods}
  \renewcommand{\arraystretch}{1.5} % Adjust the vertical spacing
  \setlength{\tabcolsep}{12pt} % Adjust the horizontal spacing
  \begin{tabular}{lc}
    \hline
    \textbf{Type of delivery} & \textbf{Frequency (\%)} \\
\hline
    \ac{cs} & 19 803 \, (27\%) \\

    Vaginal & 38 189 \, (52\%) \\

    Instrumental delivery & 15 359 \, (21\%) \\
    \hline
  \end{tabular}
\end{table}
%Vaginal      0.520634
%Cesariana    0.269976
%auxiliado    0.209390



\subsubsection{The Model}
The AUROC is presented in table \ref{tab:performancemetricsauc} for the best hyper-parameters found for each algorithm in the training data. All models used the variables indicated in table \ref{tab:obs_material_1}.
\begin{table}[htbp]
  \centering
  \caption[Performance Metrics in the training set]{Repeated Cross-validation (10x2) results in the training set with mean AUROC and 95\% Confidence Interval (CI) for best hyper-parameters found for each algorithm. Wilcoxon Test for comparing with the best performing algorithm.}
  \label{tab:performancemetricsauc}
  \renewcommand{\arraystretch}{1.5} % Adjust the vertical spacing
  \setlength{\tabcolsep}{12pt} % Adjust the horizontal spacing
  \begin{tabular}{lccc}
    \hline
    \textbf{Metric} & \textbf{AUROC} & \textbf{CI 95\%} \textit{\textbf{P value}} \\
    \hline
    XGBoost & 0.8809 & 0.8799, 0.882 & - \\  
    Decision Tree & 0.8337 & 0.8324, 0.8349 & $le$ 0.001\\
    Logistic Regression & 0.8716 & 0.8706, 0.8726 & $le$ 0.001\\
    AdaBoost & 0.8753 & 0.874, 0.8766 & $le$ 0.001\\ 
    lightgbm & 0.8805 & 0.8793, 0.8817 &  0.003\\ 
    Stochastic Gradient Descent & 0.8704 & 0.8694, 0.8713& $le$ 0.001\\ 
    Random Forest & 0.8752 & 0.8743, 0.8762& $le$ 0.001 \\  
    \hline
  \end{tabular}
\end{table}
While XGBoost was the best-performing algorithm, we selected LightGBM  \cite{lightgbm} because of its speed and lower memory requirements, which we believe are better suited for deployment in a low-hardware environment. The threshold selected for deploying the model was 0.7457 which rendered the metrics in the test set, as shown in table \ref{tab:performancemetricsthreshold}.

\begin{table}[htbp]
  \centering
\caption{Performance Metrics in the test set with chosen threshold}
\label{tab:performancemetricsthreshold}
\renewcommand{\arraystretch}{1.5} % Adjust the vertical spacing
\setlength{\tabcolsep}{12pt} % Adjust the horizontal spacing
\begin{tabular}{lc}
    \hline
    \textbf{Metric} & \textbf{Value} \\
    \hline
    Accuracy & 0.8052 \\
    Sensitivity & 0.8223 \\
    Precision & 0.9023 \\
    F1 Score & 0.8605 \\
    \hline
  \end{tabular}
\end{table}




\subsubsection{Deployment}
The purpose of this model is to serve as an API for usage within a healthcare institution and to act as a supplementary clinical decision support tool for obstetrics teams. For this to happen, a health information system must make the requests to the API. Even though a concrete, vendor-specific information model and input health information system were used, we hope to create a more interoperable clinical decision support system that can be used by every system that acts on birth and obstetrics departments. Therefore, we built it around the HL7 FHIR standard (R5 version) to simplify the method of interacting with the API. This decision, opposed as to using a proprietary model for the data, sits upon the usage of FHIR resources: Bundle and Observation for request and returning the result as a message through a custom operation called "\$predict". It is intended to publish the profiles of these objects in order to facilitate access to the API using standardized mechanisms and data models. The current build of the profiles can be seen in the published FHIR Implementation Guide where the current specifications are described in detail \url{https://joofio.github.io/obs-cdss-fhir/}. The process is illustrated in figure \ref{fig:deploy}. We deployed this model in production in a single hospital without a user interface, collecting only the data and predictions for later discussion and analysis. We collected 3231 requests. During this period, 123 (3.8\%) alarms were triggered. From this, we tried to understand the level of certainty for the decision and check the difference from the threshold of these alarms. The distance to the threshold for 73 was lower than 0.1 and was bigger than 0.1 for 50 (1.55\%) cases.


%TC:ignore

\begin{figure}[htbp]
\centering
\captionsetup{justification=centering}
\caption{Deployment and decision mechanism of the model}\label{fig:deploy} 
\includegraphics[scale=0.60]{figures/obs-model.jpg}
\end{figure}
%TC:endignore

\subsubsection{Clinical Evaluation}
The median scores given by each clinician are presented in figure 2. We also predicted the result using our model as stated in figure 2. The model misclassified only one record (4). As for the analysis of missing features for the responders, they were divided into 3 categories: 1) Existent in the dataset but not included in the model, 2) Non-existent in the dataset and 3) existent in the dataset and included but that particular information was not filled for the patient assessed. This rendered a total of 62 \% non-existent and 38 \% existent but no information was provided at that moment. No feature mentioned existed in the dataset but had not been included in the model. From the non-existent, 38 \% were new clinical assessments, 38\% were linked to information from previous births, 15\% connected in more in-depth information about provided information (i.e, motive for induction) and 11\% were related to the mother's choice (if she wanted a C-section). As for feature importance, from the 60 answers, we got 55 \% with labor being the most important factor. 15 \% answered the number of previous vaginal births, 8 \% the evolution of weight and another 8\% the number of previous C-sections. The remaining 14\% were various features, from BMI, neuroaxis techniques, gestational age and weight of the mother. Of all of these, 90 \% were in the top 10 features of the model.



%TC:ignore

\begin{figure}[htbp]
\centering
\captionsetup{justification=centering}
\caption[Obstetrics questionnaires data]{Validation data. The colour represents the actual birth type. The boxplot represents the median and \ac{iqr} of the reviewers and the X represent each patient case. Contains 6 Vaginal births and 4 \acp{cs}. * represents wrong predictions of the model. (ID: 4)}\label{fig:clinical} 
\includegraphics[scale=0.60]{figures/clinical_assessment.png}
\end{figure}
%TC:endignore




\subsubsection{Potential Financial Impact}
The financial support provided to public hospitals in Portugal is partially tied to the rate of C-sections. To assess the potential impact of this mechanism on Portuguese public hospitals, we conducted a simulation. We got data for every public hospital for the last 12 months and applied a 3.8\% reduction (the rate of warnings triggered in the new dataset) and recalculated the rate of C-sections. The increase in support was calculated by the state-mandated rate as shown in table \ref{tab:corrections}. With this new rate, we observed that implementing our tool would result in financial benefits for 30\% (11 hospitals) of the public hospitals. Specifically, five hospitals would begin receiving support instead of no support at all. Three hospitals would experience a doubling of their financial benefit, while two hospitals would see a 50\% increase. Furthermore, one hospital would receive an additional one third of financial support. If we assumed that only half of the warnings found in the new data were actually true (1.9\%) we found that only 6 hospitals would be benefited. 3 from 0 to 0.25, 2 from 0.25 to 0.50 and 1 from 0.50 to 0.75.


\begin{table}[htbp]
  \centering
  \caption[Ruleset for financial support indexed to \acsp{cs}.]{Ruleset for state-provided financial support indexed to \acp{cs}. X is the current payment of a \ac{cs} inpatient episode. Adapted from \cite{acssTermosReferenciaPara2023}}
  \label{tab:corrections}
  \renewcommand{\arraystretch}{1.2} % Adjust the vertical spacing
  \setlength{\tabcolsep}{12pt} % Adjust the horizontal spacing
  \begin{tabular}{lc}
      \hline
Rate of \acp{cs}   & Support \\
    \hline
\textless 25\%       & x       \\
{[}25\%, 26.4\%{]}   & 0.75 x   \\
{[}26.5\%, 27.9\%{]} & 0.5 x    \\
{[}28\%, 29.4\%{]}   & 0.25 x   \\
\textgreater{}29.5\% & 0      \\
    \hline
\end{tabular}
\end{table}
    \subsection{Discussion}
    % !TeX root = ../../thesis.tex

The first thing to address about this model is the number of biases that we introduced in the model by choice. We joined all vaginal delivery types into a single category (assisted and non-assisted) which introduces a bias since these delivery modes are indeed different. Secondly, the fact that we want to predict if the delivery type was wrongly chosen, mainly for the case of a C-section that did not need to be so, is also a bias. We used this approach because the initially collected data did not have the representation of such events. So the biases of possibly wrong delivery types were present in the training data. We attempted to minimize this issue by selecting a threshold that gave the model higher sensitivity than specificity so that only large probabilities would trigger an alarm for human consideration. Parallel to this, we are starting to gather labeled cases, with the help of clinicians in order to create a better training dataset. Furthermore, since the data was collected from different hospitals, differences in the data input can also occur. Even though the health information system is the same, the processes that originate the data and are being used for secondary purposes could introduce several biases in the data. This is an issue that was accepted from the start regarding the mechanism of data collection and model training. Despite this, we reached a model with a very high AUROC (~88\%, 95\%CI [0.8795, 0.8815]), which is encouraging and versus the state of the art. Moreover, assuming that more data is provided and proper labeling is done regarding the outcome variable (like a clinical evaluation of needless C-sections) is added as well, a better model could be developed. Regarding the preliminary clinical evaluation, it was only possible to get an overview of the possible comparison due to the number of responders. Despite that, the results are encouraging, since the model seems to behave better than humans with the data provided. However, this is a biased vision, since clinicians in the real world have access to more data and information than the model has. It is encouraging, but caution is advised before more tests and evaluations are done. As for the deployment, future work could be the improvement of the API in order to map all variables to an ontology like snomed CT or similar, making it easier for every system and person to access it and get a suggestion of the delivery type. Finally, we believe the assessment can be improved. A more robust clinical assessment is necessary as well as a thorough analysis of the impact of the tool in the real world, since we need to create the bridge between the results of the model and how clinical decisions are affected by it. A full cost-effectiveness analysis is also necessary to understand the real world impact of the model. One interesting result is the fact that 38\% of the answers regarding the most important data element missing from the patient record refers to data that is being collected but was missing for that specific patient, raising an important question about data input methodology, interoperability and quality. If we cannot have access to data when it matters most, it can become meaningless. Missing data is a problem of biomedical data as a whole. However, when specifically targeted at machine learning usage of this data for predicting something, we did not find any works comparing them with clinicians. However, we did find reportings of similar missing values in obstetrics data \cite{venkateshMachineLearningStatistical2020} and we also found works of similar nature using machine learning models with a robust handling of missing data such as XGBoost \cite{bitarMachineLearningAlgorithm2023} to counter this problem. This indicates that our model has the potential to counter the missing data problem as well since LightGBM can also handle missing data natively.

    \subsection{Conclusion}
    We believe that we developed a fairly robust system for alarming for possibly wrong \acp{cs}, that could have a positive impact in real-world practice. However, there are issues to tackle before doing so. There is a need for further evaluating the impact of such a system on clinical practice decisions. There could be a very wide range of reasons that could lead to a possibly sub-par decision regarding delivery type. From the mother's decision to a lack of information at the decisive moment. This system is not meant to create hurdles regarding practice or to point out defective decisions putting certain professionals under the spotlight.
All the underlying assumptions and prejudices about having autonomous systems providing support for practice must be taken into account. Nevertheless, the metrics and results so far are definitely encouraging for having a positive impact on health and economic outcomes.

    
    
    
    