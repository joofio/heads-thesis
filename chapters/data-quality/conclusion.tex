This work is still an early draft of a production-ready tool. However, we feel the work done is already a valuable insight into how to use data quality frameworks and several statistical tools in order to assess ehr data quality. This is a fundamental process not only to guarantee the quality of data for primary usage on a day-to-day but also for securing quality for secondary analysis and usage.
We believe the fact that we created an interoperable tool that was trained on real obstetrics data from 9 different hospitals and has the ability to provide a single score for a clinical record can help institutions, academics, and ehr vendors implement data quality assessment tools in their own systems and institutions.

For the next steps, we would like to further evaluate the score and its relationship with clinical usefulness. This would also include a further assessment of a  threshold for the score for defining a record that would require human attention.
