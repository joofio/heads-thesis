For completeness, we used the inverse of the percentage of nulls in the training set. 

For plausibility, several methods were applied. The first was a \ac{bn}. \acp{bn} are probabilistic graphical models that represent a set of variables and their conditional dependencies via a directed acyclic graph. These networks are particularly adept at simultaneously predicting multiple variables, offering a cohesive framework for inferring several columns within a single network structure \cite{pearl1988probabilistic}.
We used this model due to the possibility of using a single model for classifying the plausibility of all columns and due to its interpretable nature. The networks were created with the pgmpy package \cite{pgmpy}. Secondly, we added the outlier-tree method \cite{cortesExplainableOutlierDetection2020} which tries to integrate a decision tree that ”predicts” the values of each column based on the values of each other column. In the process, every time separation is evaluated, it takes observations from each branch as a homogeneous cluster to search for outliers in the predicted 1-d distribution of the column. Outliers are determined according to confidence intervals in this 1-d distribution and need to have large gaps in order to be marked as outliers in the next observation. Because it looks for outliers in the branch of the decision tree, it knows the conditions that make it a rare observation relative to other observation types corresponding to the same conditions, and these conditions are always related to target variables (as predicted by them). As such, it can only detect outliers described by decision tree logic, and unlike other methods such as isolation forests, it cannot assign outlier points to each observation, or detect outliers that are generally rare, but will always provide human-readable justification when it recognizes outliers. Therefore, these methods not only identify anomalies based on a single column/variable but also consider the context of the data, providing a more nuanced understanding of what constitutes an outlier. This contextual awareness ensures that the outliers are not merely statistical deviations but are also substantively significant within the specific framework of the target variables. We added also elliptic envelope and Local Outlier Factor as complementary models to these two. Finally, a rule system was implemented to leverage domain knowledge in the overall scoring. The system is based on the \textit{great\_expectations} package \cite{GXProactiveCollaborative}. A set of rules was defined by the team, focusing on impossible numbers present in age, weight, or relationship between variables. The rules covered plausibility and conformance. The Conformance-based were related to technical issues like the format of dates, and conformance to the value set (i.e. Robson group, bishop scores, or delivery types). Plausibility rules were linked to expected values for BMI, weight, and gestational age. We also added plausibility for the relationship between columns, namely weight across different weeks of gestation. We have also added a relationship of greatness between ultrasound weights more than 5 weeks apart. 

As for preprocessing, all null representations were standardized, we also removed features with high missing rates ($>$ 80\% ). The imputation process was performed with the median for continuous and a new category (NULLIMP) for categorical variables. 

For the usage of the \acp{bn} in particular, the continuous variables were discretized into three bins defined by quantile. We defined three as the number of bins in order to reduce the number of states in each node of the network. The evaluation was done with cross-validation with 10 splits and two repetitions for each column as the target.
The \ac{api} for serving the prediction models was developed with FastAPI. So, the methods applied in terms of the DQA framework shown in figure \ref{fig:categories} are described in the table \ref{tab:methods}.


As for  Z-Scores, they were defined for all continuous variables based on the interquartile range. Then, rows were also assessed with distance analysis, with Local Outlier Factor and Elliptic Envelope from \textit{scikit-learn} and the outlier-tree algorithm. We also added a rule engine, using the  package. Rules were defined by the team, focusing on impossible numbers present in age, weight, or relationship between variables. As for missing information was created with all the data, creating the scoring based on the inverse of the missing percentage. Missing detection was based on primary key variables. For completeness, we used the inverse of the percentage of nulls in the training set.
The API for serving the prediction models was developed with FastAPI. So, the methods applied in terms of the DQA framework shown in figure \ref{fig:categories} are described in the table 

\begin{table}[htpb]
\caption{Implemented Methods in the tool. The first column is the category or data quality dimension. The second is a subcategory of the first column if applicable and the third column is the actual method used to assess such a dimension.} \label{tab:methods}
\renewcommand{\arraystretch}{1.4}
\setlength{\tabcolsep}{10pt}

\begin{tabularx}{\textwidth}{ p{2cm} p{3.5cm} X }
\hline
 Category   & Subcategory           & Method   \\ \hline
Completeness     & N/A               & Score by the inverse percentage of missing in the train data         \\ 
Plausibility & Atemporal Plausibility & Bayesian model prediction based on the other values of row \\ 
Plausibility & Atemporal Plausibility         & Z-score for column value based on \acs{iqr} train data       \\    
Plausibility & Atemporal Plausibility           & Elliptic Envelope                       \\ 
Plausibility & Atemporal Plausibility           & Local Outlier Factor                \\ 
Conformance & Value Conformance           & Manual Rule engine                           \\ 
Plausibility & Atemporal Plausibility           & Manual Rule engine                      \\ 
Plausibility & Atemporal Plausibility           & outlier-tree                      \\ 
\hline
\end{tabularx}

\end{table}

For trying to compile all of these models into a single value, that could grasp the quality of the row or patient, a scoring method was created. The method of calculating the final score is stated in figure \ref{fig:scores}. To assess the tool’s usefulness, we implemented it in a production environment and collect metrics regarding the data being produced. Then we presented some rows (or patient's records) to selected obstetrics clinicians for them to assess how likely the information is to be suitable for usage or rank it according to the perceived quality of the record. We also compared the results with the ones from the model to make sanity checks regarding the model’s performance and adequacy. We used Kendal Tau and Average Spearman’s Rank Correlation Coefficient. Kendall Tau is a non-parametric statistic used to measure the strength and direction of the association between two ordinal variables. It calculates the difference between the number of concordant and discordant pairs of observations, normalized to ensure a value between -1 (perfect disagreement) and 1 (perfect agreement). Spearman’s rank correlation coefficient is a non-parametric measure that assesses the strength and direction of a monotonic relationship between two ranked variables. It is based on the ranked values of the variables rather than their raw data, producing a value between -1 (perfect inverse relationship) and 1 (perfect direct relationship).
We wrote all the code in Python 3.10.6 with the usage of the \textit{scikit-learn} library for preprocessing, and evaluation \cite{scikit-learn}.



