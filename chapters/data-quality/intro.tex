% !TeX root = ../../thesis.tex

With the wide spreading of healthcare information systems across all contexts of healthcare practice, the production of health-related data has followed this incremental behaviour. The potential for using this data to create new clinical knowledge and push medicine further is tempting \cite{martin-sanchezBigDataMedicine2014}.
However, to correctly use the data stored in \acp{ehr}, the quality of the data must be robust enough to sustain the clinical decisions made based on this data. Data quality cannot be construed as a linear concept; it is intrinsically dependent on the context in which it is evaluated. The quality thresholds and dimensions required to classify the quality of the data depend on the purpose that we intend to use that very same data \cite{waljiElectronicHealthRecords2019}. These uses can be very distinct and have different impacts as well. For one, we can use data to support day-to-day decisions regarding individual patients’ care \cite{verheijPossibleSourcesBias2018}. These decisions can include ones based on recorded information to understand a patient’s history, clinical decision support systems based on this data, or even using the data to help support a more macro, public health-oriented decision. Another area is using information for management purposes. The data can be used by management bodies and regulatory authorities to extract metrics regarding the quality of care or reimbursement purposes. Thirdly, data can be used for research purposes, namely observational studies and, more recently, to support clinical trials through real-world evidence analysis \cite{coreyAssessingQualitySurgical2020,verheijPossibleSourcesBias2018,wengClinicalDataQuality2020}. 
So, all the \ac{ehr} data-based decisions can only be as good as the data supporting them. Several studies have already warned about the lack of data quality in \acp{ehr} and how this can be a significant hurdle to an accurate representation of the population and potentially lead to erroneous healthcare decisions \cite{reimerDataQualityAssessment2016a,joukesImpactElectronicPaperBased2019a,huserMultisiteEvaluationData2016,zhangUnderstandingDetectingDefects2020,kramerImpactDataQuality2021,gigantiImpactDataQuality2019}.

There are several steps in the data lifecycle that can be prone to error, from data generation, where the data is registered by healthcare professionals, passing by data processing, whether inside healthcare institutions or by software engineers aiming to reuse data, to data interpretation and reuse, where investigators
try to interpret the meaning of registered data \cite{wengClinicalDataQuality2020}.
So, with all of the data’s possible uses added to the several steps that can introduce errors throughout the data lifecycle, data quality frameworks and sequential implementations can have very distinct approaches and methodologies to assess data quality. Data quality tools for checking data being registered live to support day-to-day decisions will be significantly different from one whose only purpose is to provide quality checks for research purposes. So, methodologies to tackle these issues are necessary for guaranteeing the quality of healthcare practice and the knowledge derived from \ac{ehr} data. Consequently, in this paper, we propose:
\begin{myitemize}
    \item Create a tool for identifying data quality issues in obstetrics \acp{ehr};
    \item Enlighten on the issues that can appear with a full deployment of such a tool
    \item Suggestion of a creation of a single score for data quality for comparison of high-quality and low-quality records in a database.
    \item Assess how such a tool can work in early-stage real-world scenarios and how to work with obstetricians to improve data quality.
    \item Identify data quality issues on obstetrics data
\end{myitemize}


%Data quality is a crucial aspect of the healthcare industry, as it impacts the accuracy of diagnoses, treatment plans, and patient outcomes. The reliability and accuracy of healthcare data have far-reaching consequences, including financial implications, patient safety, and legal ramifications. Inaccurate or incomplete data can lead to incorrect diagnoses, inappropriate treatments, and ultimately harm to patients. Therefore, ensuring the quality of healthcare data is essential to providing effective and safe healthcare services.

%One of the main reasons why data quality is so critical in healthcare is that healthcare data is often used to make important decisions, such as treatment plans, patient management, and resource allocation. Inaccurate or incomplete data can lead to misdiagnosis, inappropriate treatment, and increased healthcare costs. Furthermore, inaccurate data can hinder research efforts and impede the development of new treatments and therapies.

%Another key aspect of data quality in healthcare is its role in patient safety. Accurate and reliable data is essential for ensuring patient safety, particularly in areas such as medication management, clinical decision-making, and adverse event reporting. Poor data quality can lead to medication errors, adverse drug reactions, and other types of harm to patients.

%Finally, data quality is also important for legal and regulatory compliance in healthcare. Accurate and complete data is required for compliance with regulations such as HIPAA, the Affordable Care Act, and other regulatory requirements. Poor data quality can result in legal and financial penalties, as well as reputational damage for healthcare organizations.

%Overall, data quality is a crucial aspect of healthcare, with far-reaching implications for patient safety, healthcare costs, and regulatory compliance. Ensuring the quality of healthcare data requires a comprehensive approach that includes data governance, data management, data quality assurance, and ongoing monitoring and improvement efforts. By prioritizing data quality, healthcare organizations can provide better patient care, improve outcomes, and reduce costs.




