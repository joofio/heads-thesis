\ac{ai} has experienced unprecedented advancements in the last decade, leading to its integration in various domains, including medicine. It has been instrumental in transforming clinical decision-making, drug discovery, patient monitoring, and predicting disease trajectories. Despite these advancements, the "black box" nature of complex \ac{ai} models poses interpretability challenges, limiting their widespread adoption in healthcare, a field where transparency, reliability, and understanding of decision-making processes are vital. This lack of interpretability, also known as opacity, can lead to misdiagnoses, inappropriate treatment plans, and, most importantly, breaches in trust among clinicians, patients, and \ac{ai} systems.

As such, the concept of \ac{xai}, which aims to create a suite of techniques that produce more explainable models while maintaining a high level of predictive accuracy, has gained significant attention in medical \ac{ai} research. \ac{xai} seeks to bridge the gap between \ac{ai} opacity and human interpretability, and in doing so, it can enhance the transparency, reliability, and acceptance of \ac{ai} applications in the healthcare setting.

So, for this to happen, we need a new framework for applying such mechanisms. A new step that could be attached to the ones seen before in section \ref{sec:kdd} will enable human comprehension of the model's output.

Even though several grouping and taxonomies of \ac{xai} are available mentioned in \cite{adadiPeekingBlackBoxSurvey2018,linardatosExplainableAIReview2020,barredoarrietaExplainableArtificialIntelligence2020,linardatosExplainableAIReview2020,kamath2021explainable}, a simplified approach based on \cite{kamath2021explainable} will be used in order to contextualize this concept.

We can divide it into two main categories. Firstly the explanation type is divided into global and local. Local and global explanations are methods used to interpret machine learning models, especially those that are considered "black box" models, such as deep learning networks. These methods help us understand why and how a model makes certain decisions, which can be crucial in many settings for ethical, legal, and practical reasons.

Local Explanations: These involve understanding the prediction of a \ac{ml} model for a specific individual instance. They help to answer questions like: "Why did the model predict that this particular patient has cancer?" or "Why was this specific transaction flagged as fraudulent?". 

Global Explanations: These focus on understanding the model behaviour across all instances, or more broadly on a dataset-wide level. They help to answer questions like: "What features are generally important for prediction in the model?" or "What is the overall logic of the model?". 

Secondly, we have the method type, where we have 3 main subcategories related to the stage of the data science process it is applied, \textit{pre}, during, and \textit{post}-model training.

\textbf{Pre-Model \ac{xai}:} These methods involve improving the transparency and interpretability of models before they are even trained. This includes thoughtful feature engineering, \ac{eda}, and applying domain knowledge to create meaningful variables. The goal is to design a model that will be more interpretable from the onset.

\textbf{Intrinsic \ac{xai}:} This involves using machine learning models that are intrinsically explainable. These models are designed in such a way that their decision-making process is understandable by default. Examples include linear and logistic regression, cox regressions, decision trees, Naïve Bayes, \ac{bn}, and rule-based models. While these models may sometimes lack the predictive power of more complex models, they provide clear interpretability: you can directly examine the impact of the variables and understand how the model makes its predictions.
\\
\textbf{Linear Regression} is a linear approach to modelling the relationship between a dependent variable and one or more independent variables. It assumes that the relationship between these variables is linear and can be represented by a straight line. The goal is to fit the best possible line that describes this relationship by minimizing the sum of the squared differences (errors) between the observed values and the values predicted by the line. Linear regression is widely used in various fields for prediction, modelling, and determining the strength and character of the relationship between variables. It forms the basis of many more complex statistical modelling techniques.
\\
\textbf{Logistic Regression} is used to model the probability of a binary outcome that depends on one or more independent variables. Unlike linear regression, which predicts a continuous outcome, logistic regression predicts the probability of a categorical outcome (e.g., success/failure, yes/no, 1/0). The logistic function is applied to the linear combination of independent variables to ensure that the estimated probabilities are between 0 and 1. It's often used in fields like medicine, economics, and social sciences to predict the likelihood of an event occurring based on various factors.
\\
\textbf{Cox Regression} or the Cox proportional-hazard's model, is a statistical technique used for investigating the effect of several variables on the time a specified event takes to happen. In medical research, this often refers to survival times. The model allows for the estimation of hazard ratios, which describe how the hazard changes with a one-unit change in the predictor variable. The Cox model makes an assumption that the hazard ratios are constant over time, known as the proportional hazard's assumption. This model is vital for understanding how different factors influence survival or failure time and is commonly applied in epidemiological and medical research.

\textbf{Bayesian Networks}
A \ac{bn}, also known as a belief network or \ac{dag} model, is a probabilistic graphical model that represents a set of random variables and their conditional dependencies via a \ac{dag}. 

Given a set of variables \(X = \{X_1, X_2, ..., X_n\}\), the joint probability distribution is given by:

\[
P(X_1, X_2, ..., X_n) = \prod_{i=1}^{n} P(X_i | Parents(X_i))
\]

where \(Parents(X_i)\) is the set of parent variables of \(X_i\) in the network.

This formula represents the factorization of the joint distribution over \(X\), based on the graphical structure of the Bayesian network.

Now, in the Bayesian network, each node is conditional independent of its non-descendants given its parents. If we denote \(ND(X_i)\) as the set of non-descendants of \(X_i\) and \(Pa(X_i)\) as the parents of \(X_i\), the conditional independence is described as:

\[
X_i \perp ND(X_i) | Pa(X_i)
\]

This means that \(X_i\) is conditionally independent of its non-descendants given its parents. 

A common task for Bayesian networks is inference, which means computing the posterior probability of a set of query variables \(Q\), given some observed variables \(E\). That is, we want to compute \(P(Q|E)\). According to the Bayes rule, we have:

\[
P(Q|E) = \frac{P(Q,E)}{P(E)} = \frac{P(Q,E)}{\sum_{q \in Q} P(Q=q,E)}
\]

where the denominator is a normalization constant ensuring the result is a valid probability distribution. Note that performing this inference is NP-hard, which is why various approximation algorithms have been developed.

\textbf{Tree based methods}
Tree-based machine learning methods are a subset of algorithms that use a tree-like graph structure for making decisions or predictions. The most basic type is the Decision Tree, where the tree is used to go from observations about an item to conclusions about the item's target value (classification or regression). Each node in the tree represents a feature in the dataset, each branch represents a decision rule, and each leaf node represents the output value. More advanced tree-based methods include Random Forests, which build multiple Decision Trees and average their predictions for better accuracy and generalization, and gradient-boosted trees, which build trees sequentially, each one correcting the errors from the previous one.

The major advantage of tree-based methods is their ease of interpretation and understanding, especially for Decision Trees. However, a single tree is often prone to overfitting, where it performs well on the training data but poorly on unseen data. This is why ensemble methods like Random Forest and Gradient Boosting are popular; they aim to increase robustness and predictive power by combining multiple trees. These methods are widely used in various domains including but not limited to finance, healthcare, and natural language processing for tasks like classification, regression, and even unsupervised learning tasks like clustering.

\textbf{Post-Hoc \ac{xai}:} Post-hoc methods are applied after a model has been trained, to try to explain its decisions. This includes techniques like feature importance analysis, partial dependence plots, \ac{lime}, \ac{shap}, and counterfactuals. For instance, \ac{lime} can be used to create local explanations for individual predictions made by any model, and \ac{shap} values can be used to interpret the impact of features on the model's output both locally and globally. Counterfactuals try to explain a model by example, providing possible changes that would alter the outcome provided by the model.

It is to be noted that a methodology can be classified into two categories. For example, \ac{lime} is a local explanation model in a \textit{post-hoc} manner.

Despite all of this, we  have to take into account that pre-model and post-hoc methodologies are a proxy for an explanation of the models. That is why we could argue, as stated in \cite{rudinStopExplainingBlack2019} that only an intrinsically transparent model can really be the basis of \ac{xai}. While \textit{post-hoc} of pre-model methods are only a potentially unreliable proxy for an explanation.
%explicar mais disto e fechar.
